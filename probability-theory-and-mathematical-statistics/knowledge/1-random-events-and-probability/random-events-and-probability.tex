\documentclass[UTF8, 12pt]{ctexart}
% UTF8编码,ctexart现实中文
\usepackage{color}
% 使用颜色
\definecolor{orange}{RGB}{255,127,0}  
\definecolor{violet}{RGB}{192,0,255}  
\definecolor{aqua}{RGB}{0,255,255} 
\usepackage{geometry}
\setcounter{tocdepth}{4}
\setcounter{secnumdepth}{4}
% 设置四级目录与标题
\geometry{papersize={21cm,29.7cm}}
% 默认大小为A4
\geometry{left=3.18cm,right=3.18cm,top=2.54cm,bottom=2.54cm}
% 默认页边距为1英尺与1.25英尺
\usepackage{indentfirst}
\setlength{\parindent}{2.45em}
% 首行缩进2个中文字符
\usepackage{setspace}
\renewcommand{\baselinestretch}{1.5}
% 1.5倍行距
\usepackage{amssymb}
% 因为所以
\usepackage{amsmath}
% 数学公式
\usepackage[colorlinks,linkcolor=black,urlcolor=blue]{hyperref}
% 超链接
\author{Didnelpsun}
\title{随机事件与概率}
\date{}
\begin{document}
\maketitle
\pagestyle{empty}
\thispagestyle{empty}
\tableofcontents
\thispagestyle{empty}
\newpage
\pagestyle{plain}
\setcounter{page}{1}
\section{基本概念}

\subsection{随机试验}

\textcolor{violet}{\textbf{定义:}}满足三个条件的就是随机试验:

\begin{enumerate}
    \item 试验可以在相同的条件下重复进行。
    \item 试验所以可能结果都是明确可知,且不止一个。
    \item 每次试验的结果事先不确定。
\end{enumerate}

随机试验也称为\textbf{试验},并用$E_1,E_2,\cdots$来表示。

\subsection{随机事件}

\textcolor{violet}{\textbf{定义:}}一次试验中可能出现也可能补出现的结果称为\textbf{随机事件},简称\textbf{事件},并用大写字母$A,B,\cdots$来表示。

必然事件\textcolor{violet}{\textbf{定义:}}每次试验中一定发生的事件,记为$\Omega$。

不可能事件\textcolor{violet}{\textbf{定义:}}每次试验中一定不发生的事件,记为$\varnothing$。

\subsection{样本空间}

随机试验的每一个不可再分的可能结果称为\textbf{样本点},记为$\omega$,样本点的全体组成的集合称为\textbf{样本空间}或\textbf{基本事件空间},记为$\Omega$,即$\Omega=\{\omega\}$。

由一个样本点构成的事件称为\textbf{基本事件}。

随机事件$A$总是由若干个基本事件构成,即$A$是$\Omega$的子集。

样本点的个数就是基本事件的个数。

\section{事件}

\subsection{关系}

若事件$A$发生必然导致事件$B$发生,则称事件$B$\textbf{包含}事件$A$(或$A$被$B$包含),记为$A\subset B$。

如果$A\subset B$且$B\subset A$,则称事件$AB$\textbf{相等},记为$A=B$,$AB$是由完全相同的一些试验结果构成,是同一事件表面上看来两个不同说法。

若事件在事件$A$与$B$同时发生,则称为事件$A$与$B$的\textbf{积}或\textbf{交},记为$A\cap B$或$AB$。

有限个事件$A_1,A_2,\cdots,A_n$同时发生的事件为事件$A_1,A_2,\cdots,A_n$的积或交,记为$\bigcap\limits_{i=1}^nA_i$或$\bigcap\limits_{i=1}^\infty A_i$。

若$AB\neq\varnothing$,则称事件$AB$\textbf{相容},否则\textbf{互不相容}或\textbf{互斥}。如果一些事件中任意两个事件都互斥,则这些事件\textbf{两两互斥},简称互斥。

事件$AB$至少有一个发生的事件称为事件$AB$的\textbf{和}或\textbf{并},记为$A\cup B$。

有限个事件$A_1,A_2,\cdots,A_n$至少有一个发生的事件为事件$A_1,A_2,\cdots,A_n$的和或并,记为$\bigcup\limits_{i=1}^nA_i$或$\bigcup\limits_{i=1}^\infty A_i$。

事件$A$发生而事件$B$不发生的事件为事件$AB$的\textbf{差},记为$A-B$。

事件$A$不发生的事件为事件$A$的\textbf{逆事件}或\textbf{对立事件},记为$\overline{A}$。

若$\bigcup\limits_{i=1}^nA_i$或$\bigcup\limits_{i=1}^\infty A_i=\Omega$,$A_iA_j=\varnothing$(对一切的$i\neq j$,$i,j=1,2,3,\cdots,n,\cdots$),则称有限个事件$A_1,A_2,\cdots,A_n$构成一个\textbf{完备事件组}。

\subsection{运算}

定义可知:$A-B=A-AB=A\overline{B}$,$B=\overline{A}$等价于$AB=\varnothing$且$A\cup B=\Omega$。

\begin{enumerate}
    \item 吸收律:若$A\subset B$,则$A\cup B=B$,$A\cap B=A$。
    \item 交换律:$A\cup B=B\cup A$,$A\cap B=B\cap A$。
    \item 结合律:$(A\cup B)\cup C=A\cup(B\cup C)$,$(A\cap B)\cap C=A\cap(B\cap C)$。
    \item 分配律:$A\cap(B\cup C)=(A\cap B)\cup(A\cap C)$,$A\cup(B\cap C)=(A\cup B)\cap(A\cup C)$,$A\cap(B-C)=(A\cap B)-(A\cap C)$。
    \item 对偶律(德·摩根律):$\overline{A\cup B}=\overline{A}\cap\overline{B}$,$\overline{A\cap B}=\overline{A}\cup\overline{B}$。(长杠变短杠,开口换方向)
\end{enumerate}

\textbf{例题:}判断$A-(B-C)=(A-B)\cup C$是否成立。

解:$\because A-B=A\overline{B}$,$\therefore A-(B-C)=A-B\overline{C}=A\overline{B\overline{C}}=A(\overline{B}\cup C)=A\overline{B}\cup AC=(A-B)\cup AC\neq (A-B)\cup C$。

\section{概率}

\subsection{定义}

\begin{itemize}
    \item 描述性定义:将随机事件$A$发生的可能性大小的度量(非负)称为事件$A$发生的概率,记为$P(A)$。
    \item 统计性定义:在相同条件下做重复试验,事件$A$出现的次数$k$和总的试验次数$n$之比$\dfrac{k}{n}$,称为事件$A$在这$n$次试验中出现的\textbf{频率},当$n$充分大时,频率将稳定与某常数$p$附近,$n$越大频率偏离这个常数$p$的可能性越小,这个常数$p$就是事件$A$的概率。
\end{itemize}

\subsection{概率模型}

\subsubsection{古典概型}

\subsubsection{几何概型}

\subsection{性质}

\subsection{公式}

\section{独立性}

\end{document}
