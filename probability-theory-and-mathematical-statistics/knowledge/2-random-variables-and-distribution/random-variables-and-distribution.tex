\documentclass[UTF8, 12pt]{ctexart}
% UTF8编码,ctexart现实中文
\usepackage{color}
% 使用颜色
\definecolor{orange}{RGB}{255,127,0}  
\definecolor{violet}{RGB}{192,0,255}  
\definecolor{aqua}{RGB}{0,255,255} 
\usepackage{geometry}
\setcounter{tocdepth}{4}
\setcounter{secnumdepth}{4}
% 设置四级目录与标题
\geometry{papersize={21cm,29.7cm}}
% 默认大小为A4
\geometry{left=3.18cm,right=3.18cm,top=2.54cm,bottom=2.54cm}
% 默认页边距为1英尺与1.25英尺
\usepackage{indentfirst}
\setlength{\parindent}{2.45em}
% 首行缩进2个中文字符
\usepackage{setspace}
\renewcommand{\baselinestretch}{1.5}
% 1.5倍行距
\usepackage{amssymb}
% 因为所以
\usepackage{amsmath}
% 数学公式
\usepackage[colorlinks,linkcolor=black,urlcolor=blue]{hyperref}
% 超链接
\usepackage{tikz}
% 绘图
\author{Didnelpsun}
\title{随机变量及其分布}
\date{}
\begin{document}
\maketitle
\pagestyle{empty}
\thispagestyle{empty}
\tableofcontents
\thispagestyle{empty}
\newpage
\pagestyle{plain}
\setcounter{page}{1}
\section{一维随机变量}

\subsection{随机变量概念}

\textcolor{violet}{\textbf{定义:}}随机变量就是其值会随机而定的变量。设随机试验$E$的样本空间$\Omega={\omega}$,如果对每一个$\omega$都有唯一的实数$X(\omega)$与之对应,并且对任意实数$x$,$\{\omega|X(\omega)\leqslant x,\omega\in\Omega\}$是随机事件,则称定义在$\Omega$上的实值单值函数$X(\omega)$为\textbf{随机变量},记为随机变量$X$。

\subsection{分布函数}

\subsubsection{概念}

\textcolor{violet}{\textbf{定义:}}设$X$为随机变量,$x$为任意实数,称函数$F(x)=P\{X\leqslant x\}$($x\in R$且取遍所有实数)为随机变量$X$的分布函数,或称$X$服从分布$F(x)$,记为$X\sim F(x)$。(随着$x$从$-\infty$到$+\infty$,$X(\omega)$到$\varnothing$到$\Omega$)

\subsubsection{性质}

同样是分布函数的充要条件:

\begin{itemize}
    \item $F(x)$是$x$的单调不减函数,即对任意实数$x_1<x_2$,有$F(x_1)\leqslant F(x_2)$。
    \item $F(x)$是$x$的右连续函数,即对任意$x_0\in R$,有$\lim\limits_{x\to x_0^+}F(x)=F(x_0+0)=F(x_0)$。(左空心右实心)
    \item $F(-\infty)=\lim\limits_{x\to-\infty}F(x)=0$,$F(+\infty)=\lim\limits_{x\to+\infty}F(x)=1$。
\end{itemize}

\subsubsection{应用}

\begin{itemize}
    \item $P\{X\leqslant a\}=F(a)$。
    \item $P\{X<a\}=F(a-0)$。是指分布函数下该点左极限的概率值。
    \item $P(X=a)=F(a)-F(a-0)$。$\because P\{X\leqslant a\}=P\{X<a\cup X=a\}=P\{X<a\}+P\{X=a\}$,$\therefore P\{X=a\}=P\{X\leqslant a\}-P\{X<a\}=F(a)-F(a-0)$。
\end{itemize}

\section{一维离散随机变量}

\textcolor{violet}{\textbf{定义:}}若随机变量$X$只可能取有限个或可列各值$x_1,x_2,\cdots$,则称$X$为离散型随机变量。

\subsection{分布律}

\textcolor{violet}{\textbf{定义:}}$P\{X=x_i\}=p_i$,$i=1,2,\cdots$为$X$的\textbf{分布列}、\textbf{分布律}或\textbf{概率分布},记为$X\sim p_i$。

概率分布常用表格或矩阵表示:\medskip

\begin{tabular}{c|ccc}
    \hline
    $X$ & $x_1$ & $x_2$ & $\cdots$ \\ \hline
    $P$ & $p_1$ & $p_2$ & $\cdots$ \\
    \hline
\end{tabular}\,或$X\sim\left(\begin{array}{ccc}
    x_1 & x_2 & \cdots \\
    p_1 & p_2 & \cdots
\end{array}\right)$。\medskip

\subsection{性质}

数列$\{p_i\}$是离散型随机变量的概率分布的充要条件是$p_i\geqslant0$($i=1,2,\cdots$)且$\sum\limits_{i=1}^np_i=1$。

设离散型随机变量$X$的概率分布为$P\{X=x_i\}=p_i$,则$X$的分布函数$F(x)=P\{X\leqslant x\}=\sum\limits_{x_i\leqslant x}P\{X=x_i\}$,即离散型随机变量的分布律函数是一个左实右空的阶梯形函数。

$P\{X=x_i\}=P\{X\leqslant x_i\}-P\{X<x_i\}=F(x_i)-F(x_i-0)$,即某点的概率值为该点分布律值减去该点左极限的分布律值。

对实数轴上的任一集合$B$有$P\{X\in B\}=\sum\limits_{x_i\in B}P\{X=x_i\}$,特别地$P\{a<X\leqslant b\}=P\{X\leqslant b\}-P\{X\leqslant a\}=F(b)-F(a)$。

\subsection{应用}

\textbf{例题:}已知随机变量$X$的概率分布为:\medskip

\begin{tabular}{c|ccc}
    \hline
    $X$ & 1 & 2 & 3 \\ \hline
    $P\{X=k\}$ & $\theta^2$ & $2\theta(1-\theta)$ & $(1-\theta)^2$ \\
    \hline
\end{tabular} \medskip

且$P\{X\geqslant2\}=\dfrac{3}{4}$,求未知参数$\theta$与$X$的分布函数$F(x)$。

解:$\because P\{X\geqslant2\}=\dfrac{3}{4}$,$\therefore 2\theta(1-\theta)+(1-\theta)^2=\dfrac{3}{4}$,解得$\theta=\dfrac{1}{2}$,$-\dfrac{1}{2}$舍。

\subsection{分布}

\subsubsection{0-1分布}

\textcolor{violet}{\textbf{定义:}}若$X$的概率分布为$X\sim\left(\begin{array}{cc}
    1 & 0 \\
    p & 1-p
\end{array}\right)$,即$P\{X=1\}=p$,$P\{X=0\}=1-p$,则称$X$服从参数为$p$的0-1分布,记为$X\sim B(1,p)$($0<p<1$)。

0-1分布基于一次伯努利试验,$X$也称为伯努利计数变量。

\subsubsection{二项分布}

\textcolor{violet}{\textbf{定义:}}如果$X$的概率分布为$P\{X=k\}=C_n^kp^k(1-p)^{n-k}$($k=0,1,\cdots,n$,$0<p<1$),则称$X$服从参数为$(n,p)$的\textbf{二项分布},记为$X\sim B(n,p)$。

二项分布基于$n$重伯努利试验。

二项分布的分布律计算,总共进行试验$n$次,已知成功的概率为$p$,若成功了$k$次,则$k$次成功概率为$p^k$,则失败次数为$n-k$,从而$n-k$失败概率为$(1-p)^{n-k}$,因为$n$次试验都是相互独立的,所以将成功的概率与失败的概率乘在一起。又在$n$次中成功$k$次就可以了,进行排列,所以还乘上$C_n^k$。

\subsubsection{泊松分布}

\textcolor{violet}{\textbf{定义:}}如果$X$的概率分布为$P\{X=k\}=\dfrac{\lambda^k}{k!}e^{-\lambda}$($k=0,1,\cdots,n$,$\lambda>0$),则称$X$服从参数为$\lambda$的\textbf{泊松分布},记为$X\sim P(\lambda)$。

泊松分布基于某场合某单位时间内源源不断的质点来流的个数$X=k$,$\lambda$代表质点流动到来的强度。也可以代表稀有事件发生的概率。

\subsubsection{几何分布}

\textcolor{violet}{\textbf{定义:}}如果$X$的概率分布为$P\{X=k\}=(1-p)^{k-1}p$($k=0,1,\cdots,n$,$0<p<1$),则称$X$服从参数为$p$的\textbf{几何分布},记为$X\sim G(p)$。

几何分布与几何无关,代表的是$n$重伯努利试验首次成功就停止试验,试验次数可以为无穷。设$X$表示伯努利试验中事件$A$首次放生所需要的试验次数,则$X\sim G(p)$,其中$p=P(A)$。

从而根据意义,几何分布要求前$k-1$次都失败,从而概率为$(1-p)^{k-1}$,最后一次成功,所以再乘上$p$。

\subsubsection{超几何分布}

\textcolor{violet}{\textbf{定义:}}如果$X$的概率分布为$P\{X=k\}=\dfrac{C_M^kC_{N-M}^{n-k}}{C_N^n}$($\max\{0,n-N+M\}\leqslant k\leqslant\min\{MM,n\}$,$M,N,n$为正整数且$M\leqslant N$,$n\leqslant N$,$k$为整数),则称$X$服从参数为$(n,N,M)$的\textbf{超几何分布},记为$X\sim H(n,N,M)$。

超几何分布考的可能性很小,事件数就是古典概型的一个特例。

如有$N$件产品,其中$M$件正品,从而$N-M$件次品,任取$n$个,则取出$k$件正品的概率就是超几何分布。

\section{一维连续随机变量}

\textcolor{violet}{\textbf{定义:}}若随机变量$X$的分布函数可以表示为$F(x)=\int_{-\infty}^xf(t)\,\textrm{d}t$($x\in R$且取遍所有实数),其中$f(x)$是非负可积函数,则$X$为\textbf{连续型随机变量}。

\subsection{概率密度}

\textcolor{violet}{\textbf{定义:}}$f(x)$称为$X$的\textbf{概率密度函数},简称\textbf{概率密度},记为$X\sim f(x)$。

\subsection{性质}

改变$f(x)$有限各点的值$f(x)$仍是概率密度,$f(x)$为某一随机变量$X$的概率密度的充分必要条件:$f(x)\geqslant0$,且$\int_{-\infty}^{+\infty}f(x)\,\textrm{d}x=1$。

若$X$为连续型随机变量,$X\sim f(x)$,则对任意实数$c$有$P\{X=c\}=0$。

对实数轴上的任一集合$B$有$P\{X\in B\}=\int_Bf(x)\,\textrm{d}x$,特别地$P\{a<X<b\}=P\{a\leqslant X<b\}=P\{a<X\leqslant b\}=P\{a\leqslant X\leqslant b\}=\int_a^bf(x)\,\textrm{d}x=F(b)-F(a)$。

\subsection{应用}

\textbf{例题:}已知随机变量$X$的概率密度为$\left\{\begin{array}{ll}
    Ax, & 1<x<2 \\
    B, & 2\leqslant x<3 \\
    0, & \text{其他}
\end{array}\right.$,且$P\{1<X<2\}=P\{2<X<3\}$,求常数$AB$,分布函数$F(x)$以及概率$P\{2<X<4\}$。

解:由于归一性$\int_{-\infty}^{+\infty}f(x)\,\textrm{d}x=1$,$\therefore\int_1^2Ax\,\textrm{d}x+\int_2^2B\,\textrm{d}x=1$。

$\therefore\dfrac{3}{2}A+B=1$。又$P\{1<X<2\}=P\{2<X<3\}$。

$\therefore\int_1^2Ax\,\textrm{d}x=\int_2^3B\,\textrm{d}x$,即$\therefore\int_1^2Ax\,\textrm{d}x=\int_2^2B\,\textrm{d}x=\dfrac{1}{2}$,$A=\dfrac{1}{3}$,$B=\dfrac{1}{2}$。

$f(x)=\left\{\begin{array}{ll}
    \dfrac{1}{3}x, & 1<x<2 \medskip \\
    \dfrac{1}{2}, & 2\leqslant x<3 \\
    0, & \text{其他}
\end{array}\right.$,$\because F(x)=\int_{-\infty}^xf(t)\,\textrm{d}t$。

$\therefore F(x)=\left\{\begin{array}{ll}
    0, & x<1 \\
    \int_1^x\dfrac{1}{3}t\,\textrm{d}t=\dfrac{x^2}{6}-\dfrac{1}{6}, & 1\leqslant x<2 \medskip \\
    \int_1^2\dfrac{1}{3}x\,\textrm{d}x+\int_2^x\dfrac{1}{2}\,\textrm{d}x=\dfrac{1}{2}x-\dfrac{1}{2}, & 2\leqslant x<3 \\
    1, & x\geqslant3
\end{array}\right.$

\subsection{分布}

\subsubsection{均匀分布}

\textcolor{violet}{\textbf{定义:}}如果$X$的概率密度或分布函数分别为$f(x)=\left\{\begin{array}{ll}
    \dfrac{1}{b-a}, & a<x<b \\
    0, & \text{其他}
\end{array}\right.$,$F(x)=\left\{\begin{array}{ll}
    0, & x<a \\
    \dfrac{x-a}{b-a}, & a\leqslant x<b \\
    1, & x\geqslant b
\end{array}\right.$,则称$X$在区间$(a,b)$上服从\textbf{均匀分布},记为$X\sim U(a,b)$。

\begin{tikzpicture}[scale=1.5]
    \draw[-latex](-2,0) -- (2,0) node[below]{$x$};
    \draw[-latex](0,-0.25) -- (0,1.25) node[above]{$y$};
    \filldraw[black] (0,0) node[below]{$O$};
    \draw[black](-2,0) -- (-1,0) node[below]{$a$};
    \draw[black](2,0) -- (1,0) node[below]{$b$};
    \draw[black ](-1,0.6) -- (1,0.6);
    \draw[black, densely dashed](-1,0.6) -- (-1,0);
    \draw[black, densely dashed](1,0.6) -- (1,0);
    \filldraw[black] (-1,1) node{$\dfrac{1}{b-a}$};
    \filldraw[black] (0.5,1) node{$f(x)$};
\end{tikzpicture}
\hspace{2.5em}
\begin{tikzpicture}[scale=1.5]
    \draw[-latex](-2,0) -- (2,0) node[below]{$x$};
    \draw[-latex](0,-0.25) -- (0,1.25) node[above]{$y$};
    \filldraw[black] (0,0) node[below]{$O$};
    \draw[black](-2,0) -- (-1,0) node[below]{$a$};
    \draw[black](2,1) -- (1,1) node[below]{$b$};
    \draw[black](1,1) -- (-1,0);
    \filldraw[black] (-0.5,1) node{$F(x)$};
\end{tikzpicture}


\subsubsection{指数分布}

\subsubsection{正态分布}

\section{一维随机变量函数分布}

\subsection{离散型}

\subsection{连续性}

\end{document}
