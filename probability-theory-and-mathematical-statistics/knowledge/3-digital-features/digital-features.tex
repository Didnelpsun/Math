\documentclass[UTF8, 12pt]{ctexart}
% UTF8编码,ctexart现实中文
\usepackage{color}
% 使用颜色
\definecolor{orange}{RGB}{255,127,0}  
\definecolor{violet}{RGB}{192,0,255}  
\definecolor{aqua}{RGB}{0,255,255} 
\usepackage{geometry}
\setcounter{tocdepth}{4}
\setcounter{secnumdepth}{4}
% 设置四级目录与标题
\geometry{papersize={21cm,29.7cm}}
% 默认大小为A4
\geometry{left=3.18cm,right=3.18cm,top=2.54cm,bottom=2.54cm}
% 默认页边距为1英尺与1.25英尺
\usepackage{indentfirst}
\setlength{\parindent}{2.45em}
% 首行缩进2个中文字符
\usepackage{setspace}
\renewcommand{\baselinestretch}{1.5}
% 1.5倍行距
\usepackage{amssymb}
% 因为所以
\usepackage{amsmath}
% 数学公式
\usepackage[colorlinks,linkcolor=black,urlcolor=blue]{hyperref}
% 超链接
\usepackage{array}
% 表格垂直居中
\usepackage{diagbox}
% 表格斜线
\author{Didnelpsun}
\title{随机变量数字特征}
\date{}
\begin{document}
\maketitle
\pagestyle{empty}
\thispagestyle{empty}
\tableofcontents
\thispagestyle{empty}
\newpage
\pagestyle{plain}
\setcounter{page}{1}

有时候研究随机变量,其是没有具体的概率分布的,而对于这种类型我们只用研究其数学特征就可以了。

\section{一维随机变量数字特征}

\subsection{数学期望}

\subsubsection{概念}

设$X$是随机变量,$Y$是$X$的函数,$Y=g(X)$。

\textcolor{violet}{\textbf{定义:}}若$X$是离散型随机变量,其分布列为$p_i=P\{X=x_i\}$($i=1,2,\cdots$),若级数$\sum\limits_{i=1}^\infty x_ip_i$绝对收敛,则称随机变量$X$的数学期望存在,并将级数和$\sum\limits_{i=1}^\infty x_ip_i$称为随机变量$X$的\textbf{数学期望},记为$E(X)$或$EX$,即$EX=\sum\limits_{i=1}^\infty x_ip_i$,否则$X$数学期望不存在。(数学期望实际上是一种加权的合理平均值)

若级数$\sum\limits_{i=1}^\infty g(x_i)p_i$也绝对收敛,则称$Y=g(X)$的数学期望$E[g(X)]$存在,且$E[g(X)]=\sum\limits_{i=1}^\infty g(x_i)p_i$,否则$g(X)$的数学期望不存在。

\textcolor{violet}{\textbf{定义:}}若$X$是连续型随机变量,其概率密度为$f(x)$。若积分$\int_{-\infty}^{+\infty}xf(x)\,\textrm{d}x$绝对收敛,则称$X$的数学期望存在,且$EX=\int_{-\infty}^{+\infty}xf(x)\,\textrm{d}x$,否则$X$的数学期望不存在。

若积分$\int_{-\infty}^{+\infty}g(x)f(x)\,\textrm{d}x$绝对收敛,则称$g(X)$的数学期望存在,$E[g(X)]=\int_{-\infty}^{+\infty}g(x)f(x)\,\textrm{d}x$,否则$g(X)$的数学期望不存在。

\subsubsection{性质}

\begin{itemize}
    \item 对任意常数$a_i$和随机变量$X_i$($i=1,2,\cdots,n$)有$E\left(\sum\limits_{i=1}^na_iX_i\right)=\sum\limits_{i=1}^na_iEX_i$,其中$Ec=c$,$E(aX+c)=aEX+c$,$E(X\pm Y)=EX\pm EY$。
    \item 若$XY$相互独立,则$E(XY)=EX\cdot EY$,$E[g_1(X),g_2(Y)]=E[g_1(X)]\cdot E[g_2(Y)]$,一般若$X_1,X_2,\cdots,X_n$相互独立,则$E\left(\prod\limits_{i=1}^nX_i\right)=\prod\limits_{i=1}^nEX_i$,$E\left[\prod\limits_{i=1}^ng_i(X_i)\right]=\prod\limits_{i=1}^nE[g_i(X_i)]$。
\end{itemize}

\subsection{方差标准差}

\subsubsection{概念}

\textcolor{violet}{\textbf{定义:}}设$X$是随机变量,若$E[(X-EX)^2]$存在,则称$E[(X-EX)^2]$为$X$的\textbf{方差},记为$DX$,即$DX=E[(X-EX)^2]=E(X^2)-(EX)^2$。称$\sqrt{DX}$为$X$的\textbf{标准差}或\textbf{均方差},记为$\sigma(X)$,称随机变量$X^*=\dfrac{X-EX}{\sqrt{DX}}$为$X$的\textbf{标准化随机变量},此时$EX^*=0$,$DX^*=1$。

\subsubsection{性质}

\begin{itemize}
    \item $DX\geqslant0$,$E(X^2)=DX+(EX)^2\geqslant(EX)^2$。
    \item $Dc=0$。
    \item $D(aX+b)=a^2DX$。
    \item $D(X\pm Y)=DX+DY\pm2Cov(X,Y)$。
    \item 若$XY$相互独立,则$D(aX+bY)=a^2DX+b^2DY$,一般若$X_1,X_2,\cdots,X_n$相互独立,$g_i(x)$为关于$x$的连续函数,则$D\left(\sum\limits_{i=1}^na_iX_i\right)=\sum\limits_{i=1}^na_i^2DX_i$,$D\left[\sum\limits_{i=1}^ng_i(X_i)\right]=\sum\limits_{i=1}^nD[g_i(X_i)]$。
\end{itemize}

\subsection{切比雪夫不等式}

\textcolor{violet}{\textbf{定义:}}若随机变量$X$的方差$DX$存在,则对任意$\epsilon>0$,有$P\{\vert X-EX\vert\leqslant\epsilon\}\leqslant\dfrac{DX}{\epsilon^2}$或$P\{\vert X-EX\vert<\epsilon\}\geqslant1-\dfrac{DX}{\epsilon^2}$。

即一个变量不会离标准差太大距离。

可以用于估算随机变量在某范围中取值的概率,也可以证明某些收敛性问题(如数学统计章节中的一致性)。

\textbf{例题:}设$XY$为随机变量,数学期望都是2,方差分别为1和4,相关系数为0.5,尝试估计估计概率$P\{\vert X-Y\vert\geqslant6\}$。

解:令$Z=X-Y$,$\therefore EZ=E(X-Y)=EX-EY=2-2=0$,所以$P\{\vert X-Y\vert\geqslant6\}=P\{\vert X-Y-0\vert\geqslant6\}=P\{\vert Z-EZ\vert\geqslant6\}\leqslant\dfrac{DZ}{6^2}=\dfrac{3}{36}=\dfrac{1}{2}$。

\subsection{常用分布数字特征}

\begin{center}
    \begin{tabular}{|m{50pt}<{\centering}|m{220pt}<{\centering}|c|c|}
        \hline
        分布 & 分布列$p_i$或概率密度$f(x)$ & 期望 & 方差 \\ \hline
        0-1分布$B(1,p)$ & $P\{X=k\}=p^k(1-p)^{1-k}$,$k=0,1$ & $p$ & $p(1-p)$ \\ \hline
        二项分布$B(n,p)$ & $P\{X=k\}=C_n^kp^k(1-p)^{n-k}$,$k=0,\cdots,n$ & $np$ & $np(1-p)$ \\ \hline
        泊松分布$P(\lambda)$ & $P\{X=k\}=\dfrac{\lambda^k}{k!}e^{-\lambda}$,$k=0,\cdots$ & $\lambda$ & $\lambda$ \\ \hline
        几何分布$G(p)$ & $P\{X=k\}=(1-p)^{k-1}$,$p,k=1,\cdots$ & $\dfrac{1}{p}$ & $\dfrac{1-p}{p^2}$ \\ \hline
        正态分布$N(\mu,\sigma^2)$ & $f(x)=\dfrac{1}{\sqrt{2\pi}\sigma}\exp\left\{-\dfrac{(x-\mu)^2}{2\sigma^2}\right\}$,$x\in R$ & $\mu$ & $\sigma^2$ \\ \hline
        均匀分布$U(a,b)$ & $f(x)=\dfrac{1}{b-a}$,$a<x<b$ & $\dfrac{a+b}{2}$ & $\dfrac{(b-a)^2}{12}$ \\ \hline
        指数分布$E(\lambda)$ & $f(x)=\lambda e^{-\lambda x}$,$x>0$ & $\dfrac{1}{\lambda}$ & $\dfrac{1}{\lambda^2}$ \\ \hline
    \end{tabular}
\end{center}

\section{二维随机变量数字特征}

\subsection{数学期望}

\textcolor{violet}{\textbf{定义:}}若$XY$为随机变量,$g(X,Y)$为$XY$的函数,如果$(X,Y)$为离散型随机变量,其联合分布为$p_{ij}=P\{X=x_i,Y=y_i\}$($i,j=1,2,\cdots$),若级数$\sum\limits_i\sum\limits_jg(x_i,y_j)p_{ij}$绝对收敛,则$E[g(X,Y)]=\sum\limits_i\sum\limits_jg(x_i,y_j)p_{ij}$;如果$(X,Y)$为连续型随机变量,其概率密度为$f(x,y)$,若积分$\int_{-\infty}^{+\infty}\int_{-\infty}^{+\infty}g(x,y)f(x,y)\,\textrm{d}x\textrm{d}y$绝对收敛,则定义$E[g(X,Y)]=\int_{-\infty}^{+\infty}\int_{-\infty}^{+\infty}g(x,y)f(x,y)\,\textrm{d}x\textrm{d}y$。

\subsection{协方差相关系数}

\subsubsection{概念}

\textcolor{violet}{\textbf{定义:}}若随机变量$XY$的方差存在且$DX>0$,$DY>0$,则称$E[(X-EX)(Y-EY)]$为随机变量$X$与$Y$的\textbf{协方差},记为$Cov(X,Y)$,即$Cov(X,Y)=E[(X-EX)(Y-EY)]=E(XY-XEY-YEX+EXEY)=E(XY)-EX\cdot EY$。

其中$E(XY)=\left\{\begin{array}{l}
    \sum\limits_i\sum\limits_jx_iy_jP\{X=x_i,Y=y_j\} \\
    \int_{-\infty}^{+\infty}\int_{-\infty}^{+\infty}xyf(x,y)\,\textrm{d}x\textrm{d}y
\end{array}\right.$。

从定义来看,方差$DX$就是自己的协方差$Cov(X,X)$。

\textcolor{violet}{\textbf{定义:}}$\rho_{XY}=\dfrac{Cov(X,Y)}{\sqrt{DX}\sqrt{DY}}$为随机变量$XY$的\textbf{相关系数}。若$\rho_{XY}=0$,则$XY$不相干,否则相关。

相关系数是描述随机变量$XY$之间的线性关系。相关系数为0不代表没有其之间没有关系,也可能存在非线性关系。

\subsubsection{性质}

\begin{itemize}
    \item 对称性:$Cov(X,Y)=Cov(Y,X)$,$\rho_{XY}=\rho_{YX}$,$Cov(X,X)=DX$,$\rho_{XX}=1$。
    \item 线性性:$Cov(X,c)=0$,$Cov(aX+b,Y)=aCov(X,Y)$,$Cov(X_1+X_2,Y)=Cov(X_1,Y)+Cov(X_2,Y)$。一般$Cov\left(\sum\limits_{i=1}^na_iX_i,Y\right)=\sum\limits_{i=1}^nCov(X_i,Y)$。
    \item 相关系数有界性:$\vert\rho_{XY}\vert\leqslant1$。
    \item 线性关系下的相关系数:若$Y=aX+b$,则$\rho_{XY}=\left\{\begin{array}{ll}
        1, & a>0 \\
        -1, & a<0
    \end{array}\right.$。
\end{itemize}

\textbf{例题:}设随机变量$XY$的概率分布分别为:

\begin{center}
    \begin{tabular}{m{20pt}<{\centering}|m{40pt}<{\centering}m{40pt}<{\centering}}
        \hline
        $X$ & 0 & 1 \\ \hline
        $P$ & 1/3 & 2/3 \\ \hline
    \end{tabular}\qquad
    \begin{tabular}{m{20pt}<{\centering}|m{40pt}<{\centering}m{40pt}<{\centering}m{40pt}<{\centering}}
        \hline
        $Y$ & -1 & 0 & 1 \\ \hline
        $P$ & 1/3 & 1/3 & 1/3 \\ \hline
    \end{tabular}
\end{center}

且$P\{X^2=Y^2\}=1$。

(1)求随机变量$(X,Y)$的概率分布。

(2)求$Z=XY$的概率分布。

(3)求$XY$的相关系数$\rho_{XY}$。

(1)解:根据已知的题目条件可以知道对应的边缘概率分布:

\begin{center}
    \begin{tabular}{c|ccc|c}
        \diagbox{$X$}{$Y$} & -1 & 0 & 1 & $X$边缘 \\ \hline
        0 & & & & 1/3 \\ \hline
        1 & & & & 2/3 \\ \hline
        $Y$边缘 & 1/3 & 1/3 & 1/3 & 1 \\ \hline
    \end{tabular}
\end{center}

又$P\{X^2=Y^2\}=1$,所以$P\{X^2\neq Y^2\}=0$,所以$X=\pm Y$,解得:

\begin{center}
    \begin{tabular}{c|ccc|c}
        \diagbox{$X$}{$Y$} & -1 & 0 & 1 & $X$边缘 \\ \hline
        0 & 0 & 1/3 & 0 & 1/3 \\ \hline
        1 & 1/3 & 0 & 1/3 & 2/3 \\ \hline
        $Y$边缘 & 1/3 & 1/3 & 1/3 & 1 \\ \hline
    \end{tabular}
\end{center}

(2)解:$Z=XY$的可能取值为-1,0,1。所以根据表格:

$P\{Z=-1\}=P\{X=1,Y=-1\}=\dfrac{1}{3}$。

$P\{Z=1\}=P\{X=1,Y=1\}=\dfrac{1}{3}$。

$P\{Z=0\}=1-P\{Z=1\}-P\{Z=-1\}=\dfrac{1}{3}$。

(3)解:$\rho=\dfrac{Cov(X,Y)}{\sqrt{DX}\sqrt{DY}}=\dfrac{EXY-EXEY}{\sqrt{DX}\sqrt{DY}}=0$。

\section{独立性与相关性}

\begin{itemize}
    \item 独立则一定不相关,但是不相关不一定独立。
    \item 如果相关则一定不独立。
    \item 如果$(X,Y)$服从二维正态分布,则$XY$独立与$XY$不相关是充要条件。
\end{itemize}

\subsection{分布判断独立性}

都是通过分布情况判断独立性:

\begin{itemize}
    \item $F(x,y)=F_X(x)\cdot F_Y(y)$。
    \item $f(x,y)=f_X(x)\cdot f_Y(y)$。
    \item $P\{X=x_i,Y=y_j\}=P\{X=x_i\}\cdot P\{Y=y_j\}$。
\end{itemize}

\subsection{数字特征判断相关性}

通过相关系数$\rho_{XY}$来判断是否存在线性相关性。

$\rho_{XY}=0\Leftrightarrow Cov(X,Y)=0\Leftrightarrow E(XY)=EX\cdot EY\Leftrightarrow D(X\pm Y)=DX+DY$。

\subsection{基本判别流程}

当讨论随机变量$XY$的相关性独立性时:

\begin{enumerate}
    \item 计算$Cov(X,Y)=E(XY)-EXEY$判断是否为0。
    \item 当$Cov(X,Y)\neq0$时则$XY$相关不独立。
    \item 当$Cov(X,Y)=0$时则$XY$不相关。
    \item 若$P(XY)=P(X)P(Y)$则$XY$不相关但独立,否则不相关不独立。
\end{enumerate}

\textbf{例题:}设随机变量$X$的概率密度为$f(x)=\dfrac{1}{2}e^{-\vert x\vert}$,$x\in(-\infty,+\infty)$。证明$X$与$\vert X\vert$不相关且不独立。

解:$Cov(X,Y)=EXY-EXEY=EX\vert X\vert-EXE\vert X\vert$。

其中$EX=\displaystyle{\int_{-\infty}^{+\infty}}x\cdot\dfrac{1}{2}e^{-\vert x\vert}\,\textrm{d}x=0$,$EXY=\displaystyle{\int_{-\infty}^{+\infty}}x\cdot\dfrac{1}{2}e^{-\vert x\vert}\vert x\vert\,\textrm{d}x=0$。

$\therefore\rho_{XY}=0$,从而$XY$不相关。

令$X\leqslant a$,则$P\{X\leqslant a\}$。而$P\{\vert X\vert\leqslant a\}=P\{-a\leqslant X\leqslant a\}<P\{X\leqslant a\}$。

$\therefore P\{X\leqslant a,\vert X\vert\leqslant a\}=P\{\vert X\vert\leqslant a\}$,又$P\{X\leqslant a\}<1$。

$\therefore P\{X\leqslant a,\vert X\vert\leqslant a\}\neq P\{\vert X\vert\leqslant a\}\cdot P\{\vert X\vert\leqslant a\}$,所以不独立。

\end{document}
