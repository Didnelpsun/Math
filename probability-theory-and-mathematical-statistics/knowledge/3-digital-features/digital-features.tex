\documentclass[UTF8, 12pt]{ctexart}
% UTF8编码,ctexart现实中文
\usepackage{color}
% 使用颜色
\definecolor{orange}{RGB}{255,127,0}  
\definecolor{violet}{RGB}{192,0,255}  
\definecolor{aqua}{RGB}{0,255,255} 
\usepackage{geometry}
\setcounter{tocdepth}{4}
\setcounter{secnumdepth}{4}
% 设置四级目录与标题
\geometry{papersize={21cm,29.7cm}}
% 默认大小为A4
\geometry{left=3.18cm,right=3.18cm,top=2.54cm,bottom=2.54cm}
% 默认页边距为1英尺与1.25英尺
\usepackage{indentfirst}
\setlength{\parindent}{2.45em}
% 首行缩进2个中文字符
\usepackage{setspace}
\renewcommand{\baselinestretch}{1.5}
% 1.5倍行距
\usepackage{amssymb}
% 因为所以
\usepackage{amsmath}
% 数学公式
\usepackage[colorlinks,linkcolor=black,urlcolor=blue]{hyperref}
% 超链接
\author{Didnelpsun}
\title{随机变量数字特征}
\date{}
\begin{document}
\maketitle
\pagestyle{empty}
\thispagestyle{empty}
\tableofcontents
\thispagestyle{empty}
\newpage
\pagestyle{plain}
\setcounter{page}{1}

有时候研究随机变量,其是没有具体的概率分布的,而对于这种类型我们只用研究其数学特征就可以了。

\section{一维随机变量数字特征}

\subsection{数学期望}

设$X$是随机变量,$Y$是$X$的函数,$Y=g(X)$。

\textcolor{violet}{\textbf{定义:}}若$X$是离散型随机变量,其分布列为$p_i=P\{X=x_i\}$($i=1,2,\cdots$),若级数$\sum\limits_{i=1}^\infty x_ip_i$绝对收敛,则称随机变量$X$的数学期望存在,并将级数和$\sum\limits_{i=1}^\infty x_ip_i$称为随机变量$X$的\textbf{数学期望},记为$E(X)$或$EX$,即$EX=\sum\limits_{i=1}^\infty x_ip_i$,否则$X$数学期望不存在。(数学期望实际上是一种加权的合理平均值)

若级数$\sum\limits_{i=1}^\infty g(x_i)p_i$也绝对收敛,则称$Y=g(X)$的数学期望$E[g(X)]$存在,且$E[g(X)]=\sum\limits_{i=1}^\infty g(x_i)p_i$,否则$g(X)$的数学期望不存在。

\textcolor{violet}{\textbf{定义:}}若$X$是连续型随机变量,其概率密度为$f(x)$。若积分$\int_{-\infty}^{+\infty}xf(x)\,\textrm{d}x$绝对收敛,则称$X$的数学期望存在,且$EX=\int_{-\infty}^{+\infty}xf(x)\,\textrm{d}x$,否则$X$的数学期望不存在。

若积分$\int_{-\infty}^{+\infty}g(x)f(x)\,\textrm{d}x$绝对收敛,则称$g(X)$的数学期望存在,且$E[g(X)]=\int_{-\infty}^{+\infty}g(x)f(x)\,\textrm{d}x$,否则$g(X)$的数学期望不存在。

\subsection{方差标准差}

\subsection{切比雪夫不等式}

\section{二维随机变量数字特征}

\subsection{数学期望}

\subsection{协方差相关系数}

\section{独立性与相关性}

\subsection{分布判断独立性}

\subsection{数字特征判断相关性}

\end{document}
