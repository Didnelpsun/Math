\documentclass[UTF8, 12pt]{ctexart}
% UTF8编码,ctexart现实中文
\usepackage{color}
% 使用颜色
\usepackage{geometry}
\setcounter{tocdepth}{4}
\setcounter{secnumdepth}{4}
% 设置四级目录与标题
\geometry{papersize={21cm,29.7cm}}
% 默认大小为A4
\geometry{left=3.18cm,right=3.18cm,top=2.54cm,bottom=2.54cm}
% 默认页边距为1英尺与1.25英尺
\usepackage{indentfirst}
\setlength{\parindent}{2.45em}
% 首行缩进2个中文字符
\usepackage{setspace}
\renewcommand{\baselinestretch}{1.5}
% 1.5倍行距
\usepackage{amssymb}
% 因为所以
\usepackage{amsmath}
% 数学公式
\usepackage[colorlinks,linkcolor=black,urlcolor=blue]{hyperref}
% 超链接
\author{Didnelpsun}
\title{随机变量及其分布}
\date{}
\begin{document}
\maketitle
\pagestyle{empty}
\thispagestyle{empty}
\tableofcontents
\thispagestyle{empty}
\newpage
\pagestyle{plain}
\setcounter{page}{1}

\section{二项分布}

\textbf{例题:}已知随机变量$X$的概率密度为$f(x)=\left\{\begin{array}{ll}
    2x, & 0<x<1 \\
    0, \text{其他}
\end{array}\right.$,$Y$表示对$X$进行3次独立重复试验中事件$\left\{X\leqslant\dfrac{1}{2}\right\}$,求$P\{Y=2\}$。\medskip

解:已知对$X$进行独立重复试验,表示这个进行的是伯努利试验,从而$Y\sim B(n,p)$。又是3次,所以$Y\sim B(3,p)$。

只用求出这个$p$即$\left\{X\leqslant\dfrac{1}{2}\right\}$的概率就可以了。又已知$f(x)$。

$\therefore p=\left\{X\leqslant\dfrac{1}{2}\right\}=\int_0^\frac{1}{2}2x\,\textrm{d}x=\dfrac{1}{4}$。$\therefore P\{Y=2\}=B\left(3,\dfrac{1}{4}\right)=\dfrac{9}{64}$。

\section{泊松分布}

\textbf{例题:}设一本书的各页印刷错误的个数$X$服从泊松分布。已知只有一个和只有两个印刷错误的页数相同,则随机抽查的4页中无印刷错误的概率$p$为?

解:$\because P\{X=1\}=P\{X=2\}$,$\therefore\dfrac{\lambda^1}{1!}e^{-\lambda}=\dfrac{\lambda^2}{2!}e^{-\lambda}$,$\lambda=2$。

由于随机抽四页类似于伯努利试验是相互独立的,所以随机抽4页都无错误的概率为$[P\{X=0\}]^4=e^{-8}$。

\section{几何分布}

\textbf{例题:}已知随机变量$X$的概率密度为$f(x)=\left\{\begin{array}{ll}
    2^{-x}\ln2, & x>0 \\
    0, \text{其他}
\end{array}\right.$,对$X$进行独立重复观测,直到第2个大于3的观测值出现时停止,记$Y$为观测次数,求$Y$的概率分布。

解:由题目直到就停止,知道$Y\sim G(p)$。

又$p=P\{X\geqslant3\}=\int_3^{+\infty}2^{-x}\ln2\,\textrm{d}x=\dfrac{1}{8}$

这是对几何分布的变形,首先进行$k$次试验,第$k$次成功,所以要乘$p$,而因为是第2个成功,所以前面的$k-1$次中有$k-2$次失败和一次成功,所以一共$p^2(1-k)^{k-2}$。因为前面的成功的一次在$k-1$中任意一个地方就可以了,所以一共有$k-1$中可能性,要考虑到排列,所以还要乘$(k-1)$。

$\therefore P\{Y=k\}=(k-1)\left(\dfrac{1}{8}\right)^2\cdot\left(\dfrac{7}{8}\right)^{k-2}$。

\end{document}
