\documentclass[UTF8, 12pt]{ctexart}
% UTF8编码,ctexart现实中文
\usepackage{color}
% 使用颜色
\usepackage{geometry}
\setcounter{tocdepth}{4}
\setcounter{secnumdepth}{4}
% 设置四级目录与标题
\geometry{papersize={21cm,29.7cm}}
% 默认大小为A4
\geometry{left=3.18cm,right=3.18cm,top=2.54cm,bottom=2.54cm}
% 默认页边距为1英尺与1.25英尺
\usepackage{indentfirst}
\setlength{\parindent}{2.45em}
% 首行缩进2个中文字符
\usepackage{setspace}
\renewcommand{\baselinestretch}{1.5}
% 1.5倍行距
\usepackage{amssymb}
% 因为所以
\usepackage{amsmath}
% 数学公式
\usepackage[colorlinks,linkcolor=black,urlcolor=blue]{hyperref}
% 超链接
\author{Didnelpsun}
\title{随机变量及其分布}
\date{}
\begin{document}
\maketitle
\pagestyle{empty}
\thispagestyle{empty}
\tableofcontents
\thispagestyle{empty}
\newpage
\pagestyle{plain}
\setcounter{page}{1}

\section{二项分布}

\textbf{例题:}已知随机变量$X$的概率密度为$f(x)=\left\{\begin{array}{ll}
    2x, & 0<x<1 \\
    0, \text{其他}
\end{array}\right.$,$Y$表示对$X$进行3次独立重复试验中事件$\left\{X\leqslant\dfrac{1}{2}\right\}$,求$P\{Y=2\}$。\medskip

解:已知对$X$进行独立重复试验,表示这个进行的是伯努利试验,从而$Y\sim B(n,p)$。又是3次,所以$Y\sim B(3,p)$。

只用求出这个$p$即$\left\{X\leqslant\dfrac{1}{2}\right\}$的概率就可以了。又已知$f(x)$。

$\therefore p=\left\{X\leqslant\dfrac{1}{2}\right\}=\int_0^\frac{1}{2}2x\,\textrm{d}x=\dfrac{1}{4}$。$\therefore P\{Y=2\}=B\left(3,\dfrac{1}{4}\right)=\dfrac{9}{64}$。

\section{泊松分布}

\textbf{例题:}设一本书的各页印刷错误的个数$X$服从泊松分布。已知只有一个和只有两个印刷错误的页数相同,则随机抽查的4页中无印刷错误的概率$p$为?

解:$\because P\{X=1\}=P\{X=2\}$,$\therefore\dfrac{\lambda^1}{1!}e^{-\lambda}=\dfrac{\lambda^2}{2!}e^{-\lambda}$,$\lambda=2$。

由于随机抽四页类似于伯努利试验是相互独立的,所以随机抽4页都无错误的概率为$[P\{X=0\}]^4=e^{-8}$。

\section{几何分布}

\textbf{例题:}已知随机变量$X$的概率密度为$f(x)=\left\{\begin{array}{ll}
    2^{-x}\ln2, & x>0 \\
    0, \text{其他}
\end{array}\right.$,对$X$进行独立重复观测,直到第2个大于3的观测值出现时停止,记$Y$为观测次数,求$Y$的概率分布。

解:由题目直到就停止,知道$Y\sim G(p)$。

又$p=P\{X\geqslant3\}=\int_3^{+\infty}2^{-x}\ln2\,\textrm{d}x=\dfrac{1}{8}$

这是对几何分布的变形,首先进行$k$次试验,第$k$次成功,所以要乘$p$,而因为是第2个成功,所以前面的$k-1$次中有$k-2$次失败和一次成功,所以一共$p^2(1-k)^{k-2}$。因为前面的成功的一次在$k-1$中任意一个地方就可以了,所以一共有$k-1$中可能性,要考虑到排列,所以还要乘$(k-1)$。

$\therefore P\{Y=k\}=(k-1)\left(\dfrac{1}{8}\right)^2\cdot\left(\dfrac{7}{8}\right)^{k-2}$。

\section{均匀分布}

\textbf{例题:}已知随机变量$X\sim U(a,b)$($a>0$)且$P\{0<X<3\}=\dfrac{1}{4}$,$P\{X>4\}=\dfrac{1}{2}$,求$X$的概率密度以及$P\{1<X<5\}$。

解:$\because P\{X>4\}=\dfrac{1}{2}$,4在其区间中点上,$\dfrac{a+b}{2}=4$。

$\because P\{0<X<3\}=\dfrac{1}{4}$,$3$若在$a$左边则概率为0,所以必然在右边。

$\therefore P\{a<X<3\}=\dfrac{1}{4}$,$P\{<3X<4\}=1-\dfrac{1}{4}-\dfrac{1}{2}=\dfrac{1}{4}$,$\dfrac{4-3}{b-a}=\dfrac{1}{4}$。

解得$a=2$,$b=6$,$X\sim U(2,6)=f(x)=\left\{\begin{array}{ll}
    \dfrac{1}{4}, & 2<x<6 \\
    0, & \text{其他}
\end{array}\right.$。

$P\{1<X<5\}=\dfrac{5-2}{6-2}=\dfrac{3}{4}$。

\section{指数分布}

\textbf{例题:}已知随机变量$X\sim E(1)$,$a$为常数且大于0,求$P\{X\leqslant a+1|X>a\}$。

解:$P\{X\leqslant a+1|X>a\}=\dfrac{P\{a<X\leqslant a+1\}}{P\{X>a\}}=\dfrac{\int_a^{a+1}e^{-x}\,\textrm{d}x}{\int_a^{+\infty}e^{-x}\,\textrm{d}x}=1-\dfrac{1}{e}$。

也可以根据指数分布的无记忆性:$P\{X\leqslant a+1|X>a\}=1-P\{X>a+1|X>a\}=1-P\{X>1\}=P\{X\leqslant1\}=F(1)=1-\dfrac{1}{e}$。

\section{正态分布}

\textbf{例题:}已知随机变量$X\sim N(0,1)$,对给定的$\alpha$($0<\alpha>1$),数$\mu_\alpha$满足$P\{X>\mu_\alpha\}=\alpha$,若$P\{\vert X\vert<x\}=\alpha$,求$x$。

解:$P\{X>\mu_\alpha\}=\alpha$即表示$\mu_\alpha$为标准正态分布的上$\alpha$分位点。

又$P\{\vert X\vert<x\}=\alpha$,即$-x<X<x$的面积为$\alpha$,所以两边的面积各为$\dfrac{1-\alpha}{2}$,$P\{X<x\}=P\{X>x\}=\dfrac{1-\alpha}{2}$。

$\because$面积为$\alpha$的下标为$\alpha$,$\therefore$面积为$\dfrac{1-\alpha}{2}$的下标为$\dfrac{1-\alpha}{2}$,$x=\mu_\frac{1-\alpha}{2}$。

\end{document}
