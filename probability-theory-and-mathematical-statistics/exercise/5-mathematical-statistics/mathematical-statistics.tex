\documentclass[UTF8, 12pt]{ctexart}
% UTF8编码,ctexart现实中文
\usepackage{color}
% 使用颜色
\usepackage{geometry}
\setcounter{tocdepth}{4}
\setcounter{secnumdepth}{4}
% 设置四级目录与标题
\geometry{papersize={21cm,29.7cm}}
% 默认大小为A4
\geometry{left=3.18cm,right=3.18cm,top=2.54cm,bottom=2.54cm}
% 默认页边距为1英尺与1.25英尺
\usepackage{indentfirst}
\setlength{\parindent}{2.45em}
% 首行缩进2个中文字符
\usepackage{setspace}
\renewcommand{\baselinestretch}{1.5}
% 1.5倍行距
\usepackage{amssymb}
% 因为所以
\usepackage{amsmath}
% 数学公式
\usepackage[colorlinks,linkcolor=black,urlcolor=blue]{hyperref}
% 超链接
\author{Didnelpsun}
\title{数理统计}
\date{}
\begin{document}
\maketitle
\pagestyle{empty}
\thispagestyle{empty}
\tableofcontents
\thispagestyle{empty}
\newpage
\pagestyle{plain}
\setcounter{page}{1}
\section{统计量}

\textbf{例题:}已知总体$X$的期望为$EX=0$,方差$DX=\sigma^2$。从总体抽取容量为$n$的简单随机样本,其均值和方差分别为$\overline{X}$,$S^2$。记$S_k^2=\dfrac{n}{k}\overline{X}^2+\dfrac{1}{k}S^2$($k=1,2,3,4$),则()。

$A.E(S_1^2)=\sigma^2$\qquad$B.E(S_2^2)=\sigma^2$

$C.E(S_3^2)=\sigma^2$\qquad$D.E(S_4^2)=\sigma^2$

解:$E(S_k^2)=E\left(\dfrac{n}{k}\overline{X}^2+\dfrac{1}{k}S^2\right)=\dfrac{n}{k}E\overline{X}^2+\dfrac{1}{k}E(S^2)=\dfrac{n}{k}((E\overline{X})^2+D\overline{X})+\dfrac{1}{k}E(S^2)=\dfrac{n}{k}\left(0+\dfrac{\sigma^2}{n}\right)+\dfrac{1}{k}\sigma^2=\dfrac{2\sigma^2}{k}$,$\therefore k=2$。

\section{三大分布}

\subsection{\texorpdfstring{$\chi^2$分布}{}}

\textbf{例题:}设$X_1,X_2,X_3,X_4$是来自正态总体$N(0,4)$的简单随机样本,记$X=a(X_1-2X_2)^2+b(3X_3-4X_4)^2$。求$X$服从$\chi^2$分布下的参数与自由度。

解:若$X_1,X_2,X_3,X_4$同一个正态分布,所以$EX_1=EX_2=EX_3=EX_4=0$,$DX_1=DX_2=DX_3=DX_4=4$。

$E(X_1-2X_2)=EX_1-2EX_2=0$,$D(X_1-2X_2)=DX_1-4DX_2=20$。

$\therefore X_1-2X_2\sim N(0,20)$,同理$3X_3-4X_4\sim N(0,100)$。

对其标准化:$\dfrac{X_1-2X_2-0}{\sqrt{20}}\sim N(0,1)$,$\dfrac{3X_3-4X_4-0}{\sqrt{100}}\sim N(0,1)$。

若要让$X$满足$\chi^2$分布,则要将$a(X_1-2X_2)^2+b(3X_3-4X_4)^2$两项标准化。

$\therefore\dfrac{(X_1-2X_2)^2}{20}+\dfrac{(3X_3-4X_4)^2}{100}\sim\chi^2(2)$,所以$a=\dfrac{1}{20}$,$b=\dfrac{1}{100}$。

\subsection{\texorpdfstring{$t$分布}{}}

\textbf{例题:}设$X_1,X_2,\cdots,X_8$是来自正态总体$N(0,3^2)$的简单随机样本,则统计量$Y=\dfrac{X_1+X_2+X_3+X_4}{\sqrt{X_5^2+X_6^2+X_7^2+X_8^2}}$服从什么分布?

解:$\because X_1,\cdots,X_8\sim N(0,9)$,$\therefore X_1+X_2+X_3+X_4\sim N(0,36)$。

$\therefore\dfrac{X_1+X_2+X_3+X_4-0}{6}\sim N(0,1)$。

$\dfrac{X_5^2+X_6^2+X_7^2+X_8^2}{9}=\left(\dfrac{X_5-0}{3}\right)^2+\left(\dfrac{X_6-0}{3}\right)^2+\left(\dfrac{X_7-0}{3}\right)^2+\left(\dfrac{X_8-0}{3}\right)^2$\\$\sim\chi^2(4)$

$\therefore\dfrac{\dfrac{X_1+X_2+X_3+X_4-0}{6}}{\sqrt{\dfrac{X_5^2+X_6^2+X_7^2+X_8^2}{9}/4}}=\dfrac{X_1+X_2+X_3+X_4}{\sqrt{X_5^2+X_6^2+X_7^2+X_8^2}}\sim t(4)$。

\subsection{\texorpdfstring{$F$分布}{}}

\textbf{例题:}设$X_1,X_2,\cdots,X_15$是来自正态总体$N(0,3^2)$的简单随机样本,则统计量$Y=\dfrac{X_1^2+X_2^2+\cdots+X_{10}^2}{2X_{11}^2+X_{12}^2+\cdots+X_{15}^2}$服从什么分布?

解:$\because\dfrac{X_i-0}{3}\sim N(0,1)$,$\left(\dfrac{X_i-0}{3}\right)^2=\dfrac{x_i^2}{9}\sim\chi^2(1)$。

$\therefore\dfrac{X_1^2+X_2^2+\cdots+X_{10}^2}{9}\sim\chi^2(10)$,$\dfrac{X_{11}^2+X_{12}^2+\cdots+X_{15}^2}{9}\sim\chi^2(5)$。

$\therefore\dfrac{\dfrac{X_1^2+X_2^2+\cdots+X_{10}^2}{9}/10}{\dfrac{X_{11}^2+X_{12}^2+\cdots+X_{15}^2}{9}/5}=\dfrac{X_1^2+X_2^2+\cdots+X_{10}^2}{2X_{11}^2+X_{12}^2+\cdots+X_{15}^2}=Y\sim F(10,5)$。

\section{参数估计}

\subsection{矩估计}

\section{置信区间}

\textbf{例题:}一批零件的长度服从正态分布$N(\mu,\sigma^2)$,其中$\mu,\sigma^2$均未知。现从中随机抽取16个零件,测得样本均值$\overline{x}=20cm$,样本标准差为$s=1cm$,求$\mu$的置信水平为0.90的置信区间。

解:$\sigma$未知,所以使用$s$来求置信空间。

置信空间为$(\overline{X}-t_\frac{\alpha}{2}(n-1)\dfrac{S}{\sqrt{n}},\overline{X}+t_\frac{\alpha}{2}(n-1)\dfrac{S}{\sqrt{n}})$。

已知$\overline{x}=20$,$s=1$,$n=16$,$\alpha=1-0.90=0.1$。

所以置信空间为$\left(20-\dfrac{1}{4}t_{0.05}(15),20+\dfrac{1}{4}t_{0.05}(15)\right)$。

\section{假设检验}

\textbf{例题:}已知某机器生产出来的零件长度$X$(单位:$cm$)服从正态分布$N(\mu,\delta^2)$,现从中随意抽取容量为16的一个样本,测得样本均值$\overline{x}=10$,样本方差$s^2=0.16$,$t_{0.025}(15)=2.132$。

(1)求总体均值$\mu$置信水平为0.95的置信区间。

(2)在显著性水平$0.05$下检验假设$H_0:\mu=9.7$,$H_1:\mu\neq9.7$。

(1)解:根据公式直接解出置信空间$(10-0.1t_{0.025}(15),10+0.1t_{0.025}(15))=(9.7868,10.2132)$。

(2)解:根据假设$H_0$,得到拒绝域$(-\infty,9.4868]\cup[9.9132,+\infty)$。

又$\overline{X}=10$在拒绝域$[9.9132,+\infty)$上,所以假设$H_0$拒绝。

\section{两类错误}

\textbf{例题:}假定$X$是连续型随机变量,$U$是对$X$的一次观测值,关于其概率密度$f(x)$有如下假设:

$H_0:f(x)=\left\{\begin{array}{ll}
    \dfrac{1}{2}, & 0\leqslant x\leqslant2 \\
    0, & \text{其他}
\end{array}\right.$,$H_1:f(x)=\left\{\begin{array}{ll}
    \dfrac{x}{2}, & 0\leqslant x\leqslant2 \\
    0, & \text{其他}
\end{array}\right.$。

检验规则:当事件$V=\left\{U>\dfrac{3}{2}\right\}$出现时,否定假设$H_0$,接受$H_1$,求犯第一类错误概率和第二类错误概率$\alpha\beta$。

解:$\alpha=P\left\{U>\dfrac{3}{2}\bigg|H_0\right\}=\displaystyle{\int_\frac{3}{2}^2\dfrac{1}{2}\,\textrm{d}x=\dfrac{1}{4}}$。

$\beta=P\left\{U\leqslant\dfrac{3}{2}\bigg|H_1\right\}=\displaystyle{\int_0^{\frac{3}{2}}\dfrac{x}{2}\,\textrm{d}x=\dfrac{9}{16}}$。

\end{document}
