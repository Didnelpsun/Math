\documentclass[UTF8, 12pt]{ctexart}
% UTF8编码,ctexart现实中文
\usepackage{color}
% 使用颜色
\usepackage{geometry}
\setcounter{tocdepth}{4}
\setcounter{secnumdepth}{4}
% 设置四级目录与标题
\geometry{papersize={21cm,29.7cm}}
% 默认大小为A4
\geometry{left=3.18cm,right=3.18cm,top=2.54cm,bottom=2.54cm}
% 默认页边距为1英尺与1.25英尺
\usepackage{indentfirst}
\setlength{\parindent}{2.45em}
% 首行缩进2个中文字符
\usepackage{setspace}
\renewcommand{\baselinestretch}{1.5}
% 1.5倍行距
\usepackage{amssymb}
% 因为所以
\usepackage{amsmath}
% 数学公式
\usepackage[colorlinks,linkcolor=black,urlcolor=blue]{hyperref}
% 超链接
\author{Didnelpsun}
\title{随机事件与概率}
\date{}
\begin{document}
\maketitle
\pagestyle{empty}
\thispagestyle{empty}
\tableofcontents
\thispagestyle{empty}
\newpage
\pagestyle{plain}
\setcounter{page}{1}
\section{随机事件概率}

是基本事件关系的概率运算。

\textbf{例题:}已知事件$A$和$B$相互独立,$P(A)=a$,$P(B)=b$,如果事件$C$必然导致$AB$同时发生,则求$ABC$都不发生的概率。

解:首先必须理解题目的意思,并将其抽象为具体的计算式子。

$ABC$都不发生就是$A$不发生且$B$不发生且$C$不发生,用式子表达就是$\overline{A}\overline{B}\overline{C}$。

然后是分析事件$C$必然导致$AB$同时发生,$AB$同时发生就是$AB$,即$AB$比$C$的范围大,$C\subset AB$,$\overline{AB}\subset\overline{C}$,$\therefore\overline{ABC}=\overline{AB}\cap\overline{C}=\overline{AB}$。

又事件$AB$相互独立。$P(\overline{AB})=P(\overline{A})P(\overline{B})=(1-a)(1-b)$。

\section{概率模型}

\section{独立性}

\textbf{例题:}射手对同一目标独立地进行4次射击。若至少命中一次的概率为$\dfrac{15}{16}$,则求该射手对同一目标独立地进行4次射击中至少没命中一次的概率。

解:这个题目其实就是四重伯努利试验,彼此之间的概率都是独立的。令每一次命中的概率为$p$,则该次未命中的概率为$1-p$。

若至少命中一次的概率为$\dfrac{15}{16}$,则其对立事件全部不命中的概率为$1-\dfrac{15}{16}=\dfrac{1}{16}$,则$(1-p)^4=\dfrac{1}{16}$,则得到每次命中概率$p=\dfrac{1}{2}$。

求该射手对同一目标独立地进行4次射击中至少没命中一次的概率,则其对立事件为每次命中,其概率为$\left(\dfrac{1}{2}\right)^4=\dfrac{1}{16}$,则至少没命中一次的概率为$1-\dfrac{1}{16}=\dfrac{15}{16}$。

\end{document}
