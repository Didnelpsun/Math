\documentclass[UTF8, 12pt]{ctexart}
% UTF8编码,ctexart现实中文
\usepackage{color}
% 使用颜色
\usepackage{geometry}
\setcounter{tocdepth}{4}
\setcounter{secnumdepth}{4}
% 设置四级目录与标题
\geometry{papersize={21cm,29.7cm}}
% 默认大小为A4
\geometry{left=3.18cm,right=3.18cm,top=2.54cm,bottom=2.54cm}
% 默认页边距为1英尺与1.25英尺
\usepackage{indentfirst}
\setlength{\parindent}{2.45em}
% 首行缩进2个中文字符
\usepackage{setspace}
\renewcommand{\baselinestretch}{1.5}
% 1.5倍行距
\usepackage{amssymb}
% 因为所以
\usepackage{amsmath}
% 数学公式
\usepackage[colorlinks,linkcolor=black,urlcolor=blue]{hyperref}
% 超链接
\author{Didnelpsun}
\title{随机事件与概率}
\date{}
\begin{document}
\maketitle
\pagestyle{empty}
\thispagestyle{empty}
\tableofcontents
\thispagestyle{empty}
\newpage
\pagestyle{plain}
\setcounter{page}{1}

\section{排列组合}

排列公式:$A_n^m=\dfrac{n!}{(n-m)!}$。

组合公式:$C_n^m=\dfrac{n!}{(n-m)!m!}$。

\subsection{捆绑法}

要求某些元素必须在一起。

\textbf{例题:}$ABCDEF$六个人排队,要求$AB$必须在一起,问有多少种排法。

解:排法就是排列的问题。首先$AB$在一起,要么是$AB$要么是$BA$,也是一种排列,有$A_2^2$种。

然后将$AB$看作一个整体与$CDEF$进行排列,一共五个元素,进行全排列:$A_5^5$。

因为是按步骤来的,所以使用乘法:$A_2^2\cdot A_5^5=120$。

\subsection{插空法}

要求某些元素不能相邻。

\textbf{例题:}要对6个唱歌和4个舞蹈节目进行排列,要求两个舞蹈节目不能相邻,求多少种排法。

解:由于是对舞蹈进行限制,所以对唱歌的序列没有特别的限制,第一步先对唱歌进行全排列$A_6^6$。

第二步对舞蹈进行排列,由于舞蹈之间不能相邻,所以舞蹈节目必然是在两个唱歌节目之间进行插孔排序的,而唱歌6个节目一共7个空,所以排列$A_7^4$。

由于是步骤,所以乘法:$A_6^6\cdot A_7^4$。

\subsection{插板法}

与插空法类似,但是是组合进行归类,而不是排序。

\textbf{例题:}将8个完全相同的球放入三个不同的盒子,要求每个盒子至少有一个球,求一共有多少种放法。

解:相当于在七个空插入两个板。即$C_7^2=21$。

\section{随机事件概率}

是基本事件关系的概率运算。

\textbf{例题:}已知事件$A$和$B$相互独立,$P(A)=a$,$P(B)=b$,如果事件$C$必然导致$AB$同时发生,则求$ABC$都不发生的概率。

解:首先必须理解题目的意思,并将其抽象为具体的计算式子。

$ABC$都不发生就是$A$不发生且$B$不发生且$C$不发生,用式子表达就是$\overline{A}\overline{B}\overline{C}$。

然后是分析事件$C$必然导致$AB$同时发生,$AB$同时发生就是$AB$,即$AB$比$C$的范围大,$C\subset AB$,$\overline{AB}\subset\overline{C}$,$\therefore\overline{ABC}=\overline{AB}\cap\overline{C}=\overline{AB}$。

又事件$AB$相互独立。$P(\overline{AB})=P(\overline{A})P(\overline{B})=(1-a)(1-b)$。

\section{概率模型}

\subsection{古典概型}

\textbf{例题:}设袋中黑白球各一个,有放回取球,每次取一个,直到两种颜色的球都取到为止。求取球次数恰为3的概率。

解:若是看到题目中的为止,很容易想到后面的几何概率分布。但是这与几何分布不同,因为几何分布要求的停止条件是同一个事件的前几次的失败和最后一次的成功,而这个题目要求是取到黑并且取到白,而不是取到黑或取到白这两个事件,如求取到黑时取求次数为3的概率,则是几何分布。

每次可能取到黑白两种,取三次,则事件总数为$2^3=8$。而取到第三次才取到黑白两色的,则有黑黑白、白白黑两种可能性,所以概率为$\dfrac{2}{8}=\dfrac{1}{4}$。

\section{独立性}

\textbf{例题:}射手对同一目标独立地进行4次射击。若至少命中一次的概率为$\dfrac{15}{16}$,则求该射手对同一目标独立地进行4次射击中至少没命中一次的概率。

解:这个题目其实就是四重伯努利试验,彼此之间的概率都是独立的。令每一次命中的概率为$p$,则该次未命中的概率为$1-p$。

若至少命中一次的概率为$\dfrac{15}{16}$,则其对立事件全部不命中的概率为$1-\dfrac{15}{16}=\dfrac{1}{16}$,则$(1-p)^4=\dfrac{1}{16}$,则得到每次命中概率$p=\dfrac{1}{2}$。

求该射手对同一目标独立地进行4次射击中至少没命中一次的概率,则其对立事件为每次命中,其概率为$\left(\dfrac{1}{2}\right)^4=\dfrac{1}{16}$,则至少没命中一次的概率为$1-\dfrac{1}{16}=\dfrac{15}{16}$。

\end{document}
