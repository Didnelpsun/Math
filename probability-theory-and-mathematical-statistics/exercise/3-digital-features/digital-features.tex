\documentclass[UTF8, 12pt]{ctexart}
% UTF8编码,ctexart现实中文
\usepackage{color}
% 使用颜色
\usepackage{geometry}
\setcounter{tocdepth}{4}
\setcounter{secnumdepth}{4}
% 设置四级目录与标题
\geometry{papersize={21cm,29.7cm}}
% 默认大小为A4
\geometry{left=3.18cm,right=3.18cm,top=2.54cm,bottom=2.54cm}
% 默认页边距为1英尺与1.25英尺
\usepackage{indentfirst}
\setlength{\parindent}{2.45em}
% 首行缩进2个中文字符
\usepackage{setspace}
\renewcommand{\baselinestretch}{1.5}
% 1.5倍行距
\usepackage{amssymb}
% 因为所以
\usepackage{amsmath}
% 数学公式
\usepackage[colorlinks,linkcolor=black,urlcolor=blue]{hyperref}
% 超链接
\author{Didnelpsun}
\title{随机变量数字特征}
\date{}
\begin{document}
\maketitle
\pagestyle{empty}
\thispagestyle{empty}
\tableofcontents
\thispagestyle{empty}
\newpage
\pagestyle{plain}
\setcounter{page}{1}
\section{一维随机变量数字特征}

\subsection{数学期望}

\subsubsection{离散型随机变量}

\subsubsection{连续型随机变量}

\textbf{例题:}连续型随机变量$X$的概率密度为$f(x)=\dfrac{1}{\pi(1+x^2)}$($-\infty<x<+\infty$),求$EX$。

解:$EX=\int_{-\infty}^{+\infty}xf(x)\,\textrm{d}x=\int_{-\infty}^{+\infty}x\dfrac{1}{\pi(1+x^2)}\textrm{d}x=\dfrac{1}{2\pi}\int_{-\infty}^{+\infty}\dfrac{\textrm{d}(1+x^2)}{1+x^2}=\dfrac{1}{2pi}\ln(^1+x^2)|_{-\infty}^{+\infty}$。发散,所以不存在。

\subsubsection{连续型随机变量函数}

\textbf{例题:}连续型随机变量$X$的概率密度为$f(x)=\dfrac{1}{\pi(1+x^2)}$($-\infty<x<+\infty$),求$E(\min\{\vert X\vert,1\})$。

解:$E(\min\{\vert X\vert,1\})=\displaystyle{\int_{-\infty}^{+\infty}}\min\{\vert x\vert,1\}\dfrac{1}{\pi(1+x^2)}\textrm{d}x=\dfrac{2}{\pi}\int_0^{+\infty}\min\{x,1\}$\\$\dfrac{1}{1+x^2}\textrm{d}x=\dfrac{2}{\pi}\displaystyle{\int_0^1}x\dfrac{1}{1+x^2}\textrm{d}x+\dfrac{2}{\pi}\int_1^{+\infty}1\cdot\dfrac{1}{1+x^2}\textrm{d}x=\dfrac{1}{\pi}\ln(1+x^2)|_0^1+\dfrac{2}{\pi}\arctan x|_1^{+\infty}$\\$=\dfrac{1}{\pi}\ln2+\dfrac{1}{2}$。

\end{document}
