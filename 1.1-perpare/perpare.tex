\documentclass[UTF8]{ctexart}
\usepackage{geometry}
\geometry{papersize={21cm,29.7cm}}
\geometry{left=3.18cm,right=3.18cm,top=2.54cm,bottom=2.54cm}
\usepackage{indentfirst}
\setlength{\parindent}{2.45em}
\usepackage{setspace}
\onehalfspacing
\author{Didnelpsun}
\title{考研数学准备}
\begin{document}
\maketitle
\thispagestyle{empty}
\tableofcontents
\thispagestyle{empty}
\newpage
\pagestyle{plain}
\setcounter{page}{1}
\section{函数的概念与特性}
\subsection{函数}
\subsection{反函数}
\subsection{复合函数}
\subsection{有界性}
\subsection{单调性}
\subsection{奇偶性}
\subsection{周期性}
\section{函数的图像}
\subsection{直角坐标系图像}
\subsubsection{常见图像}
1. 基本初等函数与初等函数

2.分段函数
\subsubsection{图像变换}
1.平移变换

2.堆成变换

3.伸缩变换
\subsection{极坐标系图像}
\subsubsection{描点法}
1.心形线(外摆线)

2.玫瑰线

3.阿基米德螺线

4.伯努利双扭线
\subsubsection{直角坐标系下画极坐标图像}
\subsection{参数法}
\subsubsection{摆线(平摆线)}
\subsubsection{星形线(内摆线)}
\section{常用基础知识}
\subsection{数列}
\subsection{三角函数}
\subsection{指数运算法则}
\subsection{对数运算法则}
\subsection{一元二次方程基础}
\subsection{因式分解公式}
\subsection{阶乘与双阶乘}
\subsection{常用不等式}
\end{document}
