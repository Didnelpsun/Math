\documentclass[UTF8, 12pt]{ctexart}
\usepackage{color}
% 颜色
\definecolor{orange}{RGB}{255,127,0} 
\usepackage{amssymb}
% 因为所以
\usepackage{amsmath}
% 数学公式
\usepackage{geometry}
\setcounter{tocdepth}{4}
\setcounter{secnumdepth}{4}
\geometry{papersize={21cm,29.7cm}}
\geometry{left=3.18cm,right=3.18cm,top=2.54cm,bottom=2.54cm}
\usepackage{indentfirst}
\setlength{\parindent}{2.45em}
\usepackage{setspace}
\renewcommand{\baselinestretch}{1.5}
\author{Didnelpsun}
\title{考研数学准备}
\begin{document}
\maketitle
\thispagestyle{empty}
\tableofcontents
\thispagestyle{empty}
\newpage
\pagestyle{plain}
\setcounter{page}{1}
\section{函数的概念与特性}
\subsection{函数}
\begin{itemize}
    \item 函数即$y=f(x),x\in D$,x为自变量,y为因变量,D为定义域
    \item 一个x对应一个y,一个y可能对应多个x。
\end{itemize}
\subsection{反函数}
$y=f(x)$,定义域为$D$,值域为$R$,若对于每一个$y\in R$,必然存在$x\in D$使$y=f(x)$成立,则可以定义一个新函数$x=\psi(y)$,这个函数就是$y=f(x)$的\textbf{反函数},一般记作$x=f^{-1}(y)$,其定义域为$R$,值域为$D$,对于反函数,原来的函数称为\textbf{直接函数}。
\begin{enumerate}
    \item \textcolor{red}{严格单调}函数必然有反函数,即函数导数恒正或恒负必然有反函数。
    \item $x=f^{-1}(y)$与$y=f(x)$在同一坐标系中完全重合
    \item $y=f^{-1}(x)$与$y=f(x)$关于$y=x$对称。
    \item $f[f^{-1}(x)]$或$f[\psi(x)]$变为x,称为湮灭。
\end{enumerate}
\subsection{复合函数}
设$y=f(u)$的定义域为$D_1$,函数$u=g(x)$在$D$上有定义且$g(D)\in D$,则由$y=f[g(x)],x\in D$确定的函数称为由函数$u=g(x)$和函数$y=f(u)$构成的复合函数,定义域为D,u为中间变量。

\textbf{例题1:}设$f(x)=x^2$,$f[\psi(x)]=-x^2+2x+3$,且$\psi(x)\geqslant 0$,求$\psi(x)$以及定义域与值域。

广义化:$\because f(x)=x^2$,$\therefore f[\psi(x)]=\psi^2(x)=-x^2+2x+3$

又$\because\psi(x)\geqslant 0$, $\therefore\sqrt{\psi^2(x)}=\sqrt{-x^2+2x+3}=\psi(x)\geqslant 0$

$\therefore x\in[-1,3]$

$\therefore\frac{d\psi(x)}{dx}=(-x^2+2x+3)'=-2x+2=0$

$\therefore x=1$,驻点为1

又$\because(-x^2+2x+3)''=-2<0$

$\therefore$驻点为1时为最大值点,最大值为$\psi(1)=2$

又$\because\psi(-1)=\psi(3)=0$,$\therefore$最小值为0

$\therefore\psi(x)\in[0,2]$

\textcolor{orange}{注意}:$\sqrt{-x^2+2x+3}$为什么最值与$-x^2+2x+3$一致?

\textbf{例题2:}求函数$y=f(x)=\ln(x+\sqrt{x^2+1})$的反函数$f^{-1}(x)$的表达式及其定义域

首先研究$f(x)$本身,因为$\ln(x)$的定义域必然要求大于0,而任意实数x都有下面不等式成立:

$x+\sqrt{x^2+1}>x+\vert x\vert \geqslant 0$,所以$x\in R$。

而研究其奇偶性:

$f(-x)=\ln(-x+\sqrt{x^2+1})=\ln(\frac{1}{\sqrt{x^2+1}+x})=-\ln(x+\sqrt{x^2+1})=-f(x)$

所以该函数为奇函数。

对其求单调性,即通过链式法则求导:

$\frac{dy}{dx}=\frac{1}{x+\sqrt{x^2+1}}\cdot (1+\frac{2x}{2\sqrt{x^2+1}})=\frac{1}{\sqrt{x^2+1}}>0$

所以该函数严格单调增。

然后求$y$的反函数。

$$
\begin{aligned}
    \because y&=\ln(x+\sqrt{x^2+1}) \\
    e^y&=e^{\ln(x+\sqrt{x^2+1})} \\
    &=x+\sqrt{x^2+1}
\end{aligned}
$$

$$
\begin{aligned}
    \because -y&=-\ln(x+\sqrt{x^2+1}) \\
    &=\ln(\frac{1}{x+\sqrt{x^2+1}}) \\
    &=\ln(\sqrt{x^2+1}-x) \\
    e^{-y}&=\sqrt{x^2+1}-x
\end{aligned}
$$

$$
\begin{aligned}
    \therefore e^y-e^{-y}&=2x \\
    x&=\frac{e^y-e^{-y}}{2}
\end{aligned}
$$

解出了用x表示y的函数表达$x=f^{-1}(y)$,即反函数,则$f^{-1}(x)=\frac{e^x-e^{-x}}{2}$

这种曲线为一种常见曲线:

\begin{itemize}
    \item $\frac{e^x-e^{-x}}{2}$:双曲正弦。
    \item $\frac{e^x+e^{-x}}{2}$:双曲余弦。(为一种悬链线)
    \item $\ln(x+\sqrt{x^2+1})$:反双曲正弦。
    \item $\ln(x+\sqrt{x^2-1})$:反双曲余弦。
\end{itemize}

\textbf{例题3:}设$
f(x)=\left\{
\begin{array}{rcl}
\ln\sqrt{x} &  & {x\geqslant 1}\\
2x-1 & & {x< 1}
\end{array} \right. 
$,求$f[f(x)]$

\subsection{有界性}
\subsection{单调性}
\subsection{奇偶性}
\subsection{周期性}
\section{函数的图像}
\subsection{直角坐标系图像}
\subsubsection{常见图像}
\paragraph{基本初等函数与初等函数}
\paragraph{分段函数}
\subsubsection{图像变换}
\paragraph{平移变换}
\paragraph{对称变换}
\paragraph{伸缩变换}
\subsection{极坐标系图像}
\subsubsection{描点法}
\paragraph{心形线(外摆线)}
\paragraph{玫瑰线}
\paragraph{阿基米德螺线}
\paragraph{伯努利双扭线}
\subsubsection{直角坐标系下画极坐标图像}
\subsection{参数法}
\subsubsection{摆线(平摆线)}
\subsubsection{星形线(内摆线)}
\section{常用基础知识}
\subsection{数列}
\subsection{三角函数}
\subsection{指数运算法则}
\subsection{对数运算法则}
\subsection{一元二次方程基础}
\subsection{因式分解公式}
\subsection{阶乘与双阶乘}
\subsection{常用不等式}
\end{document}
