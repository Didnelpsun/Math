\documentclass[UTF8, 12pt]{ctexart}
\usepackage{color}
\usepackage{geometry}
\setcounter{tocdepth}{4}
\setcounter{secnumdepth}{4}
\geometry{papersize={21cm,29.7cm}}
\geometry{left=3.18cm,right=3.18cm,top=2.54cm,bottom=2.54cm}
\usepackage{indentfirst}
\setlength{\parindent}{2.45em}
\usepackage{setspace}
\renewcommand{\baselinestretch}{1.5}
\author{Didnelpsun}
\title{考研数学准备}
\begin{document}
\maketitle
\thispagestyle{empty}
\tableofcontents
\thispagestyle{empty}
\newpage
\pagestyle{plain}
\setcounter{page}{1}
\section{函数的概念与特性}
\subsection{函数}
\begin{itemize}
    \item 函数即$y=f(x),x\in D$,x为自变量,y为因变量,D为定义域
    \item 一个x对应一个y,一个y可能对应多个x。
\end{itemize}
\subsection{反函数}
$y=f(x)$,定义域为$D$,值域为$R$,若对于每一个$y\in R$,必然存在$x\in D$使$y=f(x)$成立,则可以定义一个新函数$x=\psi (y)$,这个函数就是$y=f(x)$的\textbf{反函数},一般记作$x=f^{-1}(y)$,其定义域为$R$,值域为$D$,对于反函数,原来的函数称为\textbf{直接函数}。
\begin{enumerate}
    \item \textcolor{red}{严格单调}函数必然有反函数,即函数导数恒正或恒负必然有反函数。
    \item $x=f^{-1}(y)$与$y=f(x)$在同一坐标系中完全重合
    \item $y=f^{-1}(x)$与$y=f(x)$关于$y=x$对称。
    \item $f[f^{-1}(x)]$或$f[\psi (x)]$变为x,称为湮灭。
\end{enumerate}
\subsection{复合函数}
设$y=f(u)$的定义域为$D_1$,函数$u=g(x)$在$D$上有定义且$g(D)\in D$,则由$y=f[g(x)],x\in D$确定的函数称为由函数$u=g(x)$和函数$y=f(u)$构成的复合函数,定义域为D,u为中间变量。
\subsection{有界性}
\subsection{单调性}
\subsection{奇偶性}
\subsection{周期性}
\section{函数的图像}
\subsection{直角坐标系图像}
\subsubsection{常见图像}
\paragraph{基本初等函数与初等函数}
\paragraph{分段函数}
\subsubsection{图像变换}
\paragraph{平移变换}
\paragraph{对称变换}
\paragraph{伸缩变换}
\subsection{极坐标系图像}
\subsubsection{描点法}
\paragraph{心形线(外摆线)}
\paragraph{玫瑰线}
\paragraph{阿基米德螺线}
\paragraph{伯努利双扭线}
\subsubsection{直角坐标系下画极坐标图像}
\subsection{参数法}
\subsubsection{摆线(平摆线)}
\subsubsection{星形线(内摆线)}
\section{常用基础知识}
\subsection{数列}
\subsection{三角函数}
\subsection{指数运算法则}
\subsection{对数运算法则}
\subsection{一元二次方程基础}
\subsection{因式分解公式}
\subsection{阶乘与双阶乘}
\subsection{常用不等式}
\end{document}
