\documentclass[UTF8]{ctexart}
% UTF8编码,ctexart现实中文
\usepackage{color}
% 使用颜色
\usepackage{geometry}
\setcounter{tocdepth}{4}
\setcounter{secnumdepth}{4}
% 设置四级目录与标题
\geometry{papersize={21cm,29.7cm}}
% 默认大小为A4
\geometry{left=3.18cm,right=3.18cm,top=2.54cm,bottom=2.54cm}
% 默认页边距为1英尺与1.25英尺
\usepackage{indentfirst}
\setlength{\parindent}{2.45em}
% 首行缩进2个中文字符
\usepackage{setspace}
\renewcommand{\baselinestretch}{1.5}
% 1.5倍行距
\author{Didnelpsun}
\title{一元函数微分学}
\begin{document}
\maketitle
\thispagestyle{empty}
\tableofcontents
\thispagestyle{empty}
\newpage
\pagestyle{plain}
\setcounter{page}{1}
\section{概念}
\subsection{引例}
\subsection{导数}
\subsection{微分}
\section{导数与微分计算}
\subsection{四则运算}
\subsection{分段函数的导数}
\subsection{复合函数的导数与微分形式不变性}
\subsection{反函数导数}
\subsection{参数方程函数导数}
\subsection{隐函数求导法}
\subsection{对数求导法}
\subsection{幂指函数求导法}
\subsection{高阶导数}
\subsubsection{归纳法}
\subsubsection{莱布尼茨公式}
\subsubsection{泰勒公式}
\subsection{变限积分求导公式}
\subsection{基本求导公式}
\end{document}
