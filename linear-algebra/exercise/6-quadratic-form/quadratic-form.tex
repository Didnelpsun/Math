\documentclass[UTF8, 12pt]{ctexart}
% UTF8编码,ctexart现实中文
\usepackage{color}
% 使用颜色
\usepackage{geometry}
\setcounter{tocdepth}{4}
\setcounter{secnumdepth}{4}
% 设置四级目录与标题
\geometry{papersize={21cm,29.7cm}}
% 默认大小为A4
\geometry{left=3.18cm,right=3.18cm,top=2.54cm,bottom=2.54cm}
% 默认页边距为1英尺与1.25英尺
\usepackage{indentfirst}
\setlength{\parindent}{2.45em}
% 首行缩进2个中文字符
\usepackage{setspace}
\renewcommand{\baselinestretch}{1.5}
% 1.5倍行距
\usepackage{amssymb}
% 因为所以
\usepackage{amsmath}
% 数学公式
\usepackage[colorlinks,linkcolor=black,urlcolor=blue]{hyperref}
% 超链接
\author{Didnelpsun}
\title{二次型}
\date{}
\begin{document}
\maketitle
\pagestyle{empty}
\thispagestyle{empty}
\tableofcontents
\thispagestyle{empty}
\newpage
\pagestyle{plain}
\setcounter{page}{1}
\section{二次型}

即最基本的将二次型式子变为矩阵形式。

\subsection{配方法}

\subsection{矩阵乘法}

由于二次型是$X^TAX$的形式,所以最后的左右两边都存在所有的$x_i$,所以可以依次把$x_i$缺的项进行补齐$x_n$与其他所有$x_i$乘积的和的形式。

\textbf{例题:}将二次型$f(x_1,x_2,x_3)=2x_1^2+2x_2^2+2x_3^2-2x_1x_2-2x_2x_3+2x_1x_3$化为矩阵。

解:$f(x_1,x_2,x_3)=2x_1^2+2x_2^2+2x_3^2-2x_1x_2-2x_2x_3+2x_1x_3=2x_1^2-2x_1x_2+2x_1x_3+2x_2^2-2x_2x_3+2x_3^2=2x_1^2-x_1x_2+x_1x_3+2x_2^2-x_1x_2-x_2x_3+2x_3^2+x_1x_3-x_2x_3=x_1(2x_1-x_2+x_3)+x_2(-x_1+2x_2-x_3)+x_3(x_1-x_2+2x_3)$

$=\left[x_1,x_2,x_3\right]\left[\begin{array}{c}
    2x_1-x_2+x_3 \\
    -x_1+2x_2-x_3 \\\
    x_1-x_2+2x_3
\end{array}\right]=[x_1,x_2,x_3]\left[\begin{array}{ccc}
    2 & -1 & 1\\
    -1 & 2 & -1\\
    1 & -1 & 2
\end{array}\right]\left[\begin{array}{c}
    x_1 \\
    x_2 \\
    x_3
\end{array}\right]$。

即$A=\left[\begin{array}{ccc}
    2 & -1 & 1\\
    -1 & 2 & -1\\
    1 & -1 & 2
\end{array}\right]$。

\section{标准形}

即将二次型式子变为平方形式,再变量更换,变成矩阵形式。

\subsection{初等变换法}

$f(x)=X^TAX$,线性变换$X=CY$,$C^TAC=\Lambda$,又$C$可逆,$\therefore C=P_1P_2\cdots P_s$,$EP_1P_2\cdots P_s=C$,$\therefore(P_1P_2\cdots P_s)^TAP_1P_2\cdots P_3=\Lambda$,

\begin{enumerate}
    \item 对$A,E$做同样的初等列变换。
    \item 对$A$做相应的初等行变换。(交换$i,j$列就要交换$i,j$行)。一套行列变换后$\Lambda$为对称矩阵。
    \item $A$化成对角矩阵时,$E$化成的就是$C$。
\end{enumerate}

$\left(\begin{array}{c}
    A \\
    E
\end{array}\right)\rightarrow\left(\begin{array}{c}
    \Lambda \\
    C
\end{array}\right)$,对整个列变换,只对$A$行变换。

$\left(\begin{array}{c}
    A \\
    E
\end{array}\right)=\left(\begin{array}{ccc}
    1 & 1 & 1 \\
    1 & 2 & 2 \\
    1 & 2 & 1 \\
    1 & 0 & 0 \\
    0 & 1 & 0 \\
    0 & 0 & 1
\end{array}\right)=\left(\begin{array}{ccc}
    1 & 0 & 1 \\
    1 & 1 & 2 \\
    1 & 1 & 1 \\
    1 & -1 & 0 \\
    0 & 0 & 0 \\
    0 & 0 & 1
\end{array}\right)=\left(\begin{array}{ccc}
    1 & 0 & 1 \\
    0 & 1 & 1 \\
    1 & 1 & 1 \\
    1 & -1 & 0 \\
    0 & 0 & 0 \\
    0 & 0 & 1
\end{array}\right)=\\\left(\begin{array}{ccc}
    1 & 0 & 0 \\
    0 & 1 & 1 \\
    1 & 1 & 0 \\
    1 & -1 & -1 \\
    0 & 0 & 0 \\
    0 & 0 & 1
\end{array}\right)=\left(\begin{array}{ccc}
    1 & 0 & 0 \\
    0 & 1 & 1 \\
    0 & 1 & 0 \\
    1 & -1 & -1 \\
    0 & 1 & 0 \\
    0 & 0 & 1
\end{array}\right)=\left(\begin{array}{ccc}
    1 & 0 & 0 \\
    0 & 1 & 0 \\
    0 & 1 & -1 \\
    1 & -1 & 0 \\
    0 & 1 & -1 \\
    0 & 0 & 1
\end{array}\right)=\left(\begin{array}{ccc}
    1 & 0 & 0 \\
    0 & 1 & 0 \\
    0 & 0 & -1 \\
    1 & -1 & 0 \\
    0 & 1 & -1 \\
    0 & 0 & 1
\end{array}\right)$

$\therefore\Lambda=\left(\begin{array}{ccc}
    1 & 0 & 0 \\
    0 & 1 & 0 \\
    0 & 0 & -1
\end{array}\right)$,$C=\left(\begin{array}{ccc}
    1 & -1 & 0 \\
    0 & 1 & -1 \\
    0 & 0 & 1
\end{array}\right)$

\subsection{可逆线性变换法}

即配方法,求可逆线性变换。

\begin{enumerate}
    \item 如果二次型有平方项,则首先从$x_1$开始往后不断配方,让最后的式子全部以平方加和的形式,从而不会有混合项。
    \item 如果二次型没有平方项,则首先令$x_1=y_1+y_2$,$x_2=y_1-y_2$,$x_i=y_i$等然后带入$f(x)$强行出现平方项,然后配方,成功后再用$z_i$替换。
    \item 如果总的完全平方项数小于变量个数,则令多余的$x_i$为$y_i$,系数为0。
\end{enumerate}

\subsubsection{平方项}

即依次对存在$x_i$的式子进行整合配方。从$x_1$开始,后面含$x_1$的都提到一起配方,然后依次按这个方法进行配方。

\textbf{例题:}将$f(x_1,x_2,x_3)=x_1^2+2x_1x_2+2x_1x_3-x_2^2-2x_2x_3-x_3^2$化为标准形并求出作的可逆线性变换。

解:首先对$x_1$进行配方,因为有$x_1$因子的式子有$x_1^2+2x_1x_2+2x_1x_3$。

所以将$x_1,x_2,x_3$全部配在一起:$(x_1+x_2+x_3)^2=x_1^2+x_2^2+x_3^2+2x_1x_2+2x_1x_3+2x_2x_3$。

所以$f(x)=(x_1+x_2+x_3)^2-2x_2^2-4x_2x_3-2x_3^2$,然后继续配$x_2$。

因为还有$-2x_2^2-4x_2x_3$,所以配成$-2(x_2+x_3)^2$,正好全部配完了。

$\therefore f(x)=(x_1+x_2+x_3)^2-2(x_2+x_3)^2$。

令$y_1=x_1+x_2+x_3$,$y_2=x_2+x_3$,补$y_3=x_3$,$\therefore f=y_1^2-2y_2^2$。

$(y_1,y_2,y_3)^T=\left(\begin{array}{ccc}
    1 & 1 & 1 \\
    0 & 1 & 1 \\
    0 & 0 & 1
\end{array}\right)(x_1,x_2,x_3)^T$,此时是$y=Dx$,但是我们要求的是$x=Cy$,所以$C=D^{-1}$,所以$D^{-1}$才是作出的可逆线性变换。

所以得到的线性变换为$\left(\begin{array}{ccc}
    1 & -1 & 0 \\
    0 & 1 & -1 \\
    0 & 0 & 1
\end{array}\right)$。

这样方法还要重新求逆,比较麻烦。实际上我们要求的是$x=Cy$,即用$y$来表示$x$,从而直接将$y$来表示$x$就可以了。

首先$y_3=x_3$,所以$x_2=y_2-x_3=y_2-y_3$,$x_1=y_1-x_2-x_3=y_1-y_2+y_3-y_3=y_1-y_2$,综上$x_1=y_1-y_2$,$x_2=y_2-y_3$,$x_3=y_3$,也得到同样结果。

\subsubsection{无平方项}

\textbf{例题:}将二次型$f(x_1,x_2,x_3)=x_1x_2+x_1x_3-x_2x_3$化为规范形,并求所用的可逆线性变换。

解:因为二次型中没有平方项式子,而如果进行配方一定会出现平方,就会产生冲突,所以希望把$x$代换称有平方的式子。

令$x_1=y_1+y_2$,$x_2=y_1-y_2$,$x_3=y_3$,代入二次型中。

$f=y_1^2-y_2^2+y_1y_3+y_2y_3-y_1y_3-+y_2y_3=y_1^2-y_2^2+2y_2y_3=y_1^2-y_2^2+2y_2y_3$。

此时由没有平方项就变成了有平方项,所以就能进行配方。

$=y_1^2-(y_2-y_3)^2+y_3^2$,继续之前的步骤,进行换元:

令$z_1=y_1$,$z_2=y_2-y_3$,$z_3=y_3$,$f=z_1^2-z_2^2+z_3^2$得到标准形。

对于$x$与$y$:$(x_1,x_2,x_3)^T=\left(\begin{array}{ccc}
    1 & 1 & 0 \\
    1 & -1 & 0 \\
    0 & 0 & 1
\end{array}\right)(y_1,y_2,y_3)^T$。$y$作为过渡变量。

将$y$转换为$z$:$(z_1,z_2,z_3)^T=\left(\begin{array}{ccc}
    1 & 0 & 0 \\
    0 & 1 & -1 \\
    0 & 0 & 1
\end{array}\right)(y_1,y_2,y_3)^T$,我们需要$x=Cz$。

$(x_1,x_2,x_3)^T=\left(\begin{array}{ccc}
    1 & 1 & 0 \\
    1 & -1 & 0 \\
    0 & 0 & 1
\end{array}\right)\left(\begin{array}{ccc}
    1 & 0 & 0 \\
    0 & 1 & -1 \\
    0 & 0 & 1
\end{array}\right)^{-1}(z_1,z_2,z_3)^T$,从而得到$C=\left(\begin{array}{ccc}
    1 & 1 & 1 \\
    1 & -1 & -1 \\
    0 & 0 & 1
\end{array}\right)$。

\subsection{正交变换法}

即求正交变换。

\textbf{例题:}将二次型$f(x_1,x_2,x_3)=2x_1^2+5x_2^2+5x_3^2+4x_1x_2-4x_1x_3-8x_2x_3$使用正交变换法化为标准形,并求所作的正交变换。

已知将二次型通过矩阵表示:$=(x_1,x_2,x_3)\left(\begin{array}{ccc}
    2 & 2 & -2 \\
    2 & 5 & -4 \\
    -2 & -4 & 5
\end{array}\right)(x_1,x_2,x_3)^T$。\medskip

这个矩阵跟第五章相似的实对称矩阵相似对角化的例题的矩阵一样。

所以直接结果:$\lambda_1=\lambda_2=1$,$\lambda_3=10$,$\eta_1'=\dfrac{\sqrt{5}}{5}(-2,1,0)^T$,$\eta_2'=\dfrac{\sqrt{5}}{15}(2,4,5)^T$,$\eta_3'=\dfrac{1}{3}(1,2,-2)^T$。

第五步:$f(x)=g(y)=y^T\Lambda y=(y_1,y_2,y_3)\left(\begin{array}{ccc}
    1 \\
     & 1 \\
     & & 10
\end{array}\right)(y_1,y_2,y_3)^T=y_1^2+y_2^2+10y_3^2$

\section{规范形}

由于只有少部分二次型能转换为规范形,所以基本上都是选择题考察。

而且因为规范形的系数必然是0或1或-1,所以不需要求,直接使用惯性定理即可求出。

\subsection{惯性定理}

多用于规范形的判断。

\textbf{例题:}二次型$f(x_1,x_2,x_3)=x_1^2+4x_2^2+4x_3^2-4x_1x_2+4x_1x_3-8x_2x_3$的规范形为()。

$A.f=z_1^2\qquad B.f=z_1^2-z_2^2\qquad C.f=z_1^2+z_2^2+z_3^2\qquad D.f=z_1^2+z_2^2-z_3^2$

解:

已知$f$的二次型矩阵表示$A=\left[\begin{array}{ccc}
    1 & -1 & 2 \\
    -2 & 4 & -4 \\
    2 & -4 & 4
\end{array}\right]$,根据特征方程$\vert\lambda E-A\vert=\lambda^2(\lambda-9)=0$,$\lambda_1=9$,$\lambda_2=\lambda_3=0$,所以根据特征值符号,正惯性系数$p=1$,负惯性系数$q=0$,所以选择$A$。

\section{合同}

\subsection{合同判断}

合同基于二次型,所以只有对称矩阵才能讨论是否合同。

二次型的合同只有两种判断方式:

\begin{enumerate}
    \item 秩相同,正(负)惯性系数相同。
    \item 正负惯性系数都相同。
\end{enumerate}

\textbf{例题:}设$A=\left[\begin{array}{ccc}
    1 & 2 & 0 \\
    2 & 1 & 0 \\
    0 & 0 & 1
\end{array}\right]$,与$A$合同的是()。

$A.\left[\begin{array}{ccc}
    1 & 0 & 0 \\
    0 & 1 & 0 \\
    0 & 0 & 1
\end{array}\right]$\;$B.\left[\begin{array}{ccc}
    1 & 0 & 0 \\
    0 & 1 & 0 \\
    0 & 0 & -1
\end{array}\right]$\;$C.\left[\begin{array}{ccc}
    1 & 0 & 0 \\
    0 & -1 & 0 \\
    0 & 0 & -1
\end{array}\right]$\;$D.\left[\begin{array}{ccc}
    -1 & 0 & 0 \\
    0 & -1 & 0 \\
    0 & 0 & -1
\end{array}\right]$ \medskip

解:从四个选项,由于是常量矩阵,所以由对角线元素的正负号可以得出这四个的惯性系数分别为$(3,0)$、$(2,1)$、$(1,2)$、$(0,3)$(前面为正惯性系数,后面为负惯性系数)。

且每个选项的秩都是3。

\subsubsection{配方法}

即将二次型配方为标准型,然后求该矩阵的秩和惯性系数。

解:经过配方$f=(x_1+2x_2)^2-3x_2^2+x_3^2$,由于有三个平方项,所以矩阵秩为3,正惯性系数为2,与$B$相同。

\subsubsection{特征值法}

即根据特征方程进行正交变换得到正负惯性系数。

解:求$A$的特征值,得到$\lambda_1=1$、$\lambda_2=3$、$\lambda_3=-1$,所以正交变换后标准形为$y_1^2+3y_2^2-y_3^2$,惯性系数与$B$相同。

\subsection{可逆矩阵}

已知$A\simeq\Lambda$,则$C^TAC=\Lambda$。即$f=x^TAx=y^T\Lambda y$,得到$x=Cy$。

\textbf{例题:}已知$A=\left[\begin{array}{ccc}
    1 & 0 & 0 \\
    0 & -4 & 0 \\
    0 & 0 & \dfrac{1}{9}
\end{array}\right]$合同于$\Lambda=\left[\begin{array}{ccc}
    1 & 0 & 0 \\
    0 & 1 & 0 \\
    0 & 0 & -1
\end{array}\right]$,求$C^TAC=\Lambda$中的$C$。

解:已知$A$,则可得二次型$f=x^TAx=[x_1,x_2,x_3]A[x_1,x_2,x_3]^T=x_1^2-4x_2^2+\dfrac{1}{9}x_3^2$,规范化让这个二次型与$\Lambda$转换的二次型相等,由于正负惯性系数相同,平方必然是正数,所以符号对齐,令$x_1^2=y_1^2$、$4x_2^2=y_3^2$、$\dfrac{1}{9}x_3^2=y_2^2$。

解得$x_1=y_1$,$x_2=\dfrac{1}{2}y_3$,$x_3=3y_2$,即$\left[\begin{array}{c}
    x_1 \\
    x_2 \\
    x_3
\end{array}\right]=\left[\begin{array}{ccc}
    1 \\
    & & \dfrac{1}{2} \\
    & 3
\end{array}\right]\left[\begin{array}{c}
    y_1 \\
    y_2 \\
    y_3
\end{array}\right]$。

所以$x=Cy$,解得$C=\left[\begin{array}{ccc}
    1 \\
    & & \dfrac{1}{2} \\
    & 3
\end{array}\right]$,此时$f=x^TAx=y^TC^TACy=y^T\Lambda y$。

\section{正定二次型}

\subsection{具体型}

\begin{enumerate}
    \item 顺序主子式全部大于0。
    \item 特征值全部大于0。
    \item 配方化为全平方和的标准型,正惯性指数$p=n$(未知数个数)。
    \item 矩阵乘法配方为完全平方和,内积$D^TD$不等于0。
\end{enumerate}

\subsection{抽象型}

\section{二次型最值}

若$A$的特征值大小排序$\lambda_1\leqslant\lambda_2\leqslant\cdots\leqslant\lambda_n$,则:

\begin{itemize}
    \item $\lambda_1x^Tx\leqslant x^TAx\leqslant\lambda_nx^Tx$。
    \item 若$x^Tx=1$,则$f_{\min}=\lambda_1$,$f_{\max}=\lambda_n$。
\end{itemize}

\end{document}
