\documentclass[UTF8, 12pt]{ctexart}
% UTF8编码,ctexart现实中文
\usepackage{color}
% 使用颜色
\usepackage{geometry}
\setcounter{tocdepth}{4}
\setcounter{secnumdepth}{4}
% 设置四级目录与标题
\geometry{papersize={21cm,29.7cm}}
% 默认大小为A4
\geometry{left=3.18cm,right=3.18cm,top=2.54cm,bottom=2.54cm}
% 默认页边距为1英尺与1.25英尺
\usepackage{indentfirst}
\setlength{\parindent}{2.45em}
% 首行缩进2个中文字符
\usepackage{setspace}
\renewcommand{\baselinestretch}{1.5}
% 1.5倍行距
\usepackage{amssymb}
% 因为所以
\usepackage{amsmath}
% 数学公式
\usepackage[colorlinks,linkcolor=black,urlcolor=blue]{hyperref}
% 超链接
\author{Didnelpsun}
\title{矩阵}
\date{}
\begin{document}
\maketitle
\pagestyle{empty}
\thispagestyle{empty}
\tableofcontents
\thispagestyle{empty}
\newpage
\pagestyle{plain}
\setcounter{page}{1}
\section{矩阵的幂}

\subsection{对应成比例}

因为矩阵运算不满足交换率但是满足结合率,且一行矩阵乘一列矩阵的乘积为一个数,所以可以推出矩阵的幂的运算方法。

这个方法要求$r(A)=1$,即对应成比例。

令$A$为$n$阶方阵,将$A$拆为$A=(a_1,a_2,\cdots,a_n)^T(b_1,b_2,\cdots,b_n)=\alpha^T\beta$,所以$A^n=\alpha^T\beta\alpha^T\beta\cdots\alpha^T\beta$,利用结合率:$\alpha^T(\beta\alpha^T)(\beta\cdots\alpha^T)\beta$,中间一共$n-1$个$\beta\alpha^T$,$\beta\alpha^T$是一个数,即$A^n=(\beta\alpha^T)^{n-1}\alpha^T\beta=(\beta\alpha^T)^{n-1}A$。\medskip

\textbf{例题:}$A=\left(\begin{array}{ccc}
    1 & 2 & 3 \\
    -2 & -4 & -6 \\
    3 & 6 & 9
\end{array}\right)$,求$A^n$。\medskip

解:$A=(1,-2,3)^T(1,2,3)$,所以$A^n=((1,2,3)(1,-2,3)^T)^n(1,-2,3)^T(1,2,3)$

$=6^{n-1}A$。

若矩阵$A$的行与列都成比例,则$A^n=[tr(A)]^{n-1}A$,$[tr(A)]=\sum a_{ii}$,即矩阵迹为对角线元素值之和。

\subsection{试算归纳}

对$A$进行试算,如$A^2$,若$A^k$是一个数量阵,那么计算$A^n$就只用找规律就可以了。

\textbf{例题:}$A=\left(\begin{array}{cccc}
    1 & -1 & -1 & -1 \\
    -1 & 1 & -1 & -1 \\
    -1 & -1 & 1 & -1 \\
    -1 & -1 & -1 & 1 \\
\end{array}\right)$,求$A^n$($n\geqslant2$)。\medskip

解:通过计算得知$A^2=4E$,这是一个数量阵。\medskip

$\therefore A^n=\left\{\begin{array}{lcl}
    4^kE, & & n=2k \\
    4^kA, & & n=2k+1
\end{array}\right.$。

\subsection{拆分矩阵}

将$A^n$拆分为两个矩阵$A^n=(B+C)^n$,其中$BC$应该是可逆的,即$BC=CB$,所以一般有一个是$E$。\medskip

\textbf{例题:}$A=\left(\begin{array}{ccc}
    1 & 1 & 0 \\
    0 & 1 & 1 \\
    0 & 0 & 1
\end{array}\right)$,求$A^n$。\medskip

解:$A=E+B=\left(\begin{array}{ccc}
    1 & 0 & 0 \\
    0 & 1 & 0 \\
    0 & 0 & 1
\end{array}\right)+\left(\begin{array}{ccc}
    0 & 1 & 0 \\
    0 & 0 & 1 \\
    0 & 0 & 0
\end{array}\right)$。\medskip

$\therefore A^n=(E+B)^n=C_n^0E^n+C_n^1E^{n-1}B+C_n^2E^{n-2}B^2+\cdots$。

又$B^2=\left(\begin{array}{ccc}
    0 & 1 & 0 \\
    0 & 0 & 1 \\
    0 & 0 & 0
\end{array}\right)\left(\begin{array}{ccc}
    0 & 1 & 0 \\
    0 & 0 & 1 \\
    0 & 0 & 0
\end{array}\right)=\left(\begin{array}{ccc}
    0 & 0 & 1 \\
    0 & 0 & 0 \\
    0 & 0 & 0
\end{array}\right)$。

$B^3=B^2B=\left(\begin{array}{ccc}
    0 & 0 & 1 \\
    0 & 0 & 0 \\
    0 & 0 & 0
\end{array}\right)\left(\begin{array}{ccc}
    0 & 1 & 0 \\
    0 & 0 & 1 \\
    0 & 0 & 0
\end{array}\right)=\left(\begin{array}{ccc}
    0 & 0 & 0 \\
    0 & 0 & 0 \\
    0 & 0 & 0
\end{array}\right)=O$。

$\therefore B^4=B^5=\cdots=O$。

$\therefore A^n=(E+B)^n=C_n^0E^n+C_n^1E^{n-1}B+C_n^2E^{n-2}B^2$。\medskip

$=\left(\begin{array}{ccc}
    1 & 0 & 0 \\
    0 & 1 & 0 \\
    0 & 0 & 1
\end{array}\right)+n\left(\begin{array}{ccc}
    0 & 1 & 0 \\
    0 & 0 & 1 \\
    0 & 0 & 0
\end{array}\right)+\dfrac{n(n-1)}{2}\left(\begin{array}{ccc}
    0 & 0 & 1 \\
    0 & 0 & 0 \\
    0 & 0 & 0
\end{array}\right)$

\section{逆矩阵}

\subsection{定义法}

找出一个矩阵$B$,使得$AB=E$,则$A$可逆,$A^{-1}=B$。

\textbf{例题:}$A$,$B$均是$n$阶方阵,且$AB=A+B$,证明$A-E$可逆,并求$(A-E)^{-1}$。

解:要证明$A-E$,就要从$AB=A+B$中尽量凑出。

$AB=A+B$变为$AB-B=A$,从而提取$(A-E)B=A$,$(A-E)BA^{-1}=E$。

但是$A^{-1}$是未知的,所以$A-E$的逆矩阵不能用$BA^{-1}$来表示。

$AB-A=B$,所以提出$A(B-E)=B$,即$A(B-E)=B-E+E$,$(A-E)(B-E)=E$,所以$A-E$的逆矩阵就是$B-E$。

\subsection{分解乘积}

将$A$分解为若干个可逆矩阵的乘积。若$A=BC$,$B$,$C$可逆,则$A$可逆,且$A^{-1}=C^{-1}B^{-1}$。

\textbf{例题:}设$A$,$B$为同阶可逆方阵,且$A^{-1}+B^{-1}$可逆,求$(A+B)^{-1}$。

解:已知$A^{-1}+B^{-1}$可以用来表示其他式子,需要求$A+B$的逆,则需要将$A+B$转为其逆。

$\because A+B=A(E+A^{-1}B)=A(B^{-1}+A^{-1})B$。

$\therefore (A+B)^{-1}=B^{-1}(B^{-1}+A^{-1})^{-1}A^{-1}$。

\subsection{分块矩阵}

对于一些分块矩阵的逆,若$A$,$B$都可逆,则:$\left[\begin{array}{cc}
    A & O \\
    O & B
\end{array}\right]^{-1}=\left[\begin{array}{cc}
    A^{-1} & O \\
    O & B^{-1}
\end{array}\right]$,$\left[\begin{array}{cc}
    O & A \\
    B & O
\end{array}\right]^{-1}=\left[\begin{array}{cc}
    O & B^{-1} \\
    A^{-1} & O
\end{array}\right]$。

\end{document}
