\documentclass[UTF8, 12pt]{ctexart}
% UTF8编码,ctexart现实中文
\usepackage{color}
% 使用颜色
\usepackage{geometry}
\setcounter{tocdepth}{4}
\setcounter{secnumdepth}{4}
% 设置四级目录与标题
\geometry{papersize={21cm,29.7cm}}
% 默认大小为A4
\geometry{left=3.18cm,right=3.18cm,top=2.54cm,bottom=2.54cm}
% 默认页边距为1英尺与1.25英尺
\usepackage{indentfirst}
\setlength{\parindent}{2.45em}
% 首行缩进2个中文字符
\usepackage{setspace}
\renewcommand{\baselinestretch}{1.5}
% 1.5倍行距
\usepackage{amssymb}
% 因为所以
\usepackage{amsmath}
% 数学公式
\usepackage[colorlinks,linkcolor=black,urlcolor=blue]{hyperref}
% 超链接
\usepackage{rotating}
% 用于旋转对象(旋转包)
\author{Didnelpsun}
\title{矩阵}
\date{}
\begin{document}
\maketitle
\pagestyle{empty}
\thispagestyle{empty}
\tableofcontents
\thispagestyle{empty}
\newpage
\pagestyle{plain}
\setcounter{page}{1}
\section{矩阵幂}

\subsection{对应成比例}

因为矩阵运算不满足交换率但是满足结合率,且一行矩阵乘一列矩阵的乘积为一个数,所以可以推出矩阵的幂的运算方法。

这个方法要求$r(A)=1$,即对应成比例。

令$A$为$n$阶方阵,将$A$拆为$A=(a_1,a_2,\cdots,a_n)^T(b_1,b_2,\cdots,b_n)=\alpha^T\beta$,所以$A^n=\alpha^T\beta\alpha^T\beta\cdots\alpha^T\beta$,利用结合率:$\alpha^T(\beta\alpha^T)(\beta\cdots\alpha^T)\beta$,中间一共$n-1$个$\beta\alpha^T$,$\beta\alpha^T$是一个数,即$A^n=(\beta\alpha^T)^{n-1}\alpha^T\beta=(\beta\alpha^T)^{n-1}A$。\medskip

\textbf{例题:}$A=\left(\begin{array}{ccc}
    1 & 2 & 3 \\
    -2 & -4 & -6 \\
    3 & 6 & 9
\end{array}\right)$,求$A^n$。\medskip

解:$A=(1,-2,3)^T(1,2,3)$,所以$A^n=((1,2,3)(1,-2,3)^T)^n(1,-2,3)^T(1,2,3)$

$=6^{n-1}A$。

若矩阵$A$的行与列都成比例,则$A^n=[tr(A)]^{n-1}A$,$[tr(A)]=\sum a_{ii}$,即矩阵迹为对角线元素值之和。

\subsection{试算归纳}

对$A$进行试算,如$A^2$,若$A^k$是一个数量阵,那么计算$A^n$就只用找规律就可以了。

\textbf{例题:}$A=\left(\begin{array}{cccc}
    1 & -1 & -1 & -1 \\
    -1 & 1 & -1 & -1 \\
    -1 & -1 & 1 & -1 \\
    -1 & -1 & -1 & 1 \\
\end{array}\right)$,求$A^n$($n\geqslant2$)。\medskip

解:通过计算得知$A^2=4E$,这是一个数量阵。\medskip

$\therefore A^n=\left\{\begin{array}{lcl}
    4^kE, & & n=2k \\
    4^kA, & & n=2k+1
\end{array}\right.$。

\subsection{行列结合}

将一个矩阵拆成$\alpha\beta^T$的形式,其中都是列向量,从而进行幂运算可以进行结合$\beta^T\alpha$为一个常数。

\textbf{例题:}设$\alpha=(1,3,-2)^T$,$\beta=(2,0,0)^T$,$A=\alpha\beta^T$,求$A^3$。

解:$\because\beta^T\alpha=[2,0,0][1,3,-2]^T=2$,$\therefore A^3=(\alpha\beta^T)(\alpha\beta^T)(\alpha\beta^T)=\alpha(\beta^T\alpha)$\\$(\beta^T\alpha)\beta^T=4\alpha\beta^T=4A$。

\subsection{拆分矩阵}

将$A^n$拆分为两个矩阵$A^n=(B+C)^n$,其中$BC$应该是可逆的,即$BC=CB$,所以一般有一个是$E$。\medskip

\textbf{例题:}$A=\left(\begin{array}{ccc}
    1 & 1 & 0 \\
    0 & 1 & 1 \\
    0 & 0 & 1
\end{array}\right)$,求$A^n$。\medskip

解:$A=E+B=\left(\begin{array}{ccc}
    1 & 0 & 0 \\
    0 & 1 & 0 \\
    0 & 0 & 1
\end{array}\right)+\left(\begin{array}{ccc}
    0 & 1 & 0 \\
    0 & 0 & 1 \\
    0 & 0 & 0
\end{array}\right)$。\medskip

$\therefore A^n=(E+B)^n=C_n^0E^n+C_n^1E^{n-1}B+C_n^2E^{n-2}B^2+\cdots$。

又$B^2=\left(\begin{array}{ccc}
    0 & 1 & 0 \\
    0 & 0 & 1 \\
    0 & 0 & 0
\end{array}\right)\left(\begin{array}{ccc}
    0 & 1 & 0 \\
    0 & 0 & 1 \\
    0 & 0 & 0
\end{array}\right)=\left(\begin{array}{ccc}
    0 & 0 & 1 \\
    0 & 0 & 0 \\
    0 & 0 & 0
\end{array}\right)$。

$B^3=B^2B=\left(\begin{array}{ccc}
    0 & 0 & 1 \\
    0 & 0 & 0 \\
    0 & 0 & 0
\end{array}\right)\left(\begin{array}{ccc}
    0 & 1 & 0 \\
    0 & 0 & 1 \\
    0 & 0 & 0
\end{array}\right)=\left(\begin{array}{ccc}
    0 & 0 & 0 \\
    0 & 0 & 0 \\
    0 & 0 & 0
\end{array}\right)=O$。

$\therefore B^4=B^5=\cdots=O$。

$\therefore A^n=(E+B)^n=C_n^0E^n+C_n^1E^{n-1}B+C_n^2E^{n-2}B^2$。\medskip

$=\left(\begin{array}{ccc}
    1 & 0 & 0 \\
    0 & 1 & 0 \\
    0 & 0 & 1
\end{array}\right)+n\left(\begin{array}{ccc}
    0 & 1 & 0 \\
    0 & 0 & 1 \\
    0 & 0 & 0
\end{array}\right)+\dfrac{n(n-1)}{2}\left(\begin{array}{ccc}
    0 & 0 & 1 \\
    0 & 0 & 0 \\
    0 & 0 & 0
\end{array}\right)$

\subsection{分块矩阵}

$\left[\begin{array}{cc}
    A & O \\
    O & B
\end{array}\right]^n=\left[\begin{array}{cc}
    A^n & O \\
    O & B^n
\end{array}\right]$。

\section{初等变换}

若$A$和$B$等价,求一个可逆矩阵$P$,使得$PA=B$。只用右乘$P=BA^{-1}$。

需要根据逻辑上的计算还原出左乘的初等矩阵。\medskip

\textbf{例题:}$A=\left(\begin{array}{ccc}
    1 & 0 & 1 \\
    -1 & -1 & 1 \\
    0 & 2 & -4
\end{array}\right)$,$B=\left(\begin{array}{ccc}
    1 & 0 & 1 \\
    0 & -1 & 2 \\
    0 & 0 & 0
\end{array}\right)$,当$A\sim B$时,求$P$使得$PA=B$。.

解:目标是将$A$变为$B$,所以第一步将第一列的第二行的-1变为0。即将第一行加到第二行。

左乘$E_{21}(1)A=\left(\begin{array}{ccc}
    1 & 0 & 0 \\
    1 & 1 & 0 \\
    0 & 0 & 1
\end{array}\right)\left(\begin{array}{ccc}
    1 & 0 & 1 \\
    -1 & -1 & 1 \\
    0 & 2 & -4
\end{array}\right)=\left(\begin{array}{ccc}
    1 & 0 & 1 \\
    0 & -1 & 2 \\
    0 & 2 & -4
\end{array}\right)=C$。\medskip

然后对第二列进行消,首先将第三行加上第二行的两倍。

$E_{32}(2)C=\left(\begin{array}{ccc}
    1 & 0 & 0 \\
    1 & 1 & 0 \\
    0 & 2 & 1
\end{array}\right)\left(\begin{array}{ccc}
    1 & 0 & 1 \\
    0 & -1 & 2 \\
    0 & 2 & -4
\end{array}\right)=\left(\begin{array}{ccc}
    1 & 0 & 1 \\
    1 & -1 & 2 \\
    0 & 0 & 0
\end{array}\right)=B$。\medskip

$\therefore E_{32}(2)E_{21}(1)A=B$。

$P=E_{32}(2)E_{21}(1)=\left(\begin{array}{ccc}
    1 & 0 & 0 \\
    0 & 1 & 0 \\
    0 & 2 & 1
\end{array}\right)\left(\begin{array}{ccc}
    1 & 0 & 0 \\
    1 & 1 & 0 \\
    0 & 0 & 1
\end{array}\right)=\left(\begin{array}{ccc}
    1 & 0 & 0 \\
    1 & 1 & 0 \\
    2 & 2 & 1
\end{array}\right)$。

\section{逆矩阵}

\subsection{定义法}

找出一个矩阵$B$,使得$AB=E$,则$A$可逆,$A^{-1}=B$。

\textbf{例题:}$A$,$B$均是$n$阶方阵,且$AB=A+B$,证明$A-E$可逆,并求$(A-E)^{-1}$。

解:要证明$A-E$,就要从$AB=A+B$中尽量凑出。

$AB=A+B$变为$AB-B=A$,从而提取$(A-E)B=A$,$(A-E)BA^{-1}=E$。

但是$A^{-1}$是未知的,所以$A-E$的逆矩阵不能用$BA^{-1}$来表示。

$AB-A=B$,所以提出$A(B-E)=B$,即$A(B-E)=B-E+E$,$(A-E)(B-E)=E$,所以$A-E$的逆矩阵就是$B-E$。

\subsection{分解乘积}

将$A$分解为若干个可逆矩阵的乘积。若$A=BC$,$B$,$C$可逆,则$A$可逆,且$A^{-1}=C^{-1}B^{-1}$。同理$(ABC)^{-1}=C^{-1}B^{-1}A^{-1}$。

\textbf{例题:}设$A$,$B$为同阶可逆方阵,且$A^{-1}+B^{-1}$可逆,求$(A+B)^{-1}$。

解:已知$A^{-1}+B^{-1}$可以用来表示其他式子,需要求$A+B$的逆,则需要将$A+B$转为其逆。

$\because A+B=A(E+A^{-1}B)=A(B^{-1}+A^{-1})B$。

$\therefore (A+B)^{-1}=B^{-1}(B^{-1}+A^{-1})^{-1}A^{-1}$。

\subsection{初等变换}

$\left[A\vdots E\right]\overset{r}{\sim}\left[E\vdots A^{-1}\right]$,$\left[\begin{array}{c}
    A \\
    E
\end{array}\right]\overset{c}{\sim}\left[\begin{array}{c}
    E \\
    A^{-1}
\end{array}\right]$。

\subsection{分块矩阵}

基于拉普拉斯展开式。

对于一些分块矩阵的逆,若$A$,$B$都可逆,则:$\left[\begin{array}{cc}
    A & O \\
    O & B
\end{array}\right]^{-1}=\left[\begin{array}{cc}
    A^{-1} & O \\
    O & B^{-1}
\end{array}\right]$,$\left[\begin{array}{cc}
    O & A \\
    B & O
\end{array}\right]^{-1}=\left[\begin{array}{cc}
    O & B^{-1} \\
    A^{-1} & O
\end{array}\right]$。\medskip

\textbf{例题:}已知$A=\left(\begin{array}{cc}
    B & O \\
    D & C
\end{array}\right)$,其中$B$为$r\times r$可逆矩阵,$C$为$s\times s$可逆矩阵,求$A^{-1}$。

解:$\because\vert A\vert=\left|\begin{array}{cc}
    B & O \\
    D & C
\end{array}\right|=\vert B\vert\vert C\vert\neq0$,所以$A$可逆,设$A^{-1}=\left(\begin{array}{cc}
    X & Y \\
    Z & W
\end{array}\right)$。

$AA^{-1}=\left(\begin{array}{cc}
    E_r & O \\
    O & E_s
\end{array}\right)=E_{r+s}$。即$\left(\begin{array}{cc}
    BX & BY \\
    DX+CZ & DY+CW
\end{array}\right)=E_{r+s}$。

$\therefore\left\{\begin{array}{l}
    BX=E \\
    BY=O \\
    DX+CZ=O \\
    DY+CW=E
\end{array}\right.$,$\left\{\begin{array}{ll}
    B^{-1}BX=B^{-1}, & X=B^{-1}\\
    B^{-1}BY=O, & Y=O \\
    CZ=-DX=-DB^{-1}, & Z=-C^{-1}DB^{-1} \\
    CW=E, & W=C^{-1}
\end{array}\right.$。

$\therefore A^{-1}=\left(\begin{array}{cc}
    B^{-1} & O \\
    -C^{-1}DB^{-1} & C^{-1}
\end{array}\right)$。\medskip

当分块矩阵为三角矩阵时,对角线为原方块矩阵的逆矩阵,非0的一角为原矩阵,再左乘同行的逆矩阵,右乘同列的逆矩阵。\medskip

$\therefore A=\left(\begin{array}{cc}
    B & D \\
    O & C
\end{array}\right)$,$A^{-1}=\left(\begin{array}{cc}
    B^{-1} & -B^{-1}DC^{-1} \\
    O & C^{-1}
\end{array}\right)$。\medskip

当分块矩阵为副对角矩阵时,对角线为对角方块矩阵的逆矩阵,非0的一角为原矩阵,再左乘同行的逆矩阵,右乘同列的逆矩阵。\medskip

$\therefore A=\left(\begin{array}{cc}
    O & B \\
    C & D
\end{array}\right)$,$A^{-1}=\left(\begin{array}{cc}
    -C^{-1}DB^{-1} & C^{-1} \\
    B^{-1} & O
\end{array}\right)$。\medskip

$\therefore A=\left(\begin{array}{cc}
    D & B \\
    C & O
\end{array}\right)$,$A^{-1}=\left(\begin{array}{cc}
    O & C^{-1} \\
    B^{-1} & -C^{-1}DB^{-1}
\end{array}\right)$。\medskip

$A=\left(\begin{array}{ccc}
    A_1 \\
     & \ddots \\
     & & A_n
\end{array}\right)$,$A^{-1}=\left(\begin{array}{ccc}
    A_1^{-1} \\
     & \ddots \\
     & & A_n^{-1}
\end{array}\right)$。\medskip

$A=\left(\begin{array}{ccc}
     & & A_1 \\
     & \ddots \\
    A_n 
\end{array}\right)$,$A^{-1}=\left(\begin{array}{ccc}
    & & A_n^{-1} \\
    & \begin{turn}{80}$\ddots$\end{turn} \\
   A_1^{-1}
\end{array}\right)$。

\textbf{例题:}已知矩阵$A$的伴随矩阵$A^*=\left[\begin{array}{cccc}
    4 & -2 & 0 & 0 \\
    -3 & 1 & 0 & 0 \\
    0 & 0 & -4 & 0 \\
    0 & 0 & 0 & -1
\end{array}\right]$,求$A$。

解:由于$A^{-1}=\dfrac{A^*}{\vert A\vert}$,所以$A=\vert A\vert(A^*)^{-1}$。已知$A^*$可知$(A^*)^{-1}$,所以重点就是求$\vert A\vert$。

又$\vert A^*\vert=\vert A\vert^{n-1}$,$\vert A^*\vert=-8$,$\vert A\vert=-2$。

所以根据分块矩阵的逆运算,可以得到$(A^*)^{-1}=\left[\begin{array}{cccc}
    -\dfrac{1}{2} & -1 & 0 & 0 \\
    -\dfrac{3}{2} & -2 & 0 & 0 \\
    0 & 0 & -\dfrac{1}{4} & 0 \\
    0 & 0 & 0 & -1
\end{array}\right]$。

所以$A=\left[\begin{array}{cccc}
    1 & 2 & 0 & 0 \\
    3 & 4 & 0 & 0 \\
    0 & 0 & \dfrac{1}{2} & 0 \\
    0 & 0 & 0 & 2
\end{array}\right]$。

\section{伴随矩阵}

伴随矩阵一般只会计算三阶以及以下。

伴随矩阵和逆矩阵往往共同参与运算,并有许多公式。

\begin{enumerate}
    \item $(A^*)^{-1}=\dfrac{A}{\vert A\vert}$。
    \item $\vert A^*\vert=\vert A\vert^{n-1}$。
    \item $(A^*)^*=\vert A\vert^{n-2}A$。
\end{enumerate}

证明关系式一,由于$A^*=\vert A\vert A^{-1}$,$(A^*)^{-1}=(\vert A\vert A^{-1})^{-1}=\dfrac{1}{\vert A\vert}(A^{-1})^{-1}=\dfrac{1}{\vert A\vert}A$。

证明关系式二,对$A^*=\vert A\vert A^{-1}$两边取行列式,得到$\vert A^*\vert=\vert\vert A\vert A^{-1}\vert=\vert A\vert^n\vert A^{-1}\vert=\dfrac{\vert A\vert^n}{\vert A\vert}=\vert A\vert^{n-1}=\dfrac{1}{\vert A^{-1}\vert^{n-1}}$。

证明关系式三,对$(A^*)^*=(A^*)^{-1}\vert A^*\vert=(\vert A\vert A^{-1})^{-1}\vert A\vert^{n-1}=\vert A\vert^{-1}A\vert A\vert^{n-1}=\vert A\vert^{n-2}A$。

\section{方阵行列式}

\subsection{两项积商}

\begin{enumerate}
    \item $\vert A^T\vert=\vert A\vert$。
    \item $\vert A^{-1}\vert=\dfrac{1}{\vert A\vert}$。
    \item $\vert\lambda A\vert=\lambda^n\vert A\vert$。
    \item $\vert AB\vert=\vert A\vert\cdot\vert B\vert=\vert BA\vert$。
    \item $\vert A^*\vert=\vert A\vert^{n-1}$。
\end{enumerate}

因为两项积商比较简单,所以基本上会变换$A$和$B$,让其变为转置或逆矩阵。

\subsection{两项和差}

两项和差需要将方阵拆分为向量组的形式,然后根据矩阵与行列式的运算法则进行运算。(注意其中的差别)

\textbf{例题:}设四阶方阵$A=[\alpha,\gamma_2,\gamma_3,\gamma_4]$,$B=[\beta,\gamma_2,\gamma_3,\gamma_4]$,其中$\alpha$、$\beta$、$\gamma_i$均为四维向量,且$\vert A\vert=5$,$\vert B\vert=-\dfrac{1}{2}$,求$\vert A+2B\vert$。

解:$=\vert[\alpha,\gamma_2,\gamma_3,\gamma_4]+2[\beta,\gamma_2,\gamma_3,\gamma_4]\vert=\vert[\alpha+2\beta,3\gamma_2,3\gamma_3,3\gamma_4]\vert=27\vert[\alpha+2\beta,\gamma_2,\gamma_3,\gamma_4]\vert=27\vert[\alpha,\gamma_2,\gamma_3,\gamma_4]\vert+54\vert[\beta,\gamma_2,\gamma_3,\gamma_4]\vert=27(\vert A\vert+2\vert B\vert)=108$。

\section{矩阵方程}

含有未知矩阵的方程就是矩阵方程,需要将方程进行恒等变形,化为$AX=B$、$XA=B$或$AXB=C$的形式。

若$A$、$B$可逆,且可以分别得到$X=A^{-1}B$,$X=BA^{-1}$,$X=A^{-1}CB^{-1}$。

\subsection{直接化简}

\textbf{例题:}设3阶方阵$A$,$B$满足$A^{-1}BA=6A+BA$,且$A=\left(\begin{array}{ccc}
    \dfrac{1}{3} & 0 & 0 \\
    0 & \dfrac{1}{4} & 0 \\
    0 & 0 & \dfrac{1}{5}
\end{array}\right)$,求$B$。

解:$A^{-1}BA=(6E+B)A$,$A^{-1}B=6E+B$,$A^{-1}B-B=6E$,$(A^{-1}-E)B=6E$。

$\therefore B=6(A^{-1}-E)^{-1}$。

\subsection{凑目标式}

有时候直接化简非常麻烦,因为所求的式子很复杂,甚至出现结果不能得到的情况。

\textbf{例题:}已知$AB=A+B$,其中$B=\left[\begin{array}{ccc}
    1 & 1 & 0 \\
    1 & 1 & 0 \\
    0 & 0 & 2
\end{array}\right]$,求$(A-E)^{-1}$。

解:已知$AB=A+B$,求$A-E$,则向目标计算。

$AB-B=A$,即$(A-E)B=A$,$(A-E)^{-1}=BA^{-1}$。因为$A$未知,所以要消去$A$。

根据$AB=A+B$,得到$AB-A=B$,即$A(B-E)=B$,$A^{-1}=(B-E)B^{-1}$。

$(A-E)^{-1}=BA^{-1}=B(B-E)B^{-1}$,然后就不知道接下来怎么办了。

我们很希望$BB^{-1}$在一起消掉,但是无论如何操作都无法完成。但是也可以通过此得到解题的启示,按$(A-E)(B-E)$去凑。

回到$(A-E)B=A$,去凑$B-E$,先尝试两边减去$E$,得到$(A-E)B-E=A-E$,正好左移右项$(A-E)(B-E)=E$,解得$(A-E)^{-1}=B-E$。

即$=\left[\begin{array}{ccc}
    0 & 1 & 0 \\
    1 & 0 & 0 \\
    0 & 0 & 1
\end{array}\right]$。

\section{矩阵秩}

\subsection{未知参数}

已知一个矩阵的秩,求其矩阵中的参数。需要将矩阵简化,使得最下面的一行除了参数没有别的非零常数。

\textbf{例题:}已知$A=\left[\begin{array}{cccc}
    1 & 1 & a & 4 \\
    1 & 0 & 2 & a \\
    -1 & a & 1 & 0
\end{array}\right]$,$r(A)=3$,求$A$。

解:首先对$A$化简:$A=\left[\begin{array}{cccc}
    1 & 1 & a & 4 \\
    0 & 1 & a-2 & 4-a \\
    0 & 0 & (a+1)(3-a) & a(a-3)
\end{array}\right]$,若$r(A)=3$,则$(a+1)(3-a)$与$a(a-3)$不全为0,所以$a\neq3$。

\subsection{矩阵运算}

给出几个矩阵,进行矩阵运算求出对应的秩。

$r(kA)=r(A)$。

$r(AB)\leqslant\min\{r(A),r(B)\}$。当且仅当$AB$满秩等号成立。

$r(A+B)\leqslant r(A|B)\leqslant r(A)+r(B)$。

$r(A^*)=\left\{\begin{array}{l}
    n, r(A)=n \\
    1, r(A)=n-1 \\
    0, r(A)<n-1
\end{array}\right.$。

\textbf{例题:}已知$A=\left[\begin{array}{cccc}
    1 & 2 & 3 & 4 \\
    2 & 3 & 4 & 5 \\
    3 & 4 & 5 & 6 \\
    4 & 5 & 6 & 7
\end{array}\right]$,$B=\left[\begin{array}{cccc}
    0 & 1 & -1 & 2 \\
    0 & -1 & 2 & 3 \\
    0 & 0 & 1 & 4 \\
    0 & 0 & 0 & 2
\end{array}\right]$,求$r(AB+2A)$。

解:$r(AB+2A)=r(A(B+2E))$。又$B+2E=\left[\begin{array}{cccc}
    2 & 1 & -1 & 2 \\
    0 & 1 & 2 & 3 \\
    0 & 0 & 3 & 4 \\
    0 & 0 & 0 & 4
\end{array}\right]$,$r(B+2E)=4$。

又$A=\left[\begin{array}{cccc}
    1 & 2 & 3 & 4 \\
    2 & 3 & 4 & 5 \\
    3 & 4 & 5 & 6 \\
    4 & 5 & 6 & 7
\end{array}\right]=\left[\begin{array}{cccc}
    1 & 2 & 3 & 4 \\
    1 & 1 & 1 & 1 \\
    1 & 1 & 1 & 1 \\
    1 & 1 & 1 & 1
\end{array}\right]=\left[\begin{array}{cccc}
    1 & 2 & 3 & 4 \\
    0 & -1 & -2 & -3 \\
    0 & 0 & 0 & 0 \\
    0 & 0 & 0 & 0
\end{array}\right]$。\medskip

所以$r(A)=2$,$r(AB+2A)=\min\{r(A),r(B+2E)\}=2$。

\section{矩阵等价}

其实求等价矩阵就是判定其秩是否相等。

\subsection{行列式变换}

注意如果携带参数,要保证乘除的参数式子不为0。

\subsection{行列式}

只有矩阵可逆,即满秩时行列式才不为0,否则为0。

\end{document}
