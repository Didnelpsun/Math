\documentclass[UTF8, 12pt]{ctexart}
% UTF8编码,ctexart现实中文
\usepackage{color}
% 使用颜色
\usepackage{geometry}
\setcounter{tocdepth}{4}
\setcounter{secnumdepth}{4}
% 设置四级目录与标题
\geometry{papersize={21cm,29.7cm}}
% 默认大小为A4
\geometry{left=3.18cm,right=3.18cm,top=2.54cm,bottom=2.54cm}
% 默认页边距为1英尺与1.25英尺
\usepackage{indentfirst}
\setlength{\parindent}{2.45em}
% 首行缩进2个中文字符
\usepackage{setspace}
\renewcommand{\baselinestretch}{1.5}
% 1.5倍行距
\usepackage{amssymb}
% 因为所以
\usepackage{amsmath}
% 数学公式
\usepackage[colorlinks,linkcolor=black,urlcolor=blue]{hyperref}
% 超链接
\author{Didnelpsun}
\title{线性方程组}
\date{}
\begin{document}
\maketitle
\pagestyle{empty}
\thispagestyle{empty}
\tableofcontents
\thispagestyle{empty}
\newpage
\pagestyle{plain}
\setcounter{page}{1}
\section{基础解系}

\subsection{方程求通解}

\subsection{通解求通解}

题目给出$\xi_i$是$Ax=0$的基础解系,然后判断这几个基础解系的变式是否还能称为基础解系,判断条件就是对这些基础解析进行初等运算(往往是加减),如果最后能凑成0则代表其线性相关,所以不能成为基础解系,否则可以。

如$\xi_1+\xi_2$、$\xi_2+x_3$、$\xi_3+\xi_1$可以成为,因为$(\xi_1+\xi_2)-(\xi_2+x_3)+(\xi_3+\xi_1)=2\xi_1\neq0$,$\xi_1-\xi_2$、$\xi_2-x_3$、$\xi_3-\xi_1$不能成为,因为$(\xi_1-\xi_2)+(\xi_2-x_3)+(\xi_3-\xi_1)=0$。

\subsection{特解求通解}

\subsection{通解判断特解}

已知特解为方程的一个解,知道通解,所以特解可以由通解线性表出,所以将通解和特解组成增广矩阵进行初等变换(如果是判断多个向量,则可以一起组成),通解矩阵的秩和增广矩阵的秩相同则代表可以线性表出,否则不能。

\subsection{线性表出}

\section{反求参数}

基本上都是给出方程组有无穷多解:

\begin{itemize}
    \item 齐次方程组:系数矩阵是降秩的;行列式值为0。
    \item 非齐次方程组:系数矩阵与增广矩阵秩相同且降秩。
\end{itemize}

\textbf{例题:}已知齐次线性方程组$\left\{\begin{array}{l}
    ax_1-3x_2+3x_3=0 \\
    x_1+(a+2)x_2+3x_3=0 \\
    2x_1+x_2-x_3=0
\end{array}\right.$有无穷多解,求参数$a$。

解:使用矩阵比较麻烦,三阶的系数矩阵可以使用行列式。

$\vert A\vert=\left\vert\begin{array}{ccc}
    a & -3 & 3 \\
    1 & a+2 & 3 \\
    2 & 1 & -1 \\
\end{array}\right\vert=\left\vert\begin{array}{ccc}
    a & 0 & 3 \\
    1 & a+5 & 3 \\
    2 & 0 & -1 \\
\end{array}\right\vert=(a+5)(a+6)=0$。

解得$a=-5$或$a=-6$。

\section{公共解}

\end{document}