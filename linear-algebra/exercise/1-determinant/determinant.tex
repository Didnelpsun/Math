\documentclass[UTF8, 12pt]{ctexart}
% UTF8编码,ctexart现实中文
\usepackage{color}
% 使用颜色
\usepackage{geometry}
\setcounter{tocdepth}{4}
\setcounter{secnumdepth}{4}
% 设置四级目录与标题
\geometry{papersize={21cm,29.7cm}}
% 默认大小为A4
\geometry{left=3.18cm,right=3.18cm,top=2.54cm,bottom=2.54cm}
% 默认页边距为1英尺与1.25英尺
\usepackage{indentfirst}
\setlength{\parindent}{2.45em}
% 首行缩进2个中文字符
\usepackage{setspace}
\renewcommand{\baselinestretch}{1.5}
% 1.5倍行距
\usepackage{amssymb}
% 因为所以
\usepackage{amsmath}
% 数学公式
\usepackage[colorlinks,linkcolor=black,urlcolor=blue]{hyperref}
% 超链接
\author{Didnelpsun}
\title{行列式}
\date{}
\begin{document}
\maketitle
\pagestyle{empty}
\thispagestyle{empty}
\tableofcontents
\thispagestyle{empty}
\newpage
\pagestyle{plain}
\setcounter{page}{1}
\section{逆序}

\section{因式项}

\section{证明行列式值}

与计算行列式值的体型不同的是,其行列式的值是固定给出的,一方面虽然约束了解题思路,一方面也给出了解题的方向,需要结果与给定值“靠近”。

\section{计算行列式值}

包含直接计算行列式的值和已知行列式值计算参数值两种体型,基本上求解方式一致。

\section{代数余子式}

\end{document}
