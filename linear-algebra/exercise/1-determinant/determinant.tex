\documentclass[UTF8, 12pt]{ctexart}
% UTF8编码,ctexart现实中文
\usepackage{color}
% 使用颜色
\usepackage{geometry}
\setcounter{tocdepth}{4}
\setcounter{secnumdepth}{4}
% 设置四级目录与标题
\geometry{papersize={21cm,29.7cm}}
% 默认大小为A4
\geometry{left=3.18cm,right=3.18cm,top=2.54cm,bottom=2.54cm}
% 默认页边距为1英尺与1.25英尺
\usepackage{indentfirst}
\setlength{\parindent}{2.45em}
% 首行缩进2个中文字符
\usepackage{setspace}
\renewcommand{\baselinestretch}{1.5}
% 1.5倍行距
\usepackage{amssymb}
% 因为所以
\usepackage{amsmath}
% 数学公式
\usepackage[colorlinks,linkcolor=black,urlcolor=blue]{hyperref}
% 超链接
\author{Didnelpsun}
\title{行列式}
\date{}
\begin{document}
\maketitle
\pagestyle{empty}
\thispagestyle{empty}
\tableofcontents
\thispagestyle{empty}
\newpage
\pagestyle{plain}
\setcounter{page}{1}
\section{逆序}

逆序一般只会考一个数列的逆序数,一般以自然数从小到大为标准次序。

对于逆序数的计算一般是数,假设一共有$n$项,则需要依次从$i$向后判断各项与当前项的大小,最后相加。

\subsection{有穷排列}

对于给出几个数字的有限排列,只需要直接计算即可。

\textbf{例题:}求2413的逆序数。

2的逆序有21一个。4的逆序与41、43两个。1无逆序数,所以一共逆序数为3。

\subsection{无穷排列}

\textbf{例题:}求$13\cdots(2n-1)(2n)(2n-2)\cdots2$的逆序数。

这个序列分为两个部分,第一个是前面的$13\cdots(2n-1)$部分,这个部分无逆序。

第二个部分是后面的$(2n)(2n-2)\cdots2$,这个序列是全部逆序的,所以考虑其第二个内部一共有$n$个数,从前往后依次有$n,(n-1),\cdots,1$个逆序,所以逆序数为$\dfrac{n(n-1)}{2}$。

然后是考虑第二个部分对于第一个部分的逆序。$2n-2$对$2n-1$产生一个逆序,到最后的2对前面的$3\cdots(2n-1)$都产生了逆序一共$n-1$个,所以一共$\dfrac{n(n-1)}{2}$个逆序。

所以最后一共加起来与$n(n-1)$个逆序。

\section{因式项}

需要求出带有某些因子的因式项,其实就是对顺序的排列组合,若已经给出某些因式,则因式项的其他因子就必须是其他数值。

且还要考虑因式项的正负号,即选择的值序列的逆序数。

\textbf{例题:}写出四阶行列式中含有$a_{11}a_{23}$的因式项。

因为是四阶行列式,且含有$a_{11}a_{23}$,所以余下来的$a_{3?}$和$a_{4?}$中的$?$只有2和4可选。

若是$a_{11}a_{23}a_{32}a_{44}$,则列坐标序列为$1324$,从而逆序数为1,所以该项为$-a_{11}a_{23}a_{32}a_{44}$。

若是$a_{11}a_{23}a_{34}a_{42}$,则列坐标序列为$1342$,从而逆序数为2,所以该项为$a_{11}a_{23}a_{34}a_{42}$。

$\therefore\,-a_{11}a_{23}a_{32}a_{44}+a_{11}a_{23}a_{34}a_{42}$。

\section{证明行列式值}

与计算行列式值的题型不同的是,其行列式的值是固定给出的,一方面虽然约束了解题思路,一方面也给出了解题的方向,需要结果与给定值“靠近”。

\textbf{例题:}

\section{计算行列式值}

包含直接计算行列式的值和已知行列式值计算参数值两种体型,基本上求解方式一致。

\section{代数余子式}

\end{document}
