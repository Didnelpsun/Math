\documentclass[UTF8, 12pt]{ctexart}
% UTF8编码,ctexart现实中文
\usepackage{color}
% 使用颜色
\usepackage{geometry}
\setcounter{tocdepth}{4}
\setcounter{secnumdepth}{4}
% 设置四级目录与标题
\geometry{papersize={21cm,29.7cm}}
% 默认大小为A4
\geometry{left=3.18cm,right=3.18cm,top=2.54cm,bottom=2.54cm}
% 默认页边距为1英尺与1.25英尺
\usepackage{indentfirst}
\setlength{\parindent}{2.45em}
% 首行缩进2个中文字符
\usepackage{setspace}
\renewcommand{\baselinestretch}{1.5}
% 1.5倍行距
\usepackage{amssymb}
% 因为所以
\usepackage{amsmath}
% 数学公式
\usepackage[colorlinks,linkcolor=black,urlcolor=blue]{hyperref}
% 超链接
\author{Didnelpsun}
\title{相似}
\date{}
\begin{document}
\maketitle
\pagestyle{empty}
\thispagestyle{empty}
\tableofcontents
\thispagestyle{empty}
\newpage
\pagestyle{plain}
\setcounter{page}{1}

特征值往往与前面的内容进行混合考察。

\section{特征值与特征向量}

首先根据$\vert\lambda E-A\vert=0$求出$\lambda$,然后把$\lambda$逐个带入$(\lambda E-A)x=0$,根据齐次方程求解方法进行初等变换求出基础解系。这个基础解系就是当前特征值的特征向量。

\subsection{迹}

\textbf{例题:}已知$A$是3阶方阵,特征值为1,2,3,求$\vert A\vert$的元素$a_{11},a_{22},a_{33}$的代数余子式$A_{11},A_{22},A_{33}$的和$\sum\limits_{i=1}^3A_{ii}$。

解:首先代数余子式的和$A_{11},A_{22},A_{33}$一般在行列式展开定理中使用,但是这里给出的不是一行或一列的代数余子式,而是主对角线上的代数余子式,这就无法使用代数余子式来表达行列式的值了。

而另一个提到代数余子式的地方就是伴随矩阵$A^*$,所求的正好是伴随矩阵的迹$tr(A^*)=A_{11}+A_{22}+A_{33}$。

又根据特征值性质,特征值的和为矩阵的迹,特征值的积为矩阵行列式的值,所以$tr(A^*)=A_{11}+A_{22}+A_{33}=\lambda_1^*+\lambda_2^*+\lambda_3^*$

$=\sum\limits_{i=1}^3\dfrac{\vert A\vert}{\lambda_i}=\sum\limits_{i=1}^3\dfrac{\lambda_1\lambda_2\lambda_3}{\lambda_i}=\lambda_2\lambda_3+\lambda_1\lambda_3+\lambda_1\lambda_2=2+3+6=11$。

\subsection{逆矩阵}

通过相关式子将逆矩阵转换为原矩阵。同一个向量的逆矩阵的特征值是原矩阵的特征值的倒数。

\textbf{例题:}已知$\overrightarrow{\alpha}=(a,1,1)^T$是矩阵$A=\left[\begin{array}{ccc}
    -1 & 2 & 2 \\
    2 & a & -2 \\
    2 & -2 & -1
\end{array}\right]$的逆矩阵的特征向量,则求$\overrightarrow{\alpha}$在矩阵$A$中对应的特征值。

解:由于$\overrightarrow{\alpha}$是$A^{-1}$的特征向量,所以令此时的特征值为$\lambda_0$,则定义$\lambda_0\overrightarrow{\alpha}=A^{-1}\overrightarrow{\alpha}$,$\lambda_0A\overrightarrow{\alpha}=\overrightarrow{\alpha}$。

即$\lambda_0\left[\begin{array}{ccc}
    -1 & 2 & 2 \\
    2 & a & -2 \\
    2 & -2 & -1
\end{array}\right]\left[\begin{array}{c}
    a \\
    1 \\
    1 \\
\end{array}\right]=\left[\begin{array}{c}
    a \\
    1 \\
    1 \\
\end{array}\right]$,即$\lambda_0\left[\begin{array}{ccc}
    -a & 2 & 2 \\
    2a & a & -2 \\
    2a & -2 & -1
\end{array}\right]=\left[\begin{array}{c}
    a \\
    1 \\
    1 \\
\end{array}\right]$。\medskip

即根据矩阵代表的是方程组,得到$\lambda_0(4-a)=a$,$\lambda_0(3a-2)=1$,$\lambda_0(2a-3)=1$。

又$\lambda_0\neq0$,$3a-2=2a-3$,$a=-1$,则$\lambda_0=-\dfrac{1}{5}$。

所以矩阵$A$对应的特征值为$-5$。

\subsection{抽象型}

题目只会给对应的式子,来求对应的特征向量或特征值。需要记住特征值的关系式然后与给出的式子上靠拢,不会很复杂。

\textbf{例题:}已知$A$为三阶矩阵,且矩阵$A$各行元素之和均为5,则求$A$必然存在的特征向量。

解:由于是抽象型,所以没有实际的数据,就不能求出固定的特征值,$\lambda\xi=A\xi$。

又矩阵$A$各行元素之和均为5,所以转换为方程组:\medskip

$\left\{\begin{array}{l}
    A_{11}+A_{12}+A_{13}=5 \\
    A_{21}+A_{22}+A_{23}=5 \\
    A_{31}+A_{32}+A_{33}=5
\end{array}\right.$,转为矩阵:$\left[\begin{array}{ccc}
    A_{11} & A_{12} & A_{13} \\
    A_{21} & A_{22} & A_{23} \\
    A_{31} & A_{32} & A_{33}
\end{array}\right]\left[\begin{array}{c}
    1 \\
    1 \\
    1 \\
\end{array}\right]=5\left[\begin{array}{c}
    1 \\
    1 \\
    1 \\
\end{array}\right]$。\medskip

即$\xi=(1,1,1)^T$。

\subsection{可逆矩阵}

使用可逆矩阵相似对角化的性质。若$A\sim B$,则$P^{-1}AP=B$。$B$为纯量阵。且$B$的迹为$A$的特征值。$P$为特征向量。\medskip

\textbf{例题:}已知$P^{-1}AP=\left[\begin{array}{ccc}
    1 \\
     & 1 \\
     & & -1
\end{array}\right]$,$P=(\alpha_1,\alpha_2,\alpha_3)$可逆,求$A$关于特征值$\lambda=1$的特征向量。

解:根据$P^{-1}AP=\Lambda$,所以$P$为特征向量,$1,1,-1$为特征值。

所以$A$关于$\lambda=1$的特征向量为$\alpha_1$或$\alpha_2$。而某一特征值的全部特征向量构成特征向量子空间,所以$\lambda=1$的特征向量为$k_1\alpha_1+k_2\alpha_2$。

\subsection{实对称矩阵}

实对称矩阵的不同特征值的特征向量相互正交($B^TA=0$)。

\textbf{例题:}已知$A$为三阶实对称矩阵,特征值为$1,3,-2$,其中$\alpha_1=(1,2,-2)^T$,$\alpha_2=(4,-1,a)^T$分别属于特征值$\lambda=1$,$\lambda=3$的特征向量。求$A$属于特征值$\lambda=-2$的特征向量。

解:令$A$属于特征值$\lambda=-2$的特征向量为$(x_1,x_2,x_3)^T$。

根据实对称矩阵的正交性质。

$\alpha_1^T\alpha_2=4-2-2a=0$,$\alpha_2^T\alpha_3=4x_1-x_2+ax_3=0$,$\alpha_3^T\alpha_1=x_1+2x_2-2x_3=0$。

$a=1$,$4x_1-x_2+x_3=0$,$x_1+2x_2-2x_3=0$,解得基础解系$(0,1,1)^T$,$\alpha_3=(0,k,k)^T$($k\neq0$)。

\section{相似理论}

\subsection{判断相似对角化}

可以使用相似对角化的四个条件,但是最基本的使用还是$A$有$n$个无关的特征向量$\xi$。

\begin{enumerate}
    \item 判断是否为实对称矩阵,实对称矩阵必然相似于对角矩阵。
    \item 特征值是否都是实单根,相似于对角矩阵。
    \item 特征值为$n$重根,对应$n$个线性无关的特征向量,则相似于对角矩阵。如果小于则不相似。
\end{enumerate}

\subsection{反求参数}

常用方法:

\begin{itemize}
    \item 若$A\sim B$,则$\vert A\vert=\vert B\vert$,$r(A)=r(B)$,$tr(A)=tr(B)$,$\lambda_A=\lambda_B$,通过等式计算参数。
    \item 若$\xi$是$A$属于特征值$\lambda$的特征向量,则有$A\xi=\lambda\xi$,建立若干等式或方程组来计算参数。
    \item 若$\lambda$是$A$的特征值,则与$\vert\lambda E-A\vert=0$,通过该等式计算参数。
\end{itemize}

\subsubsection{具体矩阵}

\textbf{例题:}已知$A=\left(\begin{array}{ccc}
    2 & 0 & 0 \\
    0 & 0 & 1 \\
    0 & 1 & x
\end{array}\right)$,$B=\left(\begin{array}{ccc}
    2 \\
     & y \\
     & & -1
\end{array}\right)$,且$A\sim B$,求参数。\medskip

首先可以利用迹相等,则$2+0+x=2+y-1$,行列式值相等,则$-2=-2y$,解得$x=0$,$y=1$。

\subsubsection{对角矩阵}

首先要计算其特征值,再根据特征值反代特征方程,根据向量的构成判定秩的大小。

\textbf{例题:}已知$A=\left(\begin{array}{ccc}
    0 & 0 & 1 \\
    x & 1 & y \\
    1 & 0 & 0
\end{array}\right)$相似于对角矩阵,求$xy$关系式。

解:已知相似,即$P^{-1}AP=\Lambda$,则需要求$A$的特征值和特征向量。

根据特征关系式$\vert E\lambda-A\vert=0$,即$\left\vert\begin{array}{ccc}
    \lambda & 0 & -1 \\
    -x & \lambda-1 & -y \\
    -1 & 0 & \lambda
\end{array}\right\vert=(\lambda-1)(\lambda^2-1)=(\lambda-1)^2(\lambda+1)=0$,即有特征值$\lambda_1=\lambda_2=1$,$\lambda_3=1$。

此时有二重特征值,所以应该有两个线性无关的特征向量,即对于$(E-A)x=0$有两个线性无关的解向量,所以该矩阵的秩为$3-2=1$。

$E-A=\left(\begin{array}{ccc}
    1 & 0 & -1 \\
    -x & 1 & -y \\
    -1 & 0 & 1
\end{array}\right)=\left(\begin{array}{ccc}
    1 & 0 & -1 \\
    0 & 0 & -x-y \\
    0 & 0 & 0
\end{array}\right)$。

所以当$r(E-A)=1$时$x+y=0$。

\subsection{反求矩阵}

若有可逆矩阵$P$,使得$P^{-1}AP=\Lambda$,则:

$P$即是$A$特征向量的拼合。

\begin{itemize}
    \item $A=P\Lambda P^{-1}$。
    \item $A^k=P\Lambda^kP^{-1}$。
    \item $f(A)=Pf(\Lambda)P^{-1}$。
\end{itemize}

\textbf{例题:}已知$A=\left(\begin{array}{ccc}
    2 & x & 1 \\
    0 & 3 & 0 \\
    3 & -6 & 0
\end{array}\right)$相似于对角矩阵,求$A^{100}$。\medskip

解:首先$A\sim\Lambda$,所以$A$能相似对角化。

$\vert\lambda E-A\lambda=\left|\begin{array}{ccc}
    \lambda-2 & -x & -1 \\
    0 & \lambda-3 & 0 \\
    -3 & 6 & \lambda
\end{array}\right|=(\lambda-3)^2(\lambda+1)=0$。$\lambda_1=\lambda_2=3$,$\lambda_3=-1$。

所以对于$\lambda_1=\lambda_2=3$时,需要$s=2$,从而$r(A)=1$,对应成比例。

代入3:$(3E-A)x=0$,$\left(\begin{array}{ccc}
    1 & -x & -1 \\
    0 & 0 & 0 \\
    -3 & 6 & 3
\end{array}\right)=0$,所以$\dfrac{-1}{3}=\dfrac{-x}{6}$,$x=2$。

解得$\xi_1=(1,0,1)^T$,$\xi_2=(2,1,0)^T$,$\xi_3=(1,0,-3)^T$。

令$P=(\xi_1,\xi_2,\xi_3)$,所以$A=P\Lambda P^{-1}$,$A^{100}=P\Lambda^{100}P^{-1}$。

\subsection{相似性}

% \begin{itemize}
%     \item 定义法:若存在可逆矩阵$P$,使得$P^{-1}AP=B$,则$A\sim B$。
%     \item 性质:若$A\sim B$,则$r(A)=r(B)$,$\vert A\vert=\vert B\vert$,$tr(A)=tr(B)$,$\lambda_A=\lambda_B$。
%     \item 传递性:$A\sim\Lambda$,$\Lambda\sim B$,则$A\sim B$。
% \end{itemize}

\begin{enumerate}
    \item 首先要判断特征值是否相等(利用$\vert\lambda E-A\vert$)。
    \item 判断矩阵是否可以相似对角化(先根据特征值拼出$\Lambda$,然后求矩阵的秩是否也有同样数量的特征向量)。
\end{enumerate}

\textbf{例题:}已知矩阵$A=\left[\begin{array}{ccc}
    3 & 1 & 2 \\
    0 & 2 & a \\
    0 & 0 & 3
\end{array}\right]$和对角矩阵相似,求$a$。\medskip

解:由于$A$是对角矩阵,所以特征值为其迹$\lambda=(3,2,3)$。特征值有二重根。

已知$A\sim\Lambda$,$\lambda=3$有两个线性无关的特征向量。即$(3E-A)x=0$有两个线性无关的解。即$r(3E-A)=1$。

$3E-A=\left[\begin{array}{ccc}
    0 & -1 & -2 \\
    0 & 1 & -a \\
    0 & 0 & 0
\end{array}\right]=\left[\begin{array}{ccc}
    0 & -1 & -2 \\
    0 & 0 & -a-2 \\
    0 & 0 & 0
\end{array}\right]$,$\therefore a=-2$。

\subsection{抽象型}

\textbf{例题:}设$A$是三阶矩阵,$\alpha_1,\alpha_2,\alpha_3$是三维线性无关的列向量,且$A\alpha_1=\alpha_2+\alpha_3$,$A\alpha_2=\alpha_1+\alpha_3$,$A\alpha_3=\alpha_1+\alpha_2$,求$A$相似的矩阵。

解:$A\sim\Lambda$,则$P^{-1}AP=\Lambda$。

$A(\alpha_1,\alpha_2,\alpha_3)=(\alpha_2+\alpha_3,\alpha_1+\alpha_3,\alpha_1+\alpha_2)=(\alpha_1,\alpha_2,\alpha_3)\left[\begin{array}{ccc}
    0 & 1 & 1 \\
    1 & 0 & 1 \\
    1 & 1 & 0
\end{array}\right]$。

记$P=(\alpha_1,\alpha_2,\alpha_3)$,$B=\left[\begin{array}{ccc}
    0 & 1 & 1 \\
    1 & 0 & 1 \\
    1 & 1 & 0
\end{array}\right]$。\medskip

又$\alpha_1,\alpha_2,\alpha_3$是三维线性无关的列向量,$\therefore\vert\alpha_1,\alpha_2,\alpha_3\vert\neq0$,所以$P$可逆。

$\therefore AP=PB$,$P^{-1}AP=B$,$A\sim B$。

\subsection{正交相似}

\textbf{例题:}已知$A$是三阶实对称矩阵,若正交矩阵$Q$使得$Q^{-1}AQ=\left[\begin{array}{ccc}
    3 & 0 & 0 \\
    0 & 3 & 0 \\
    0 & 0 & 6
\end{array}\right]$,如果$\alpha_1=(1,0,-1)^T$和$\alpha_2=(0,1,1)^T$是矩阵$A$属于特征值$\lambda=3$的特征向量,求$Q$。

解:首先由正交矩阵就可以知道各特征值正交。令$\alpha_3=(x_1,x_2,x_3)^T$。对应的$\lambda_3=6$。

$\alpha_3^T\alpha_1=x_1-x_3=0$,$\alpha_3^T\alpha_2=x_2+x_3=0$,求$\lambda_3$的特征值,则不如令$x_3=1$,则解得$\alpha_3=(1,-1,1)^T$。

这样$Q=\left[\begin{array}{ccc}
    1 & 0 & 1 \\
    0 & 1 & -1 \\
    -1 & 1 & 1
\end{array}\right]$,还需要将$Q$正交单位化。可知$\alpha_3$根据正交规律求出来,一定是正交的,而$\alpha_1^T\alpha_2=-1\neq0$所以需要正交。

令$\beta_1=\alpha_1=(1,0,-1)^T$,$\beta_2=\alpha_2-\dfrac{<\alpha_2,\beta_1>}{<\beta_1,\beta_1>}\beta_1=(0,1,1)^T+\dfrac{1}{2}(1,0,-1)^T=(\dfrac{1}{2},1,\dfrac{1}{2})^T$。

最后对整个$Q$进行单位化:$\gamma_1=\dfrac{1}{\sqrt{2}}(1,0,-1)^T$,$\gamma_2=\dfrac{1}{\sqrt{6}}(1,2,1)^T$,$\gamma_3=\dfrac{1}{\sqrt{3}}(1,-1,1)^T$。

\end{document}
