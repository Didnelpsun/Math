\documentclass[UTF8, 12pt]{ctexart}
% UTF8编码,ctexart现实中文
\usepackage{color}
% 使用颜色
\usepackage{geometry}
\setcounter{tocdepth}{4}
\setcounter{secnumdepth}{4}
% 设置四级目录与标题
\geometry{papersize={21cm,29.7cm}}
% 默认大小为A4
\geometry{left=3.18cm,right=3.18cm,top=2.54cm,bottom=2.54cm}
% 默认页边距为1英尺与1.25英尺
\usepackage{indentfirst}
\setlength{\parindent}{2.45em}
% 首行缩进2个中文字符
\usepackage{setspace}
\renewcommand{\baselinestretch}{1.5}
% 1.5倍行距
\usepackage{amssymb}
% 因为所以
\usepackage{amsmath}
% 数学公式
\usepackage[colorlinks,linkcolor=black,urlcolor=blue]{hyperref}
% 超链接
\author{Didnelpsun}
\title{相似}
\date{}
\begin{document}
\maketitle
\pagestyle{empty}
\thispagestyle{empty}
\tableofcontents
\thispagestyle{empty}
\newpage
\pagestyle{plain}
\setcounter{page}{1}

特征值往往与前面的内容进行混合考察。

\section{特征值与迹}

\textbf{例题:}已知$A$是3阶方阵,特征值为1,2,3,求$\vert A\vert$的元素$a_{11},a_{22},a_{33}$的代数余子式$A_{11},A_{22},A_{33}$的和$\sum\limits_{i=1}^3A_{ii}$。

解:首先代数余子式的和$A_{11},A_{22},A_{33}$一般在行列式展开定理中使用,但是这里给出的不是一行或一列的代数余子式,而是主对角线上的代数余子式,这就无法使用代数余子式来表达行列式的值了。

而另一个提到代数余子式的地方就是伴随矩阵$A^*$,所求的正好是伴随矩阵的迹$tr(A^*)=A_{11}+A_{22}+A_{33}$。

又根据特征值性质,特征值的和为矩阵的迹,特征值的积为矩阵行列式的值,所以$tr(A^*)=A_{11}+A_{22}+A_{33}=\lambda_1^*+\lambda_2^*+\lambda_3^*$

$=\sum\limits_{i=1}^3\dfrac{\vert A\vert}{\lambda_i}=\sum\limits_{i=1}^3\dfrac{\lambda_1\lambda_2\lambda_3}{\lambda_i}=\lambda_2\lambda_3+\lambda_1\lambda_3+\lambda_1\lambda_2=2+3+6=11$。

\section{相似对角化}

\section{判断相似对角化}

可以使用相似对角化的四个条件,但是最基本的使用还是$A$有$n$个无关的特征向量$\xi$。

\end{document}
