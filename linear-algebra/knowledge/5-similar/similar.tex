\documentclass[UTF8, 12pt]{ctexart}
% UTF8编码,ctexart现实中文
\usepackage{color}
% 使用颜色
\definecolor{orange}{RGB}{255,127,0} 
\usepackage{geometry}
\setcounter{tocdepth}{4}
\setcounter{secnumdepth}{4}
% 设置四级目录与标题
\geometry{papersize={21cm,29.7cm}}
% 默认大小为A4
\geometry{left=3.18cm,right=3.18cm,top=2.54cm,bottom=2.54cm}
% 默认页边距为1英尺与1.25英尺
\usepackage{indentfirst}
\setlength{\parindent}{2.45em}
% 首行缩进2个中文字符
\usepackage{setspace}
\renewcommand{\baselinestretch}{1.5}
% 1.5倍行距
\usepackage{amssymb}
% 因为所以
\usepackage{amsmath}
% 数学公式
\usepackage[colorlinks,linkcolor=black,urlcolor=blue]{hyperref}
% 超链接
\author{Didnelpsun}
\title{相似}
\date{}
\begin{document}
\maketitle
\pagestyle{empty}
\thispagestyle{empty}
\tableofcontents
\thispagestyle{empty}
\newpage
\pagestyle{plain}
\setcounter{page}{1}

主要包括特征值与特征向量,相似矩阵,对角矩阵。

这里的矩阵都是指方阵。

\section{特征值与特征向量}

\subsection{定义}

设$A$是$n$阶矩阵,$\lambda$是一个数,若存在$n$维非零列向量$\xi\neq0$,使得$A\xi=\lambda\xi$,则$\lambda$是$A$的特征值,$\xi$是$A$的对应于特征值$\lambda$的特征向量。

\subsection{性质}

\subsubsection{特征值性质}

设$A=(a_{ij})_{n\times n}$,$\lambda_i$($i=1,2,\cdots,n$)是$A$的特征值,则:

\begin{enumerate}
    \item $\sum\limits_{i=1}^n\lambda_i=\sum\limits_{i=1}^n=tr(A)$。主对角线元素和即矩阵的迹。
    \item $\prod\limits_{i=1}^n\lambda_i=\vert A\vert$。
\end{enumerate}

\subsubsection{特征向量性质}

\begin{enumerate}
    \item $k$重特征值$\lambda$至多只有$k$个线性无关的特征向量。
    \item 若$\xi_1$和$\xi_2$是$A$的属于不同特征值$\lambda_1$和$\lambda_2$的特征向量,则$\xi_1$和$\xi_2$线性无关。
    \item 若$\xi_1$和$\xi_2$是$A$的属于同特征值$\lambda$的特征向量,则$k_1\xi_1+k_2\xi_2$($k_1k_2$不同时为0)仍是$A$的属于特征值$\lambda$的特征向量。
\end{enumerate}

证明性质二:利用定义法,首先$A\xi_1=\lambda_1\xi_1$,$A\xi_2=\lambda_2\xi_2$。

要证明两个特征向量线性无关,则证明$k_1\xi_1+k_2\xi_2=0$时$k_1=k_2=0$。

$Ak_1\xi_1+Ak_2\xi_2=k_1\lambda_1\xi_1+k_2\lambda_1\xi_2=0$。又$k_1\xi_1+k_2\xi_2=\lambda_1k_1\xi_1+\lambda_1k_2\xi_2=0$,

两式相减:$k_2(\lambda_2-\lambda_1)\xi_2=0$,且$\lambda_1\neq\Lambda_2$,$\xi_2\neq0$,$\therefore k_2=0$。

代入$k_1\xi_1+k_2\xi_2=0$,即$k_1\xi_1=0$,又$\xi_1\neq0$,$\therefore k_1=0$。

\subsection{运算}

$\because\lambda\xi-A\xi=0$,$\therefore(\lambda E-A)\xi=0$,又$\xi\neq0$,$\therefore(\lambda E-A)x=0$有非零解。

从而$\lambda E-A$所表示的方阵线性相关,为降秩,从而$\vert\lambda E-A\vert=0$。

$\vert\lambda E-A\vert=0$也称为特征方程或是特征多项式,解出的$\lambda_i$就是特征值。

将$\lambda_i$代回原方程,所有非零的解就是$\xi$。

\subsubsection{具体型}

若矩阵$A$为对角线矩阵,则特征值为对角线上元素。

\textcolor{orange}{注意:}特征向量因为要求不为0,所以需要$k\neq0$。

\textcolor{orange}{注意:}得到多重特征值时要全部写出,$\lambda_1=\lambda_2=\cdots=\lambda_n=\lambda$。

\textbf{例题:}求$A=\left(\begin{array}{ccc}
    2 & -2 & 0 \\
    -2 & 1 & -2 \\
    0 & -2 & 0
\end{array}\right)$的特征值与特征向量。

$\vert\lambda E-A\vert=\left|\begin{array}{ccc}
    \lambda_2 & 2 & 0 \\
    2 & \lambda-1 & 2 \\
    0 & 2 & \Lambda
\end{array}\right|=(\lambda-2)(\lambda-1)\lambda-4\lambda-4(\lambda-2)=\lambda^3-3\lambda^2-6\lambda+8=(\lambda+2)(\lambda-1)(\lambda-4)=0$。

$\therefore\lambda_1=-2$,$\lambda_2=1$,$\lambda=4$。

当计算$\vert\lambda E-A\vert$时往往难点就是从多项式中解出$\lambda$,对于$f(\lambda)=a_k\lambda^k+\cdots+a_1\lambda+a_0=0$,可以使用试根法:

\begin{enumerate}
    \item 若$a_0=0$,$\lambda=0$就是其根。
    \item 若$a_k+\cdots+a_1+a_0=0$,$\lambda=1$就是其根。
    \item 若$a_0+a_2+\cdots+a_{2k}=a_1+a_3\cdots+a_{2k-1}$,$\lambda=-1$就是其根。
    \item 若$a_k=1$,且系数都是整数,则有理根是整数,且均为$a_0$的因子。
\end{enumerate}

对于第四个,如$\lambda^3-4\lambda^2+3\lambda+2=0$,2的因子为$\pm1$和$\pm2$,分别代入得到一根为2。

\subsubsection{抽象型}

\begin{enumerate}
    \item 利用定义,寻找$A\xi=\lambda\xi$,$\xi\neq0$,$\lambda$是$A$的特征值,$\xi$是$A$属于$\lambda$的特征向量。
    \item 根据$\vert\lambda E-A\vert=0$计算出对应的$\lambda$值,再计算$\xi$的值。
\end{enumerate}

\begin{tabular}{|c|c|c|c|c|c|c|c|c|}
    \hline
    矩阵 & $A$ & $kA$ & $A^k$ & $f(A)$ & $A^{-1}$ & $A^*$ & $P^{-1}AP$ & $A^T$ \\ \hline
    特征值 & $\lambda$ & $k\lambda$ & $\lambda^k$ & $f(\lambda)$ & $\lambda^{-1}$ & $\vert A\vert/\lambda$ & $\lambda$ & $\lambda$ \\ \hline
    特征向量 & $\xi$ & $\xi$ & $\xi$ & $\xi$ & $\xi$ & $\xi$ & $P^{-1}\xi$ & 无关 \\
    \hline
\end{tabular} \medskip

\textbf{例题:}设$A$为$n$阶矩阵,且$A^T=A$(此时$A$就是幂等矩阵)。

(1)求$A$的特征值可能的取值。

(2)证明$E+A$是可逆矩阵。

(1)解:$\because A^2=A$,$\therefore f(A)=A^2-A=0$,$f(\lambda)=\lambda^2-\lambda=0$,$\lambda_1=0$,$\lambda_2=1$。

值得注意的是这里求的$\lambda$是可能的取值,因为不同的矩阵特征值不同,只有通过$\vert\lambda E-A\vert=0$的值才是真实的特征值。

\section{相似}

\subsection{矩阵相似}

\subsubsection{定义}

\subsubsection{性质}

\subsection{相似对角化}

\subsubsection{定义}

\subsubsection{对角化条件}

\subsubsection{步骤}

\subsection{应用}

\subsection{相似对角化}

\subsection{反向问题}

\subsubsection{参数}

\subsubsection{矩阵}

\subsection{幂与函数}

\end{document}
