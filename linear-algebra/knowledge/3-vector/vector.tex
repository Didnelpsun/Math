\documentclass[UTF8, 12pt]{ctexart}
% UTF8编码,ctexart现实中文
\usepackage{color}
% 使用颜色
\definecolor{orange}{RGB}{255,127,0} 
\definecolor{violet}{RGB}{192,0,255} 
\definecolor{aqua}{RGB}{0,255,255} 
\usepackage{geometry}
\setcounter{tocdepth}{4}
\setcounter{secnumdepth}{4}
% 设置四级目录与标题
\geometry{papersize={21cm,29.7cm}}
% 默认大小为A4
\geometry{left=3.18cm,right=3.18cm,top=2.54cm,bottom=2.54cm}
% 默认页边距为1英尺与1.25英尺
\usepackage{indentfirst}
\setlength{\parindent}{2.45em}
% 首行缩进2个中文字符
\usepackage{setspace}
\renewcommand{\baselinestretch}{1.5}
% 1.5倍行距
\usepackage{amssymb}
% 因为所以
\usepackage{amsmath}
% 数学公式
\usepackage[colorlinks,linkcolor=black,urlcolor=blue]{hyperref}
% 超链接
\usepackage{pifont}
% 圆圈序号
\author{Didnelpsun}
\title{向量}
\date{}
\begin{document}
\maketitle
\pagestyle{empty}
\thispagestyle{empty}
\tableofcontents
\thispagestyle{empty}
\newpage
\pagestyle{plain}
\setcounter{page}{1}

线性代数的主要研究对象就是向量,行列式与矩阵都是由向量组成的向量组。

\section{向量与向量组}

\subsection{向量的定义与运算}

$n$维向量\textcolor{violet}{\textbf{定义:}}$n$个数构成的一个有序数组$[a_1,a_2,\cdots,a_n]$称为一个$n$维向量,记为$\alpha=[a_1,a_2,\cdots,a_n]$,并称$\alpha$为$n$维行向量,$\alpha^T$为$n$维列向量,$a_i$为向量$\alpha$的$i$个分量。

若$\alpha$与$\beta$都是$n$维向量,且对应元素相等,则$\alpha=\beta$。

$\alpha+\beta=[a_1+b_1,a_2+b_2,\cdots,a_n+b_n]$。

$k\alpha=[ka_1,ka_2,\cdots,ka_3]$。

\subsection{向量组的线性概念}

线性组合\textcolor{violet}{\textbf{定义:}}$m$个$n$维向量$\alpha_1,\alpha_2,\cdots,\alpha_m$以及$m$个数$k_1,k_2,\cdots,k_m$,则向量$k_1\alpha_1+k_2\alpha_2+\cdots+k_m\alpha_m$就是向量组$a_1,a_2,\cdots,a_m$的线性组合。

线性表出\textcolor{violet}{\textbf{定义:}}若向量$\beta$能表示成向量组$\alpha_1,\alpha_2,\cdots,a_m$的线性组合,则存在$m$个数$k_1,k_2,\cdots,k_m$,使得$\beta=k_1\alpha_1+k_2\alpha_2+\cdots+k_m\alpha_m$,则成向量$\beta$能被向量组$a_1,a_2,\cdots,a_m$线性表出。否则不能被线性表出。

线性相关\textcolor{violet}{\textbf{定义:}}对$m$个$n$维向量$a_1,a_2,\cdots,a_m$,存在一组不全为0的数$k_1,k_2,\cdots,k_m$,使得$k_1\alpha_1+k_2\alpha_2+\cdots+k_m\alpha_m=0$,则称$a_1,a_2,\cdots,a_m$线性相关。

含有零向量或成比例向量的向量组必然线性相关。

线性无关\textcolor{violet}{\textbf{定义:}}对$m$个$n$维向量$a_1,a_2,\cdots,a_m$,不存在一组不全为0的数$k_1,k_2,\cdots,k_m$,使得$k_1\alpha_1+k_2\alpha_2+\cdots+k_m\alpha_m=0$,即仅当$k_1=k_2=\cdots=k_m=0$才成立,则称$a_1,a_2,\cdots,a_m$线性无关。

两个非零向量,不成比例向量的向量必然线性无关。

\subsection{线性相关性}

\begin{enumerate}
    \item 向量组$\alpha_1,\alpha_2,\cdots,\alpha_n$($n\geqslant2$)线性相关的充要条件是向量组中至少有一个向量可由其他$n-1$个向量线性表出。若$\alpha_1,\alpha_2,\cdots,\alpha_n$线性无关的充要条件是向量组的任何一个向量都不能被其他$n-1$个向量线性表出。
    \item 向量组$\alpha_1,\alpha_2,\cdots,\alpha_n$线性无关,而$\beta,\alpha_1,\alpha_2,\cdots,\alpha_n$线性相关,则$\beta$可由$\alpha_1,\alpha_2,\cdots,\alpha_n$线性表示,且表示方法唯一。
    \item 向量组$\alpha_1,\alpha_2,\cdots,\alpha_n$可由向量组$\beta_1,\beta_2,\cdots,\beta_s$线性表示,且$n>s$,则$\alpha_1,\alpha_2,\cdots,\alpha_n$线性相关。(以少表多,多的相关)若向量组$\alpha_1,\alpha_2,\cdots,\alpha_n$可由向量组$\beta_1,\beta_2,\cdots,\beta_s$线性表示,$\alpha_1,\alpha_2,\cdots,\alpha_n$线性无关,则$n\leqslant s$。
    \item 设$m$个$n$维向量$\alpha_1,\alpha_2,\cdots,\alpha_m$,其中$\alpha_1=[a_{11},a_{12},\cdots,a_{m1}]^T$,$\cdots$,$\alpha_m=[a_{1m},a_{2m},\cdots,a_{mm}]^T$,则向量组$\alpha_1,\alpha_2,\cdots,\alpha_m$线性相关的充要条件是齐次线性方程$Ax=0$有非零解,其中$A=[\alpha_1,\alpha_2,\cdots,\alpha_m]$,$x=[x_1,x_2,\cdots,x_m]^T$。$m$个$n$维向量$\alpha_1,\alpha_2,\cdots,\alpha_m$线性无关的充要条件是齐次线性方程$Ax=0$只有零解。
    \item 向量$\beta$可由向量组$\alpha_1,\alpha_2,\cdots,\alpha_n$表出,则向量组$\alpha_1x_1+\alpha_2x_2+\cdots+\alpha_nx_n=[\alpha_1,\alpha_2,\cdots,\alpha_n][x_1,x_2,\cdots,x_n]^T=\beta$有解,即$r([\alpha_1,\alpha_2,\cdots,\alpha_n])=r([\alpha_1,\alpha_2,\cdots,\alpha_n,\beta])$。否则则不能表出,则方程无解,$r([\alpha_1,\alpha_2,\cdots,\alpha_n])+1=r([\alpha_1,\alpha_2,\cdots,\alpha_n,\beta])$
    \item 向量组$\alpha_1,\alpha_2,\cdots,\alpha_n$存在一部分向量线性相关,则整个向量组线性相关。若$\alpha_1,\alpha_2,\cdots,\alpha_n$线性无关,则任意一部分向量组线性无关。
    \item 设$m$个$n$维向量$\alpha_1,\alpha_2,\cdots,\alpha_m$线性无关,则把这些向量中每个各任意添加$s$个分量所得到的新向量组($n+s$维)$\alpha_1^*,\alpha_2^*,\cdots,\alpha_m^*$也是线性无关的;如果$\alpha_1,\alpha_2,\cdots,\alpha_m$线性相关,则每个各去掉相同的若干分量得到的新向量组也线性相关。(原来无关延长无关,原来相关缩短相关)
\end{enumerate}

\section{极大线性无关组}

\subsection{概念}

极大线性无关组\textcolor{violet}{\textbf{定义:}}在向量组$\alpha_1,\alpha_2,\cdots,\alpha_n$中,若存在部分$a_i,a_j,\cdots,a_k$满足:\ding{172}$a_i,a_j,\cdots,a_k$线性无关;\ding{173}向量组中任一向量$a_s$($i=1,2,\cdots,n$)均可由$a_i,a_j,\cdots,a_k$线性表出,则称向量组$a_i,a_j,\cdots,a_k$为原向量组的极大线性无关组。

不包含无用约束方程的最简方程组的系数矩阵就是极大线性无关组。

向量组的极大线性无关组一般不唯一,只由一个零向量组成的向量组不存在极大线性无关组,一个线性无关向量组的极大线性无关组就是其本身。

\section{向量组秩}

向量组构成矩阵的秩等于行向量组的秩等于列向量组的秩。

若$A$通过初等行变换为$B$,则$AB$的行向量组是等价向量组,任何对应的部分列向量组都具有同样的线性相关性。

若向量组$B$均可由$A$线性表出,则$r(B)\leqslant r(A)$。

\section{等价向量组}

任何一个组都可以由其极大线性无关组来代表。

\subsection{定义}

设两个向量组$\alpha_1,\alpha_2,\cdots,\alpha_n$和$\beta_1,\beta_2,\cdots,\beta_m$,若这两个向量组可以互相线性表出,则称其为等价向量组,记为$\alpha\cong\beta$。

具有的性质:

\begin{enumerate}
    \item $A\cong A$(反身性)。
    \item $A\cong B$,则$B\cong A$(对称性)。
    \item $A\cong B$,$B\cong C$,则$A\cong C$(传递性)。
\end{enumerate}

向量组和其极大线性无关组是等价向量组。

\subsection{判定}

若$r(A)=r(B)=r(A|B)$,则向量组等价。

\subsection{与等价矩阵区别}

对于矩阵而言,若$A\cong B$,则$AB$同型且$r(A)=r(B)$。

对于向量组而言,若$A\cong B$,则$AB$同维(行数相同)且$r(A)=r(B)=r(A|B)$。

\section{向量空间}

\subsection{基本概念}

若$\xi_1,\xi_2,\cdots,\xi_n$是$n$维向量空间$R^n$中的线性无关的有序向量组,则任意向量$\alpha\in R^n$均可由$\xi_1,\xi_2,\cdots,\xi_n$线性表出,记为$\alpha=a_1\xi_1+a_2\xi_2+\cdots+a_n\xi_n$,类似一个极大线性无关组,则称有序向量组$\xi_1,\xi_2,\cdots,\xi_n$为$R^n$的一个\textbf{基},基向量的个数$n$为向量空间的\textbf{维数},而$[a_1,a_2,\cdots,a_n]([a_1,a_2,\cdots,a_n]^T)$为向量$\alpha$在基$\xi_1,\xi_2,\cdots,\xi_n$下的\textbf{坐标},或称为$\alpha$的坐标行列向量。

\subsection{基变换与坐标变换}

若$\eta_1,\eta_2,\cdots,\eta_n$和$\xi_1,\xi_2,\cdots,\xi_n$是$R^n$中两个基,且有关系:$[\eta_1,\eta_2,\cdots,\eta_n]=[\xi_1,\xi_2,\cdots,\xi_n]C_{n\times n}$,则这个式子称为基$\xi_1,\xi_2,\cdots,\xi_n$到基$\eta_1,\eta_2,\cdots,\eta_n$的\textbf{基变换公式},矩阵$C$就是基$\xi_1,\xi_2,\cdots,\xi_n$到基$\eta_1,\eta_2,\cdots,\eta_n$的\textbf{过渡矩阵},$C$可逆,$C$的第$i$列就是$\eta_i$在基$\xi_1,\xi_2,\cdots,\xi_n$下的坐标列向量。

$\alpha$在基$\xi_1,\xi_2,\cdots,\xi_n$和基$\eta_1,\eta_2,\cdots,\eta_n$下坐标分别为$x=[x_1,x_2,\cdots,x_n]^T$,$y=[y_1,y_2,\cdots,y_n]^T$,即$\alpha=[\xi_1,\xi_2,\cdots,\xi_n]x=[\eta_1,\eta_2,\cdots,\eta_n]y$。又$C$是基$\xi_1,\xi_2,\cdots,\xi_n$到基$\eta_1,\eta_2,\cdots,\eta_n$的过渡矩阵,则$[\xi_1,\xi_2,\cdots,\xi_n]=[\eta_1,\eta_2,\cdots,\eta_n]C$,则$\alpha=[\xi_1,\xi_2,\cdots,\xi_n]x=[\eta_1,\eta_2,\cdots,\eta_n]y=[\xi_1,\xi_2,\cdots,\xi_n]Cy$,从而$x=Cy$或$y=C^{-1}x$,这个就是\textbf{坐标变换公式}。

\end{document}
