\documentclass[UTF8, 12pt]{ctexart}
% UTF8编码,ctexart现实中文
\usepackage{color}
% 使用颜色
\definecolor{orange}{RGB}{255,127,0} 
\definecolor{violet}{RGB}{192,0,255} 
\definecolor{aqua}{RGB}{0,255,255} 
\usepackage{geometry}
\setcounter{tocdepth}{4}
\setcounter{secnumdepth}{4}
% 设置四级目录与标题
\geometry{papersize={21cm,29.7cm}}
% 默认大小为A4
\geometry{left=3.18cm,right=3.18cm,top=2.54cm,bottom=2.54cm}
% 默认页边距为1英尺与1.25英尺
\usepackage{indentfirst}
\setlength{\parindent}{2.45em}
% 首行缩进2个中文字符
\usepackage{setspace}
\renewcommand{\baselinestretch}{1.5}
% 1.5倍行距
\usepackage{amssymb}
% 因为所以
\usepackage{amsmath}
% 数学公式
\usepackage[colorlinks,linkcolor=black,urlcolor=blue]{hyperref}
% 超链接
\author{Didnelpsun}
\title{向量}
\date{}
\begin{document}
\maketitle
\pagestyle{empty}
\thispagestyle{empty}
\tableofcontents
\thispagestyle{empty}
\newpage
\pagestyle{plain}
\setcounter{page}{1}

线性代数的主要研究对象就是向量,行列式与矩阵都是由向量组成的向量组。

\section{向量与向量组}

\subsection{向量的定义与运算}

$n$维向量\textcolor{violet}{\textbf{定义:}}$n$个数构成的一个有序数组$[a_1,a_2,\cdots,a_n]$称为一个$n$维向量,记为$\alpha=[a_1,a_2,\cdots,a_n]$,并称$\alpha$为$n$维行向量,$\alpha^T$为$n$维列向量,$a_i$为向量$\alpha$的$i$个分量。

若$\alpha$与$\beta$都是$n$维向量,且对应元素相等,则$\alpha=\beta$。

$\alpha+\beta=[a_1+b_1,a_2+b_2,\cdots,a_n+b_n]$。

$k\alpha=[ka_1,ka_2,\cdots,ka_3]$。

\subsection{向量组的线性概念}

线性组合$\textcolor{violet}{\textbf{定义:}}$$m$个$n$维向量$\alpha_1,\alpha_2,\cdots,\alpha_m$以及$m$个数$k_1,k_2,\cdots,k_m$,则向量$k_1\alpha_1+k_2\alpha_2+\cdots+k_m\alpha_m$就是向量组$a_1,a_2,\cdots,a_m$的线性组合。

线性表出$\textcolor{violet}{\textbf{定义:}}$若向量$\beta$能表示成向量组$\alpha_1,\alpha_2,\cdots,a_m$的线性组合,则存在$m$个数$k_1,k_2,\cdots,k_m$,使得$\beta=k_1\alpha_1+k_2\alpha_2+\cdots+k_m\alpha_m$,则成向量$\beta$能被向量组$a_1,a_2,\cdots,a_m$线性表出。否则不能被线性表出。

线性相关$\textcolor{violet}{\textbf{定义:}}$对$m$个$n$维向量$a_1,a_2,\cdots,a_m$,存在一组不全为0的数$k_1,k_2,\cdots,k_m$,使得$k_1\alpha_1+k_2\alpha_2+\cdots+k_m\alpha_m=0$,则称$a_1,a_2,\cdots,a_m$线性相关。

含有零向量或成比例向量的向量组必然线性相关。

线性无关$\textcolor{violet}{\textbf{定义:}}$对$m$个$n$维向量$a_1,a_2,\cdots,a_m$,不存在一组不全为0的数$k_1,k_2,\cdots,k_m$,使得$k_1\alpha_1+k_2\alpha_2+\cdots+k_m\alpha_m=0$,即仅当$k_1=k_2=\cdots=k_m=0$才成立,则称$a_1,a_2,\cdots,a_m$线性无关。

两个非零向量,不成比例向量的向量必然线性无关。

\end{document}
