\documentclass[UTF8, 12pt]{ctexart}
% UTF8编码,ctexart现实中文
\usepackage{color}
% 使用颜色
\definecolor{orange}{RGB}{255,127,0} 
\definecolor{violet}{RGB}{192,0,255}  
\definecolor{aqua}{RGB}{0,255,255} 
\usepackage{geometry}
\setcounter{tocdepth}{4}
\setcounter{secnumdepth}{4}
% 设置四级目录与标题
\geometry{papersize={21cm,29.7cm}}
% 默认大小为A4
\geometry{left=3.18cm,right=3.18cm,top=2.54cm,bottom=2.54cm}
% 默认页边距为1英尺与1.25英尺
\usepackage{indentfirst}
\setlength{\parindent}{2.45em}
% 首行缩进2个中文字符
\usepackage{setspace}
\renewcommand{\baselinestretch}{1.5}
% 1.5倍行距
\usepackage{amssymb}
% 因为所以
\usepackage{amsmath}
% 数学公式
\usepackage[colorlinks,linkcolor=black,urlcolor=blue]{hyperref}
% 超链接
\usepackage{multicol}
% 分栏
\usepackage{arydshln}
% 增广矩阵长虚线
\usepackage{pifont}
% 圆圈序号
\setlength{\dashlinegap}{2pt}
\setlength{\dashlinedash}{2pt}
\author{Didnelpsun}
\title{线性方程组}
\date{}
\begin{document}
\maketitle
\pagestyle{empty}
\thispagestyle{empty}
\tableofcontents
\thispagestyle{empty}
\newpage
\pagestyle{plain}
\setcounter{page}{1}
\section{基本概念}

矩阵是根据线性方程组得到。线性方程组和向量组本质上是一致的。

\subsection{线性方程组与矩阵}

\begin{multicols}{2}
    
    $\begin{cases}
        a_{11}x_1+\cdots+a_{1n}x_n=0 \\
        \cdots \\
        a_{m1}x_1+\cdots+a_{mn}x_n=0
    \end{cases}$ \medskip
    
    $n$元齐次线性方程组。

    $\begin{cases}
        a_{11}x_1+\cdots+a_{1n}x_n=b_1 \\
        \cdots \\
        a_{m1}x_1+\cdots+a_{mn}x_n=b_n
    \end{cases}$ \medskip
    
    $n$元非齐次线性方程组。

\end{multicols}

$m$是方程个数,即方程组行数,$n$是方程未知数个数,即类似方程组的列数。

对于齐次方程,$x_1=\cdots=x_n=0$一定是其解,称为其\textbf{零解},若有一组不全为零的解,则称为其\textbf{非零解}。其一定有零解,但是不一定有非零解。

对于非齐次方程,只有$b_1\cdots b_n$不全为零才是。\medskip

令\textbf{系数矩阵}$A_{m\times n}=\left(
    \begin{array}{ccc}
        a_{11} & \cdots & a_{1n} \\
        \cdots \\
        a_{m1} & \cdots & a_{mn}
    \end{array}
\right)$,\textbf{未知数矩阵}$x_{n\times 1}=\left(
    \begin{array}{c}
        x_1 \\
        \cdots \\
        x_n
    \end{array}
\right)$,\textbf{常数项矩阵}$b_{m\times 1}=\left(
    \begin{array}{c}
        b_1 \\
        \cdots \\
        b_m
    \end{array}
\right)$,\textbf{增广矩阵}$B_{m\times(n+1)}=\left(
    \begin{array}{c:c}
        \begin{matrix}
            a_{11} & \cdots & a_{1n}\\
            \cdots \\
            a_{m1} & \cdots & a_{mn}
        \end{matrix}&
        \begin{matrix}
            b_1\\
            \\
            b_n
        \end{matrix}
    \end{array}
\right)$。

所以$AX=\left(
    \begin{array}{c}
        a_11x_1+\cdots+a_{1n}x_n \\
        \cdots \\
        a_{m1}x_1+\cdots+a_{mn}x_n
    \end{array}
\right)$。

从而$AX=b$等价于$\begin{cases}
    a_{11}x_1+\cdots+a_{1n}x_n=b_1 \\
    \cdots \\
    a_{m1}x_1+\cdots+a_{mn}x_n=b_n
\end{cases}$,当$b=O$就是齐次线性方程。

从而矩阵可以简单表示线性方程。

\subsection{矩阵乘法与线性变换}

矩阵乘法实际上就是线性方程组的线性变换,将一个变量关于另一个变量的关系式代入原方程组,得到与另一个变量的关系。

$\begin{cases}
    y_1=a_{11}x_1+a_{12}x_2+\cdots+a_{1s}x_s \\
    \cdots \\
    y_m=a_{m1}x_1+a_{m2}x_2+\cdots+a_{ms}x_s
\end{cases}\begin{cases}
    x_1=b_{11}t_1+b_{12}t_2+\cdots+b_{1n}t_n \\
    \cdots \\
    x_s=b_{s1}t_1+b_{s2}t_2+\cdots+b_{sn}t_n
\end{cases}$\medskip

原本是线性方程分别是$y$与$x$和$x$与$t$的关系式,而如果将$t$关于$x$的关系式代入$x$关于$y$的关系式中,就会得到$t$关于$y$的关系式:\medskip

$\begin{cases}
    y_1=a_{11}(b_{11}t_1+\cdots+b_{1n}t_n)+\cdots+a_{1s}(b_{s1}t_1+b_{s2}t_2+\cdots+b_{sn}t_n) \\
    \cdots \\
    y_m=a_{m1}(b_{11}t_1+\cdots+b_{1n}t_n)+\cdots+a_{ms}(b_{s1}t_1+b_{s2}t_2+\cdots+b_{sn}t_n)
\end{cases}$

$=\begin{cases}
    y_1=(a_{11}b_{11}+\cdots+a_{1s}b_{s1})t_1+\cdots+(a_{11}b_{1n}+\cdots+a_{1s}b_{sn})t_n \\
    \cdots \\
    y_m=(a_{m1}b_{11}+\cdots+a_{ms}b_{s1})t_1+\cdots+(a_{m1}b_{1n}+\cdots+a_{ms}b_{sn})t_m
\end{cases}$ \medskip

这可以看作上面两个线性方程组相乘,也可以将线性方程组表示为矩阵,进行相乘就得到乘积,从而了解矩阵乘积与线性方程组的关系:\medskip


$\left(\begin{array}{ccc}
    a_{11} & \cdots & a_{1s} \\
    \vdots & \ddots & \vdots \\
    a_{m1} & \cdots & a_{ms}
\end{array}\right)_{m\times s}\left(\begin{array}{ccc}
    b_{11} & \cdots & a_{1n} \\
    \vdots & \ddots & \vdots \\
    b_{s1} & \cdots & b_{sn}
\end{array}\right)_{s\times n}$

$=\left(\begin{array}{ccc}
    a_{11}b_{11}+\cdots+a_{1s}b_{s1} & \cdots & a_{11}b_{1n}+\cdots+a_{1s}b_{sn} \\
    \vdots & \ddots & \vdots \\
    a_{m1}b_{11}+\cdots+a_{ms}b_{s1} & \cdots & a_{m1}b_{1n}+\cdots+a_{ms}b_{sn}
\end{array}\right)_{m\times n}\text{。}$

\subsection{线性方程组的解}

对于一元一次线性方程:$ax=b$:

\begin{itemize}
    \item 当$a\neq 0$时,可以解得$x=\dfrac{b}{a}$。
    \item 当$a=0$时,若$b\neq 0$时,无解,若$b=0$时,无数解。
\end{itemize}

当推广到多元一次线性方程组:$Ax=b$,如何求出$x$这一系列的$x$的解?

从数学逻辑上看,已知多元一次方程,有$m$个约束方程,有$n$个未知数,假定$m\leqslant n$。

当$m<n$时,就代表有更多的未知变量不能被方程约束,从而有$n-m$个自由变量,所以就是无数解,解组中其他解可以由自由变量来表示。无穷多解需要一个解来代表其他解,这个解就是\textbf{基础解系}。

当$m=n$时代表约束与变量数量相等,此时又要分三种情况。

当所有的约束条件其中存在线性相关,即一部分约束条件可以由其他约束表示,则代表这部分约束条件是没用的,实际上的约束条件变少,从而情况等于$m<n$,结果是无数解。

当所有的约束条件不存在线性相关,但是一部分约束条件互相矛盾,则约束条件下就无法解出解,从而结果是无实数解。

当所有的约束条件不存在线性相关,且相互之间不存在矛盾情况,这时候才会解出一个实数解,从而结果是有唯一实解。

若使用矩阵来解决线性方程组的问题,其系数矩阵$A_{m\times n}$。

对于$A\neq O$,则$Ax=b$,若存在一个矩阵$B_{n\times n}$类似$\dfrac{1}{a}$,使得$BAx=Bb$,解得$Ex=x=Bb$,这个$B$就是$A$的逆矩阵。

对于$A=O$即不可逆,需要判断$b$是否为0,若不是则无实数解,若是则无穷解,这种判断需要用到增广矩阵,需要用到矩阵的秩判断。

取自由变量时必须要保证取完后的矩阵行列式不为0,否则自由变量不能表示其他向量。

\subsection{线性方程组的矩阵解表示}

已知对于线性方程组$\begin{cases}
    a_{11}x_1+\cdots+a_{1n}x_n=b_1 \\
    \cdots \\
    a_{m1}x_1+\cdots+a_{mn}x_n=b_n
\end{cases}$。

按乘积表示为$A_{m\times n}x_{n\times 1}=b_{m\times 1}$,然后将$A$按列分块,$x$按行分块:\medskip

$(a_1,a_2,\cdots,a_n)\left(\begin{array}{c}
    x_1 \\
    x_2 \\
    \vdots \\
    x_n
\end{array}\right)=b\text{,}\left(\begin{array}{c}
    a_{11} \\
    a_{21} \\
    \vdots \\
    a_{m1}
\end{array}\right)x_1+\cdots+\left(\begin{array}{c}
    a_{1n} \\
    a_{2n} \\
    \vdots \\
    a_{mn}
\end{array}\right)x_n=\left(\begin{array}{c}
    b_1 \\
    b_2 \\
    \vdots \\
    b_m
\end{array}\right)\text{。}$

这三种都是解的表示方法。

\section{具体线性方程}

\subsection{齐次方程组}

即$Ax=0$。其中$A$有$m$行$n$列。

\subsubsection{有解条件}

必有一个零解。

有解条件讨论是否列满秩问题,即方程组是否能约束全部变量。

对系数矩阵进行行变换,若$r(A)=m$,即使行满秩若$m<n$则列不满秩,那么还是无法约束所有变量;若$r(A)=n$,即使行不满秩但是列满秩,所以还是能约束所有变量。

当$r(A)=n$时,即$\alpha_1,\alpha_2,\cdots,\alpha_n$线性无关,则方程组有唯一零解。

当$r(A)=r<n$时,即$\alpha_1,\alpha_2,\cdots,\alpha_n$线性相关,则方程具有无穷多个非零解,具有$n-r$个线性无关解(自由变量)。

\subsubsection{解的性质}

若$A\xi_1=0$,$A\xi_2=0$,则$A(k_1\xi_1+k_2\xi_2)=0$。

\subsubsection{解的结构}

基础解系\textcolor{violet}{\textbf{定义:}}假如$\xi_1,\xi_2,\cdots,\xi_{n-r}$满足:\ding{172}是方程组$Ax=0$的解;\ding{173}线性无关;\ding{174}方程组$Ax=0$的任一解均可由$\xi_1,\xi_2,\cdots,\xi_{n-r}$线性表出,则称$\xi_1,\xi_2,\cdots,\xi_{n-r}$为$Ax=0$的\textbf{基础解系}。

当$r(A)<n$时讨论基础解系。

通解\textcolor{violet}{\textbf{定义:}}设$\xi_1,\xi_2,\cdots,\xi_{n-r}$是$Ax=0$的基础解系,则$k_1\xi_1+k_2\xi_2+\cdots+k_{n-r}\xi_{n-r}$是方程组$Ax=0$的通解,$k_1,k_2,\cdots,k_{n-r}$为任意常数。

\subsubsection{求解过程}

\begin{enumerate}
    \item 将系数矩阵$A$作为\textbf{初等行变换}后化为阶梯形矩阵或最简阶梯形矩阵$B$,因为初等行变换将方程组化为同解方程组,所以$Ax=0$与$Bx=0$同解,只需解$Bx=0$,设$r(A)=r$。其中$A$为$m$行$n$列,$m$为约束方程组个数,$n$为变量个数。
    \item 在$B$中按列找到一个秩为$r$的子矩阵,即在每排阶梯都选出一列组合成子矩阵,则剩余列位置的未知数就是自由变量。(极大线性无关组)
    \item 按基础解析定义求出$\xi_1,\xi_2,\cdots,\xi_{n-r}$,并写出通解。
\end{enumerate}

\textbf{例题:}求齐次线性方程组$\left\{\begin{array}{l}
    x_1+x_2-3x_4-x_5=0 \\
    x_1-x_2+2x_3-x_4=0 \\
    4x_1-2x_2+6x_3+3x_4-4x_5=0 \\
    2x_1+4x_2-2x_3+4x_4-7x_5=0
\end{array}\right.$的通解。

解:系数矩阵$A=\left(\begin{array}{ccccc}
    1 & 1 & 0 & -3 & -1 \\
    1 & -1 & 3 & -1 & 0 \\
    4 & -2 & 6 & 3 & -4 \\
    2 & 4 & -2 & 4 & -7
\end{array}\right)$,然后对其行变换,得到:

$=\left(\begin{array}{ccccc}
    1 & 1 & 0 & -3 & -1 \\
    0 & -2 & 2 & 2 & 1 \\
    0 & 0 & 0 & 3 & -1 \\
    0 & 0 & 0 & 0 & 0
\end{array}\right)=\left(\begin{array}{ccccc}
    1 & 0 & 1 & 0 & -\dfrac{7}{6} \medskip \\
    0 & 1 & -1 & 0 & -\dfrac{5}{6} \medskip \\
    0 & 0 & 0 & 1 & -\dfrac{1}{3} \\
    0 & 0 & 0 & 0 & 0
\end{array}\right)$,$r(A)=3$。

然后找子矩阵,第一台阶选$C_1$,第二台阶选$C_2$或$C_3$,第三台阶选$C_4$或$C_5$,随便找一个,如$(C_1,C_2,C_4)$为子矩阵,则$C_3$,$C_5$所代表的未知数$x_3$,$x_5$就是自由变量。

所以选择两个分量$\xi_1=(\xi_{11},\xi_{12},\xi_{13},\xi_{14},\xi_{15})^T$和$\xi_2=(\xi_{21},\xi_{22},\xi_{23},\xi_{24},\xi_{25})^T$作为基础解系。

因为此时选择$x_3$,$x_5$为自由变量,所以$x_3$和$x_5$所对应的$\xi_{13}$、$\xi_{15}$、$\xi_{23}$、$\xi_{25}$可以任意取,但是为了保证秩为2,所以让$\xi_{13}=1$、$\xi_{15}=0$、$\xi_{23}=0$、$\xi_{25}=1$。这四个分量组成的矩阵线性无关,原矩阵线性无关,延长矩阵线性无关,从而$\xi_1$和$\xi_2$必然线性无关。

所以此时已经给定两组解,一种是$\xi_1$的$x_3=1$,$x_5=0$,另一种是$\xi_2$的$x_3=0$,$x_5=1$,这样就只有三个未知数和三个方程,分别代入$A$矩阵所代表的方程组中(代入行阶梯矩阵就可以,不用代入最简行阶梯矩阵):

$\left\{\begin{array}{l}
    1\cdot x_1+1\cdot x_2+0\cdot x_3-3\cdot x_4-1\cdot x_5=0 \\
    0\cdot x_1-2\cdot x_2+2\cdot x_3+2\cdot x_4+1\cdot x_5=0 \\
    0\cdot x_1+0\cdot x_2+0\cdot x_3+3\cdot x_4-1\cdot x_5=0
\end{array}\right.$,分别代入:

$\xi_1$:$\left\{\begin{array}{l}
    1\cdot x_1+1\cdot x_2+0\cdot1-3\cdot x_4-1\cdot0=0 \\
    0\cdot x_1-2\cdot x_2+2\cdot1+2\cdot x_4+1\cdot0=0 \\
    0\cdot x_1+0\cdot x_2+0\cdot1+3\cdot x_4-1\cdot0=0
\end{array}\right.$,$\xi_1=(-1,1,1,0,0)^T$。

$\xi_2$:$\left\{\begin{array}{l}
    1\cdot x_1+1\cdot x_2+0\cdot0-3\cdot x_4-1\cdot1=0 \\
    0\cdot x_1-2\cdot x_2+2\cdot0+2\cdot x_4+1\cdot1=0 \\
    0\cdot x_1+0\cdot x_2+0\cdot0+3\cdot x_4-1\cdot1=0
\end{array}\right.$,$\xi_2=(7,5,0,2,6)^T$。\medskip

所以通解为$k_1\xi_1+k_2\xi_2=k_1(-1,1,1,0,0)^T+k_2(7,5,0,2,6)^T$。

\subsection{非齐次方程组}

即$Ax=b$,$b$为不全为0的列向量。

\subsubsection{有解条件}

$A=[\alpha_1,\alpha_2,\cdots,\alpha_n]$,其中$\alpha_j=[a_{1j},a_{2j},\cdots,a_{mj}]^T$,$j=1,2,\cdots,n$。

当$r(A)\neq r([A,b])$时($r(A)+1=r([A,b])$),即$b$不能被$A$线性表出,则方程组无解。

当$r(A)=r([A,b])=n$时,即$b$能被$A$线性表出,$A$线性无关,$[A,b]$线性相关,矩阵列满秩,则方程组有唯一解。

当$r(A)=r([A,b])=r<n$时,即$b$能被$A$线性表出,$A$线性相关,矩阵列降秩,则方程组有无穷多解。

\subsubsection{解的性质}

若$\eta_1,\eta_2,\eta$是非齐次线性方程组$Ax=b$的解,$\xi$是对应齐次线性方程组$Ax=0$的解,则:

\ding{172}$\eta_1-\eta_2$是$Ax=0$的通解。

\ding{173}$k\xi+\eta$是$Ax=b$的解。

\subsubsection{求解过程}

将系数矩阵和常数项矩阵合并为一个增广矩阵,对增广矩阵进行行变换变为阶梯形矩阵,求出对应齐次线性方程组的通解,最后假设一个非齐次线性方程组的特解。

\begin{enumerate}
    \item 写出$Ax=b$的导出方程组$Ax=0$并求出其通解$k_1\xi_1+k_2\xi_2+\cdots+k_{n-r}\xi_{n-r}$。
    \item 求出$Ax=b$的一个特解$\eta$。
    \item $Ax=b$的通解为$k_1\xi_1+k_2\xi_2+\cdots+k_{n-r}\xi_{n-r}+\eta$。
\end{enumerate}

\textbf{例题:}求非齐次线性方程组$\left\{\begin{array}{l}
    x_1+5x_2-x_3-x_4=-1 \\
    x_1-2x_2+x_3+3x_4=3 \\
    3x_1+8x_2-x_3+x_4=1 \\
    x_1-9x_2+3x_3+7x_4=7
\end{array}\right.$的通解。

解:对方程组提取出增广矩阵并进行行变换:\medskip

$\left(\begin{array}{c:c}
    \begin{matrix}
        1 & 5 & -1 & -1 \\
        1 & -2 & 1 & 3 \\
        3 & 8 & -1 & 1 \\
        1 & -9 & 3 & 7
    \end{matrix}&
    \begin{matrix}
        -1 \\
        3 \\
        1 \\
        7
    \end{matrix}
\end{array}\right)=\left(\begin{array}{c:c}
    \begin{matrix}
        1 & 5 & -1 & -1 \\
        0 & -7 & 2 & 4 \\
        0 & 0 & 0 & 0 \\
        0 & 0 & 0 & 0
    \end{matrix}&
    \begin{matrix}
        -1 \\
        4 \\
        0 \\
        0
    \end{matrix}
\end{array}\right)$。\medskip

然后求齐次方程的通解:找两列作为子矩阵,如$x_1$,$x_2$,则$x_3$,$x_4$作为自由变量,设两个$\xi_1=(\xi_{11},\xi_{12},1,0)^T$和$\xi_2=(\xi_{21},\xi_{22},0,1)^T$。

解得$\xi_1=(-3,2,7,0)^T$,$\xi_2=(-13,4,0,7)^T$(为了得到整数通解都乘了7)。

通解为$k_1\xi_1+k_2\xi_2=k_1(-3,2,7,0)^T+k_2(-13,4,0,7)^T$。

然后求其非齐次的特解,让两个自由变量为0减少计算,即$\eta=(\eta_1,\eta_2,0,0)^T$代入方程得到$\eta=\left(\dfrac{13}{7},-\dfrac{4}{7},0,0\right)^T$。

所以通解为$k_1(-3,2,7,0)^T+k_2(-13,4,0,7)^T+\left(\dfrac{13}{7},-\dfrac{4}{7},0,0\right)^T$。

\textcolor{orange}{注意:}通解的向量可以同乘一个数,因为其表示的是一个关系而不是具体数,但是特解不能同乘一个数,因为其表示的是一个具体的数。

\subsection{克拉默法则}

克拉默法则本来是矩阵中的运算法则,但是与方程组有更密切的关系,所以放到线性方程组中。

\textcolor{aqua}{\textbf{定理:}}若$Ax=b$的系数矩阵$A$的行列式$\vert A\vert\neq0$,则方程有唯一解,且$x_i=\dfrac{\vert A_i\vert}{\vert A\vert}$,其中$A_i$为把系数矩阵$A$的第$i$列的元素用方程组右侧的常数项代替后所得到的$n$阶矩阵。

\section{抽象线性方程}

\subsection{解的判定}

$Ax=0$,总有解,至少有零解。

$A_{m\times n}x=0$,当$r(A)=n$时,只有零解;当$r(A)<n$时,无穷多解。

$A_{m\times n}x=b$时,当$r(A)=r([A,b])+1\neq r([A,b])$时,无解;当$r(A)=r([A,b])=n$时,有唯一解;当$r(A)=r([A,b])=r<n$时,无穷多解。

当$Ax=0$只有零解时,$r(A)=n$,当$Ax=0$有无穷多解时,$r(A)=r<n$,都不能判定$r(A)$与$r([A,b])$的关系,若以$Ax=b$可能有解也可能无解。

当$Ax=b$有唯一解时,$r(A)=r([A,b])=n$,所以$Ax=0$列满秩,只有零解。

当$Ax=b$有无穷多解时,$r(A)=r([A,b])=r<n$,则$Ax=0$有无穷多解。

当$A$行满秩,则$r(A)=r([A,b])$,则$Ax=\beta$必有解,因为原来无关,延长无关。

所以已知非齐次解情况能推出齐次解情况,但是反之不能。

\subsection{解的性质}

非齐次通解=齐次的通解+非齐次一个特解。

\textbf{例题:}$r(A_{4\times4})=2$,$\eta_1,\eta_2,\eta_3$为$Ax=b$的三个解向量,其中具有如下关系:

$\left\{\begin{array}{l}
    \eta_1-\eta_2=(-1,0,3,-4)^T \\
    \eta_1+\eta_2=(3,2,1,-2)^T \\
    \eta_3+2\eta_2=(5,1,0,3)^T
\end{array}\right.$,求$Ax=b$的通解。

解:$s=n-r(A)=4-2=2$,所以通解的基础解系中有两个分量$\xi_1$和$\xi_2$。

所以需要解$Ax=0$,又存在三个解向量,所以$A\eta_1=A\eta_2=A\eta_3=b$,所以$A(\eta_1-\eta_2)=0$,所以$\eta_1-\eta_2=(-1,0,3,-4)^T$就是其中一个解,所以令$\xi_1=\eta_1-\eta_2=(-1,0,3,-4)^T$。

然后根据所给出的$\eta$进行凑,$A(\eta_1+\eta_2)=2b=A(3,2,1,-2)^T$,$A(\eta_3+2\eta_2)=3b=A(5,1,0,3)^T$。所以$3A(\eta_1+\eta_2)-2A(\eta_3+2\eta_2)=0$,所以$A(3(\eta_1+\eta_2)-2(\eta_3+2\eta_2))=0$,所以令$\xi_2=3(\eta_1+\eta_2)-2(\eta_3+2\eta_2)=(-1,4,3,-12)^T$。

最后找一个特解,$\because A(\eta_1+\eta_2)=2b$,$\therefore A\left(\dfrac{\eta_1+\eta_2}{2}\right)=b$,$\dfrac{\eta_1+\eta_2}{2}=\left(\dfrac{3}{2},1,\dfrac{1}{2},-1\right)^T$就是一个特解。

所以通解为$k_1(-1,0,3,-4)^T+k_2(-1,4,3,-12)^T+\left(\dfrac{3}{2},1,\dfrac{1}{2},-1\right)^T$

\subsection{基础解系}

对于$A_{m\times n}x=0$,$r(A)=r$,若向量组$\alpha_1,\alpha_2,\cdots,\alpha_s$满足:\ding{172}$A\alpha_i=0$,$i=1,2,\cdots,s$;\ding{173}$\alpha_1,\alpha_2,\cdots,\alpha_s$线性无关;\ding{174}$s=n-r$,则称$\alpha_1,\alpha_2,\cdots,\alpha_s$为$Ax=0$的基础解系。

\textbf{例题:}设$\xi_1,\xi_2,\xi_3$是方程组$Ax=0$的基础解系,则下列向量组也是方程组$Ax=0$的基础解系的是()。

$A.\xi_1-\xi_2$,$\xi_2-\xi_3$,$\xi_3-\xi_1$\qquad$B.\xi_1+\xi_2$,$\xi_2-\xi_3$,$\xi_3+\xi_1$

$C.\xi_1+\xi_2-\xi_3$,$\xi_1+2\xi_2+\xi_3$,$2\xi_1+3\xi_2$\qquad$D.\xi_1+\xi_2$,$\xi_2+\xi_3$,$\xi_3+\xi_1$

解:需要判断基础解系是否线性无关,需要对应的行列式值非0。\medskip

对于$D$:$(\xi_1+\xi_2$,$\xi_2+\xi_3$,$\xi_3+\xi_1)=(\xi_1,\xi_2,\xi_3)\left(\begin{array}{ccc}
    1 & 0 & 1 \\
    1 & 1 & 0 \\
    0 & 1 & 1
\end{array}\right)\neq0$,所以$D$线性无关,从而为基础解系。

\textbf{例题:}设$\xi_1=[1,-2,3,1]^T$,$\xi_2=[2,0,5,-2]^T$是齐次线性方程组$A_{3\times4}x=0$的解,且$r(A)=2$,则下列向量中是其解向量的是()。

$A.\alpha_1=[1,-2,3,2]^T$\qquad$B.\alpha_2=[0,0.5,-2]^T$

$C.\alpha_3=[-1,-6,-1,7]^T$\qquad$D.\alpha_4=[1,6,1,6]^T$

解:若$\xi_1$和$\xi_2$为$Ax=0$的基,所以$\xi_1$和$\xi_2$应该能表示其解向量。

所以将$\xi_1$和$\xi_2$与$\alpha_1,\alpha_2,\alpha_3,\alpha_4$分别联立为矩阵,进行初等行变换,查看是否有解,即新增广矩阵必须秩为2。

$ABD$选项增广矩阵的秩都为3,所以不能表示,而只有$C$的为2,所以$C$可以表示。

\subsection{系数矩阵列向量与解}

对于齐次方程而言,其解是让$A$的线性组合为零向量时线性组合的系数,对于非齐次而言解是$b$由$A$线性表出的表出系数。

所以方程的解就是描述列向量组之间数量关心的系数。

\textbf{例题:}已知$A=[\alpha_1,\alpha_2,\alpha_3,\alpha_4]$,其中$\alpha_1,\alpha_2,\alpha_3,\alpha_4$是四维列向量,且$\alpha_1=2\alpha_2+\alpha_3$,$r(A)=3$,若$\beta=\alpha_1+2\alpha_2+3\alpha_3+4\alpha_4$,求线性方程组$Ax=\beta$的通解。

解:$\because\alpha_1=2\alpha_2+\alpha_3$,$1\alpha_1-2\alpha_2-1\alpha_3+0\alpha_4=0$,即$A(1,-2,-1,0)^T=0$。

又$r(A_{4\times4})=4$,$s=n-r(A)=4-3=1$,$\therefore\xi=(1,-2,-1,0)^T$。

所以特解为$\beta$的系数:$(1,2,3,4)^T$,通解为$k(1,-2,-1,0)^T+(1,2,3,4)^T$。

\section{公共解}

\subsection{待定系数法}

\begin{enumerate}
    \item 求两个方程组解的交集部分。可以联立两个方程求解。
    \item 求出$A_{m\times n}x=0$的通解$k_1\xi_1+k_2\xi_2+\cdots+k_s\xi_s$,这些$k$本来是独立的,然后代入$B_{m\times n}x=0$,求出$k_i(i=1,2,\cdots,s)$之间的关系,再代回$A_{m\times n}x=0$的通解中就得到公共解。
    \item 给出$A_{m\times n}x=0$的通解与$B_{m\times n}x=0$的通解联立:$k_1\xi_1+k_2\xi_2+\cdots+k_s\xi_s=l_1\eta_1+l_2\eta_2+\cdots+l_s\eta_s=0$,能解出$k_i$和$l_i$。
\end{enumerate}

这种方法可以求出公共解,不过比较麻烦。

如果已经给出原方程的基础解系而没有给出矩阵,则这个方法解出公共解较好。

\textbf{例题:}已知线性方程组$A=\left\{\begin{array}{l}
    x_1+x_2=0 \\
    x_2-x_4=0
\end{array}\right.$,$B=\left\{\begin{array}{l}
    x_1-x_2+x_3=0 \\
    x_2-x_3+x_4=0
\end{array}\right.$,求方程组的公共解。

解:$A=\left(\begin{array}{cccc}
    1 & 1 & 0 & 0 \\
    0 & 1 & 0 & -1
\end{array}\right)$,$B=\left(\begin{array}{cccc}
    1 & -1 & 1 & 0 \\
    0 & 1 & -1 & 1
\end{array}\right)$。\medskip

两个秩都为2,选择前两个分量为基子矩阵,后两个为通解分量。

$\xi_1=(0,0,1,0)^T$,$\xi_2=(-1,1,0,1)^T$,$\eta_1=(0,1,1,0)^T$,$\eta_2=(-1,-1,0,1)^T$。

$k_1\xi_1+k_2\xi_2=k_1(0,0,1,0)^T+k_2(-1,1,0,1)^T=(-k_2,k_2,k_1,k_2)^T$。

$l_1\eta_1+l_2\eta_2=l_1(0,1,1,0)^T+l_2(-1,-1,0,1)^T=(-l_2,l_1-l_2,l_1,l_2)^T$。

令$(-k_2,k_2,k_1,k_2)^T=(-l_2,l_1-l_2,l_1,l_2)^T$,所以解得$2k_2=k_1$。

公共解为$(-k_2,k_2,2k_2,k_2)^T=k_2(-1,1,2,1)^T$。

\subsection{矩阵法}

要求$A$和$B$的非零公共解,即求联立矩阵$\left(\begin{array}{c}
    A \\
    B
\end{array}\right)x=0$的非零解。对这个矩阵求出基础解系。

如果直接给出矩阵,则这种方法可以不用求出基础解系就能得到公共解。

\section{同解方程组}

\subsection{性质}

若$A_{m\times n}x=0$和$B_{s\times n}x=0$有完全相同的解,就是同解方程组。

$\therefore r(A)=r(B)=r([A,B]^T)$。

$A$与$A^TA$同解。

\subsection{代入法}

先求一个方程组的通解,然后把这个通解代入到第二个方程组中,不用管$k$的取值(因为$k$为任意数,所以直接令其为0)直接求出对应参数。

\textbf{例题:}线性方程组$A=\left\{\begin{array}{l}
    x_1+3x_3+5x_4=0 \\
    x_1-x_2-2x_3+2x_4=0 \\
    2x_1-x_2+x_3+3x_4=0
\end{array}\right.$,在其基础上加一个方程$B=\left\{\begin{array}{l}
    x_1+3x_3+5x_4=0 \\
    x_1-x_2-2x_3+2x_4=0 \\
    2x_1-x_2+x_3+3x_4=0 \\
    4x_4+ax_2+bx_3+13x_4=0
\end{array}\right.$,$ab$满足什么条件,$AB$是同解方程组。

解:$B$在$A$的基础上增加一个方程,即多增加了约束,从而$B$的解一定为$A$的解的子集。所以只要$A$的解也满足$B$的解就是同解方程组。

$A=\left(\begin{array}{cccc}
    1 & 0 & 3 & 5 \\
    0 & -1 & -5 & -3 \\
    0 & 0 & 0 & -4
\end{array}\right)$,$s=n-r=4-3=1$,$\xi=(-3,-5,1,0)^T$,$k\xi=k(-3,-5,1,0)^T=(-3k,-5k,k,0)^T$。

所以这个对于$B$而言必然满足前三行,若要整体满足,就也要满足$B$的第四行,所以直接代入第四行:$4(-3k)+a(-5k)+bk+0=k(-12-5a+b)=0$。

又$k$为任意数,所以$-12-5a+b=0$,即$b=5a+12$。

\textbf{例题:}设$A$为$n$阶实矩阵,$A^T$是$A$的转置矩阵,证明方程组$\Lambda:Ax=0$和$\Upsilon:A^TAx=0$是同解方程组。

证明:若$\gamma$为$\Lambda$的唯一解,则$A\gamma=0$,则$A^TA\gamma=A^T0=0$,$\therefore\gamma$也为$\Upsilon$的解。

若$\eta$为$\Upsilon$的唯一解,则$A^TA\eta=0$,$\eta^TA^TA\eta=(A\eta)^TA\eta=\Vert A\eta\Vert^2=0$,所以$A\eta=0$,从而$\eta$也为$\Lambda$的解。

所以同解,所以其两个矩阵的基解等价。

\textcolor{aqua}{\textbf{定理:}}$r(A)=r(A^T)=r(A^TA)=r(AA^T)$。

\end{document}
