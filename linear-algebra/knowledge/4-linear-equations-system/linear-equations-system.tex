\documentclass[UTF8, 12pt]{ctexart}
% UTF8编码,ctexart现实中文
\usepackage{color}
% 使用颜色
\usepackage{geometry}
\setcounter{tocdepth}{4}
\setcounter{secnumdepth}{4}
% 设置四级目录与标题
\geometry{papersize={21cm,29.7cm}}
% 默认大小为A4
\geometry{left=3.18cm,right=3.18cm,top=2.54cm,bottom=2.54cm}
% 默认页边距为1英尺与1.25英尺
\usepackage{indentfirst}
\setlength{\parindent}{2.45em}
% 首行缩进2个中文字符
\usepackage{setspace}
\renewcommand{\baselinestretch}{1.5}
% 1.5倍行距
\usepackage{amssymb}
% 因为所以
\usepackage{amsmath}
% 数学公式
\usepackage[colorlinks,linkcolor=black,urlcolor=blue]{hyperref}
% 超链接
\usepackage{multicol}
% 分栏
\usepackage{arydshln}
% 增广矩阵长虚线
\author{Didnelpsun}
\title{线性方程组}
\date{}
\begin{document}
\maketitle
\pagestyle{empty}
\thispagestyle{empty}
\tableofcontents
\thispagestyle{empty}
\newpage
\pagestyle{plain}
\setcounter{page}{1}
\section{基本概念}

矩阵是根据线性方程组得到。线性方程组和向量组本质上是一致的。

\subsection{线性方程组与矩阵}

\begin{multicols}{2}
    
    $\begin{cases}
        a_{11}x_1+\cdots+a_{1n}x_n=0 \\
        \cdots \\
        a_{m1}x_1+\cdots+a_{mn}x_n=0
    \end{cases}$ \medskip
    
    $n$元齐次线性方程组。

    $\begin{cases}
        a_{11}x_1+\cdots+a_{1n}x_n=b_1 \\
        \cdots \\
        a_{m1}x_1+\cdots+a_{mn}x_n=b_n
    \end{cases}$ \medskip
    
    $n$元非齐次线性方程组。

\end{multicols}

$m$是方程个数,即方程组行数,$n$是方程未知数个数,即类似方程组的列数。

对于齐次方程,$x_1=\cdots=x_n=0$一定是其解,称为其\textbf{零解},若有一组不全为零的解,则称为其\textbf{非零解}。其一定有零解,但是不一定有非零解。

对于非齐次方程,只有$b_1\cdots b_n$不全为零才是。\medskip

令\textbf{系数矩阵}$A_{m\times n}=\left(
    \begin{array}{ccc}
        a_{11} & \cdots & a_{1n} \\
        \cdots \\
        a_{m1} & \cdots & a_{mn}
    \end{array}
\right)$,\textbf{未知数矩阵}$x_{n\times 1}=\left(
    \begin{array}{c}
        x_1 \\
        \cdots \\
        x_n
    \end{array}
\right)$,\textbf{常数项矩阵}$b_{m\times 1}=\left(
    \begin{array}{c}
        b_1 \\
        \cdots \\
        b_m
    \end{array}
\right)$,\textbf{增广矩阵}$B_{m\times(n+1)}=\left(
    \begin{array}{c:c}
        \begin{matrix}
            a_{11} & \cdots & a_{1n}\\
            \cdots \\
            a_{m1} & \cdots & a_{mn}
        \end{matrix}&
        \begin{matrix}
            b_1\\
            \\
            b_n
        \end{matrix}
    \end{array}
\right)$。

所以$AX=\left(
    \begin{array}{c}
        a_11x_1+\cdots+a_{1n}x_n \\
        \cdots \\
        a_{m1}x_1+\cdots+a_{mn}x_n
    \end{array}
\right)$。

从而$AX=b$等价于$\begin{cases}
    a_{11}x_1+\cdots+a_{1n}x_n=b_1 \\
    \cdots \\
    a_{m1}x_1+\cdots+a_{mn}x_n=b_n
\end{cases}$,当$b=O$就是齐次线性方程。

从而矩阵可以简单表示线性方程。

\subsection{矩阵乘法与线性变换}

矩阵乘法实际上就是线性方程组的线性变换,将一个变量关于另一个变量的关系式代入原方程组,得到与另一个变量的关系。

$\begin{cases}
    y_1=a_{11}x_1+a_{12}x_2+\cdots+a_{1s}x_s \\
    \cdots \\
    y_m=a_{m1}x_1+a_{m2}x_2+\cdots+a_{ms}x_s
\end{cases}\begin{cases}
    x_1=b_{11}t_1+b_{12}t_2+\cdots+b_{1n}t_n \\
    \cdots \\
    x_s=b_{s1}t_1+b_{s2}t_2+\cdots+b_{sn}t_n
\end{cases}$\medskip

原本是线性方程分别是$y$与$x$和$x$与$t$的关系式,而如果将$t$关于$x$的关系式代入$x$关于$y$的关系式中,就会得到$t$关于$y$的关系式:\medskip

$\begin{cases}
    y_1=a_{11}(b_{11}t_1+\cdots+b_{1n}t_n)+\cdots+a_{1s}(b_{s1}t_1+b_{s2}t_2+\cdots+b_{sn}t_n) \\
    \cdots \\
    y_m=a_{m1}(b_{11}t_1+\cdots+b_{1n}t_n)+\cdots+a_{ms}(b_{s1}t_1+b_{s2}t_2+\cdots+b_{sn}t_n)
\end{cases}$

$=\begin{cases}
    y_1=(a_{11}b_{11}+\cdots+a_{1s}b_{s1})t_1+\cdots+(a_{11}b_{1n}+\cdots+a_{1s}b_{sn})t_n \\
    \cdots \\
    y_m=(a_{m1}b_{11}+\cdots+a_{ms}b_{s1})t_1+\cdots+(a_{m1}b_{1n}+\cdots+a_{ms}b_{sn})t_m
\end{cases}$ \medskip

这可以看作上面两个线性方程组相乘,也可以将线性方程组表示为矩阵,进行相乘就得到乘积,从而了解矩阵乘积与线性方程组的关系:\medskip


$\left(\begin{array}{ccc}
    a_{11} & \cdots & a_{1s} \\
    \vdots & \ddots & \vdots \\
    a_{m1} & \cdots & a_{ms}
\end{array}\right)_{m\times s}\left(\begin{array}{ccc}
    b_{11} & \cdots & a_{1n} \\
    \vdots & \ddots & \vdots \\
    b_{s1} & \cdots & b_{sn}
\end{array}\right)_{s\times n}$

$=\left(\begin{array}{ccc}
    a_{11}b_{11}+\cdots+a_{1s}b_{s1} & \cdots & a_{11}b_{1n}+\cdots+a_{1s}b_{sn} \\
    \vdots & \ddots & \vdots \\
    a_{m1}b_{11}+\cdots+a_{ms}b_{s1} & \cdots & a_{m1}b_{1n}+\cdots+a_{ms}b_{sn}
\end{array}\right)_{m\times n}\text{。}$

\subsection{线性方程组的解}

对于一元一次线性方程:$ax=b$:

\begin{itemize}
    \item 当$a\neq 0$时,可以解得$x=\dfrac{b}{a}$。
    \item 当$a=0$时,若$b\neq 0$时,无解,若$b=0$时,无数解。
\end{itemize}

当推广到多元一次线性方程组:$Ax=b$,如何求出$x$这一系列的$x$的解?

从数学逻辑上看,已知多元一次方程,有$m$个约束方程,有$n$个未知数,假定$m\leqslant n$。

当$m<n$时,就代表有更多的未知变量不能被方程约束,从而有$n-m$个自由变量,所以就是无数解,解组中其他解可以由自由变量来表示。

当$m=n$时代表约束与变量数量相等,此时又要分三种情况。

当所有的约束条件其中存在线性相关,即一部分约束条件可以由其他约束表示,则代表这部分约束条件是没用的,实际上的约束条件变少,从而情况等于$m<n$,结果是无数解。

当所有的约束条件不存在线性相关,但是一部分约束条件互相矛盾,则约束条件下就无法解出解,从而结果是无实数解。

当所有的约束条件不存在线性相关,且相互之间不存在矛盾情况,这时候才会解出一个实数解,从而结果是有唯一实解。

若使用矩阵来解决线性方程组的问题,其系数矩阵$A_{m\times n}$。

对于$A\neq O$,则$Ax=b$,若存在一个矩阵$B_{n\times n}$类似$\dfrac{1}{a}$,使得$BAx=Bb$,解得$Ex=x=Bb$,这个$B$就是$A$的逆矩阵。

对于$A=O$即不可逆,需要判断$b$是否为0,若不是则无实数解,若是则无穷解,这种判断需要用到增广矩阵,需要用到矩阵的秩判断。

\subsection{线性方程组的矩阵解表示}

已知对于线性方程组$\begin{cases}
    a_{11}x_1+\cdots+a_{1n}x_n=b_1 \\
    \cdots \\
    a_{m1}x_1+\cdots+a_{mn}x_n=b_n
\end{cases}$。

按乘积表示为$A_{m\times n}x_{n\times 1}=b_{m\times 1}$,然后将$A$按列分块,$x$按行分块:\medskip

$(a_1,a_2,\cdots,a_n)\left(\begin{array}{c}
    x_1 \\
    x_2 \\
    \vdots \\
    x_n
\end{array}\right)=b\text{,}\left(\begin{array}{c}
    a_{11} \\
    a_{21} \\
    \vdots \\
    a_{m1}
\end{array}\right)x_1+\cdots+\left(\begin{array}{c}
    a_{1n} \\
    a_{2n} \\
    \vdots \\
    a_{mn}
\end{array}\right)x_n=\left(\begin{array}{c}
    b_1 \\
    b_2 \\
    \vdots \\
    b_m
\end{array}\right)\text{。}$

这三种都是解的表示方法。

\section{具体线性方程}

\subsection{齐次方程组}

\subsubsection{有解条件}

\subsubsection{解的性质}

\subsubsection{基础解系}

\subsubsection{求解过程}

\subsection{非齐次方程组}

\subsubsection{有解条件}

\subsubsection{解的性质}

\subsubsection{求解过程}

\section{抽象线性方程}

\subsection{解的判定}

\subsection{解的性质}

\subsection{求解过程}

\section{公共解}

\section{通解方程组}

\end{document}
