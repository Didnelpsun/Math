\documentclass[UTF8, 12pt]{ctexart}
% UTF8编码,ctexart现实中文
\usepackage{color}
% 使用颜色
\usepackage{geometry}
\setcounter{tocdepth}{4}
\setcounter{secnumdepth}{4}
% 设置四级目录与标题
\geometry{papersize={21cm,29.7cm}}
% 默认大小为A4
\geometry{left=3.18cm,right=3.18cm,top=2.54cm,bottom=2.54cm}
% 默认页边距为1英尺与1.25英尺
\usepackage{indentfirst}
\setlength{\parindent}{2.45em}
% 首行缩进2个中文字符
\usepackage{setspace}
\renewcommand{\baselinestretch}{1.5}
% 1.5倍行距
\usepackage{amssymb}
% 因为所以
\usepackage{amsmath}
% 数学公式
\usepackage[colorlinks,linkcolor=black,urlcolor=blue]{hyperref}
% 超链接
\author{Didnelpsun}
\title{二次型}
\date{}
\begin{document}
\maketitle
\pagestyle{empty}
\thispagestyle{empty}
\tableofcontents
\thispagestyle{empty}
\newpage
\pagestyle{plain}
\setcounter{page}{1}
\section{二次型}

\subsection{定义}

$n$元变量$x_1,x_2,\cdots,x_n$的二次齐次多项式:

$
\begin{array}{cr}
    f(x_1,x_2,\cdots,x_n)= & a_{11}x_1^2+2a_{12}x_1x_2+\cdots+2a_{1n}x_1x_n \\
    & +a_{22}x_2^x+\cdots+2a_{2n}x_2x_n \\
    & \cdots \\
    & +a_{nn}x_n^2
\end{array}
$

这就是\textbf{$n$元二次型},简称\textbf{二次型}。

$\because x_ix_j=x_jx_i$,令$a_{ij}=a_{ji}$($i,j=1,2,\cdots,n$),则$2a_{ij}x_ix_j=a_{ij}x_ix_j+a_{ji}x_jx_i$:

$f(x_1,x_2,\cdots,x_n)=a_{11}x_1^2+a_{12}x_1x_2+\cdots+a_{1n}x_1x_n+a_{21}x_2x_1+a_{22}x_2^2+a_{2n}x_2x_n+\cdots+a_{n1}x_nx_1+a_{n2}x_nx_2+\cdots+a_{nn}x_n^2$,这个式子就是完全展开式。

$f(x_1,x_2,\cdots,x_n)=\sum\limits_{i=1}^n\sum\limits_{j=1}^na_{ij}x_ix_j$,这个就是和式。

\subsection{矩阵表示}

二次型可以用矩阵来表示,即$f(x)=x^TAx$。其中$x$是列向量。

矩阵表示的重点就是找到中间的$A$,$A$是$f$的二次型矩阵。

方法是:$A$的主对角线元素$a_{ii}$是$x_i^2$的对应系数,$a_{ij}$与$a_{ji}$是混合项$x_ix_j$的系数的一半。

如一个二次型$f(x_1,x_2,x_3)=2x_1^2+5x_2^2+5x_3^2+4x_1x_2-4x_1x_3-8x_2x_3$

$=(x_1,x_2,x_3)\left(\begin{array}{ccc}
    2 & 2 & -2 \\
    2 & 5 & -4 \\
    -2 & -4 & 5
\end{array}\right)(x_1,x_2,x_3)^T$。

所以可以发现二次型矩阵就是一个对称矩阵,所以只要能写出二次型的就一定存在一个对称矩阵,就一定可以相似对角化。

\section{标准型与规范型}

\subsection{配方法}

\subsection{正交变换法}

\subsection{合同变换}

\subsubsection{线性变换}

对于$n$元二次型$f(x_1,x_2,\cdots,x_n)$,若令$\left\{\begin{array}{l}
    x_1=c_{11}y_1+c_{12}y_2+\cdots+c_{1n}y_n \\
    x_2=c_{21}y_1+c_{22}y_2+\cdots+c_{2n}y_n \\
    \cdots \\
    x_n=c_{n1}y_1+c_{n2}y_2+\cdots+c_{nn}y_n
\end{array}\right.$

记$x=(x_1,x_2,\cdots,x_n)$,$C=\left(\begin{array}{cccc}
    c_{11} & c_{12} & \cdots & c_{1n} \\
    c_{21} & c_{22} & \cdots & c_{2n} \\
    \vdots & \vdots & \ddots & \vdots \\
    c_{n1} & c_{n2} & \cdots & c_{nn}
\end{array}\right)$,$y=(y_1,y_2,\cdots,y_n)$,则上式写为$x=Cy$称为$y_1,y_2,\cdots,y_n$到$x_1,x_2,\cdots,x_n$的\textbf{线性变换}。

若线性变换的系数矩阵$C$可逆,即$\vert C\vert\neq0$,则称为\textbf{可逆线性变换}。

若$f(x)=x^TAx$,令$x=Cy$,则$f(x)=(Cy)^TA(Cy)=y^T(C^TAC)y$,记$B=C^TAC$,则$f(x)=y^TBy=g(y)$,此时二次型$f(x)=x^TAx$通过线性变换$x=Cy$得到一个新二次型$g(y)=y^TBy$。即将二次型用$x$表示换成用$y$表示。

$x^TAx=y^TBy$这种改变表示方法的变换就是合同变换。

\subsubsection{定义}

二次型$f(x)$与$g(y)$的系数矩阵$A$与$B$满足$B=C^TAC$,这种关系就是\textbf{合同变换}。

设$AB$为$n$阶实对称矩阵,若存在可逆矩阵$C$,使得$C^TAC=B$,则称$AB$\textbf{合同},记为$A\simeq B$,此时$f(x)$与$g(x)$为\textbf{合同二次型}。

\subsection{惯性定理}

\section{正定二次型}

\subsection{定义}

\subsection{性质}

\subsection{判定}

\end{document}
