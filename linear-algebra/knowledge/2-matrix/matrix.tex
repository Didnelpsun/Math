\documentclass[UTF8, 12pt]{ctexart}
% UTF8编码,ctexart现实中文
\usepackage{color}
% 使用颜色
\usepackage{geometry}
\setcounter{tocdepth}{4}
\setcounter{secnumdepth}{4}
% 设置四级目录与标题
\geometry{papersize={21cm,29.7cm}}
% 默认大小为A4
\geometry{left=3.18cm,right=3.18cm,top=2.54cm,bottom=2.54cm}
% 默认页边距为1英尺与1.25英尺
\usepackage{indentfirst}
\setlength{\parindent}{2.45em}
% 首行缩进2个中文字符
\usepackage{setspace}
\renewcommand{\baselinestretch}{1.5}
% 1.5倍行距
\usepackage{amssymb}
% 因为所以
\usepackage{amsmath}
% 数学公式
\usepackage[colorlinks,linkcolor=black,urlcolor=blue]{hyperref}
% 超链接
\usepackage{multicol}
% 分栏
\author{Didnelpsun}
\title{矩阵}
\date{}
\begin{document}
\maketitle
\pagestyle{empty}
\thispagestyle{empty}
\tableofcontents
\thispagestyle{empty}
\newpage
\pagestyle{plain}
\setcounter{page}{1}

矩阵本质是一个表格。

\section{矩阵定义}

$m\times n$矩阵是由$m\times n$个数$a_{ij}$(元素)排成的$m$行$n$列的数表。

元素是实数的矩阵称为实矩阵,元素是复数的矩阵是复矩阵。

行数列数都为$n$的就是$n$阶矩阵或方阵,记为$A_n$。

行矩阵或行向量:只有一行的矩阵$A=(a_1a_2\cdots a_n)$。

列矩阵或列向量:只有一列的矩阵$B=
\left(\begin{array}{c} 
    b_1 \\
    b_2 \\
    \cdots \\
    b_m
\end{array}\right)$。

同型矩阵:两个矩阵行数、列数相等。

相等矩阵:是同型矩阵,且对应元素相等的矩阵。记为$A=B$。

零矩阵:元素都是零的矩阵,记为$O$,但是不同型的零矩阵不相等。

\begin{multicols}{2}
    
    
    对角矩阵或对角阵:从左上角到右下角的直线(对角线)以外的元素都是0的矩阵,记为$\varLambda=\textrm{diag}(\lambda_1,\lambda_2,\cdots,\lambda_n)$。

    $\varLambda=\left(
        \begin{array}{cccc}
            \lambda_1 & 0 & \cdots & 0 \\
        \end{array}
    \right)$

    单位矩阵或单位阵:$\lambda_1=\lambda_2=\cdots=\lambda_n=1$的对角矩阵,记为$E$。



\end{multicols}

\end{document}
