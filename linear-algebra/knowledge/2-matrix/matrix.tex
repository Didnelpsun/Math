\documentclass[UTF8, 12pt]{ctexart}
% UTF8编码,ctexart现实中文
\usepackage{color}
% 使用颜色
\definecolor{orange}{RGB}{255,127,0} 
\definecolor{violet}{RGB}{192,0,255} 
\definecolor{aqua}{RGB}{0,255,255} 
\usepackage{geometry}
\setcounter{tocdepth}{4}
\setcounter{secnumdepth}{4}
% 设置四级目录与标题
\geometry{papersize={21cm,29.7cm}}
% 默认大小为A4
\geometry{left=3.18cm,right=3.18cm,top=2.54cm,bottom=2.54cm}
% 默认页边距为1英尺与1.25英尺
\usepackage{indentfirst}
\setlength{\parindent}{2.45em}
% 首行缩进2个中文字符
\usepackage{setspace}
\renewcommand{\baselinestretch}{1.5}
% 1.5倍行距
\usepackage{amssymb}
% 因为所以
\usepackage{amsmath}
% 数学公式
\usepackage[colorlinks,linkcolor=black,urlcolor=blue]{hyperref}
% 超链接
\usepackage{multicol}
% 分栏
\usepackage{arydshln}
\setlength{\dashlinegap}{1pt}
\setlength{\dashlinedash}{1pt}
% 阶梯矩阵的虚线
\author{Didnelpsun}
\title{矩阵}
\date{}
\begin{document}
\maketitle
\pagestyle{empty}
\thispagestyle{empty}
\tableofcontents
\thispagestyle{empty}
\newpage
\pagestyle{plain}
\setcounter{page}{1}

矩阵本质是一个表格。

\section{矩阵定义}

\textcolor{violet}{\textbf{定义:}}$m\times n$矩阵是由$m\times n$个数$a_{ij}$(元素)排成的$m$行$n$列的数表。

元素是实数的矩阵称为\textbf{实矩阵},元素是复数的矩阵是\textbf{复矩阵}。

行数列数都为$n$的就是\textbf{$n$阶矩阵}或\textbf{方阵},记为$A_n$。

行矩阵或行向量\textcolor{violet}{\textbf{定义:}}只有一行的矩阵$A=(a_1a_2\cdots a_n)$。

列矩阵或列向量\textcolor{violet}{\textbf{定义:}}只有一列的矩阵$B=
\left(\begin{array}{c} 
    b_1 \\
    b_2 \\
    \cdots \\
    b_m
\end{array}\right)$。

同型矩阵\textcolor{violet}{\textbf{定义:}}两个矩阵行数、列数相等。

相等矩阵\textcolor{violet}{\textbf{定义:}}是同型矩阵,且对应元素相等的矩阵。记为$A=B$。

零矩阵\textcolor{violet}{\textbf{定义:}}元素都是零的矩阵,记为$O$,但是不同型的零矩阵不相等。

\begin{multicols}{2}
    
    
    对角矩阵或对角阵\textcolor{violet}{\textbf{定义:}}从左上角到右下角的直线(对角线)以外的元素都是0的矩阵,记为$\varLambda=\textrm{diag}(\lambda_1,\lambda_2,\cdots,\lambda_n)$。

    $\varLambda=\left(
        \begin{array}{cccc}
            \lambda_1 & 0 & \cdots & 0 \\
            0 & \lambda_2 & \cdots & 0 \\
            \vdots & \vdots & \vdots & \vdots \\
            0 & 0 & \cdots & \lambda_n
        \end{array}
    \right)$

    单位矩阵或单位阵\textcolor{violet}{\textbf{定义:}}$\lambda_1=\lambda_2=\cdots=\lambda_n=1$的对角矩阵,记为$E$。这种线性变换叫做恒等变换,$AE=A$。 \medskip

    $E=\left(
        \begin{array}{cccc}
            1 & 0 & \cdots & 0 \\
            0 & 1 & \cdots & 0 \\
            \vdots & \vdots & \vdots & \vdots \\
            0 & 0 & \cdots & 1
        \end{array}
    \right)$

\end{multicols}

\section{矩阵运算}

\subsection{矩阵加法减法}

设与两个矩阵都是同型矩阵$m\times n$,$A=(a_{ij})$和$B=(b_{ij})$,则其加法就是$A+B$。

$A+B=\left(
    \begin{array}{cccc}
        a_{11}+b_{11} & a_{12}+b_{12} & \cdots & a_{1n}+b_{1n} \\
        a_{21}+b_{21} & a_{22}+b_{22} & \cdots & a_{2n}+b_{2n} \\
        \vdots & \vdots & \vdots & \vdots \\
        a_{m1}+b_{m1} & a_{m2}+b_{m2} & \cdots & a_{m+n}+b_{m+n}
    \end{array}
\right)$

\begin{itemize}
    \item $A+B=B+A$。
    \item $(A+B)+C=A+(B+C)$。
\end{itemize}

若$-A=(-a_{ij})$,则$-A$是$A$的负矩阵,$A+(-A)=O$。

从而矩阵的减法为$A-B=A+(-B)$。

\subsection{数乘矩阵}

数$\lambda$与矩阵$A$的乘积记为$\lambda A$或$A\lambda$,规定:

$\lambda A=A\lambda=\left(
    \begin{array}{cccc}
        \lambda a_{11} & \lambda a_{12} & \cdots & \lambda a_{1n} \\
        \lambda a_{21} & \lambda a_{22} & \cdots & \lambda a_{2n} \\
        \vdots & \vdots & \ddots & \vdots \\
        \lambda a_{m1} & \lambda a_{m2} & \cdots & \lambda a_{mn}
    \end{array}
\right)$

假设$A$、$B$都是$m\times n$的矩阵,$\lambda$、$\mu$为数:

\begin{itemize}
    \item $(\lambda\mu)A=\lambda(\mu A)$。
    \item $(\lambda+\mu)A=\lambda A+\mu A$。
    \item $\lambda(A+B)=\lambda A+\lambda B$。
\end{itemize}

矩阵加法与数乘矩阵都是矩阵的线性运算。

\subsection{矩阵相乘}

设$A=(a_{ij})$是一个$m\times s$的矩阵,$B=(b_{ij})$是一个$s\times n$的矩阵,那么$A\times B=AB=C_{m\times n}=(c_{ij})$。即:$c_{ij}=a_{i1}b_{1j}+a_{i2}b_{2j}+\cdots+a_{is}b_{sj}=\sum\limits_{k=1}^sa_{ik}b_{kj}\,\text{(}i=1,2,\cdots,m;j=1,2,\cdots,n\text{)}$。

即前一个矩阵的行乘后一个矩阵的列就得到当前元素的值。

所以按此定义一个$1\times s$行矩阵与$s\times 1$列矩阵的乘积就是一个1阶方针即一个数:

$(a_{i1},a_{i2},\cdots,a_{is})\left(
    \begin{array}{c}
        b_{1j} \\
        b_{2j} \\
        \cdots \\
        b_{sj}
    \end{array}
\right)=a_{i1}b_{1j}+a_{i2}b_{2j}+\cdots+a_{is}b_{sj}=\sum\limits_{k=1}^sa_{ik}b_{kj}=c_{ij}$。

从而$AB=C$的$c_{ij}$就是$A$的第$i$行与$B$的$j$列的乘积。

当$A$左边乘$P$为$PA$,称为\textbf{左乘}$P$,若右边乘$P$为$AP$,则称为\textbf{右乘}$P$。

\textcolor{orange}{注意:}只有左矩阵的列数等于右矩阵的行数才能相乘。

只有$AB$都是方阵的时候才能$AB$与$BA$。

矩阵的左乘与右乘不一定相等,即$AB\neq BA$。

\textcolor{violet}{\textbf{定义:}}若方阵$AB$乘积满足$AB=BA$,则表示其是\textbf{可交换}的。

$A\neq O$,$B\neq O$,但是不能推出$AB\neq O$或$BA\neq O$。

$AB=O$不能推出$A=O$或$B=O$。

$A(X-Y)=O$当$A\neq O$也不能推出$X=Y$。

\begin{itemize}
    \item $(AB)C=A(BC)$。
    \item $\lambda(AB)=(\lambda A)B=A(\lambda B)$。
    \item $A(B+C)=AB+AC$。
    \item $(B+C)A=BA+CA$。
    \item $EA=AE=A$。
\end{itemize}

$\lambda E$称为\textbf{纯量阵},$(\lambda E_n)A_n=\lambda A_n=A_n(\lambda E_n)$。

若$A_{m\times s}$,$B_{s\times n}=(\beta_1,\cdots,\beta_s)$,其中$\beta$为$n$行的列矩阵,则:

$AB=A(\beta_1,\cdots,\beta_s)=(A\beta_1,\cdots,A\beta_n)$。

\subsection{矩阵幂}

只有方阵才能连乘,从而只有方阵才有幂。

若$A$是$n$阶方阵,所以:

$A^1=A\text{,}A^2=A^1A^1\text{,}\cdots\text{,}A^{k+1}=A^kA^1$

\begin{itemize}
    \item $A^kA^l=A^{k+l}$。
    \item $(A^k)^l=A^{kl}$。
\end{itemize}

因为矩阵乘法一般不满足交换率,所以$(AB)^k\neq A^kB^k$。只有$AB$可交换时才相等。

若$A\neq 0$不能推出$A^k\neq 0$,如:

$A=\left(
    \begin{array}{cc}
        0 & 2 \\
        0 & 0
    \end{array}
\right)\neq 0$。$A^2=\left(
    \begin{array}{cc}
        0 & 2 \\
        0 & 0
    \end{array}
\right)\left(
    \begin{array}{cc}
        0 & 2 \\
        0 & 0
    \end{array}
\right)=\left(
    \begin{array}{cc}
        0 & 0 \\
        0 & 0
    \end{array}
\right)=O$。

$A=\left(
    \begin{array}{ccc}
        0 & 1 & 1 \\
        0 & 0 & 1 \\
        0 & 0 & 0
    \end{array}
\right)$,$A^3=O$。

矩阵幂可以同普通多项式进行处理。

如$f(x)=a_nx^n+\cdots+a_1x+n$,对于$A$就是$f(A)=a_nA^n+\cdots+a_1A+a_nE$。

$f(A)=A^2-A-6E=(A+2E)(A-3E)$。

\subsection{矩阵转置}

把矩阵$A$的行换成同序数的列就得到一个新矩阵,就是$A$的转置矩阵$A^T$。若$A$为$m\times n$,则$A^T$为$n\times m$。

\begin{itemize}
    \item $(A^T)^T=A$。
    \item $(A+B)^T=A^T+B^T$。
    \item $(\lambda A)^T=\lambda A^T$。
    \item $(AB)^T=B^TA^T$。
    \item 若$m=n$,$\vert A\vert=\vert A^T\vert$。
\end{itemize}

对称矩阵或对称阵\textcolor{violet}{\textbf{定义:}}元素以对角线为对称轴对应相等,$A=A^T$。

\subsection{方阵行列式}

由$n$阶方阵$A$的元素所构成的行列式称为矩阵$A$的行列式,记为$\textrm{det}\,A$或$\vert A\vert$。

\begin{itemize}
    \item $\vert A^T\vert=\vert A\vert$。
    \item $\vert\lambda A\vert=\lambda^n\vert A\vert$。
    \item $\vert AB\vert=\vert A\vert\cdot\vert B\vert=\vert BA\vert$。
\end{itemize}

一般而言:$\vert A+B\vert\neq\vert A\vert+\vert B\vert$,$\vert A\vert\neq O\nRightarrow\vert A\vert\neq0$,$A\neq B\nRightarrow\vert A\vert\neq\vert B\vert$。

伴随矩阵或伴随阵\textcolor{violet}{\textbf{定义:}}行列式$\vert A\vert$各个元素的代数余子式$A_{ij}$转置构成的矩阵。

$A^*=\left(
    \begin{array}{cccc}
        A_{11} & A_{21} & \cdots & A_{n1} \\
        A_{12} & A_{22} & \cdots & A_{n2} \\
        \vdots & \vdots & \ddots & \vdots \\
        A_{1n} & A_{2n} & \cdots & A_{nn}
    \end{array}
\right)$

其中$AA^*=A^*A=\vert A\vert E$。

\subsection{分块矩阵}

在行列式的时候提到了分块行列式,分块行列式计算时要求对应的零行列式必须是行列数相等的,而对于分块矩阵而言则不要求,且不一定要零矩阵。

对于行列数较多的矩阵常使用\textbf{分块法},将大矩阵化为小矩阵。将矩阵用横纵线分为多个小矩阵,每个矩阵成为矩阵的\textbf{子块},以子块为元素的矩阵就是\textbf{分块矩阵}。

\subsubsection{分块矩阵计算}

分块矩阵的计算法则与普通矩阵计算类似。

\textcolor{aqua}{\textbf{定理:}}若$AB$矩阵行列数相同,采用相同的分块法,则

$A=\left(
    \begin{array}{ccc}
        A_{11} & \cdots & A_{1r} \\
        \vdots & & \vdots \\
        A_{s1} & \cdots & A_{sr}
    \end{array}
\right)\text{,}B=\left(
    \begin{array}{ccc}
        B_{11} & \cdots & B_{1r} \\
        \vdots & & \vdots \\
        B_{s1} & \cdots & B_{sr}
    \end{array}
\right)$

$A+B=\left(
    \begin{array}{ccc}
        A_{11}+B_{11} & \cdots & A_{1r}+B_{1r} \\
        \vdots & & \vdots \\
        A_{s1}+B_{s1} & \cdots & A_{sr}+B_{sr}
    \end{array}
\right)\text{。}$

\textcolor{aqua}{\textbf{定理:}}设$A=\left(
    \begin{array}{ccc}
        A_{11} & \cdots & A_{1r} \\
        \vdots & & \vdots \\
        A_{s1} & \cdots & A_{sr}
    \end{array}
\right)$,$\lambda$为数,则$\lambda A=\left(
    \begin{array}{ccc}
        \lambda A_{11} & \cdots & \lambda A_{1r} \\
        \vdots & & \vdots \\
        \lambda A_{s1} & \cdots & \lambda A_{sr}
    \end{array}
\right)$。\medskip

\textcolor{aqua}{\textbf{定理:}}若$A_{m\times l}$,$B_{l\times n}$,采用相同的分块法,则

$A=\left(
    \begin{array}{ccc}
        A_{11} & \cdots & A_{1t} \\
        \vdots & & \vdots \\
        A_{s1} & \cdots & A_{st}
    \end{array}
\right)\text{,}B=\left(
    \begin{array}{ccc}
        B_{11} & \cdots & B_{1t} \\
        \vdots & & \vdots \\
        B_{t1} & \cdots & B_{sr}
    \end{array}
\right)$

$AB=\left(
    \begin{array}{ccc}
        C_{11} & \cdots & C_{1r} \\
        \vdots & & \vdots \\
        C_{s1} & \cdots & C_{sr}
    \end{array}
\right)\text{,}C_{ij}=\sum\limits_{k=1}^tA_{ik}B_{kj}\text{。}$

\textcolor{aqua}{\textbf{定理:}}设$A=\left(
    \begin{array}{ccc}
        A_{11} & \cdots & A_{1r} \\
        \vdots & & \vdots \\
        A_{s1} & \cdots & A_{sr}
    \end{array}
\right)$,则$A^T=\left(
    \begin{array}{ccc}
        A_{11}^T & \cdots & A_{s1}^T \\
        \vdots & & \vdots \\
        A_{1r}^T & \cdots & A_{sr}^T
    \end{array}
\right)$。 \medskip

\textcolor{aqua}{\textbf{定理:}}设$A$为$n$阶方阵,$A$的分块矩阵只有对角线上才有非零子块且都是方阵,其余子块都是零矩阵,即$A=\left(
    \begin{array}{cccc}
        A_1 & & & O \\
         & A_2 & \\
         & & \ddots & \\
        O & & & A_s
    \end{array}
\right)$,称为\textbf{分块对角矩阵}。$\vert A\vert=\vert A_1\vert\vert A_2\vert\cdots\vert A_s\vert$。

若$\vert A_i\vert\neq0$,则$\vert A\vert\neq0$,且$A^{-1}=\left(
    \begin{array}{cccc}
        A_1^{-1} & & & O \\
         & A_2^{-1} & \\
         & & \ddots & \\
        O & & & A_s^{-1}
    \end{array}
\right)$。

\subsubsection{按行按列分块}

对于$m\times n$的矩阵$A$,其$n$列称为$A$的$n$个列向量,若第$j$列记为$a_j=\left(
    \begin{array}{c}
        a_{1j} \\
        a_{2j} \\
        \vdots \\
        a_{mj}
    \end{array}
\right)$,则$A$可以按列分块为$A=(a_1,a_2,\cdots,a_n)$。

其$m$行称为$A$的$m$个行向量,若第$i$行记为$a_i^T=(a_{i1},a_{i2},\cdots,a_{in})$,则$A$可以按行分块为$A=\left(\begin{array}{c}
    a_1^T \\
    a_2^T \\
    \vdots \\
    a_{m}^T
\end{array}\right)$。

若对于$A_{m\times s}$与$B_{s\times n}$的乘积矩阵$AB=C=(c_{ij})_{m\times n}$,若将$A$按行分为$m$块,$B$按列分为$n$块,则有:

$AB=\left(
    \begin{array}{c}
        a_1^T \\
        a_2^T \\
        \vdots \\
        a_{m}^T
    \end{array}
\right)(b_1,b_2,\cdots,b_n)$

$=\left(
    \begin{array}{cccc}
        a_1^Tb_1 & a_1^Tb_2 & \cdots & a_1^Tb_n \\
        a_2^Tb_1 & a_2^Tb_2 & \cdots & a_2^Tb_n \\
        \vdots & \vdots & \ddots & \vdots \\
        a_{m}^Tb_1 & a_{m}^Tb_2 & \cdots & a_{m}^Tb_n
    \end{array}
\right)=(c_{ij})_{m\times n}\text{。}$

其中:$c_{ij}=a_i^Tb_j=(a_{i1},a_{i2},\cdots,a_{is})\left(\begin{array}{c}
    b_{1j} \\
    b_{2j} \\
    \vdots \\
    b_{sj}
\end{array}\right)=\sum\limits_{k=1}^s=a_{ik}b_{kj}\text{。}$

\textcolor{aqua}{\textbf{定理:}}$A=O$的充要条件是$A^TA=O$。

证明:$\because A=O$,$\therefore A^T=O$,$A^TA=O$。

设$A=(a_{ij})_{m\times n}$,将$A$按列分块为$A=(a_1,a_2,\cdots,a_n)$,则

$A^TA=\left(
    \begin{array}{c}
        a_1^T \\
        a_2^T \\
        \vdots \\
        a_{m}^T
    \end{array}
\right)(a_1,a_2,\cdots,a_n)=\left(
    \begin{array}{cccc}
        a_1^Ta_1 & a_1^Ta_2 & \cdots & a_1^Ta_n \\
        a_2^Ta_1 & a_2^Ta_2 & \cdots & a_2^Ta_n \\
        \vdots & \vdots & \ddots & \vdots \\
        a_{m}^Ta_1 & a_{m}^Ta_2 & \cdots & a_{m}^Ta_n
    \end{array}
\right)\text{。}$

所以$A^TA$的元为$a^T_ia_j$,又$\because A^TA=O$,$\therefore a^T_ia_j=0$($i,j=1,2,\cdots n$)。

$\therefore a^T_ja_j=0$($j=1,2,\cdots n$),对角线元素全部为0。

且$a^T_ja_j=\left(
    \begin{array}{cccc}
        a_1^Ta_1 & & & \\
         & a_2^Ta_2 & & \\
         & & \ddots & \\
         & & & a_{m}^Ta_n
    \end{array}
\right)=(a_{1j},a_{2j},\cdots,a_{mj})\left(\begin{array}{c}
    a_{1j} \\
    a_{2j} \\
    \vdots \\
    a_{mj}
\end{array}\right)$

$=a_{1j}^2+a_{2j}^2+\cdots+a_{mj}^2=0$,所以$a_{1j}=a_{2j}=\cdots+a_{mj}=0$。

$\therefore A=O$。

\section{线性方程组}

矩阵是根据线性方程组得到。

\subsection{线性方程组与矩阵}

\begin{multicols}{2}
    
    $\begin{cases}
        a_{11}x_1+\cdots+a_{1n}x_n=0 \\
        \cdots \\
        a_{m1}x_1+\cdots+a_{mn}x_n=0
    \end{cases}$ \medskip
    
    $n$元齐次线性方程组。

    $\begin{cases}
        a_{11}x_1+\cdots+a_{1n}x_n=b_1 \\
        \cdots \\
        a_{m1}x_1+\cdots+a_{mn}x_n=b_n
    \end{cases}$ \medskip
    
    $n$元非齐次线性方程组。

\end{multicols}

对于齐次方程,$x_1=\cdots=x_n=0$一定是其解,称为其\textbf{零解},若有一组不全为零的解,则称为其\textbf{非零解}。其一定有零解,但是不一定有非零解。

对于非齐次方程,只有$b_1\cdots b_n$不全为零才是。

令\textbf{系数矩阵}$A_{m\times n}=\left(
    \begin{array}{ccc}
        a_{11} & \cdots & a_{1n} \\
        \cdots \\
        a_{m1} & \cdots & a_{mn}
    \end{array}
\right)$,\textbf{未知数矩阵}$x_{n\times 1}=\left(
    \begin{array}{c}
        x_1 \\
        \cdots \\
        x_n
    \end{array}
\right)$,\textbf{常数项矩阵}$b_{m\times 1}=\left(
    \begin{array}{c}
        b_1 \\
        \cdots \\
        b_m
    \end{array}
\right)$,\textbf{增广矩阵}$B_{m\times(n+1)}=\left(
    \begin{array}{c:c}
        \begin{matrix}
            a_{11} & \cdots & a_{1n}\\
            \cdots \\
            a_{m1} & \cdots & a_{mn}
        \end{matrix}&
        \begin{matrix}
            b_1\\
            \\
            b_n
        \end{matrix}
    \end{array}
\right)$。

所以$AX=\left(
    \begin{array}{c}
        a_11x_1+\cdots+a_{1n}x_n \\
        \cdots \\
        a_{m1}x_1+\cdots+a_{mn}x_n
    \end{array}
\right)$。

从而$AX=b$等价于$\begin{cases}
    a_{11}x_1+\cdots+a_{1n}x_n=b_1 \\
    \cdots \\
    a_{m1}x_1+\cdots+a_{mn}x_n=b_n
\end{cases}$,当$b=O$就是齐次线性方程。

从而矩阵可以简单表示线性方程。

\subsection{矩阵乘法与线性变换}

矩阵乘法实际上就是线性方程组的线性变换,将一个变量关于另一个变量的关系式代入原方程组,得到与另一个变量的关系。

$\begin{cases}
    y_1=a_{11}x_1+a_{12}x_2+\cdots+a_{1s}x_s \\
    \cdots \\
    y_m=a_{m1}x_1+a_{m2}x_2+\cdots+a_{ms}x_s
\end{cases}\begin{cases}
    x_1=b_{11}t_1+b_{12}t_2+\cdots+b_{1n}t_n \\
    \cdots \\
    x_s=b_{s1}t_1+b_{s2}t_2+\cdots+b_{sn}t_n
\end{cases}$

原本是线性方程分别是$y$与$x$和$x$与$t$的关系式,而如果将$t$关于$x$的关系式代入$x$关于$y$的关系式中,就会得到$t$关于$y$的关系式:

$\begin{cases}
    y_1=a_{11}(b_{11}t_1+\cdots+b_{1n}t_n)+\cdots+a_{1s}(b_{s1}t_1+b_{s2}t_2+\cdots+b_{sn}t_n) \\
    \cdots \\
    y_m=a_{m1}(b_{11}t_1+\cdots+b_{1n}t_n)+\cdots+a_{ms}(b_{s1}t_1+b_{s2}t_2+\cdots+b_{sn}t_n)
\end{cases}$

$=\begin{cases}
    y_1=(a_{11}b_{11}+\cdots+a_{1s}b_{s1})t_1+\cdots+(a_{11}b_{1n}+\cdots+a_{1s}b_{sn})t_n \\
    \cdots \\
    y_m=(a_{m1}b_{11}+\cdots+a_{ms}b_{s1})t_1+\cdots+(a_{m1}b_{1n}+\cdots+a_{ms}b_{sn})t_m
\end{cases}$

这可以看作上面两个线性方程组相乘,也可以将线性方程组表示为矩阵,进行相乘就得到乘积,从而了解矩阵乘积与线性方程组的关系:


$\left(\begin{array}{ccc}
    a_{11} & \cdots & a_{1s} \\
    \vdots & \ddots & \vdots \\
    a_{m1} & \cdots & a_{ms}
\end{array}\right)_{m\times s}\left(\begin{array}{ccc}
    b_{11} & \cdots & a_{1n} \\
    \vdots & \ddots & \vdots \\
    b_{s1} & \cdots & b_{sn}
\end{array}\right)_{s\times n}$

$=\left(\begin{array}{ccc}
    a_{11}b_{11}+\cdots+a_{1s}b_{s1} & \cdots & a_{11}b_{1n}+\cdots+a_{1s}b_{sn} \\
    \vdots & \ddots & \vdots \\
    a_{m1}b_{11}+\cdots+a_{ms}b_{s1} & \cdots & a_{m1}b_{1n}+\cdots+a_{ms}b_{sn}
\end{array}\right)_{m\times n}\text{。}$

\subsection{线性方程组的解}

对于一元一次线性方程:$ax=b$:

\begin{itemize}
    \item 当$a\neq 0$时,可以解得$x=\dfrac{b}{a}$。
    \item 当$a=0$时,若$b\neq 0$时,无解,若$b=0$时,无数解。
\end{itemize}

当推广到多元一次线性方程组:$Ax=b$,如何求出$x$这一系列的$x$的解?

从数学逻辑上看,已知多元一次方程,有$m$个约束方程,有$n$个未知数,假定$m\leqslant n$。

当$m<n$时,就代表有更多的未知变量不能被方程约束,从而有$n-m$个自由变量,所以就是无数解,解组中其他解可以由自由变量来表示。

当$m=n$时代表约束与变量数量相等,此时又要分三种情况。

当所有的约束条件其中存在线性相关,即一部分约束条件可以由其他约束表示,则代表这部分约束条件是没用的,实际上的约束条件变少,从而情况等于$m<n$,结果是无数解。

当所有的约束条件不存在线性相关,但是一部分约束条件互相矛盾,则约束条件下就无法解出解,从而结果是无实数解。

当所有的约束条件不存在线性相关,且相互之间不存在矛盾情况,这时候才会解出一个实数解,从而结果是有唯一实解。

若使用矩阵来解决线性方程组的问题,其系数矩阵$A_{m\times n}$。

对于$A\neq O$,则$Ax=b$,若存在一个矩阵$B_{n\times n}$类似$\dfrac{1}{a}$,使得$BAx=Bb$,解得$Ex=x=Bb$,这个$B$就是$A$的逆矩阵。

对于$A=O$即不可逆,需要判断$b$是否为0,若不是则无实数解,若是则无穷解,这种判断需要用到增广矩阵,需要用到矩阵的秩判断。

\subsection{线性方程组的矩阵解表示}

已知对于线性方程组$\begin{cases}
    a_{11}x_1+\cdots+a_{1n}x_n=b_1 \\
    \cdots \\
    a_{m1}x_1+\cdots+a_{mn}x_n=b_n
\end{cases}$。

按乘积表示为$A_{m\times n}x_{n\times 1}=b_{m\times 1}$,然后将$A$按列分块,$x$按行分块:

$(a_1,a_2,\cdots,a_n)\left(\begin{array}{c}
    x_1 \\
    x_2 \\
    \vdots \\
    x_n
\end{array}\right)=b\text{,}\left(\begin{array}{c}
    a_{11} \\
    a_{21} \\
    \vdots \\
    a_{m1}
\end{array}\right)x_1+\cdots+\left(\begin{array}{c}
    a_{1n} \\
    a_{2n} \\
    \vdots \\
    a_{mn}
\end{array}\right)x_n=\left(\begin{array}{c}
    b_1 \\
    b_2 \\
    \vdots \\
    b_m
\end{array}\right)\text{。}$

这三种都是解的表示方法。

\section{逆矩阵}

\textcolor{violet}{\textbf{定义:}}逆矩阵类比倒数,若对于$n$阶矩阵$A$,有一个$n$阶矩阵$B$,使得$AB=BA=E$,则$A$可逆,$B$是$A$的逆矩阵也称为逆阵,且逆矩阵唯一,记为$B=A^{-1}$。

\textcolor{aqua}{\textbf{定理:}}若矩阵$A$可逆,则$\vert A\vert\neq 0$。

证明:若$A$可逆,则$AA^{-1}=E$,所以$\vert A\vert\cdot\vert A^{-1}\vert=\vert E\vert=1$,$\vert A\vert\neq 0$。

可以类比普通数字,若$a$有一个倒数$\dfrac{1}{a}$,则$a\neq 0$,否则无法倒。

\textcolor{aqua}{\textbf{定理:}}若$\vert A\vert\neq 0$,则$A$可逆,且$A^{-1}=\dfrac{1}{\vert A\vert}A^*$。

证明:$\because AA^*=A^*A=\vert A\vert E$,又$\vert A\vert\neq 0$,$A\dfrac{1}{\vert A\vert}A^*=\dfrac{1}{\vert A\vert}A^*A=E$。

按逆矩阵定义,当$A$可逆,与$A^{-1}=\dfrac{1}{\vert A\vert}A^*$。

当$\vert A\vert=0$时,$A$为\textbf{奇异矩阵},否则是\textbf{非奇异矩阵}。

\textcolor{aqua}{\textbf{定理:}}矩阵是可逆矩阵的必要条件是非奇异矩阵。

\textcolor{aqua}{\textbf{定理:}}若$AB=E$或$BA=E$,则$B=A^{-1}$。

\begin{itemize}
    \item 若$A$可逆,则$(A^{-1})^{-1}=A$。
    \item 若$A$可逆,数$\lambda\neq0$,则$(\lambda A)^{-1}=\dfrac{1}{\lambda}A^{-1}$。
    \item 若$AB$为同阶矩阵且都可逆,则$(AB)^{-1}=B^{-1}A^{-1}$。
    \item 若$A$可逆,则$(A^T)^{-1}=(A^{-1})^T$。
    \item 若$A$可逆,$\lambda\mu$为整数时,$A^\lambda A^\mu=A^{\lambda+\mu}$,$(A^\lambda)^\mu=A^{\lambda\mu}$。
\end{itemize}

\section{矩阵初等变换}

求逆矩阵可以使用伴随矩阵来求,但是只针对三阶以及以下的矩阵,若阶数过高则会十分困难。可以使用矩阵初等变换来实现求逆矩阵。且初等变换还可以用来求线性方程组的解。

\subsection{初等变换}

矩阵的三种初等行变换:

\begin{enumerate}
    \item 对换两行(对换$ij$两行,记为$r_i\leftrightarrow r_j$)。
    \item 以数$k\neq0$乘某一行中的所有元(第$i$行乘$k$,记为$r_i\times k$),对角线元素全部为0。
    \item 把某一行所有元的$k$倍加到另一行对应元上(第$j$行的$k$倍加上第$i$行上,记为$r_i+kr_j$)。
\end{enumerate}

把对应的行换为列就得到初等列变换,将$r$改为$c$。其逆变换也是一种初等变换。初等行变换和初等列变换都是\textbf{初等变换}。

\textcolor{violet}{\textbf{定义:}}若$A$经过有限次行变换得到$B$,则称$AB$行等价,记为$A\overset{r}{\thicksim}B$;若$A$经过有限次列变换得到$B$,则称$AB$行等价,记为$A\overset{c}{\thicksim}B$;若$A$经过有限次初等变换得到$B$,则称$AB$行等价,记为$A\thicksim B$。

矩阵之间的等价关系:

\begin{enumerate}
    \item 反身性:$A\thicksim A$。
    \item 对称性:若$A\thicksim B$,则$B\thicksim A$。
    \item 传递性:若$A\thicksim B$,$B\thicksim C$,则$A\thicksim C$。
\end{enumerate}

若是解方程组,则使用初等行变换解不会发生改变,若使用初等列变换则会改变解。

\subsection{阶梯型矩阵}

将方程式用增广矩阵表示,然后通过初等行变换就可以对方程式进行消元。得到如下类型的矩阵结果,类似三角行列式,如:

\begin{multicols}{2}
    
    $
    \left(
    \begin{array}{@{} c c c c c @{}}
        \multicolumn{1}{: c}{1} & 2 & -1 &  3 &  4 \\
      \cdashline{1-1}
      0 & \multicolumn{1}{: c}{1} &  3 & -2 & -1 \\
      \cdashline{2-5}
      0 & 0 &  0 &  0 &  0 \\
      0 & 0 &  0 &  0 &  0
    \end{array}
    \right)
    $

    竖线区分零元素与非零元素,每行的竖线右方第一个元素,称为该非零行的\textbf{首非零元}。

\end{multicols}

行阶梯形矩阵\textcolor{violet}{\textbf{定义:}}非零行在零行的上面,非零行的首非零元素所在列在上一行首非零元素所在列的右边的非零矩阵。

行最简形矩阵\textcolor{violet}{\textbf{定义:}}非零行的首非零元素为1,首非零元所在列其他的元全部为0的行阶梯矩阵。

对于任何矩阵都能通过初等列变换变为行阶梯形矩阵和行最简形矩阵,再通过列变换可以变为\textbf{标准形}:左上角是一个单位矩阵,其他元全部是0。

\subsection{初等变换性质}

\textcolor{aqua}{\textbf{定理:}}设$AB$都是$m\times n$矩阵:\begin{enumerate}
    \item $A\overset{r}{\thicksim}B$的充要条件是存在$m$阶可逆矩阵$P$,使得$PA=B$。
    \item $A\overset{c}{\thicksim}B$的充要条件是存在$n$阶可逆矩阵$Q$,使得$AQ=B$。
    \item $A\thicksim B$的充要条件是存在$m$阶可逆矩阵$P$和$n$阶可逆矩阵$Q$,使得$PAQ=B$。
\end{enumerate}

初等矩阵\textcolor{violet}{\textbf{定义:}}由单位矩阵$E$经过一次初等变换得到的矩阵。

初等矩阵包括:\begin{enumerate}
    \item 第$ij$行对换:$E_m(ij)A$,第$ij$列变换:$AE_n(ij)$。
    \item 数$k$乘第$i$行:$E_m(i(k))A$,数$k$乘第$i$列:$AE_n(i(k))$。
    \item 数$k$乘第$j$行加到$i$行:$E_m(ij(k))A$,数$k$乘第$j$列加到$i$列:$AE_n(ij(k))$。
\end{enumerate}

\textcolor{aqua}{\textbf{定理:}}设$A$是一个$m\times n$矩阵,对$A$进行一次初等行变换,相当于在$A$左乘对应$m$阶初等矩阵;对$A$进行一次列变换,相当于在$A$右乘对应$n$阶初等矩阵。

\textcolor{aqua}{\textbf{定理:}}方阵$A$可逆的充分必要条件是存在有限个初等矩阵$P_i$使得$A=\prod\limits_{i=1}^nP_i$。

\textcolor{aqua}{\textbf{定理:}}方阵$A$可逆的充要条件是$A\overset{r}{\thicksim}E$。(即$A$方阵所代表的线性方程组能通过初等计算得到最后的解)

\subsection{初等行变换求逆}

已知$A^{-1}=\dfrac{A^*}{\vert A\vert}$,但是伴随矩阵计算非常麻烦,并且若矩阵在三阶以上计算就很难办到,所以还有一种方法,就是若该矩阵$A$是可逆矩阵,就将$AX=B$的增广矩阵$(A,B)$化为最简形矩阵,从而得到方程解。

$(A\vdots B)\overset{r}{\thicksim}(E\vdots A^{-1})$,$\left(\begin{array}{c}
    A \\
    \cdots \\
    B
\end{array}\right)\overset{c}{\thicksim}\left(\begin{array}{c}
    E \\
    \cdots \\
    A^{-1}
\end{array}\right)$。

\textbf{例题:}求解矩阵方程$AX=B$,$A=\left(\begin{array}{ccc}
    2 & 1 & -3 \\
    1 & 2 & -2 \\
    -1 & 3 & 2
\end{array}\right)$,$B=\left(\begin{array}{cc}
    1 & -1 \\
    2 & 0 \\
    -2 & 5
\end{array}\right)$。

解:因为$AX=B$,所以左乘$A^{-1}$:$A^{-1}AX=EX=A^{-1}B$,增广矩阵行变换:

$(A,B)=\left(\begin{array}{ccccc}
    2 & 1 & -3 & 1 & -1 \\
    1 & 2 & -2 & 2 & 0 \\
    -1 & 3 & 2 & -2 & 5
\end{array}\right)\thicksim\left(\begin{array}{ccccc}
    1 & 2 & -2 & 2 & 0 \\
    0 & -3 & 1 & -3 & -1 \\
    0 & 5 & 0 & 0 & 5
\end{array}\right)$

$\thicksim\left(\begin{array}{ccccc}
    1 & 2 & -2 & 2 & 0 \\
    0 & 1 & 0 & 0 & 1 \\
    0 & 0 & 1 & -3 & 2
\end{array}\right)\thicksim\left(\begin{array}{ccccc}
    1 & 0 & 0 & -4 & 2 \\
    0 & 1 & 0 & 0 & 1 \\
    0 & 0 & 1 & -3 & 2
\end{array}\right)$,从而$X=\left(\begin{array}{cc}
    -4 & 2 \\
    0 & 1 \\
    -3 & 2
\end{array}\right)$。

\section{矩阵秩}

秩的本质就是组成矩阵的线性无关的向量个数。

\end{document}