\documentclass[UTF8, 12pt]{ctexart}
% UTF8编码,ctexart现实中文
\usepackage{color}
% 使用颜色
\definecolor{orange}{RGB}{255,127,0} 
\definecolor{aqua}{RGB}{0,255,255} 
\usepackage{geometry}
\setcounter{tocdepth}{4}
\setcounter{secnumdepth}{4}
% 设置四级目录与标题
\geometry{papersize={21cm,29.7cm}}
% 默认大小为A4
\geometry{left=3.18cm,right=3.18cm,top=2.54cm,bottom=2.54cm}
% 默认页边距为1英尺与1.25英尺
\usepackage{indentfirst}
\setlength{\parindent}{2.45em}
% 首行缩进2个中文字符
\usepackage{setspace}
\renewcommand{\baselinestretch}{1.5}
% 1.5倍行距
\usepackage{amssymb}
% 因为所以
\usepackage{amsmath}
% 数学公式
\usepackage[colorlinks,linkcolor=black,urlcolor=blue]{hyperref}
% 超链接
\usepackage{multicol}
% 分栏
\author{Didnelpsun}
\title{矩阵}
\date{}
\begin{document}
\maketitle
\pagestyle{empty}
\thispagestyle{empty}
\tableofcontents
\thispagestyle{empty}
\newpage
\pagestyle{plain}
\setcounter{page}{1}

矩阵本质是一个表格。

\section{矩阵定义}

$m\times n$矩阵是由$m\times n$个数$a_{ij}$(元素)排成的$m$行$n$列的数表。

元素是实数的矩阵称为实矩阵,元素是复数的矩阵是复矩阵。

行数列数都为$n$的就是$n$阶矩阵或方阵,记为$A_n$。

行矩阵或行向量:只有一行的矩阵$A=(a_1a_2\cdots a_n)$。

列矩阵或列向量:只有一列的矩阵$B=
\left(\begin{array}{c} 
    b_1 \\
    b_2 \\
    \cdots \\
    b_m
\end{array}\right)$。

同型矩阵:两个矩阵行数、列数相等。

相等矩阵:是同型矩阵,且对应元素相等的矩阵。记为$A=B$。

零矩阵:元素都是零的矩阵,记为$O$,但是不同型的零矩阵不相等。

\begin{multicols}{2}
    
    
    对角矩阵或对角阵:从左上角到右下角的直线(对角线)以外的元素都是0的矩阵,记为$\varLambda=\textrm{diag}(\lambda_1,\lambda_2,\cdots,\lambda_n)$。

    $\varLambda=\left(
        \begin{array}{cccc}
            \lambda_1 & 0 & \cdots & 0 \\
            0 & \lambda_2 & \cdots & 0 \\
            \vdots & \vdots & \vdots & \vdots \\
            0 & 0 & \cdots & \lambda_n
        \end{array}
    \right)$

    单位矩阵或单位阵:$\lambda_1=\lambda_2=\cdots=\lambda_n=1$的对角矩阵,记为$E$。这种线性变换叫做恒等变换,$AE=A$。

    $E=\left(
        \begin{array}{cccc}
            1 & 0 & \cdots & 0 \\
            0 & 1 & \cdots & 0 \\
            \vdots & \vdots & \vdots & \vdots \\
            0 & 0 & \cdots & 1
        \end{array}
    \right)$

\end{multicols}

\section{矩阵运算}

\subsection{矩阵加法减法}

设与两个矩阵都是同型矩阵$m\times n$$A=(a_{ij})$和$B=(b_{ij})$,则其加法就是$A+B$。

$$A+B=\left(
    \begin{array}{cccc}
        a_{11}+b_{11} & a_{12}+b_{12} & \cdots & a_{1n}+b_{1n} \\
        a_{21}+b_{21} & a_{22}+b_{22} & \cdots & a_{2n}+b_{2n} \\
        \vdots & \vdots & \vdots & \vdots \\
        a_{m1}+b_{m1} & a_{m2}+b_{m2} & \cdots & a_{m+n}+b_{m+n}
    \end{array}
\right)$$

\begin{itemize}
    \item $A+B=B+A$。
    \item $(A+B)+C=A+(B+C)$。
\end{itemize}

若$-A=(-a_{ij})$,则$-A$是$A$的负矩阵,$A+(-A)=O$。

从而矩阵的减法为$A-B=A+(-B)$。

\subsection{数乘矩阵}

数$\lambda$与矩阵$A$的乘积记为$\lambda A$或$A\lambda$,规定:

$$\lambda A=A\lambda=\left(
    \begin{array}{cccc}
        \lambda a_{11} & \lambda a_{12} & \cdots & \lambda a_{1n} \\
        \lambda a_{21} & \lambda a_{22} & \cdots & \lambda a_{2n} \\
        \vdots & \vdots & \ddots & \vdots \\
        \lambda a_{m1} & \lambda a_{m2} & \cdots & \lambda a_{mn}
    \end{array}
\right)$$

假设$A$、$B$都是$m\times n$的矩阵,$\lambda$、$\mu$为数:

\begin{itemize}
    \item $(\lambda\mu)A=\lambda(\mu A)$。
    \item $(\lambda+\mu)A=\lambda A+\mu A$。
    \item $\lambda(A+B)=\lambda A+\lambda B$。
\end{itemize}

矩阵加法与数乘矩阵都是矩阵的线性运算。

\subsection{矩阵相乘}

设$A=(a_{ij})$是一个$m\times s$的矩阵,$B=(b_{ij})$是一个$s\times n$的矩阵,那么$A\times B=AB=C_{m\times n}=(c_{ij})$。即:

$$c_{ij}=a_{i1}b_{1j}+a_{i2}b_{2j}+\cdots+a_{is}b_{sj}=\sum\limits_{k=1}^sa_{ik}b_{kj}\,\text{(}i=1,2,\cdots,m;j=1,2,\cdots,n\text{)}$$

所以按此定义一个$1\times s$行矩阵与$s\times 1$列矩阵的乘积就是一个1阶方针即一个数:

$(a_{i1},a_{i2},\cdots,a_{is})\left(
    \begin{array}{c}
        b_{1j} \\
        b_{2j} \\
        \cdots \\
        b_{sj}
    \end{array}
\right)=a_{i1}b_{1j}+a_{i2}b_{2j}+\cdots+a_{is}b_{sj}=\sum\limits_{k=1}^sa_{ik}b_{kj}=c_{ij}$。

从而$AB=C$的$c_{ij}$就是$A$的第$i$行与$B$的$j$列的乘积。

\textcolor{orange}{注意:}只有左矩阵的列数等于右矩阵的行数才能相乘。

只有$AB$都是方阵的时候才能$AB$与$BA$。

矩阵的左乘与右乘不一定相等,即$AB\neq BA$。

若方阵$AB$乘积满足$AB=BA$,则表示其是可交换的。

$A\neq O$,$B\neq O$,但是不能推出$AB\neq O$或$BA\neq O$。

$AB=O$不能推出$A=O$或$B=O$。

$A(X-Y)=O$当$A\neq O$也不能推出$X=Y$。

\begin{itemize}
    \item $(AB)C=A(BC)$。
    \item $\lambda(AB)=(\lambda A)B=A(\lambda B)$。
    \item $A(B+C)=AB+AC$。
    \item $(B+C)A=BA+CA$。
    \item $EA=AE=A$。
\end{itemize}

$\lambda E$称为纯量阵,$(\lambda E_n)A_n=\lambda A_n=A_n(\lambda E_n)$。

若$A_{m\times s}$,$B_{s\times n}=(\beta_1,\cdots,\beta_s)$,其中$\beta$为$n$行的列矩阵,则:

$AB=A(\beta_1,\cdots,\beta_s)=(A\beta_1,\cdots,A\beta_n)$。

\subsection{矩阵幂}

只有方阵才能连乘,从而只有方阵才有幂。

若$A$是$n$阶方阵,所以:

$$A^1=A\text{,}A^2=A^1A^1\text{,}\cdots\text{,}A^{k+1}=A^kA^1$$

\begin{itemize}
    \item $A^kA^l=A^{k+l}$。
    \item $(A^k)^l=A^{kl}$。
\end{itemize}

因为矩阵乘法一般不满足交换率,所以$(AB)^k\neq A^kB^k$。只有$AB$可交换时才相等。

若$A\neq 0$不能推出$A^k\neq 0$,如:

$A=\left(
    \begin{array}{cc}
        0 & 2 \\
        0 & 0
    \end{array}
\right)\neq 0$。$A^2=\left(
    \begin{array}{cc}
        0 & 2 \\
        0 & 0
    \end{array}
\right)\left(
    \begin{array}{cc}
        0 & 2 \\
        0 & 0
    \end{array}
\right)=\left(
    \begin{array}{cc}
        0 & 0 \\
        0 & 0
    \end{array}
\right)=O$。

$A=\left(
    \begin{array}{ccc}
        0 & 1 & 1 \\
        0 & 0 & 1 \\
        0 & 0 & 0
    \end{array}
\right)$,$A^3=O$。

矩阵幂可以同普通多项式进行处理。

如$f(x)=a_nx^n+\cdots+a_1x+n$,对于$A$就是$f(A)=a_nA^n+\cdots+a_1A+a_nE$。

$f(A)=A^2-A-6E=(A+2E)(A-3E)$。

\subsection{矩阵转置}

把矩阵$A$的行换成同序数的列就得到一个新矩阵,就是$A$的转置矩阵$A^T$。若$A$为$m\times n$,则$A^T$为$n\times m$。

\begin{itemize}
    \item $(A^T)^T=A$。
    \item $(A+B)^T=A^T+B^T$。
    \item $(\lambda A)^T=\lambda A^T$。
    \item $(AB)^T=B^TA^T$。
\end{itemize}

对称矩阵或对称阵:元素以对角线为对称轴对应相等,$A=A^T$。

\subsection{方阵行列式}

由$n$阶方阵$A$的元素所构成的行列式称为矩阵$A$的行列式,记为$\textrm{det}\,A$或$\vert A\vert$。

\subsection{线性方程组与矩阵}

\begin{itemize}
    \item $\vert A^T\vert=\vert A\vert$。
    \item $\vert\lambda A\vert=\lambda^n\vert A\vert$。
    \item $\vert AB\vert=\vert A\vert\cdot\vert B\vert=\vert BA\vert$。
\end{itemize}

伴随矩阵或伴随阵:行列式$\vert A\vert$各个元素的代数余子式$A_{ij}$构成的矩阵。

$$A^*=\left(
    \begin{array}{cccc}
        A_{11} & A_{21} & \cdots & A_{n1} \\
        A_{12} & A_{22} & \cdots & A_{n2} \\
        \vdots & \vdots & \ddots & \vdots \\
        A_{1n} & A_{2n} & \cdots & A_{nn}
    \end{array}
\right)$$

其中$AA^*=A^*A=\vert A\vert E$。

\section{线性方程组}

矩阵是根据线性方程组得到。

\subsection{线性方程组与矩阵}

\begin{multicols}{2}
    
    $\begin{cases}
        a_{11}x_1+\cdots+a_{1n}x_n=0 \\
        \cdots \\
        a_{m1}x_1+\cdots+a_{mn}x_n=0
    \end{cases}$ \medskip
    
    $n$元齐次线性方程组。

    $\begin{cases}
        a_{11}x_1+\cdots+a_{1n}x_n=b_1 \\
        \cdots \\
        a_{m1}x_1+\cdots+a_{mn}x_n=b_n
    \end{cases}$ \medskip
    
    $n$元非齐次线性方程组。

\end{multicols}

对于齐次方程,$x_1=\cdots=x_n=0$一定是其解,称为其零解,若有一组不全为零的解,则称为其非零解。其一定有零解,但是不一定有非零解。

对于非齐次方程,只有$b_1\cdots b_n$不全为零才是。

令系数矩阵$A_{m\times n}=\left(
    \begin{array}{ccc}
        a_{11} & \cdots & a_{1n} \\
        \cdots \\
        a_{m1} & \cdots & a_{mn}
    \end{array}
\right)$,未知数矩阵$X_{n\times 1}=\left(
    \begin{array}{c}
        x_1 \\
        \cdots \\
        x_n
    \end{array}
\right)$,常数项矩阵$b_{m\times 1}=\left(
    \begin{array}{c}
        b_1 \\
        \cdots \\
        b_m
    \end{array}
\right)$,增广矩阵$B_{m\times(n+1)}=\left(
    \begin{array}{cccc}
        a_{11} & \cdots & a_{1n} & b_1\\
        \cdots \\
        a_{m1} & \cdots & a_{mn} & b_n
    \end{array}
\right)$。



所以$AX=\left(
    \begin{array}{c}
        a_11x_1+\cdots+a_{1n}x_n \\
        \cdots \\
        a_{m1}x_1+\cdots+a_{mn}x_n
    \end{array}
\right)$。

从而$AX=b$等价于$\begin{cases}
    a_{11}x_1+\cdots+a_{1n}x_n=b_1 \\
    \cdots \\
    a_{m1}x_1+\cdots+a_{mn}x_n=b_n
\end{cases}$,当$b=O$就是齐次线性方程。

从而矩阵可以简单表示线性方程。

\subsection{线性方程组的解}

对于一元一次线性方程:$ax=b$:

\begin{itemize}
    \item 当$a\neq 0$时,可以解得$x=\dfrac{b}{a}$。
    \item 当$a=0$时,若$b\neq 0$时,无解,若$b=0$时,无数解。
\end{itemize}

当推广到多元一次线性方程组:$AX=b$,如何求出$X$这一系列的$x$的解?

从数学逻辑上看,已知多元一次方程,有$m$个约束方程,有$n$个未知数,假定$m\leqslant n$。

当$m<n$时,就代表有更多的未知变量不能被方程约束,从而有$n-m$个自由变量,所以就是无数解,解组中其他解可以由自由变量来表示。

当$m=n$时代表约束与变量数量相等,此时又要分三种情况。

当所有的约束条件其中存在线性相关,即一部分约束条件可以由其他约束表示,则代表这部分约束条件是没用的,实际上的约束条件变少,从而情况等于$m<n$,结果是无数解。

当所有的约束条件不存在线性相关,但是一部分约束条件互相矛盾,则约束条件下就无法解出解,从而结果是无实数解。

当所有的约束条件不存在线性相关,且相互之间不存在矛盾情况,这时候才会解出一个实数解,从而结果是有唯一实解。

若使用矩阵来解决线性方程组的问题,其系数矩阵$A_{m\times n}$。

对于$A\neq O$,则$AX=b$,若存在一个矩阵$B_{n\times n}$类似$\dfrac{1}{a}$,使得$BAX=Bb$,解得$EX=X=Bb$,这个$B$就是$A$的逆矩阵。

对于$A=O$即不可逆,需要判断$b$是否为0,若不是则无实数解,若是则无穷解,这种判断需要用到增广矩阵,需要用到矩阵的秩判断。

\section{逆矩阵}

逆矩阵类比倒数,若对于$n$阶矩阵$A$,有一个$n$阶矩阵$B$,使得$AB=BA=E$,则$A$可逆,$B$是$A$的逆矩阵也称为逆阵,且逆矩阵唯一,记为$B=A^{-1}$。

\textcolor{aqua}{\textbf{定理:}}若矩阵$A$可逆,则$\vert A\vert\neq 0$。

证明:若$A$可逆,则$AA^{-1}=E$,所以$\vert A\vert\cdot\vert A^{-1}\vert=\vert E\vert=1$,$\vert A\vert\neq 0$。

可以类比普通数字,若$a$有一个倒数$\dfrac{1}{a}$,则$a\neq 0$,否则无法倒。

\textcolor{aqua}{\textbf{定理:}}若$\vert A\vert\neq 0$,则$A$可逆,且$A^{-1}=\dfrac{1}{\vert A\vert}A^*$。

\end{document}
