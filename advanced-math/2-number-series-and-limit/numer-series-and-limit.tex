\documentclass[UTF8]{ctexart}
\usepackage{color}
% 使用颜色
\definecolor{orange}{RGB}{255,127,0} 
\definecolor{violet}{RGB}{192,0,255} 
\definecolor{aqua}{RGB}{0,255,255} 
\usepackage{geometry}
\setcounter{tocdepth}{5}
\setcounter{secnumdepth}{5}
% 设置四级目录与标题
\geometry{papersize={21cm,29.7cm}}
% 默认大小为A4
\geometry{left=3.18cm,right=3.18cm,top=2.54cm,bottom=2.54cm}
% 默认页边距为1英尺与1.25英尺
\usepackage{indentfirst}
\setlength{\parindent}{2.45em}
% 首行缩进2个中文字符
\usepackage{amssymb}
% 因为所以与其他数学拓展
\usepackage{amsmath}
% 数学公式
\usepackage{setspace}
\renewcommand{\baselinestretch}{1.5}
% 1.5倍行距
\usepackage{pifont}
% 圆圈序号
\author{Didnelpsun}
\title{数列与极限}
\date{}
\begin{document}
\maketitle
\thispagestyle{empty}
\tableofcontents
\thispagestyle{empty}
\newpage
\pagestyle{plain}
\setcounter{page}{1}

极限就是一个无限逼近某个值的过程。如$\dfrac{n}{n+1}$这个分式在$n$无限增大的时候会无限逼近1,这个1叫做极限值,所以写成$\lim_{n\to\infty}\dfrac{n}{n+1}=1$。

所以从另一个方面更精确的指出一个数$N>0$,使得数列下标大于$N$的项与极限值之间的距离始终保持在$(0,\epsilon)$之间,即$\dfrac{1}{n+1}<\epsilon$,即$n>\dfrac{1}{\epsilon}-1$,所以任意正数都能得到从$N>\dfrac{1}{\epsilon}-1$项开始之后都有$\left\vert\dfrac{n}{n+1}-1\right\vert<\epsilon$。

即无论给出多么小的$\epsilon$,总可以找到一项从该项之后函数值与极限值之间的差小于$\epsilon$,即更接近这个极限值而不式其他任何值,所以该数列趋向于极限值。

所以以后的基本流程,令$x_n$为通项,$a$为极限值,$\epsilon$为任意正数。

\begin{enumerate}
    \item 写出$\vert x_n-a|<\epsilon$。
    \item 反解出项数$n<g(\epsilon)$。
    \item 取$N=[g(\epsilon)]+1$,所以令$n>N$就可以证明。
\end{enumerate}

\textbf{例题1:}用定义证明$\lim_{x\to\infty}\left[1+\dfrac{(-1)^n}{n}\right]=1$

证明:

\ding{172}计算距离:$\left\vert 1+\dfrac{(-1)^n}{n}-1\right\vert=\left\vert\dfrac{(-1)^n}{n}\right\vert<\epsilon$。

\ding{173}解得到:$\dfrac{1}{n}<\epsilon$,反解为$n>\dfrac{1}{\epsilon}$。

\ding{174}取整:$N=\left[\dfrac{1}{\epsilon}\right]+1$。

$\therefore\forall\epsilon>0$,当$n>N$时,就有$n>\dfrac{1}{\epsilon}$,使得$\left\vert 1+\dfrac{(-1)^n}{n}-1\right\vert=\left\vert\dfrac{(-1)^n}{n}\right\vert<\epsilon$。

$\therefore$证明完毕。

\textbf{例题2:}用定义证明$\lim_{n\to\infty}q^n=0$($q$为常数且$\vert q\vert<1$)。

证明:

\ding{172}$\vert q^n-0\vert<\epsilon$。

\ding{173}$\vert q^n\vert<\epsilon$,取对数进行反解$n\ln\vert q\vert<\ln\epsilon$,又因为$\vert q\vert<1$,所以$\ln\vert q\vert<0$,所以得到$n>\dfrac{\ln\epsilon}{\ln\vert q\vert}$。(若$\epsilon>1$则$n$就是负数,这样条件必然成立)

\ding{174}取$N=\left[\dfrac{\ln\epsilon}{\ln\vert q\vert}\right]+1$。

$\therefore$当$n>N$时,必然$n>\dfrac{\ln\epsilon}{\ln\vert q\vert}$,有$\vert q^n-0\vert<\epsilon$。

故$\lim_{n\to\infty}q^n=0$。

\section{定义}

通过定义可以证明极限。

\textcolor{violet}{\textbf{定义:}}设$\{x_n\}$为一数列,若存在常数$a$,对于不论任意小的$\epsilon>0$,总存在正整数$N$,使$n>N$时,$\vert x_n-a\vert<\epsilon$恒成立,则常数$a$为数列$\{x_n\}$的极限,或$\{x_n\}$收敛于$a$,记为:$\lim_{x\to\infty}x_n=a$或$x_n\to a(n\to\infty)$。

常用语言($\epsilon-N$语言):$\lim_{x\to\infty}x_n=a\Leftrightarrow\forall\epsilon>0,\exists N\in N_+$,当$n>N$时,恒有$\vert x_n-a\vert<\epsilon$。

如果不存在该数$a$,则称数列$x_n$发散。

\subsection{数列绝对值}

\textbf{例题3:}证明若$\lim_{x\to\infty}a_n=A$,则$\lim_{x\to\infty}\vert a_n\vert=\vert A\vert$。

因为$\lim_{n\to\infty}a_n=A\Leftrightarrow\forall\epsilon>0,\exists N>0,\text{当}n>N$,恒有$\vert a_n-A\vert<\epsilon$。

又由重要不等式$\vert\vert a\vert-\vert b\vert\vert\leqslant\vert a-b\vert$,所以$\vert\vert a_n-\vert A\vert\vert\leqslant\epsilon$。

所以恒成立,证明完毕。

从这个题推出:$\lim_{n\to\infty}a_n=0\Leftrightarrow\lim_{n\to\infty}\vert a_n\vert=0$。所以如果我们以后需要证明某一数列极限为0,可以证明数列绝对值极限0,而数列绝对值绝对时大于等于0的,所以由夹逼准则,其中小的一头已经固定为0了,所以只用找另一个偏大的数列夹逼所证明数列就可以了。

\subsection{子数列}

\textcolor{violet}{\textbf{定义:}}从数列${a_n}:a_1,a_2,\cdots,a_n,\cdots$中选取无穷多项并按原来顺序组成的新数列就称为原数列的子列,记为$\{a_{n_k}\}:a_{n_1},a_{n_2},\cdots,a_{n_k},\cdots$。

若$n_k$分别取奇数和偶数,则得到奇数项数列与偶数项数列。

\textcolor{aqua}{\textbf{定理:}}若数列$\{a_n\}$收敛,则其任何子列$\{a_{n_k}\}$也收敛,且极限值相同。

所以对于其变式我们用到更多:

\begin{enumerate}
    \item 若一个数列$\{a_n\}$能找到一个发散的子列,那该数列发散。
    \item 若一个数列$\{a_n\}$能找到两个极限值不同的收敛子列,那么这个数列发散。
    \item 若一个数列$\{a_n\}$,则其奇数子列与偶数子列都收敛于同一个值。
\end{enumerate}

例如对于数列$\{(-1)^n\}$,能找到其奇数子列收敛于-1,偶数子列收敛于1,所以收敛值不同,原数列发散。

\section{性质}
\subsection{唯一性}

\textcolor{violet}{\textbf{定义:}}若数列$\{x_n\}$收敛于$a$,则$a$是唯一的。

\subsection{有界性}

\textcolor{violet}{\textbf{定义:}}若数列$\{x_n\}$极限存在,则数列有界。

\subsection{保号性}

较重要。也称为脱帽法。

\textcolor{violet}{\textbf{定义:}}若数列$\{x_n\}$存在极限$\lim_{n\to\infty}a_n=a\neq 0$,则存在正整数$N$,当$n>N$时$a_n$都与$a$同号。

简单来说,就是极限大于0,后面一部分数列大于0,极限小于0,后面一部分数列小于0。

推论,戴帽法:若数列$\{a_n\}$从某项开始$a_n\geqslant b$,且$\lim_{n\to\infty}a_n=a$,则$a\geqslant b$。这里一定要带等号。

\section{运算规则}

若$\lim_{n\to\infty}x_n=a$,$\lim_{n\to\infty}y_n=b$则:

\begin{enumerate}
    \item $\lim_{n\to\infty}x_n\pm y_n=a\pm b$。
\end{enumerate}

\textbf{例题4:}若$\lim_{n\to\infty}(a_n+b_n)=1$且$\lim_{n\to\infty}(a_n-b_n)=3$,计算$\lim_{n\to\infty}a_n$与$\lim_{n\to\infty}b_n$。

首先是不能通过运算法则第一条将两个条件直接加减的,因为不能保证两个极限是否都存在。

所以必须先令$u_n=a_n+b_n$,$v_n=a_n-b_n$,所以$\lim_{n\to\infty}u_n=1$,$\lim_{n\to\infty}v_n=3$。

因为这两个极限都存在,所以可以进行运算。

相加得到$\lim_{n\to\infty}(u_n+v_n)=2\lim_{n\to\infty}a_n=4$。

所以得到$\lim_{n\to\infty}a_n=2$。同理$\lim_{n\to\infty}(u_n-v_n)$得到$\lim_{n\to\infty}b_n=-1$。

\section{夹逼准则}

\textcolor{violet}{\textbf{定义:}}若数列$\{x_n\}$,$\{y_n\}$,$\{z_n\}$满足以下条件:

$y_n\leqslant x_n\leqslant z_n(n=1,2,3,\cdots)$;$\lim_{n\to\infty}y_n=a,\lim_{n\to\infty}z_n=a$。

则数列$x_n$极限存在且$\lim_{n\to\infty}x_n=a$。

\textbf{例题5:}求极限$\lim_{n\to\infty}\left(\dfrac{n}{n^2+1}+\dfrac{n}{n^2+2}+\cdots+\dfrac{n}{n^2+n}\right)$。

使用夹逼准则:$\dfrac{n^2}{n^2+n}<\sum_{i=1}^n\dfrac{n}{n^2+i}<\dfrac{n^2}{n^2+1}$

又$\lim_{n\to\infty}\dfrac{n^2}{n^2+1}=\lim_{n\to\infty}\dfrac{n^2/n^2}{n^2/n^2+1/n^2}=\lim_{n\to\infty}\dfrac{1}{1+\dfrac{1}{n^2}}=1$

且$\lim_{n\to\infty}\dfrac{n^2}{n^2+n}=\lim_{n\to\infty}\dfrac{1}{1+\dfrac{1}{n}}=1$。

由夹逼准则,原式的极限为1。

对于分式的放缩主要在于分母的放缩,不变分子,分母变小原式变大,分母变大原式变小。然后分子分母除以最高项得到逼向0的极限。

\section{单调有界准则}

也称为魏尔施特拉斯准则,该部分最重要。

\textcolor{violet}{\textbf{定义:}}单调有界数列必有极限,即若$\{x_n\}$单调增加(减少)且有上界(下界),则极限存在。

该部分需要证明两个地方:

\begin{enumerate}
    \item 数列单调:$x_{n+1}-x_n$与0的关系,或$\dfrac{x_{n+1}}{x_n}$与1的关系。
    \item 有界:$\vert x_n\vert\leqslant M$是否存在。
\end{enumerate}

见到\textcolor{orange}{递推式(迭代式)}$a_{n+1}=f(a_n)$,一般都要用单调有界准则。单调性通过减或除进行计算,有界性通过不等式来计算。

\textbf{例题6:}已知$a_1=a>0$,证明$a_{n+1}=\dfrac{1}{2}\left(a_n+\dfrac{2}{a_n}\right)$的极限存在并求出。

$\because a_1=a>0$,且递推式中没有负数与减的操作,所以$a_n>0$。

由重要不等式$\dfrac{a+b}{2}\geqslant\sqrt{ab}$,所以$a_{n+1}=\dfrac{1}{2}\left(a_n+\dfrac{2}{a_n}\right)\geqslant\sqrt{a_n\cdot\dfrac{2}{a_n}})=\sqrt{2}$

$\therefore$数列$\{a_n\}$有下界$\sqrt{2}$。

又$a_{n+1}-a_n=\dfrac{2-a_n^2}{2a_n}$,且由上面证明已知$a_n^2\geqslant\sqrt{2}$,所以该式子小于等于0。

$\therefore a_{n+1}\leqslant a_n$,得到数列单调减少。

由单调有界准则,$\lim_{n\to\infty}a_n$存在并记为$A$。

将$A$代入递推式并两边求极限:$A=\dfrac{1}{2}(A+\dfrac{2}{A})$,得到$A=\pm\sqrt{2}$。

又因为保号性,数列下界为$\sqrt{2}$,所以$A=\sqrt{2}$。

\textbf{例题7:}求证$x_{n+1}=\sin x_n$极限存在,$0<x_1<\pi$。

由三角函数中的不等式$\sin x<x$。

\ding{172}当$n=1$,$\because 0<x_1<\pi$,$\therefore 0<\sin x_1<1$,$\therefore 0<x_2=sin x_1<x<\pi$。

\ding{173}假设$0<x_n=\sin x_{n-1}<\pi$。

\ding{174}$\therefore 0<x_{n+1}=\sin x_n<x_n<\pi$。

\ding{175}故$\{x_n\}\searrow$且有下界0。

$\therefore\lim_{n\to\infty}x_n$存在,并记为$A$。

对两边取极限:$A=\sin A$,所以$A=0$。

$\therefore\lim_{n\to\infty}x_n=0$。

\textbf{例题8:}证明$a_n=\dfrac{1}{1^2}+\dfrac{1}{2^2}+\cdots+\dfrac{1}{n^2}$存在极限。

因为是递推式,所以一般使用单调有界准则。

\ding{172}$a_{n+1}=\dfrac{1}{1^2}+\dfrac{1}{2^2}+\cdots+\dfrac{1}{n^2}+\dfrac{1}{(n+1)^2}$。

$\Rightarrow a_{n+1}-a_n=\dfrac{1}{(n+2)^2}>0\Rightarrow\{a_n\}\nearrow$

$
\begin{aligned}
    \text{\ding{173}}a_n & =\dfrac{1}{1\cdot 1}+\dfrac{1}{2\cdot 2}+\cdots+\dfrac{1}{n\cdot n} \\
    & \text{裂项相消} \\
    < & 1+\dfrac{1}{1\cdot 2}+\cdots+\dfrac{1}{(n-1)\cdot(n)} \\
    = & 1+(1-\dfrac{1}{2})+(\dfrac{1}{2}-\dfrac{1}{3})+\cdots+(\dfrac{1}{n-1}-\dfrac{1}{n}) \\
    = & 2-\dfrac{1}{n} \\
    < & 2 \text{ (上界)}
\end{aligned}
$

单调增且有上界,所以必然有极限。

\section{直接计算法}

通过单调有界准则不一定能简单计算出结果,因为如果要计算极限,就无法使用,对于迭代式方程也可以直接计算,不过该种类型题目难度较大。

\textbf{例题9:}数列$\{a_n\}$满足$a_0=0,a_1=1,2a_{n+1}=a_n+a_{n-1},n=1,2,\cdots$。计算$\lim_{n\to\infty}a_n$。

首先看题目,给出的递推式设计到二阶递推,即存在三个数列变量,所以我们必须先求出对应的数列表达式。因为这个表达式涉及三个变量,所以尝试对其进行变型:

$
\begin{aligned}
    a_{n+1}-a_n & =\dfrac{a_{n-1}-a_n}{2} \\
    & =\left(-\dfrac{1}{2}\right)(a_n-a_{n-1}) \\
    & =\left(-\dfrac{1}{2}\right)^2(a_{n-1}-a_{n-2}) \\
    & =\cdots \\
    & =\left(-\dfrac{1}{2}\right)^n(a_1-a_0) \\ 
    & = \left(-\dfrac{1}{2}\right)^n
\end{aligned}
$

然后得到了$a_{n+1}-a_n=\left(-\dfrac{1}{2}\right)^n$,而需要求极限,所以使用列项相消法的逆运算:

$
\begin{aligned}
    a_n & = (a_n-a_{n-1})+(a_{n-1}-a_{n-2})+\cdots+(a_1-a_0)+a_0 \\
     & = \left(-\dfrac{1}{2}\right)^{n-1} + \left(-\dfrac{1}{2}\right)^{n-2} + \cdots + \left(-\dfrac{1}{2}\right)^0 \\
     & = \dfrac{1\cdot\left(1-\left(-\dfrac{1}{2}\right)^n\right)}{1-\left(-\dfrac{1}{2}\right)} \\
     & = \dfrac{2}{3}\left[1-\left(-\dfrac{1}{2}\right)^n\right] \\
    \lim_{n\to\infty}a_n & =\dfrac{2}{3}
\end{aligned}
$

\end{document}
