\documentclass[UTF8]{ctexart}
\usepackage{color}
% 使用颜色
\usepackage{geometry}
\setcounter{tocdepth}{5}
\setcounter{secnumdepth}{5}
% 设置四级目录与标题
\geometry{papersize={21cm,29.7cm}}
% 默认大小为A4
\geometry{left=3.18cm,right=3.18cm,top=2.54cm,bottom=2.54cm}
% 默认页边距为1英尺与1.25英尺
\usepackage{indentfirst}
\setlength{\parindent}{2.45em}
% 首行缩进2个中文字符
\usepackage{amssymb}
% 因为所以与其他数学拓展
\usepackage{amsmath}
% 数学公式
\usepackage{setspace}
\renewcommand{\baselinestretch}{1.5}
% 1.5倍行距
\usepackage{pifont}
% 圆圈序号
\author{Didnelpsun}
\title{数列与极限}
\begin{document}
\maketitle
\thispagestyle{empty}
\tableofcontents
\thispagestyle{empty}
\newpage
\pagestyle{plain}
\setcounter{page}{1}

极限就是一个无限逼近某个值的过程。如$\dfrac{n}{n+1}$这个分式在$n$无限增大的时候会无限逼近1,这个1叫做极限值,所以写成$\lim_{n\to\infty}\dfrac{n}{n+1}=1$。

所以从另一个方面更精确的指出一个数$N>0$,使得数列下标大于$N$的项与极限值之间的距离始终保持在$(0,\epsilon)$之间,即$\dfrac{1}{n+1}<\epsilon$,即$n>\dfrac{1}{\epsilon}-1$,所以任意正数都能得到从$N>\dfrac{1}{\epsilon}-1$项开始之后都有$\left\vert\dfrac{n}{n+1}-1\right\vert<\epsilon$。

即无论给出多么小的$\epsilon$,总可以找到一项从该项之后函数值与极限值之间的差小于$\epsilon$,即更接近这个极限值而不式其他任何值,所以该数列趋向于极限值。

所以以后的基本流程,令$x_n$为通项,$a$为极限值,$\epsilon$为任意正数。

\begin{enumerate}
    \item 写出$\vert x_n-a|<\epsilon$。
    \item 反解出项数$n<g(\epsilon)$。
    \item 取$N=[g(\epsilon)]+1$,所以令$n>N$就可以证明。
\end{enumerate}

\textbf{例题1:}用定义证明$\lim_{x\to\infty}\left[1+\dfrac{(-1)^n}{n}\right]=1$

证明:

\ding{172}计算距离:$\left\vert 1+\dfrac{(-1)^n}{n}-1\right\vert=\left\vert\dfrac{(-1)^n}{n}\right\vert<\epsilon$。

\ding{173}解得到:$\dfrac{1}{n}<\epsilon$,反解为$n>\dfrac{1}{\epsilon}$。

\ding{174}取整:$N=\left[\dfrac{1}{\epsilon}\right]+1$。

$\therefore\forall\epsilon>0$,当$n>N$时,就有$n>\dfrac{1}{\epsilon}$,使得$\left\vert 1+\dfrac{(-1)^n}{n}-1\right\vert=\left\vert\dfrac{(-1)^n}{n}\right\vert<\epsilon$。

$\therefore$证明完毕。

\textbf{例题2:}用定义证明$\lim_{n\to\infty}q^n=0$($q$为常数且$\vert q\vert<1$)。

证明:

\ding{172}$\vert q^n-0\vert<\epsilon$。

\ding{173}$\vert q^n\vert<\epsilon$,取对数进行反解$n\ln\vert q\vert<\ln\epsilon$,又因为$\vert q\vert<1$,所以$\ln\vert q\vert<0$,所以得到$n>\dfrac{\ln\epsilon}{\ln\vert q\vert}$。(若$\epsilon>1$则$n$就是负数,这样条件必然成立)

\ding{174}取$N=\left[\dfrac{\ln\epsilon}{\ln\vert q\vert}\right]+1$。

$\therefore$当$n>N$时,必然$n>\dfrac{\ln\epsilon}{\ln\vert q\vert}$,有$\vert q^n-0\vert<\epsilon$。

故$\lim_{n\to\infty}q^n=0$。

\section{定义}

通过定义可以证明极限。

设$\{x_n\}$为一数列,若存在常数$a$,对于不论任意小的$\epsilon>0$,总存在正整数$N$,使$n>N$时,$\vert x_n-a\vert<\epsilon$恒成立,则常数$a$为数列$\{x_n\}$的极限,或$\{x_n\}$收敛于$a$,记为:$\lim_{x\to\infty}x_n=a$或$x_n\to a(n\to\infty)$。

常用语言($\epsilon-N$语言):$\lim_{x\to\infty}x_n=a\Leftrightarrow\forall\epsilon>0,\exists N\in N_+$,当$n>N$时,恒有$\vert x_n-a\vert<\epsilon$。

如果不存在该数$a$,则称数列$x_n$发散。

\subsection{数列与数列绝对值}

\textbf{例题3:}证明若$\lim_{x\to\infty}a_n=A$,则$\lim_{x\to\infty}\vert a_n\vert=\vert A\vert$。

因为$\lim_{n\to\infty}a_n=A\Leftrightarrow\forall\epsilon>0,\exists N>0,\text{当}n>N$,恒有$\vert a_n-A\vert<\epsilon$。

又由重要不等式$\vert\vert a\vert-\vert b\vert\vert\leqslant\vert a-b\vert$,所以$\vert\vert a_n-\vert A\vert\vert\leqslant\epsilon$。

所以恒成立,证明完毕。

从这个题推出:$\lim_{n\to\infty}a_n=0\Leftrightarrow\lim_{n\to\infty}\vert a_n\vert=0$。所以如果我们以后需要证明某一数列极限为0,可以证明数列绝对值极限0,而数列绝对值绝对时大于等于0的,所以由夹逼准则,其中小的一头已经固定为0了,所以只用找另一个偏大的数列夹逼所证明数列就可以了。

\subsection{数列与子数列}

\section{性质}
\subsection{唯一性}
\subsection{有界性}
\subsection{保号性}
\section{运算规则}
\section{夹逼准则}
\section{单调有界准则}

该部分最重要。
\end{document}
