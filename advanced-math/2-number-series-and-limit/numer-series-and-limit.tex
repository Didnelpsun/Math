\documentclass[UTF8]{ctexart}
\usepackage{color}
% 使用颜色
\definecolor{orange}{RGB}{255,127,0} 
\definecolor{violet}{RGB}{192,0,255} 
\definecolor{aqua}{RGB}{0,255,255} 
\usepackage{geometry}
\setcounter{tocdepth}{5}
\setcounter{secnumdepth}{5}
% 设置四级目录与标题
\geometry{papersize={21cm,29.7cm}}
% 默认大小为A4
\geometry{left=3.18cm,right=3.18cm,top=2.54cm,bottom=2.54cm}
% 默认页边距为1英尺与1.25英尺
\usepackage{indentfirst}
\setlength{\parindent}{2.45em}
% 首行缩进2个中文字符
\usepackage{amssymb}
% 因为所以与其他数学拓展
\usepackage{amsmath}
% 数学公式
\usepackage{setspace}
\renewcommand{\baselinestretch}{1.5}
% 1.5倍行距
\usepackage{pifont}
% 圆圈序号
\author{Didnelpsun}
\title{数列与极限}
\date{}
\begin{document}
\maketitle
\thispagestyle{empty}
\tableofcontents
\thispagestyle{empty}
\newpage
\pagestyle{plain}
\setcounter{page}{1}


\section{夹逼准则}



\section{单调有界准则}



\section{直接计算法}

通过单调有界准则不一定能简单计算出结果,因为如果要计算极限,就无法使用,对于迭代式方程也可以直接计算,不过该种类型题目难度较大。

\textbf{例题9:}数列$\{a_n\}$满足$a_0=0,a_1=1,2a_{n+1}=a_n+a_{n-1},n=1,2,\cdots$。计算$\lim_{n\to\infty}a_n$。

首先看题目,给出的递推式设计到二阶递推,即存在三个数列变量,所以我们必须先求出对应的数列表达式。因为这个表达式涉及三个变量,所以尝试对其进行变型:

$
\begin{aligned}
    a_{n+1}-a_n & =\dfrac{a_{n-1}-a_n}{2} \\
    & =\left(-\dfrac{1}{2}\right)(a_n-a_{n-1}) \\
    & =\left(-\dfrac{1}{2}\right)^2(a_{n-1}-a_{n-2}) \\
    & =\cdots \\
    & =\left(-\dfrac{1}{2}\right)^n(a_1-a_0) \\ 
    & = \left(-\dfrac{1}{2}\right)^n
\end{aligned}
$

然后得到了$a_{n+1}-a_n=\left(-\dfrac{1}{2}\right)^n$,而需要求极限,所以使用列项相消法的逆运算:

$
\begin{aligned}
    a_n & = (a_n-a_{n-1})+(a_{n-1}-a_{n-2})+\cdots+(a_1-a_0)+a_0 \\
     & = \left(-\dfrac{1}{2}\right)^{n-1} + \left(-\dfrac{1}{2}\right)^{n-2} + \cdots + \left(-\dfrac{1}{2}\right)^0 \\
     & = \dfrac{1\cdot\left(1-\left(-\dfrac{1}{2}\right)^n\right)}{1-\left(-\dfrac{1}{2}\right)} \\
     & = \dfrac{2}{3}\left[1-\left(-\dfrac{1}{2}\right)^n\right] \\
    \lim_{n\to\infty}a_n & =\dfrac{2}{3}
\end{aligned}
$

\end{document}
