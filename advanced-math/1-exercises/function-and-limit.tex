\documentclass[UTF8]{ctexart}
% UTF8编码,ctexart现实中文
\usepackage{color}
% 使用颜色
\usepackage{geometry}
\setcounter{tocdepth}{4}
\setcounter{secnumdepth}{4}
% 设置四级目录与标题
\geometry{papersize={21cm,29.7cm}}
% 默认大小为A4
\geometry{left=3.18cm,right=3.18cm,top=2.54cm,bottom=2.54cm}
% 默认页边距为1英尺与1.25英尺
\usepackage{indentfirst}
\setlength{\parindent}{2.45em}
% 首行缩进2个中文字符
\usepackage{setspace}
\renewcommand{\baselinestretch}{1.5}
% 1.5倍行距
\usepackage{amssymb}
% 因为所以
\usepackage{amsmath}
% 数学公式
\author{Didnelpsun}
\title{函数与极限练习题}
\date{}
\begin{document}
\maketitle
\thispagestyle{empty}
\tableofcontents
\thispagestyle{empty}
\newpage
\pagestyle{plain}
\setcounter{page}{1}

\section{求极限值}

\subsection{等价无穷小替换}

当看到复杂的式子,且当$x\to 0$时,使用等价无穷小进行替换。

求$\lim_{x\to 0}\dfrac{(e^{x^2}-1)(\sqrt{1+x}-\sqrt{1-x})}{[\ln(1-x)+\ln(1+x)]\sin\dfrac{x}{x+1}}$。

在明显的部分由等价无穷小的式子得到:$e^{x^2}-1\sim x^2$,$\sin\dfrac{x}{x+1}=\dfrac{x}{x+1}$。

并注意在积或商的时候不能把对应的部分替换为0,如分母部分的$[\ln(1-x)+\ln(1+x)]$就无法使用$\ln(1+x)\sim x$替换为$-x+x$,这样底就是0了,无法求得最后的极限。

这时可以尝试变形,如对数函数相加等于对数函数内部式子相乘:$\ln(1-x)+\ln(1+x)=\ln(1-x^2)\sim-x^2$。

对于无法直接得出变换式子的,可以对对应参数进行凑,以达到目标的可替换的等价无穷小。

对分子部分的$\sqrt{1+x}-\sqrt{1-x}$无法使用$(1+x)^a-1\sim ax$替换为$(1+x)^{\frac{1}{2}}-1-[(1-x)^{\frac{1}{2}}-1]\sim\dfrac{1}{2}x+\dfrac{1}{2}x=x$。

将所有替换的无穷小代入原式:$=\lim_{x\to 0}\dfrac{x^2\cdot x}{-x^2\cdot\dfrac{x}{1+x}}=\lim_{x\to 0}-(1+x)=-1$。

\subsection{幂指类型}

当出现$f(x)^{g(x)}$的类似幂函数与指数函数类型的式子,需要使用$u^v=e^{v\ln u}$。

求$\lim_{x\to\infty}$




\end{document}
