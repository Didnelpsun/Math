\documentclass[UTF8]{ctexart}
\usepackage{color}
% 使用颜色
\definecolor{orange}{RGB}{255,127,0} 
\definecolor{violet}{RGB}{192,0,255} 
\definecolor{aqua}{RGB}{0,255,255} 
\usepackage{geometry}
\setcounter{tocdepth}{5}
\setcounter{secnumdepth}{5}
% 设置四级目录与标题
\geometry{papersize={21cm,29.7cm}}
% 默认大小为A4
\geometry{left=3.18cm,right=3.18cm,top=2.54cm,bottom=2.54cm}
% 默认页边距为1英尺与1.25英尺
\usepackage{indentfirst}
\setlength{\parindent}{2.45em}
% 首行缩进2个中文字符
\usepackage{amssymb}
% 因为所以与其他数学拓展
\usepackage{amsmath}
% 数学公式
\usepackage{setspace}
\renewcommand{\baselinestretch}{1.5}
% 1.5倍行距
\usepackage{pifont}
% 圆圈序号
\author{Didnelpsun}
\title{函数与极限}
\begin{document}
\maketitle
\thispagestyle{empty}
\tableofcontents
\thispagestyle{empty}
\newpage
\pagestyle{plain}
\setcounter{page}{1}
\section{邻域}
\subsection{一维}

邻域\textcolor{violet}{\textbf{定义:}}以点$x_0$为中心的任何开区间为点$x_0$的邻域,记为$U(x_0)$。

$\delta$邻域\textcolor{violet}{\textbf{定义:}}设$\delta$为一正数,则称开区间$(x_0-\delta,x_0+\delta)$为点$x_0$的$\delta$邻域,记作$U(x_0,\delta)$。$x_0$称为邻域的中心,$\delta$为邻域的半径。

去心$\delta$邻域就是去除$x_0$的$\delta$邻域,记为$\mathring{U}(x_0,\delta)$,左$\delta$邻域就是左侧的去心$\delta$邻域,记为$U^+(x_0,\delta)$,右$\delta$邻域就是右侧的去心$\delta$邻域,记为$U^-(x_0,\delta)$。

\subsection{二维}

邻域\textcolor{violet}{\textbf{定义:}}设点$P_0(x_0,y_0)$为$xOy$平面上的一点,$\delta$为某一个正数,与点$P_0(x_0,y_0)$的距离小于$\delta$的点$P(x,y)$的全体,称为点$P_0$的$\delta$邻域,记为$U(P_0,\delta)$。

同理可以得到去心$\delta$邻域的定义。

$\delta$邻域的几何意义:以$P_0(x_0,y_0)$为中心,以$\delta>0$为半径的圆内部所有的点。

函数的邻域就是一个区间,所以比如函数在某点的某邻域内有定义,就是说明函数在这个点的附近有定义,这个附近的距离没有必要说明。

\section{定义}

设函数$f(x)$在点$x_0$的某一个去心邻域有定义,若存在常数$A$,对于任意给定的$\epsilon>0$,总存在正数$\delta$,使得当$0<\vert x-x_0\vert<\delta$式,对应的函数值$f(x)$都满足不等式$\vert f(x)-A\vert <\epsilon$,则$A$就是函数$f(x)$当$x\to x_0$时的极限,记作$\lim_{x\to x_0}f(x)=A$或$f(x)\rightarrow A(x\rightarrow x_0)$。

写成$\epsilon-\delta$语言:$\lim_{x\to x_0}f(x)=A\Leftrightarrow\forall\epsilon>0,\exists\delta>0,\text{当}0<\vert x-x_0\vert<\delta$时,有$\vert f(x)-A\vert\epsilon$。

而对于趋向无穷时,写成$\epsilon-X$语言:$\lim_{x\to\infty}f(x)=A\Leftrightarrow\forall\epsilon>0,\exists X>0,\text{当}\vert x\vert>X$时,有$\vert f(x)-A\vert<\epsilon$。

\textcolor{orange}{注意:}这里的趋向分为六种:$x\to x_0$、$x\to x_0^+$、$x\to x_0^-$、$x\to\infty$、$x\to\infty^+$、$x\to\infty^-$。

\subsection{单侧极限}

当$x\to x_0^-$存在的极限称为左极限,当$x\to x_0^+$存在的极限称为右极限。

\subsection{函数极限存在条件}

函数存在的充要条件是:

\begin{enumerate}
    \item $\lim_{x\to x_0}f(x)\Leftrightarrow\lim_{x\to x_0^-}f(x)=\lim_{x\to x_0^+}f(x)=A$。
    \item 函数脱帽法:$\lim_{x\to x_0}f(x)\Leftrightarrow f(x)=A+\alpha(x),\lim_{x\to x_0}\alpha(x)=0$,后面的$\alpha(x)$就是函数与极限值的误差。
\end{enumerate}

\section{性质}

任何$x$的趋向三个性质都是成立的。

\subsection{唯一性}

若极限存在,则极限唯一。

\subsection{局部有界性}

若极限存在且为$A$,则存在正常数$M$和$\delta$,使得当$0<\vert x-x_0\vert<\delta$时,有$\vert f(x)\vert\leqslant M$。

\subsection{局部保号性}

若极限存在,则存在常数$\delta>0$,使得当$0<\vert x-x_0\vert<\delta$时,$f(x)$与$A$同号。

简单来说,函数值在$x\to x_0$时函数值与极限值同号。

证明局部保号性:

首先根据极限存在定义:$\forall\epsilon>0,\exists\delta>0,0<\vert x-x_0\vert<\delta$时,恒有$\vert f(x)-A\vert<\epsilon$。

$\Rightarrow -\epsilon<f(x)-A<\epsilon$

$\Rightarrow A-\epsilon<f(x)<A+\epsilon$

任意取$\epsilon=\dfrac{A}{2}>0\Rightarrow f(x)>A-\dfrac{A}{2}=\dfrac{A}{2}>0$

证明完毕。

关于$\epsilon$的取值问题,为什么不能取到令结果为负的值,因为请注意这个取值得到的区间并不是$f(x)$的范围,而是对$f(x)$所在区间的陈述,其是无尽逼近$A$的,所以取多大的区间都无所谓。

推论:若函数值在$x\to x_0$时都非负或非正,极限值为$A$,那么$A$与此时函数值同号。不能去除等号。

关于三个性质要注意自变量取值的双向性,所以需要留意下面几个函数:

\begin{enumerate}
    \item $\lim_{x\to\infty}e^x$不存在,因为$\lim_{x\to +\infty}e^x=+\infty$,$\lim_{x\to -\infty}e^x=0$。
    \item $\lim_{x\to 0}\dfrac{\sin x}{\vert x\vert}$不存在,因为$\lim_{x\to 0^+}\dfrac{\sin x}{\vert x\vert}=1$,$\lim_{x\to 0^-}\dfrac{\sin x}{\vert x\vert}=-1$。
    \item $\lim_{x\to\infty}\arctan x$不存在,因为$\lim_{x\to +\infty}\arctan x=\dfrac{\pi}{2}$,$\lim_{x\to -\infty}\arctan x=-\dfrac{\pi}{2}$。
    \item $\lim_{x\to 0}[x]$不存在,因为$\lim_{x\to 0^+}[x]=0$,$\lim_{x\to 0^-}[x]=-1$
\end{enumerate}

\section{运算法则}
\section{夹逼准则}
\section{洛必达法则}
\section{泰勒公式}

非常重要。

\subsection{定义}
\subsection{泰勒展开}
\section{归结原则}
\section{无穷小比阶}
\section{连续}
\subsection{连续点的定义}
\section{间断}
\subsection{间断点定义}
\subsection{间断点分类}
\subsubsection{可去间断点}
\subsubsection{跳跃间断点}
\subsubsection{无穷间断点}
\subsubsection{振荡间断点}
\end{document}
