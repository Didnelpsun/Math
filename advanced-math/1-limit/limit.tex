\documentclass[UTF8, 12pt]{ctexart}
% UTF8编码,ctexart现实中文
\usepackage{color}
% 使用颜色
\definecolor{orange}{RGB}{255,127,0} 
\usepackage{geometry}
\setcounter{tocdepth}{4}
\setcounter{secnumdepth}{4}
% 设置四级目录与标题
\geometry{papersize={21cm,29.7cm}}
% 默认大小为A4
\geometry{left=3.18cm,right=3.18cm,top=2.54cm,bottom=2.54cm}
% 默认页边距为1英尺与1.25英尺
\usepackage{indentfirst}
\setlength{\parindent}{2.45em}
% 首行缩进2个中文字符
\usepackage{setspace}
\renewcommand{\baselinestretch}{1.5}
% 1.5倍行距
\usepackage{amssymb}
% 因为所以
\usepackage{amsmath}
% 数学公式
\usepackage{pifont}
% 圆圈序号
\usepackage{mathtools}
% 有字的长箭头
\usepackage[colorlinks,linkcolor=black,urlcolor=blue]{hyperref}
% 超链接
\author{Didnelpsun}
\title{极限练习题}
\date{}
\begin{document}
\maketitle
\thispagestyle{empty}
\tableofcontents
\thispagestyle{empty}
\newpage
\pagestyle{plain}
\setcounter{page}{1}

\section{极限类型}

七种:$\dfrac{0}{0},\dfrac{\infty}{\infty},0\cdot\infty,\infty-\infty,\infty^0,0^0,1^\infty$。

\bigskip

\ding{172}其中$\dfrac{0}{0}$为洛必达法则的基本型。$\dfrac{\infty}{\infty}$可以类比$\dfrac{0}{0}$的处理方式。$0\cdot\infty$可以转为$\dfrac{0}{\dfrac{1}{\infty}}=\dfrac{0}{0}=\dfrac{\infty}{\dfrac{1}{0}}=\dfrac{\infty}{\infty}$。设置分母有原则,简单因式才下放。(简单:幂函数,e为底的指数函数)

\bigskip

\ding{173}$\infty-\infty$可以提取公因式或通分,即和差化积。

\bigskip

\ding{174}$\infty^0,0^0,1^\infty$,就是幂指函数。

\bigskip

$
u^v=e^{v\ln u}=\left\{
\begin{array}{lcl}
    \infty^0 & \rightarrow & e^{0\cdot+\infty} \\
    0^0 & \rightarrow & e^{0\cdot-\infty} \\
    1^\infty & \rightarrow & e^{\infty\cdot 0} \\
\end{array} \right.
$

$\therefore \lim u^v=e^{\lim v\cdot\ln u}=e^{\lim v(u-1)}$

综上,无论什么样的四则形式,都必须最后转换为商的形式。

\section{常用化简运算}

\subsection{对数法则}

\textbf{例题:}求$\lim_{x\to 0}\dfrac{(e^{x^2}-1)(\sqrt{1+x}-\sqrt{1-x})}{[\ln(1-x)+\ln(1+x)]\sin\dfrac{x}{x+1}}$。

注意在积或商的时候不能把对应的部分替换为0,如分母部分的$[\ln(1-x)+\ln(1+x)]$就无法使用$\ln(1+x)\sim x$替换为$-x+x$,这样底就是0了,无法求得最后的极限。

这时可以尝试变形,如对数函数相加等于对数函数内部式子相乘:$\ln(1-x)+\ln(1+x)=\ln(1-x^2)\sim-x^2$。

\subsection{指数法则}

求$\lim_{n\to\infty}n\left[\left(1+\dfrac{1}{n}\right)^{\frac{n}{2}}-\sqrt{e}\right]$。

首先对于幂指函数需要取指数,所以$\left(1+\dfrac{1}{n}\right)^{\frac{n}{2}}=e^{\frac{n}{2}\ln(1+\frac{1}{n})}$。

而后面的多一个$\sqrt{e}$导致整个式子变为一个复杂的式子,而与$e^x$相关的是$e^x-1\sim x$。

所以$e^{\frac{n}{2}\ln(1+\frac{1}{n})}-\sqrt{e}=e^{\frac{1}{2}}\cdot\left(e^{\frac{n}{2}\ln(1+\frac{1}{n})-\frac{1}{2}}-1\right)=e^{\frac{1}{2}}\cdot\left[\dfrac{n}{2}\ln(1+\dfrac{1}{n})-\dfrac{1}{2}\right]$。

综上:

$
\begin{aligned}
    & \lim_{n\to\infty}n\left[\left(1+\dfrac{1}{n}\right)^{\frac{n}{2}}-\sqrt{e}\right] \\
    & =\lim_{n\to\infty}n\left(e^{\frac{n}{2}\ln(1+\frac{1}{n})}-\sqrt{e}\right) \\
    & =\lim_{n\to\infty}n\left[e^{\frac{1}{2}}\cdot\left(e^{\frac{n}{2}\ln(1+\frac{1}{n})-\frac{1}{2}}-1\right)\right] \\
    & =\dfrac{e^{\frac{1}{2}}}{2}\lim_{n\to\infty}n^2\left[\ln\left(1+\frac{1}{n}\right)-\dfrac{1}{n}\right] \\
    & =\dfrac{e^{\frac{1}{2}}}{2}\lim_{n\to\infty}\dfrac{\dfrac{1}{n}-\dfrac{1}{2n^2}-\dfrac{1}{n}}{\dfrac{1}{n^2}} \\
    & =\dfrac{e^{\frac{1}{2}}}{2}\cdot\left(-\dfrac{1}{2}\right) \\
    & =-\dfrac{\sqrt{e}}{4}
\end{aligned}
$

\subsection{三角函数关系式}

\textbf{例题:}求极限$\lim_{x\to 0}\left(\dfrac{1}{\sin^2x}-\dfrac{\cos^2x}{x^2}\right)$。

$
\begin{aligned}
    & \lim_{x\to 0}\left(\dfrac{1}{\sin^2x}-\dfrac{\cos^2x}{x^2}\right) \\
    & = \lim_{x\to 0}\dfrac{x^2-\sin^2x\cos^2x}{\sin^2x\cdot x^2} (\sin x\sim x)\\
    & = \lim_{x\to 0}\dfrac{x^2-\sin^2x\cos^2x}{x^4} (\sin x\cos x\sim\dfrac{1}{2}\sin 2x)\\
    & = \lim_{x\to 0}\dfrac{x^2-\dfrac{1}{4}\sin^22x}{x^4} \\
\end{aligned}
$

$
\begin{aligned}
    & = \lim_{x\to 0}\dfrac{2x-\dfrac{1}{4}\cdot 2\sin 2x\cdot\cos 2x\cdot 2}{4x^3} (\sin x\cos x\sim\dfrac{1}{2}\sin 2x)\\
    & = \lim_{x\to 0}\dfrac{2x-\dfrac{1}{2}\sin 4x}{4x^3} \\
    & = \lim_{x\to 0}\dfrac{2-\dfrac{1}{2}\cos 4x\cdot 4}{12x^2} \\
    & = \dfrac{1}{6}\lim_{x\to 0}\dfrac{1-\cos 4x}{x^2} (1-\cos x\sim \dfrac{1}{2}x^2)\\
    & = \dfrac{4}{3}
\end{aligned}
$

\subsection{提取常数因子}

提取常数因子就是提取出能转换为常数的整个极限式子的因子。这个因子必然在自变量的趋向时会变为非0的常数,那么这个式子就可以作为常数提出。

\subsection{提取公因子}

当作为商的极限式子上下都具有公因子时可以提取公因子然后相除,从而让未知数集中在分子或分母上。

\textcolor{orange}{注意:}提取公因子的时候应该注意开平方等情况下符号的问题。如果极限涉及倒正负两边则必须都讨论。

当趋向为负且式子中含有根号的时候最好提取负因子,从而让趋向变为正。

\subsection{幂指函数}

当出现$f(x)^{g(x)}$的类似幂函数与指数函数类型的式子,需要使用$u^v=e^{v\ln u}$。

\textbf{例题:}求极限$\lim_{x\to+\infty}(x+\sqrt{1+x^2})^{\frac{1}{x}}$。

$
\begin{aligned}
    & \lim_{x\to+\infty}(x+\sqrt{1+x^2})^{\frac{1}{x}} \\
    & =e^{\lim_{x\to+\infty}\frac{\ln(x+\sqrt{1+x^2})}{x}} \left(\ln(x+\sqrt{1+x^2})'=\dfrac{1}{\sqrt{1+x^2}}\right) \\
    & =e^{\lim_{x\to+\infty}\frac{1}{\sqrt{1+x^2}}} \\
    & =e^0 \\
    & =1
\end{aligned}
$

\subsection{有理化}

当遇到带有根号的式子可以使用等价无穷小,但是只针对形似$(1+x)a-1\sim ax$的式子,而针对$x^a\pm x^b$的式子则无法替换,必须使用有理化来将单个式子变为商的形式。

如$\sqrt{a}\pm\sqrt{b}=\dfrac{a+b}{\sqrt{a}\mp\sqrt{b}}$。

\textbf{例题:}求极限$\lim_{x\to-\infty}x(\sqrt{x^2+100}+x)$。

首先定性分析:$\lim_{x\to-\infty}x\cdot(\sqrt{x^2+100}+x)$。

在$x\to-\infty$趋向时,$x$就趋向无穷大,而$\sqrt{x^2+100}$为一次,所以$\sqrt{x^2+100}+x$趋向0。

又$\sqrt{x^2+100}$在$x\to-\infty$时本质为根号差,所以有理化:

$
\begin{aligned}
    & \lim_{x\to-\infty}x(\sqrt{x^2+100}+x) \\
    & \lim_{x\to-\infty}x\dfrac{x^2+100-x^2}{\sqrt{x^2+100}-x} \\
    & = \lim_{x\to-\infty}\dfrac{100x}{\sqrt{x^2+100}-x} \\
    & \xRightarrow{\text{令}x=-t}\lim_{t\to+\infty}\dfrac{-100t}{\sqrt{t^2+100}+t} \\
    & = \lim_{t\to+\infty}\dfrac{-100}{\sqrt{1+\dfrac{100}{t^2}}+1} \\
    & = -50
\end{aligned}
$

\subsection{换元法}

换元法本身没什么技巧性,主要是更方便计算。最重要的是获取到共有的最大因子进行替换。

\textbf{例题:}求极限$\lim_{x\to 1^-}\ln x\ln(1-x)$。

当$x\to 1^-$时,$\ln x$趋向0,$\ln(1-x)$趋向$-\infty$。

又$x\to 0$,$\ln(1+x)\sim x$,所以$x\to 1$,$\ln x\sim x-1$:

$
\begin{aligned}
    & \lim_{x\to 1^-}\ln x\ln(1-x) \\
    & = \lim_{x\to 1^-}(x-1)\ln(1-x) \\
    & \xRightarrow{令t=1-x} =-\lim_{t\to 0^+}t\ln t \\
    & = -\lim_{t\to 0^+}\dfrac{\ln t}{\dfrac{1}{t}} \\
    & = -\lim_{t\to 0^+}\dfrac{\dfrac{1}{t}}{-\dfrac{1}{t^2}} \\
    & = \lim_{t\to 0^+}t \\
    & = 0
\end{aligned}
$

\subsection{倒代换}

\subsubsection{含有分式}

当极限式子中含有分式中一般都需要用其倒数,把分式换成整式方便计算。

\textbf{例题:}求极限$\lim_{x\to 0}\dfrac{e^{-\frac{1}{x^2}}}{x^{100}}$

$
\begin{aligned}
    & \lim_{x\to 0}\dfrac{e^{-\frac{1}{x^2}}}{x^{100}} \\
    & = \lim_{x\to 0}\dfrac{e^{-\frac{1}{x^2}}\cdot 2x^{-3}}{100x^{99}} \\
    & = \lim_{x\to 0}\dfrac{1}{50}\lim_{x\to 0}\dfrac{e^{-\frac{1}{x^2}}}{x^{102}}
\end{aligned}
$

\bigskip

使用洛必达法则下更复杂,因为分子的幂次为负数,导致求导后幂次绝对值越来越大,不容易计算。

使用倒代换再洛必达降低幂次,令$\dfrac{1}{x^2}=t$。

$
\begin{aligned}
    & \lim_{x\to 0}\dfrac{e^{-\frac{1}{x^2}}}{x^{100}} \\
    & = \lim_{t\to+\infty}\dfrac{e^{-t}}{t^{-50}} \\
    & = \lim_{t\to+\infty}\dfrac{t^{50}}{e^t} \\
    & = \lim_{t\to+\infty}\dfrac{50t^{49}}{e^t} \\
    & = \cdots \\
    & = \lim_{t\to+\infty}\dfrac{50!}{e^t} \\
    & = 0
\end{aligned}
$

\textbf{例题:}求极限$\lim_{x\to+\infty}[x^2(e^{\frac{1}{x}}-1)-x]$。

该式子含有分数,所以尝试使用倒数代换:

$
\begin{aligned}
    & \lim_{x\to+\infty}[x^2(e^{\frac{1}{x}}-1)-x] \\
    & \xRightarrow{\text{令}x=\frac{1}{t}}\lim_{t\to 0^+}\left(\dfrac{e^t-1}{t^2}-\dfrac{1}{t}\right) \\
    & \lim_{t\to 0^+}\dfrac{e^t-1-t}{t^2} \\
    & \xRightarrow{\text{泰勒展开}e^t}\lim_{t\to 0^+}\dfrac{\dfrac{1}{2}t^2}{t^2} \\
    & =\dfrac{1}{2}
\end{aligned}
$

\subsubsection{\texorpdfstring{$\infty-\infty$}\ 型}

\subsection{拆项}

当极限式子中出现不知道项数的$n$时,一般需要使用拆项,把项重新组合。一般的组合是根据等价无穷小。

\textbf{例题:}求极限$\lim_{x\to 0}\left(\dfrac{e^x+e^{2x}+\cdots+e^{nx}}{n}\right)^{\frac{e}{x}}$。($n\in N^+$)

$
\begin{aligned}
    & \lim_{x\to 0}\left(\dfrac{e^x+e^{2x}+\cdots+e^{nx}}{n}\right)^{\frac{e}{x}} \\
    & =e^{\lim_{x\to 0}\frac{e}{x}\ln\left(\frac{e^x+e^{2x}+\cdots+e^{nx}}{n}\right)} \\
    & =e^{\lim_{x\to 0}\frac{e}{x}\left(\frac{e^x+e^{2x}+\cdots+e^{nx}}{n}-1\right)} \\
    & =e^{\lim_{x\to 0}\frac{e}{x}\left(\frac{e^x+e^{2x}+\cdots+e^{nx}-n}{n}\right)} \\
    & =e^{\frac{e}{n}\lim_{x\to 0}\left(\frac{e^x-1}{x}+\frac{e^{2x}-1}{x}+\cdots+\frac{e^{nx}-1}{x}\right)} \\
    & =e^{\frac{e}{n}[1+2+\cdots+n]} \\
    & =e^{\frac{e}{n}\cdot\frac{n(1+n)}{2}} \\
    & =e^{\frac{e(1+n)}{2}}
\end{aligned}
$

\section{基本计算方式}

课本上极限计算可以使用的主要计算方式:

\subsection{基础四则运算}

\subsection{两个重要极限}

\subsection{导数定义}

\subsection{等价无穷小替换}

当看到复杂的式子,且不论要求的极限值的趋向,而只要替换的式子是$\Delta\to 0$时的无穷小,就使用等价无穷小进行替换。

\textcolor{orange}{注意:}替换的必然是整个求极限的乘或除的因子,一般加减法与部分的因子不能进行等价无穷小替换。

对于无法直接得出变换式子的,可以对对应参数进行凑,以达到目标的可替换的等价无穷小。

\subsection{夹逼准则}

夹逼准则可以用来证明不等式也可以用来计算极限。但是最重要的是找到能夹住目标式子的两个式子。

\textbf{例题:}求极限$\lim_{x\to 0}x\left[\dfrac{10}{x}\right]$,其中$[\cdot]$为取整符号。

取整函数公式:$x-1<[x]\leqslant x$,所以$\dfrac{10}{x}-1<\left[\dfrac{10}{x}\right]\leqslant\dfrac{10}{x}$。

当$x>0$时,$x\to 0^+$,两边都乘以10,$10-x<x\cdot\left[\dfrac{10}{x}\right]\leqslant x\cdots\dfrac{10}{x}=10$,而左边在$x\to 0^+$时极限也为10,所以夹逼准则,中间$x\cdot\left[\dfrac{10}{x}\right]$极限也为10。

当$x>0$时,$x\to 0^-$,同样也是夹逼准则得到极限为10。

$\therefore \lim_{x\to 0}x\left[\dfrac{10}{x}\right]=10$。

\subsection{拉格朗日中值定理}

对于形如$f(a)-f(b)$的极限式子就可以使用拉格朗日中值定理,这个$f(x)$为任意的函数。

\textbf{例题:}求极限$\lim_{n\to\infty}n^2\left(\arctan\dfrac{2}{n}-\arctan\dfrac{2}{n+1}\right)$。

因为式子不算非常复杂,其实也可以通过洛必达法则来完成,但是求导会很复杂。而$\arctan x$可以认定为$f(x)$。

从而$\arctan\dfrac{2}{n}-\arctan\dfrac{2}{n+1}$为$f(\dfrac{2}{n})-f(\dfrac{2}{n+1})=f'(\xi)\left(\dfrac{2}{n}-\dfrac{2}{n+1}\right)$。

其中$\dfrac{2}{n+1}<\xi<\dfrac{2}{n}$,而当$n\to\infty$时,$f'(\xi)=\dfrac{1}{1+\xi^2}\to 1$。

$\therefore\arctan\dfrac{2}{n}-\arctan\dfrac{2}{n+1}\sim\dfrac{2}{n}-\dfrac{2}{n+1}=\dfrac{2}{n(n+1)}$。

$\therefore\lim_{n\to\infty}n^2\left(\arctan\dfrac{2}{n}-\arctan\dfrac{2}{n+1}\right)=\lim_{n\to\infty}n^2\cdot\dfrac{2}{n(n+1)}=2$。

\subsection{洛必达法则}

洛必达法则的本质是降低商形式的极限式子的幂次。

洛必达在处理一般的极限式子比较好用,但是一旦式子比较复杂最好不要使用洛必达法则,最好是对求导后有规律或幂次较低的式子进行上下求导。

对于幂次高的式子必然使用洛必达法则。

\subsection{泰勒公式}

泰勒公式一般会使用趋向0的麦克劳林公式,且一般只作为极限计算的一个小部分,用来替代一个部分。

且一般只有麦克劳林公式表上的基本初等函数才会使用倒泰勒公式,复合函数最好不要使用。

\textbf{例题:}求极限$\lim_{x\to 0}\dfrac{\arcsin x-\arctan x}{\sin x-\tan x}$。

分析:该题目使用洛必达法则会比较麻烦且难以计算,所以先考虑是否能用泰勒展开:

$x\to 0$,$\sin x=x-\dfrac{1}{6}x^3+o(x^3)$,$\tan x=x+\dfrac{1}{3}x^3+o(x^3)$,$\arcsin x=x+\dfrac{1}{6}x^3+o(x^3)$,$\arctan x=x-\dfrac{1}{3}x^3+o(x^3)$。

$\therefore \sin x-\tan x=-\dfrac{1}{2}x^3+o(x^3)$,$\arcsin x-\arctan x=\dfrac{1}{2}x^3+o(x^3)$

$\therefore \text{原式}=\dfrac{\dfrac{1}{x}x^3+o(x^3)}{-\dfrac{1}{2}x^3+o(x^3)}=-1$。

\section{极限计算形式}

极限相关计算形式主要分为下面六种:

\begin{enumerate}
    \item 未定式:直接根据式子计算极限值。
    \item 极限转换:根据已知的极限值计算目标极限值。
    \item 求参数:已知式子的极限值,计算式子中未知的参数。
    \item 极限存在性:根据式子以及极限存在性计算极限或参数。
    \item 函数连续性:根据连续性与附加条件计算极限值或参数。
    \item 迭代式数列:根据数列迭代式计算极限值。
    \item 变限积分:根据变限积分计算极限值。
\end{enumerate}

\subsection{极限转换}

\subsubsection{整体换元}

最常用的方式就是将目标值作为一个部分,然后对已知的式子进行替换。

\textbf{例题:}已知$\lim_{x\to 0}\dfrac{\ln(1-x)+xf(x)}{x^2}=0$,求$\lim_{x\to 0}\dfrac{f(x)-1}{x}$。

令目标$\dfrac{f(x)-1}{x}=t$,$\therefore f(x)=tx+1$。

$
\begin{aligned}
    & \lim_{x\to 0}\dfrac{\ln(1-x)+xf(x)}{x^2} \\
    & =\lim_{x\to 0}\dfrac{\ln(1-x)+tx^2+x}{x^2} (\text{泰勒展开}) \\
    & =\lim_{x\to 0}\dfrac{-x-\dfrac{x^2}{2}+tx^2+x}{x^2} \\
    & =\lim_{x\to 0}\dfrac{\left(t-\dfrac{1}{2}\right)x^2}{x^2} \\
    & ==\lim_{x\to 0}\left(t-\dfrac{1}{2}\right) \\
    & =0
\end{aligned}
$

$\therefore\lim_{x\to 0}t=\lim_{x\to 0}\dfrac{f(x)-1}{x}=\dfrac{1}{2}$。

\subsubsection{关系转换}

\textbf{例题:}如果$\lim_{x\to 0}\dfrac{x-\sin x+f(x)}{x^4}$存在,则$\lim_{x\to 0}\dfrac{x^3}{f(x)}$为常数多少?

由$\lim_{x\to 0}\dfrac{x\sin x+f(x)}{x^4}=A$,而目标是$x^3$,所以需要变形:

$
\begin{aligned}
    & \lim_{x\to 0}\dfrac{x\sin x+f(x)}{x^4}=A \\
    & \lim_{x\to 0}\dfrac{x\sin x+f(x)\cdot x}{x^4}=A\cdot\lim_{x\to 0}x=0 \\
    & \lim_{x\to 0}\dfrac{x-\sin x}{x^3}+\lim_{x\to 0}\dfrac{f(x)}{x^3}=0 \\
    & \text{泰勒展开:}x-\sin x=\dfrac{1}{6}x^3 \\
    & \lim_{x\to 0}\dfrac{f(x)}{x^3}=-\dfrac{1}{6} \\
    & \lim_{x\to 0}\dfrac{x^3}{f(x)}=-6
\end{aligned}
$

\subsubsection{脱帽法}

$\lim_{x\to x_0}f(x)\Leftrightarrow f(x)=A+\alpha(x),\lim_{x\to x_0}\alpha(x)=0$。

\textbf{例题:}如果$\lim_{x\to 0}\dfrac{x-\sin x+f(x)}{x^4}$存在,则$\lim_{x\to 0}\dfrac{x^3}{f(x)}$为常数多少?

由$\lim_{x\to 0}\dfrac{x\sin x+f(x)}{x^4}=A$脱帽:$\dfrac{x\sin x+f(x)}{x^4}=A+\alpha$。

得到:$f(x)=Ax^4+\alpha\cdot x^4-(x-\sin x)$。

反代入:$\lim_{x\to 0}\dfrac{f(x)}{x^3}=\lim_{x\to 0}\dfrac{Ax^4+\alpha\cdot x^4-x+\sin x}{x^3}=0+0-\dfrac{1}{6}=-\dfrac{1}{6}$。

$\therefore \lim_{x\to 0}\dfrac{x^3}{f(x)}=-6$。

\subsection{求参数}

因为求参数类型的题目中式子是未知的,所以求导后也是未知的,所以一般不要使用洛必达法则,而使用泰勒展开。

一般极限式子右侧等于一个常数,或是表明高阶或低阶。具体的关系参考无穷小比阶。

\textbf{例题:}设$\lim_{x\to 0}\dfrac{\ln(1+x)-(ax+bx^2)}{x^2}=2$,求常数a,b。

根据泰勒展开式:$x\to 0,\ln(1+x)=x-\dfrac{x^2}{x}+o(x^2)$。

$
\begin{aligned}
    & \lim_{x\to 0}\dfrac{\ln(1+x)-(ax+bx^2)}{x^2}=2 \\
    & =\lim_{x\to 0}\dfrac{(1-a)x-\left(\dfrac{1}{2}+b\right)x^2+o(x^2)}{x^2}=2\neq 0 \\
    & 1-a=0;-\left(\dfrac{1}{2}+b\right)=2 \\
    & \therefore a=1;b=-\dfrac{5}{2}
\end{aligned}
$

\textcolor{orange}{注意:}根据泰勒公式,$x-\ln(1+x)\sim\dfrac{1}{2}x^2\sim 1-\cos x$。

\subsection{极限存在性}

\subsection{函数连续性}

函数的连续性代表:极限值=函数值。

\textbf{例题:}函数在$f(x)$在$x=1$处连续,且$f(1)=1$,求$\lim_{x\to+\infty}\ln\left[2+f\left(x^{\frac{1}{x}}\right)\right]$。

根据题目,所求的$\lim_{x\to+\infty}\ln\left[2+f\left(x^{\frac{1}{x}}\right)\right]$中,唯一未知的且会随着$x\to+\infty$而变换就是$f\left(x^{\frac{1}{x}}\right)$。如果我们可以求出这个值就可以了。

而我们对于$f(x)$的具体的关系是未知的,只知道$f(1)=1$。那么先需要考察$\lim_{x\to+\infty}x^{\frac{1}{x}}$的整数最大值。

$
\begin{aligned}
    & \lim_{x\to+\infty}x^{\frac{1}{x}} \\
    & e^{\lim_{x\to+\infty}\frac{\ln x}{x}} \\
    & e^{\lim_{x\to+\infty}\frac{1}{x}} \\
    & =e^0 \\
    & =1
\end{aligned}
$

$\therefore\lim_{x\to+\infty}f(x^{\frac{1}{x}})=f(1)=1$。

\subsection{迭代式数列}

\subsubsection{关系式变形}

最重要的是将迭代式进行变形。

\textbf{例题:}数列$\{a_n\}$满足$a_0=0,a_1=1,2a_{n+1}=a_n+a_{n-1},n=1,2,\cdots$。计算$\lim_{n\to\infty}a_n$。

首先看题目,给出的递推式设计到二阶递推,即存在三个数列变量,所以我们必须先求出对应的数列表达式。因为这个表达式涉及三个变量,所以尝试对其进行变型:

$
\begin{aligned}
    a_{n+1}-a_n & =\dfrac{a_{n-1}-a_n}{2} \\
    & =\left(-\dfrac{1}{2}\right)(a_n-a_{n-1}) \\
    & =\left(-\dfrac{1}{2}\right)^2(a_{n-1}-a_{n-2}) \\
    & =\cdots \\
    & =\left(-\dfrac{1}{2}\right)^n(a_1-a_0) \\ 
    & = \left(-\dfrac{1}{2}\right)^n
\end{aligned}
$

然后得到了$a_{n+1}-a_n=\left(-\dfrac{1}{2}\right)^n$,而需要求极限,所以使用列项相消法的逆运算:

$
\begin{aligned}
    a_n & = (a_n-a_{n-1})+(a_{n-1}-a_{n-2})+\cdots+(a_1-a_0)+a_0 \\
     & = \left(-\dfrac{1}{2}\right)^{n-1} + \left(-\dfrac{1}{2}\right)^{n-2} + \cdots + \left(-\dfrac{1}{2}\right)^0 \\
     & = \dfrac{1\cdot\left(1-\left(-\dfrac{1}{2}\right)^n\right)}{1-\left(-\dfrac{1}{2}\right)} \\
     & = \dfrac{2}{3}\left[1-\left(-\dfrac{1}{2}\right)^n\right] \\
    \lim_{n\to\infty}a_n & =\dfrac{2}{3}
\end{aligned}
$

\subsubsection{单调有界准则}

\subsection{变限积分极限}

已知更改区间限制的积分$s(x)=\int_{\varphi_1(x)}^{\varphi_2(x)}g(t)\rm{d}x$,$s'(x)=g[\varphi_2(x)]\cdot\varphi_2'(x)-g[\varphi_1(x)]\cdot\varphi_1'(x)$。


\end{document}
