\documentclass[UTF8]{ctexart}
% UTF8编码,ctexart现实中文
\usepackage{color}
% 使用颜色
\definecolor{orange}{RGB}{255,127,0} 
\definecolor{violet}{RGB}{192,0,255} 
\usepackage{geometry}
\setcounter{tocdepth}{4}
\setcounter{secnumdepth}{4}
% 设置四级目录与标题
\geometry{papersize={21cm,29.7cm}}
% 默认大小为A4
\geometry{left=3.18cm,right=3.18cm,top=2.54cm,bottom=2.54cm}
% 默认页边距为1英尺与1.25英尺
\usepackage{indentfirst}
\setlength{\parindent}{2.45em}
% 首行缩进2个中文字符
\usepackage{amssymb}
% 因为所以与其他数学拓展
\usepackage{amsmath}
% 数学公式
\usepackage{setspace}
\renewcommand{\baselinestretch}{1.5}
% 1.5倍行距
\author{Didnelpsun}
\title{一元函数微分学}
\begin{document}
\maketitle
\thispagestyle{empty}
\tableofcontents
\thispagestyle{empty}
\newpage
\pagestyle{plain}
\setcounter{page}{1}
\section{概念}
\subsection{引例}

设$f(x)$下$x$在$x_0$的邻域内,$\alpha$为切线所成夹角。

$\tan\alpha=f'(x_0)=\lim_{x\to x_0}\dfrac{f(x)-f(x_0)}{x-x_0}=k$

\subsection{导数}

设$y=f(x)$定义在区间$I$上,让自变量在$x=x_0$处加一个增量$\Delta x$,其中$x_0\in I$,$x_0+\Delta x\in I$,则可得函数的增量$\Delta y=f(x_0+\Delta x)-f(x_0)$。若函数增量$\Delta y$与自变量增量$\Delta x$的比值在$\Delta x\to 0$时的极限存在,则称函数$y=f(x)$在$x_0$处可导,并称这个极限为$y=f(x)$在点$x_0$处的导数,记作$f'(x)$,即$f'(x)=\lim_{\Delta x\to 0}\dfrac{\Delta y}{\Delta x}=\lim_{\Delta x\to 0}\dfrac{f(x_0+\Delta x)-f(x_0)}{\Delta x}$。

下面三句话等价:

\begin{enumerate}
    \item $y=f(x)$在$x_0$处可导。
    \item $y=f(x)$在$x_0$处导数存在。
    \item $f'(x)=A$。($A$为有限数)
\end{enumerate}

单侧导数分为左导数和右导数。

$f'_-(x)=\lim_{\Delta x\to 0^-}\dfrac{\Delta y}{\Delta x}=\lim_{\Delta x\to 0}\dfrac{f(x_0+\Delta x)-f(x_0)}{\Delta x}$

$f'_+(x)=\lim_{\Delta x\to 0^+}\dfrac{\Delta y}{\Delta x}=\lim_{\Delta x\to 0}\dfrac{f(x_0+\Delta x)-f(x_0)}{\Delta x}$

所以$f(x)$在$x_0$处可导的充要条件是其左导数和右导数存在且相等。

若$f(x)$在$x_0$的左右,如$y=\vert x\vert$在$0$的左右出现了单侧的不同的切线,那这个$x_0$就是一个\textbf{角点},该角点处不可导。

若$f(x)$在$x_0$处导数为无穷,如$y=x^{\frac{1}{3}}$在$0$处利用导数的极限定义计算得到为正无穷,那么该点的导数为无穷导数,在考研中被认为是不存在的。

\textbf{例题1:}证明若$f(x)$为可导的偶函数,则$f'(x)$为奇函数,若$f(x)$为可导的奇函数,则$f'(x)$为偶函数。

该证明是准备部分的定理。

首先已知$f(-x)=f(x)$,证明$f'(-x)=-f'(x)$。

$\therefore$

$
\begin{aligned}
    f'(-x) &=\lim_{\Delta x\to 0}\dfrac{f(-x+\Delta x)-f(-x)}{\Delta x} \\
    & =\lim_{\Delta x\to 0}\dfrac{f(x+(-\Delta x))}{\Delta x} \\
    & =-\lim_{-\Delta x\to 0}\dfrac{f(x+(-\Delta x))}{-\Delta x} \\
    & =-f'(x)
\end{aligned}
$

同理得证$f(-x)=-f(x)\Rightarrow f'(-x)=f'(x)$。

\textbf{例题2:}证明$f(x)$为可导的周期为$T$的周期函数,则$f'(x)$也是以$T$为周期的周期函数。

已知$f(x+T)=f(x)$,求证$f'(x+T)=f'(x)$。

$\therefore f'(x+T)=\lim_{\Delta x\to 0}\dfrac{f(x+T+\Delta x)-f(x+T)}{\Delta x}=\lim_{\Delta x\to 0}\dfrac{f(x+\Delta x)-f(x)}{\Delta x}=f'(x)$。

\textbf{例题3:}设$f(x)$是二阶可导的以2为周期的奇函数,且$f(\dfrac{1}{2})>0$,$f'(\dfrac{1}{2})>0$,比较$f(-\dfrac{1}{2})$、$f'(\dfrac{3}{2})$、$f''(0)$的大小。

$\because f(x)$为二阶奇函数,$\therefore f(x)\text{奇函数}\Rightarrow f'(x)\text{偶函数}\Rightarrow f''(x)\text{奇函数}\Rightarrow f''(0)=0$。

$\therefore f(-\dfrac{1}{2})=-f(\dfrac{1}{2})<0$。

$\because f(x)T=2\Rightarrow f'(x)T=2$,$\therefore f'(\dfrac{3}{2})=f'(\dfrac{3}{2}-2)=f'(-\dfrac{1}{2})=f'(\dfrac{1}{2})>0$。

$\therefore f'(\dfrac{3}{2})>f''(0)>f(-\dfrac{1}{2})$。

\textbf{例题4:}证明$(uv)'=u'v+uv'$。

令$f(x)=u(x)v(x)$。

$
\begin{aligned}
    & (u\cdot v)' \\
    & =f'(x) \\
    & =\lim_{\Delta x\to 0}\dfrac{f(x+\Delta x)-f(x)}{\Delta x} \\
    & =\lim_{\Delta x\to 0}\dfrac{u(x+\Delta x)v(x+\Delta x)-u(x)v(x)}{\Delta x} \\
    & =\lim_{\Delta x\to 0}\dfrac{u(x+\Delta x)v(x+\Delta x)-u(x)v(x+\Delta x)+u(x)v(x+\Delta x)-u(x)v(x)}{\Delta x} \\
    & =\lim_{\Delta x\to 0}\dfrac{u(x+\Delta x)-u(x)}{\Delta x}v(x+\Delta x) +\lim_{\Delta x\to 0}\dfrac{v(x+\Delta x)-v(x)}{\Delta x}u(x) \\
    & =u'(x)v(x)+v'(x)u(x)
\end{aligned}
$

\textbf{例题5:}证明可导必连续。

已知连续定义:$\lim_{\Delta x\to 0}f(x+\Delta x)=f(x)$,即$\lim_{\Delta x\to 0}f(x+\Delta x)-f(x)=0$。

可导定义:$f'(x)=\lim_{\Delta x\to 0}\dfrac{f(x+\Delta x)-f(x)}{\Delta x} = A$

$
\begin{aligned}
    & \lim_{\Delta x\to 0}f(x+\Delta x)-f(x) \\
    & =\lim_{\Delta x\to 0}\dfrac{f(x+\Delta x)-f(x)}{\Delta x}\cdot\Delta x \\
    & =A\cdot 0 \\
    & =0
\end{aligned}
$

\textbf{例题6:}若$f(x)$在$x=x_0$处连续,且$\lim_{x\to x_0}\dfrac{f(x)}{x-x_0}=A$,则$f(x_0)=0$且$f'(x_0)=A$。

证明:$\because\text{连续,}\therefore f(x_0)=\lim_{x\to x_0}f(x)=\lim_{x\to x_0}\dfrac{f(x)}{x-x_0}(x-x_0)=A\cdot 0=0$。

又$f'(x_0)=\lim_{x\to x_0\dfrac{f(x)-f(x_0)}{x-x_0}}=\lim_{x\to x_0}\dfrac{f(x)}{x-x_0}=A$。

如$\lim_{x\to 1}\dfrac{f(x)}{x-1}=2$且$f(x)$连续,可以推出$f(1)=0$与$f'(1)=2$。

高阶导数\textcolor{violet}{\textbf{定义:}}$f^n(x_0)=\lim_{\Delta x\to 0}\dfrac{f^{n-1}(x_0+\Delta x)-f^{n-1}(x_0)}{\Delta x}$,其中$n\geqslant 2$且$n\in N^+$,$f^{n-1}(x)$在$x_0$的某领域内有定义,$x_0+\Delta x$也在该邻域内。

\subsection{微分}

有一个边长为$x$的正方形,变化了$\Delta x$,其面积$\Delta S=(x+\Delta x)^2-x^2=2x\Delta x+(\Delta x)^2$。

当$\Delta x\to 0$时,将这个变化定义为$2x\cdot\Delta x+o(\Delta x)$,前项为线性主部,后面为误差。这个就是$S$的微分。

增量$\Delta y=f(x_0+\Delta)-f(x_0)=A\Delta x+o(\Delta x)$,这个$A\Delta x$定义为$\rm{d}y$,叫做$y$的微分。

$\therefore \rm{d}y\vert_{x=x_0}=A\Delta x=y'(x_0)\cdot\Delta x=y'(x_0)\cdot\rm{d}x$

由此,可导必可微,可微必可导。

\section{导数与微分计算}
\subsection{四则运算}
\subsection{分段函数的导数}
\subsection{复合函数的导数与微分形式不变性}
\subsection{反函数导数}
\subsection{参数方程函数导数}
\subsection{隐函数求导法}
\subsection{对数求导法}
\subsection{幂指函数求导法}
\subsection{高阶导数}
\subsubsection{归纳法}
\subsubsection{莱布尼茨公式}
\subsubsection{泰勒公式}
\subsection{变限积分求导公式}
\subsection{基本求导公式}
\end{document}
