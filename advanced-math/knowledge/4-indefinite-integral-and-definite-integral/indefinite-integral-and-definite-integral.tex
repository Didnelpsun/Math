\documentclass[UTF8, 12pt]{ctexart}
% UTF8编码,ctexart现实中文
\usepackage{color}
% 使用颜色
\definecolor{orange}{RGB}{255,127,0} 
\definecolor{violet}{RGB}{192,0,255} 
\definecolor{aqua}{RGB}{0,255,255} 
\usepackage{geometry}
\setcounter{tocdepth}{4}
\setcounter{secnumdepth}{4}
% 设置四级目录与标题
\geometry{papersize={21cm,29.7cm}}
% 默认大小为A4
\geometry{left=3.18cm,right=3.18cm,top=2.54cm,bottom=2.54cm}
% 默认页边距为1英尺与1.25英尺
\usepackage{indentfirst}
\setlength{\parindent}{2.45em}
% 首行缩进2个中文字符
\usepackage{setspace}
\renewcommand{\baselinestretch}{1.5}
% 1.5倍行距
\usepackage{amssymb}
% 因为所以
\usepackage{amsmath}
% 数学公式
\usepackage[colorlinks,linkcolor=black,urlcolor=blue]{hyperref}
% 超链接
\author{Didnelpsun}
\title{不定积分与定积分}
\date{}
\begin{document}
\maketitle
\pagestyle{empty}
\thispagestyle{empty}
\tableofcontents
\thispagestyle{empty}
\newpage
\pagestyle{plain}
\setcounter{page}{1}
\section{不定积分}

积分就是导数的逆运算。$\int f(x)\,\rm{d}$$x=F(x)+C$,$F'(x)=f(x)$。 

所以积分运算就可以将原来求导的方式进行逆运算。其中隐函数求导法与参数方程求导法都可以看作复合函数求导法则的变式。

函数和差的求导法则的逆运算,就是分项积分法。

复合函数的求导法则的逆运算,就是换元积分法。

函数乘积的求导法则的逆运算,就是分部积分法。

\subsection{换元积分法}

\subsubsection{第一类换元法(凑微分法)}

\textcolor{aqua}{\textbf{定理:}}$\int f(u)\,\rm{d}$$u=F(u)+C$,则$\int f[\varphi(x)]\varphi'(x)\,\rm{d}$$x=\int f[\varphi(x)]\,\rm{d}$$\varphi(x)=F[\varphi(x)]+C$。

$\int\dfrac{\rm{d}\textit{x}}{a^2+x^2}=\dfrac{1}{a}\int\dfrac{\rm{d}\dfrac{\textit{x}}{\textit{a}}}{1+\left(\dfrac{x}{a}\right)^2}=\dfrac{1}{a}\arctan\dfrac{x}{a}+C$,$\int\dfrac{\rm{d}\textit{x}}{\sqrt{a^2-x^2}}=\int\dfrac{\rm{d}\dfrac{\textit{x}}{\textit{a}}}{\sqrt{1-\left(\dfrac{x}{a}\right)^2}}=\arcsin\dfrac{x}{a}+C$。

如$\int\dfrac{x}{\sqrt{1+x^2}}\rm{d}\textit{x}=\dfrac{1}{2}\int\dfrac{\rm{d}(1+\textit{x}^2)}{\sqrt{1+x^2}}=\sqrt{1+x^2}+C$。\medskip

凑微分法适用于式子比较简单的情况,所凑微分的形式必须符合一个简单积分公式的式子,且有一定的式子可以提出来到微分号后面。

\textbf{例题:}

$\int(1+3x)^{100}\,\rm{d}$$x=\dfrac{1}{3}\int(1+3x)^{100}\,\rm{d}$$(1+3x)=\dfrac{1}{303}(1+3x)^{101}+C$。

$\int\cos^2x\,\rm{d}$$x=\dfrac{1}{2}\int(1+\cos 2\textit{x})\,\rm{d}$$x=\dfrac{1}{2}\left(x+\dfrac{1}{2}\sin 2x\right)+C$。

$\int\cos^3x\,\rm{d}$$x=\int\cos^2\,\rm{d}$$\sin x=\int(1-\sin^2x)\,\rm{d}$$\sin x=\sin x-\dfrac{1}{3}\sin^3x+C$。

$\int\dfrac{\rm{d}\textit{x}}{x\sqrt{1+\ln x}}=\int\dfrac{\rm{d}(1+\ln\textit{x})}{\sqrt{1+\ln x}}=2\sqrt{1+\ln x}+C$。

$\int\dfrac{\rm{d}\textit{x}}{\sqrt{x}(1+x)}=2\int\dfrac{\rm{d}\sqrt{\textit{x}}}{1+(\sqrt{x})^2}=2\arctan\sqrt{x}+C$。

$\int\dfrac{\arcsin\sqrt{x}}{\sqrt{x(1-x)}}\,\rm{d}$$x=\int\dfrac{\arcsin\sqrt{x}}{1-x}\cdot\dfrac{\rm{d}\textit{x}}{\sqrt{x}}$$=2\int\dfrac{\arcsin\sqrt{x}}{1-(\sqrt{x})^2}\,\rm{d}$$\sqrt{x}$

$=2\int\arcsin\sqrt{x}\,\rm{d}$$\arcsin\sqrt{x}=(\arcsin\sqrt{x})^2+C$。

\subsubsection{第二类换元法}

\textcolor{aqua}{\textbf{定理:}}设$x=\varphi(t)$为单调可导函数,且$\varphi'(t)\neq 0$,$\int f[\varphi(t)\varphi'(t)]\,\rm{d}$$t=F(t)+C$,则$\int f(x)\rm{d}$$x=\int f[\varphi(t)\varphi'(t)]\,\rm{d}$$t=F(t)+C=F[\varphi^{-1}(x)]+C$。

第二类换元法适用于无法适用第一类换元法的情况,但是最重要的还是对于中间变量的取值,这个中间变量必须要让原式子更简单,且还要注意到变量取值范围。

\textcolor{orange}{注意:}$\varphi'(t)\neq 0$是为了保证中间变量函数具有反函数,而单调函数必然有反函数,所以只要能证明这个中间变量函数必然单调,那么其实$\varphi'(t)$也可以等于0。

\textbf{例题:}求$\int\sqrt{a^2-x^2}\,\rm{d}$$x(a>0)$。

首先看题目,如果使用凑微分法,那必须从式子中提取出一个式子放到微分后面,且提取后的式子满足一个简单的积分公式。

这个式子一般就只能提取出$x$到平方号外面,但是提取后式子仍不能变为一个简单微分公式,所以说第一种凑微分法就无法使用,就只能使用第二类换元法。

这个式子是一个平方取开平方的式子,所以取中间变量后最好让这个式子能被开平方。又涉及到一个常数$a$,所以我们很容易就想到是否可以通过三角函数来作为中间变量。

所以取$x=a\sin t$,从而$\sqrt{a^2-x^2}=a\cos t$。

并且还要注意到这个$t$的取值范围。

因为$x=\varphi(t)$是一个单调可导的函数。所以$\sin t$必须取在单调区间上。

又$\sqrt{a^2-x^2}$要求$-a\leqslant x\leqslant a$,$-a\leqslant a\sin t\leqslant a$,从而$-1\leqslant\sin t\leqslant 1$。

且$\varphi'(t)\neq 0$,所以$\cos t\neq 0$。

所以综上三个条件从而得到一个$t$的定义域:$t\in\left[-\dfrac{\pi}{2},0\right)\bigcup\left(0,\dfrac{\pi}{2}\right]$。

但是在$\left[-\dfrac{\pi}{2},\dfrac{\pi}{2}\right]$上$\varphi'(t)=a\sin t$是严格单调递增的,单调函数必然存在反函数,所以$\varphi'(t)$可以等于0,从而$t\in\left[-\dfrac{\pi}{2},\dfrac{\pi}{2}\right]$。

$\int\sqrt{a^2-x^2}\,\rm{d}$$x=a\int\cos t\,\rm{d}$$a\sin t=a^2\int\cos^2t\rm{d}$$t=\dfrac{a^2}{2}\int(1+\cos 2t)\rm{d}$$t=\dfrac{a^2}{2}\left(t+\dfrac{1}{2}\sin 2t\right)+C=\dfrac{a^2}{2}\left(\arcsin\dfrac{x}{a}+\dfrac{x}{a}\sqrt{1-\dfrac{x^2}{a^2}}\right)+C$。


\subsection{分部积分法}

\section{定积分}
\end{document}
