\documentclass[UTF8, 12pt]{ctexart}
% UTF8编码,ctexart现实中文
\usepackage{color}
% 使用颜色
\definecolor{orange}{RGB}{255,127,0} 
\definecolor{violet}{RGB}{192,0,255} 
\definecolor{aqua}{RGB}{0,255,255} 
\usepackage{geometry}
\setcounter{tocdepth}{4}
\setcounter{secnumdepth}{4}
% 设置四级目录与标题
\geometry{papersize={21cm,29.7cm}}
% 默认大小为A4
\geometry{left=3.18cm,right=3.18cm,top=2.54cm,bottom=2.54cm}
% 默认页边距为1英尺与1.25英尺
\usepackage{indentfirst}
\setlength{\parindent}{2.45em}
% 首行缩进2个中文字符
\usepackage{setspace}
\renewcommand{\baselinestretch}{1.5}
% 1.5倍行距
\usepackage{amssymb}
% 因为所以
\usepackage{amsmath}
% 数学公式
\usepackage[colorlinks,linkcolor=black,urlcolor=blue]{hyperref}
% 超链接
\usepackage{pifont}
% 圆圈序号
\author{Didnelpsun}
\title{无穷级数}
\date{}
\begin{document}
\maketitle
\pagestyle{empty}
\thispagestyle{empty}
\tableofcontents
\thispagestyle{empty}
\newpage
\pagestyle{plain}
\setcounter{page}{1}
\section{常数项级数}

\subsection{概念}

级数的经典悖论为芝诺悖论。

\subsubsection{基本概念}

\textcolor{violet}{\textbf{定义:}}给定义一个无穷数列$u_1,u_2,\cdots,u_n,\cdots$,将其各项用加号连起来的得到的记号$\sum\limits_{n=1}^\infty u_n$,即$\sum\limits_{n=1}^\infty u_n=u_1+u_2+\cdots+u_n+\cdots$叫做\textbf{无穷级数},简称\textbf{级数},其中$u_n$称为该级数的\textbf{通项}。

若$u_n$是常数而不是函数,则$\sum\limits_{n=1}^\infty u_n$就被称为\textbf{常数项无穷级数},简称\textbf{常数项级数}。

$S_n=u_1+u_2+\cdots+u_n$称为级数的\textbf{部分和},$\{S_n\}$是级数的\textbf{部分和数量}。

\textcolor{violet}{\textbf{定义:}}若$\lim\limits_{n\to\infty}S_n=S$,则$\sum\limits_{n=1}^\infty u_n$\textbf{收敛},并称$S$为该收敛级数$\sum\limits_{n=1}^\infty u_n$的\textbf{和};若$\lim\limits_{n\to\infty}S_n$不存在或为$\pm\infty$,则$\sum\limits_{n=1}^\infty u_n$\textbf{发散}。

研究$\sum\limits_{n=1}^\infty u_n$收敛还是发散,就是研究级数$\sum\limits_{n=1}^\infty u_n$的敛散性。

在级数$\sum\limits_{n=1}^\infty u_n$去掉前$m$项,得$\sum\limits_{n=m+1}^\infty u_n=u_{m+1}+u_{m+2}+\cdots$,称为级数$\sum\limits_{n=1}^\infty u_n$的\textbf{$m$项后余项}

\subsubsection{性质}

\begin{enumerate}
    \item 线性性质:若级数$\sum\limits_{n=1}^\infty u_n$,$\sum\limits_{n=1}^\infty v_n$均收敛,且其和分别为$S$,$T$,则任给常数$a,b$,有$\sum\limits_{n=1}^\infty(au_n+bv_n)$也收敛,且其和为$aS+bT$,即$\sum\limits_{n=1}^\infty(au_n+bv_n)=a\sum\limits_{n=1}^\infty u_n+b\sum\limits_{n=1}^\infty v_n$。(收敛±发散=发散,发散±发散=不确定)
    \item 若级数$\sum\limits_{n=1}^\infty u_n$收敛,则其任意$m$项后余项$\sum\limits_{n=m+1}^\infty u_n$也收敛;若存在$m$项后余项$\sum\limits_{n=m+1}^\infty u_n$收敛,则$\sum\limits_{n=1}^\infty u_n$也收敛。
    \item 对收敛级数加括号仍然收敛,但是加括号收敛原级数不一定收敛。如果原级数加括号发散,则原级数发散。
    \item 级数收敛必要条件:若级数$\sum\limits_{n=1}^\infty u_n$收敛,则$\lim\limits_{n\to\infty}u_n=0$。
\end{enumerate}

证明性质三:$u_n=S_n-S_{n-1}$,所以$\lim\limits_{n\to\infty}=\lim\limits_{n\to\infty}(S_n-S_{n-1})=\lim\limits_{n\to\infty}S_n-\lim\limits_{n\to\infty}S_{n-1}=S-S=0$。极限为0不一定收敛。

\subsection{级数敛散性判别}

\subsubsection{正项级数}

\paragraph{概念} \leavevmode \medskip

\textcolor{violet}{\textbf{定义:}}若通项$u_n\geqslant0$,$n=1,2,\cdots$,则$\sum\limits_{n=1}^\infty u_n$为\textbf{正项级数}。

所以和项一定是递增的,由数列极限的单调有界准则如果和有上界则极限存在。

\paragraph{收敛原则} \leavevmode \medskip

\textcolor{aqua}{\textbf{定理:}}正项级数$\sum\limits_{n=1}^\infty u_n$收敛的充要条件是其部分和数列$\{S_n\}$有界。(某一函数在固定区间内变化率是有界的,则变化范围是有界的)

证明:必要性:由于$u_n\geqslant0$,$\therefore S_n=u_1+u_2+\cdots+u_n\geqslant0$,且$S_1\leqslant S_2\leqslant\cdots\leqslant S_n\leqslant\cdots$,$\{S_n\}$单调不减且下界为0。当$\sum\limits_{n=1}^\infty u_n$收敛时,$\lim\limits_{n\to\infty}S_n$存在,则$\{S_n\}$必有上界。有上界下界则$\{S_n\}$有界。(某一函数在固定区间内变化率是有界的,则变化范围是有界的)

充分性:由于$\{S_n\}$单调不减,所以根据单调有界准则,$\{S_n\}$收敛,即$\lim\limits_{n\to\infty}S_n$存在,于是$\sum\limits_{n=1}^\infty u_n$收敛。

基本就是使用放缩法判断是否有界。

\textbf{例题:}判断级数$\sum\limits_{n=1}^\infty\dfrac{1}{\sqrt{n}}$的敛散性。

解:$S_n=1+\dfrac{1}{\sqrt{2}}+\dfrac{1}{\sqrt{3}}+\cdots+\dfrac{1}{\sqrt{n}}>n\dfrac{1}{\sqrt{n}}=\sqrt{n}$,当$n\to\infty$时$\sqrt{n}\to\infty$,无上界所以发散。

由于收敛原则很多时候都不能方便使用,所以出现了以下几种解决方法。

\paragraph{比较判别法} \leavevmode \medskip

\textcolor{aqua}{\textbf{定理:}}给出两个正项级数$\sum\limits_{n=1}^\infty u_n$,$\sum\limits_{n=1}^\infty v_n$,若从某项开始有$u_n\leqslant v_n$成立,则:\ding{172}若$\sum\limits_{n=1}^\infty v_n$收敛,则$\sum\limits_{n=1}^\infty u_n$也收敛;\ding{173}若$\sum\limits_{n=1}^\infty u_n$发散,则$\sum\limits_{n=1}^\infty v_n$也发散。

即大的收敛小的也收敛,小的发散大的也发散。

\textbf{例题:}判断调和级数$\sum\limits_{n=1}^\infty\dfrac{1}{n}$的敛散性。

解:$\because x>0$,$x>\ln(1+x)$,$\therefore\dfrac{1}{n}>\ln\left(1+\dfrac{1}{n}\right)$。

又对于$\ln\left(1+\dfrac{1}{n}\right)=\ln\dfrac{n+1}{n}=\ln(n+1)-\ln n$。

$S_n=\ln\dfrac{2}{1}+\ln\dfrac{3}{2}+\cdots+\ln\dfrac{n+1}{n}=\ln2-\ln1+\ln3-\ln2+\cdots+\ln(n+1)-\ln n=\ln(n+1)$。当$\ln(n+1)$在$n\to\infty$时,$S_n\to\infty$。

所以$\sum\limits_{n=1}^\infty\ln\left(1+\dfrac{1}{n}\right)$发散,则$\sum\limits_{n=1}^\infty\dfrac{1}{n}$也发散。

\paragraph{比较判别法极限性质} \leavevmode \medskip

是比较判别法的推论,利用极限的阶数来比较。

给出两个正项级数$\sum\limits_{n=1}^\infty u_n$,$\sum\limits_{n=1}^\infty v_n$,$v_n\neq0$,且$\lim\limits_{n\to\infty}\dfrac{u_n}{v_n}=A$:

\begin{enumerate}
    \item 若$A=0$,则当$\sum\limits_{n=1}^\infty v_n$收敛时,$\sum\limits_{n=1}^\infty u_n$也收敛。
    \item 当$A=+\infty$,当$\sum\limits_{n=1}^\infty v_n$发散时,$\sum\limits_{n=1}^\infty u_n$也发散。
    \item 若$0<A<+\infty$,则$\sum\limits_{n=1}^\infty v_n$与$\sum\limits_{n=1}^\infty u_n$具有相同敛散性。
\end{enumerate}

\textbf{例题:}判断$\sum\limits_{n=1}^\infty\left(\dfrac{1}{n}-\sin\dfrac{1}{n}\right)$敛散性。

解:令$\dfrac{1}{n}=x$,$n\to\infty$所以$x\to0^+$。当$x\to0^+$,$x-\sin x\sim\dfrac{1}{6}x^3$。

$\therefore\lim\limits_{n\to\infty}\dfrac{\dfrac{1}{n}-\sin\dfrac{1}{n}}{\dfrac{1}{n^3}}=\dfrac{1}{6}\neq0$,所以$\dfrac{1}{n}-\sin\dfrac{1}{n}$与$\dfrac{1}{n^3}$具有相同敛散性。

根据$p$级数定理,$p=3>1$,所以$\sum\limits_{n=1}^\infty\left(\dfrac{1}{n}-\sin\dfrac{1}{n}\right)$收敛。

\paragraph{比值判别法} \leavevmode \medskip

也称为达朗贝尔判别法。由等比级数推断出部分级数的敛散性只与自己的参数有关,根据自己的通项的商进行比较。

\textcolor{aqua}{\textbf{定理:}}给出一正项级数$\sum\limits_{n=1}^\infty u_n$,若$\lim\limits_{n\to\infty}\dfrac{u_{n+1}}{u_n}=\rho$,则:\ding{172}若$\rho<1$,则$\sum\limits_{n=1}^\infty u_n$收敛;\ding{173}若$\rho>1$,则$\sum\limits_{n=1}^\infty u_n$发散。

\textcolor{orange}{注意:}$\rho=1$时无法根据此判断$\sum\limits_{n=1}^\infty u_n$敛散性,如$\sum\limits_{n=1}^\infty\dfrac{1}{n}$发散,但$\sum\limits_{n=1}^\infty\dfrac{1}{n^2}$收敛。

\textbf{例题:}判断级数$\sum\limits_{n=1}^\infty\dfrac{\vert a\vert^nn!}{n^n}$的敛散性,其中$a$为非零常数。

解:记$u_n=\dfrac{\vert a\vert^nn!}{n^n}$,$\lim\limits_{n\to\infty}\dfrac{u_{n+1}}{u_n}=\vert a\vert\lim\limits_{n\to\infty}\left(\dfrac{n}{n+1}\right)^n=\vert a\vert e^{\lim\limits_{n\to\infty}n\ln\frac{n}{n+1}}=\vert a\vert e^{\lim\limits_{n\to\infty}n(\frac{n}{n+1}-1)}=\vert a\vert e^{\lim\limits_{n\to\infty}(\frac{-n}{n+1}-1)}=\vert a\vert e^{-1}=\dfrac{\vert a\vert}{e}$。

若$0<\vert a\vert<e$,所以收敛;若$\vert a\vert>e$,所以发散;若$\vert a\vert=e$,则回代得到比值$e\left(\dfrac{n}{n+1}\right)^n=\dfrac{e}{(1+\dfrac{1}{n})^n}\to1^+$,且$u_1=e$,$\therefore u_n>u_1>0$,所以发散。

\paragraph{根值判别法} \leavevmode \medskip

也称为柯西判别法。由比值判别法类比而来。

\textcolor{aqua}{\textbf{定理:}}给出正项级数$\sum\limits_{n=1}^\infty u_n$,若$\lim\limits_{n\to\infty}\sqrt[n]{u_n}=\rho$,则\ding{172}若$\rho<1$,则$\sum\limits_{n=1}^\infty u_n$收敛;若$\rho>1$,则$\sum\limits_{n=1}^\infty u_n$发散。

同理$\rho=1$也会失效。

\textbf{例题:}判断级数$\sum\limits_{n=1}^\infty\left(n\sin\dfrac{1}{n}\right)^{n^3}$的敛散性。

解:记$u=\left(n\sin\dfrac{1}{n}\right)^{n^3}$,则$\lim\limits_{n\to\infty}\sqrt[n]{u_n}=\lim\limits_{n\to\infty}\left(n\sin\dfrac{1}{n}\right)^{n^2}=e^{\lim\limits_{n\to\infty}n^(n\sin\frac{1}{n}-1)}=e^{\lim\limits_{n\to\infty}\dfrac{\sin\frac{1}{n}-\frac{1}{n}}{\frac{1}{n^3}}}=e^{-\frac{1}{6}}<1$,所以收敛。

\paragraph{积分判别法} \leavevmode \medskip

\textcolor{aqua}{\textbf{定理:}}设$f(x)$是在$[1,+\infty)$上单调递减且非负的连续函数,$a_n=f(n)$,则$\sum\limits_{n=1}^\infty a_n$与$\int_1^{+\infty}f(x)\,\textrm{d}x$同敛散。

\textbf{例题:}证明$p$级数$\sum\limits_{n=1}^\infty\dfrac{1}{n^p}$当$p>1$的敛散性。

证明:令$f(x)=\dfrac{1}{x^p}$,又$\int_1^{+\infty}\dfrac{\textrm{d}x}{x^p}$在$p>1$收敛,在$p\leqslant1$发散,所以得到原级数敛散性。

\subsubsection{交错级数}

\paragraph{概念} \leavevmode \medskip

\textcolor{violet}{\textbf{定义:}}若级数各项\textbf{正负相间}出现,则这样的级数是\textbf{交错级数},一般写为$\sum\limits_{n=1}^\infty(-1)^{n-1}u_n=u_1-u_2+u_3-u_4+\cdots+(-1)^{n-1}u_n+\cdots$,其中$u_n>0$。

\paragraph{莱布尼兹判别法} \leavevmode \medskip

\textcolor{violet}{\textbf{定义:}}给出一交错级数$\sum\limits_{n=1}^\infty(-1)^{n-1}u_n$,$u_n>0$,$n=1,2,\cdots$,若$\{u_n\}$\textbf{单调不增}$u_n\geqslant u_{n+1}$且$\lim\limits_{n\to\infty}=0$,则该级数收敛。反过来则不行。

如$\sum\limits_{n=1}^\infty\dfrac{(-1)^{n-1}}{2^{n+(-1)^n}}$收敛,但是里面的$u_n$并不递减,由根值判别法加绝对值可知$u_n$为不递减。

\textbf{例题:}判断交错调和级数$\sum\limits_{n=1}^\infty(-1)^{n-1}\dfrac{1}{n}$的敛散性。

解:$\because\lim\limits_{n\to\infty}u_n=\lim\limits_{n\to\infty}\dfrac{1}{n}=0$。

且$\dfrac{1}{n}>\dfrac{1}{n+1}$,所以级数收敛。

\textbf{例题:}判断级数$\sum\limits_{n=1}^\infty\sin(\pi\sqrt{n^2+a^2})$的敛散性,其中$a$为非零常数。

解:$\because\sin(\alpha+n\pi)=(-1)^n\sin\alpha$。$\therefore\sin(\pi\sqrt{n^2+a^2})=$

$\sin(\pi\sqrt{n^2+a^2-n\pi+n\pi})=(-1)^n\sin\left(\dfrac{a^2\pi}{\sqrt{n^2+a^2}+n}\right)$。

记$u_n=\sin\left(\dfrac{a^2\pi}{\sqrt{n^2+a^2}+n}\right)$,又$n\to\infty$时$\dfrac{a^2\pi}{\sqrt{n^2+a^2}+n}\to0^+$且单调不增,$\sin x$在$x\to0^+$时也是单调函数,所以$\lim\limits_{n\to\infty}u_n=0$且单调不增。

所以收敛。

\textbf{例题:}判断级数$\sum\limits_{n=1}^\infty(-1)^n\dfrac{\ln(1+n)}{1+n})$的敛散性。

解:$\lim\limits_{n\to\infty}u_n=\lim\limits_{n\to\infty}\dfrac{\ln(1+n)}{1+n}=\lim\limits_{x\to+\infty}\dfrac{\ln(1+x)}{1+x}=\lim\limits_{x\to+\infty}\dfrac{1}{1+x}=0$。

对$\dfrac{\ln(1+n)}{1+n})$进行比较有些麻烦,所以令$f(x)=\dfrac{\ln(1+x)}{1+x})$。

$f'(x)=\dfrac{1-\ln(1+x)}{(1+x)^2}$,当$x\to+\infty$时,$f'(x)<0$,$\{u_n\}$单调减少,所以收敛。

\subsubsection{任意项级数}

\paragraph{概念} \leavevmode \medskip

\textcolor{violet}{\textbf{定义:}}若级数$\sum\limits_{n=1}^\infty(-1)^{n-1}u_n$各项可为正可为负,可为零,则这种级数就是\textbf{任意项级数}。

给任意项级数每一项加上绝对值$\sum\limits_{n=1}^\infty(-1)^{n-1}\vert u_n\vert$,就得到了正项级数,称为原级数的\textbf{绝对值级数}。

\paragraph{绝对收敛} \leavevmode \medskip

\textcolor{violet}{\textbf{定义:}}设$\sum\limits_{n=1}^\infty(-1)^{n-1}u_n$为任意项级数,若$\sum\limits_{n=1}^\infty(-1)^{n-1}\vert u_n\vert$收敛,则称$\sum\limits_{n=1}^\infty$\\$(-1)^{n-1}u_n$\textbf{绝对收敛}。

\paragraph{条件收敛} \leavevmode \medskip

\textcolor{violet}{\textbf{定义:}}设$\sum\limits_{n=1}^\infty(-1)^{n-1}u_n$为任意项级数,若$\sum\limits_{n=1}^\infty(-1)^{n-1}u_n$收敛,但$\sum\limits_{n=1}^\infty(-1)^{n-1}$\\$\vert u_n\vert$发散,则称$\sum\limits_{n=1}^\infty(-1)^{n-1}u_n$\textbf{条件收敛}。

\textcolor{aqua}{\textbf{定理:}}若$\sum\limits_{n=1}^\infty(-1)^{n-1}\vert u_n\vert$收敛,则$\sum\limits_{n=1}^\infty(-1)^{n-1}u_n$必收敛。(绝对收敛则收敛)

\textcolor{aqua}{\textbf{定理:}}收敛级数的项任意加括号后所得的新级数仍收敛,且其和不变。

\textcolor{aqua}{\textbf{定理:}}若原级数绝对收敛,不论将其项如何排列,则所得的新级数也收敛,且其和不变。(绝对收敛的级数具有可交换性)

\textcolor{aqua}{\textbf{定理:}}条件收敛的级数的所有正项(或负项)构成的级数一定发散。

\textcolor{aqua}{\textbf{定理:}}$\sum\limits_{n=1}^\infty\vert b_n\vert$收敛,则$\sum\limits_{n=1}^\infty b_n^2$收敛。($b_n$收敛则不能得到)

\textbf{例题:}若级数$\sum\limits_{n=1}^\infty u_n$收敛,则下面级数必收敛的是()。

$A.\sum\limits_{n=1}^\infty(-1)\dfrac{u_n}{n}$\qquad$B.\sum\limits_{n=1}^\infty u_n^2$\qquad$C.\sum\limits_{n=1}^\infty(u_{2n-1}-u_{2n})$\qquad$D.\sum\limits_{n=1}^\infty(u_n+u_{n+1})$

解:对于$A$,取$u_n=(-1)^n\dfrac{1}{\ln n}$,则原来$\dfrac{u_n}{n}=(-1)^n\dfrac{1}{\ln n}$收敛,但是乘上$(-1)^n$就不一定收敛,得到$\dfrac{1}{n\ln n}$。

\textcolor{aqua}{\textbf{定理:}}广义$p$级数:$\sum\limits_{n=2}^\infty\dfrac{1}{n(\ln n)^p}\left\{\begin{array}{l}
    p>1, \text{收敛} \\
    p\leqslant1, \text{发散}
\end{array}\right.$。($n=1$无意义,从$n=2$开始不影响其敛散性)

所以$A$发散。

对于$B$,取$u_n=(-1)^n\dfrac{1}{\sqrt{n}}$,则$u_n^2=\dfrac{1}{n}$,调和级数不收敛。

对于$C$,取$u_n=(-1)^{n-1}\dfrac{1}{n}$,则得到$u_{2n-1}-u_{2n}=\dfrac{1}{n}$,调和级数不收敛。

对于$D$,由于$u_n$收敛,则$u_{n+1}$也收敛,所以相加也收敛,选$D$。

\textcolor{aqua}{\textbf{定理:}}若$u_n^2$收敛,则$\dfrac{u_n}{n}$绝对收敛。

证明:因为不等式$\vert a\vert\vert b\vert\leqslant\dfrac{\vert a\vert^2+\vert b\vert^2}{2}$,$\therefore0\leqslant\vert u_n\dfrac{1}{n}\vert\leqslant\dfrac{u_n^2+\dfrac{1}{n^2}}{2}$。

且$u_n^2$收敛,则$\dfrac{u_n^2+\dfrac{1}{n^2}}{2}$也收敛,根据性质得证。

\section{幂级数}

\subsection{概念}

\subsubsection{定义}

\textcolor{violet}{\textbf{定义:}}设函数列$\{u_n(x)\}$定义在区间$I$上,称$u_1(x)+u_2(x)+\cdots+u_n(x)+\cdots$为定义在区间$I$上的\textbf{函数项级数},记为$\sum\limits_{n=1}^\infty u_n(x)$,当$x$取确定的值$x_0$时,$\sum\limits_{n=1}^\infty$成为常数项级数$\lim\limits_{n=1}^\infty u_n(x_0)$。

\textcolor{violet}{\textbf{定义:}}若$\sum\limits_{n=1}^\infty u_0(x)$的一般项$u_0(x)$为$n$次幂函数,则称$\sum\limits_{n=1}^\infty u_0(x)$为\textbf{幂级数},是一种常用的函数项级数,一般形式为$\sum\limits_{n=0}^\infty a_n(x-x_0)^n=a_0+a_1(x-x_0)+a_2(x-x_0)^2+\cdots+a_n(x-x_0)^n+\cdots$,其标准形式为$\sum\limits_{n=0}^\infty a_nx^n=a_0+a_1x+a_2x^2+\cdots+a_nx^x+\cdots$,其中$a_n$($n=0,1,2,\cdot$)为\textbf{幂级数的系数}。

幂级数也称为泰勒级数,与泰勒展开式一样的结构。

\textcolor{violet}{\textbf{定义:}}若给定$x_0\in I$,有$\sum\limits_{n=1}^\infty u_0(x)$收敛,则称点$x_0$为幂级数$\sum\limits_{n=1}^\infty u_0(x)$的\textbf{收敛点};若给定$x_0\in I$,有$\sum\limits_{n=1}^\infty u_0(x)$发散,则点$x_0$为幂级数$\sum\limits_{n=1}^\infty u_0(x)$的\textbf{发散点}。

\subsubsection{阿贝尔定理}

\textcolor{violet}{\textbf{定义:}}当幂级数$\sum\limits_{n=0}^\infty a_nx^n$在点$x=x_1$($x_1\neq0$)处收敛时,对于满足$\vert x\vert<\vert x_1\vert$的一切$x$,幂级数\textbf{绝对收敛};当幂级数$\sum\limits_{n=0}^\infty a_nx^n$在$x=x_2$($x_2\neq0$)处发散时,对于满足$\vert x\vert>\vert x_2\vert$的一切$x$,幂级数\textbf{发散}。

所以一定存在一个点$R$,在$\vert x\vert<\vert R\vert$中绝对收敛,在$\vert x\vert>\vert R\vert$中发散,$R$称为\textbf{收敛半径}。对于点$\pm R$需要代入幂级数变成常数项级数进行计算,判别其敛散性。

\subsubsection{收敛域}

\textcolor{violet}{\textbf{定义:}}函数项级数$\sum\limits_{n=1}^\infty u_0(x)$的所有收敛点的集合就是其\textbf{收敛域}。

\paragraph{具体型} \leavevmode \medskip

收敛域的求法:

\begin{enumerate}
    \item 若$\lim\limits_{n\to\infty}\left\vert\dfrac{a_{n+1}}{a_n}\right\vert$或$\lim\limits_{n\to\infty}\sqrt[n]{\vert a_n\vert}=\rho$,则收敛半径$R=\left\{\begin{array}{ll}
        \dfrac{1}{\rho}, & \rho\neq0 \\
        +\infty, & \rho=0 \\
        0, & \rho=+\infty
    \end{array}\right.$。
    \item 开区间$(-R,R)$为幂级数$\sum\limits_{n=0}^\infty a_nx^n$的收敛区间。
    \item 代入$R$判断该点的敛散性,最后组合得到收敛域。
\end{enumerate}

但是这种方法有一点不方便,如若只知道$a_n$和$a_{n+2}$的关系则求$\rho=\dfrac{1}{R}$比较麻烦。

收敛域的统一求法:

\begin{enumerate}
    \item 取绝对值$\vert u_0(x)\vert\geqslant0$,从而可以使用正项级数的判别法。
    \item 根据比值判别法或根值判别法,求$\lim\limits_{n\to\infty}\dfrac{\vert u_{n+1}(x)\vert}{\vert u_n(x)\vert}=\rho$或$\lim\limits_{n\to\infty}\sqrt[n]{\vert u_n(x)\vert}=\rho$,令其小于1,得到收敛区间$x\in(a,b)$。
    \item 单独讨论$x=a$,$x=b$处的敛散性,得到收敛域。
\end{enumerate}

\textcolor{aqua}{\textbf{定理:}}若幂级数$\sum\limits_{n=0}^\infty a_nx^n$在点$x=x_0$处条件收敛,则点$x_0$在幂级数收敛区间的端点上。

\textbf{例题:}求幂级数$\sum\limits_{n=1}^\infty\dfrac{x^n}{n}$的收敛域。

解:令$\vert u_n(x)\vert=\left\vert\dfrac{x^n}{n}\right\vert$。由于含有$x^n$,所以使用比值判别法。

$\therefore\lim\limits_{n\to\infty}\dfrac{\vert u_{n+1}(x)\vert}{\vert u_n(x)\vert}=\lim\limits_{x\to\infty}\dfrac{\vert x^{n+1}\vert}{n+1}\dfrac{n}{\vert x^n\vert}=\lim\limits_{n\to\infty}\dfrac{n}{n+1}\vert x\vert=\vert x\vert$。\medskip

令其小于1,即$\vert x\vert<1$,$-1<x<1$。

当$x=-1$时,$\sum\limits_{n=1}^\infty(-1)^\dfrac{1}{n}$收敛。当$x=1$,$\sum\limits_{n=1}^\infty\dfrac{1}{n}$发散。

所以$x$收敛域为$[-1,1)$。

\paragraph{抽象型} \leavevmode \medskip

\textcolor{aqua}{\textbf{定理:}}根据阿贝尔定理,已知$\lim\limits_{n=0}^\infty a_0(x-x_0)^n$在某点$x_1$($x_1\neq x_0$)的敛散性,确定该幂级数的收敛半径可分为三种情况:

\begin{enumerate}
    \item 若在$x_1$处收敛,则收敛半径$R\geqslant\vert x_1-x_0\vert$。
    \item 若在$x_1$处发散,则收敛半径$R\leqslant\vert x_1-x_0\vert$。
    \item \textcolor{orange}{注意:}若在$x_1$处条件收敛,则$R=\vert x_1-x_0\vert$。
\end{enumerate}

\textcolor{aqua}{\textbf{定理:}}已知$\sum a_n(x-x_1)^n$的敛散性,讨论$\sum b_n(x-x_2)^m$的敛散性:

\begin{enumerate}
    \item $(x-x_1)^n$与$(x-x_2)^m$的转换一般通过初等变形来完成,包括\ding{172}平移收敛区间;\ding{173}提出或乘以因式$(x-x_0)^k$等。
    \item $a_n$与$b_n$的转换一般通过微积分变形来完成,包括\ding{172}对级数逐项求导;\ding{173}对级数逐项积分等。
    \item 以下三种情况,级数收敛半径不变,收敛域要具体代入点讨论:\begin{enumerate}
        \item 对级数提出或乘以因式$(x-x_0)^k$或进行平移等,收敛半径不变。
        \item 对级数逐项求导,收敛半径不变,收敛域可能缩小。
        \item 对级数逐项积分,收敛半径不变,收敛域可能扩大。
    \end{enumerate}
\end{enumerate}

\textbf{例题:}设$\sum\limits_{n=1}^\infty a_n(x+1)^n$在点$x=1$处条件收敛,则幂级数$\sum\limits_{n=1}^\infty na_n(x-1)^n$在点$x=2$处()。

$A.$绝对收敛\qquad$B.$条件收敛\qquad$C.$发散\qquad$D.$敛散性不确定

解:$\sum\limits_{n=1}^\infty a_n(x+1)^n=\sum\limits_{n=1}^\infty a_n(x-(-1))^n$,所以$x_0=-1$。

又$x=1$处条件收敛,所以$R=1-(-1)=2$。从而$\sum\limits_{n=1}^\infty a_n(x+1)^n$的收敛区间为$(-3,1)$。

$\sum\limits_{n=1}^\infty a_n(x+1)^n$要转换为$\sum\limits_{n=1}^\infty na_n(x-1)^n$,则首先中心点要从-1移动到1,$a_n(x+1)^n\to a_n(x-1)^n$,由于平移不改变收敛半径,所以$a_n(x-1)^n$收敛区间为$(1,3)$。

然后要将$a_n(x-1)^n$变为$na_n(x-1)^n$,需要进行求导得到$na_n(x-1)^{n-1}$,求导收敛半径不变,所以收敛区间依然为$(1,3)$。最后还要乘上$(x-1)$得到$na_n(x-1)^n$就是所求,收敛区间依然为$(1,3)$。

而在$x=2$在收敛区间内,必然绝对收敛,所以选$A$。

\subsection{函数展开为幂级数}

\subsubsection{概念}

\textcolor{violet}{\textbf{定义:}}若函数$f(x)$在$x=x_0$处存在任意阶导数,则称$f(x_0)+f'(x_0)(x-x_0)+\dfrac{f''(x_0)}{2!}(x-x_0)^2+\cdots+\dfrac{f^{(n)}(x_0)}{n!}(x-x_0)^n+\cdots$为函数$f(x)$在$x_0$处的\textbf{泰勒级数},则$f(x)=\sum\limits_{n=0}^\infty\dfrac{f^{(n)}(x_0)}{n!}(x-x_0)^n$。

当$x_0=0$时,称$f(0)+f'(0)x+\dfrac{f''(0)}{2!}x^2+\cdots+\dfrac{f^{(n)}(0)}{n!}x^n+\cdots$为函数$f(x)$的\textbf{麦克劳林级数},若收敛,则$f(x)=\sum\limits_{n=0}^\infty\dfrac{f^{(n)}(0)}{n!}x^n$。

都是函数展开成幂级数。

\textcolor{aqua}{\textbf{定理:}}已知$f(x)$在$x=x_0$处任意阶可导,则$\sum\limits_{n=0}^\infty\dfrac{f^{(n)}(x_0)}{n!}(x-x_0)^n$在$(x_0-R,x_0+R)$上收敛于$f(x)$与$\lim\limits_{n\to\infty}R_n(x)=0$等价。其中$R_n(x)=\dfrac{f^{(n+1)}(\xi)}{(n+1)!}(x-x_0)^{n+1}$在$f(x)$在$x_0$处的泰勒公式$f(x)=\sum\limits_{k=0}^n\dfrac{f^{(n)}(x_0)}{k!}(x-x_0)^k+R_n(x)$中的余项。

\subsubsection{重要展开式}

$x$的取值指其幂指数的收敛域。第七个幂函数问题较复杂,收敛区间与$\alpha$取值有关。

\begin{enumerate}
    \item $e^x=\sum\limits_{n=0}^\infty\dfrac{x^n}{n!}=1+x+\dfrac{x^2}{2!}+\dfrac{x^n}{n!}+\cdots$,$-\infty<x<+\infty$。
    \item $\dfrac{1}{1+x}=\sum\limits_{n=0}^\infty(-1)^nx^n=1-x+x^2-x^3+\cdots+(-1)^nx^n+\cdots$,$-1<x<1$。
    \item $\dfrac{1}{1-x}=\sum\limits_{n=0}^\infty x^n=1+x+x^2+\cdots+x^n+\cdots$,$-1<x<1$。
    \item $\ln(1+x)=\sum\limits_{n=0}^\infty(-1)^{n-1}\dfrac{x^n}{n}=x-\dfrac{x^2}{2}+\dfrac{x^3}{3}+\cdots+(-1)^{n-1}\dfrac{x^n}{n}+\cdots$,$-1<x\leqslant1$。
    \item $\sin x=\sum\limits_{n=0}^\infty(-1)^n\dfrac{x^{2x+1}}{(2n+1)!}=x-\dfrac{x^3}{3!}+\dfrac{x^5}{5!}+(-1)^n\dfrac{x^{2n+1}}{(2n+1)!}+\cdots$,$-\infty<x<+\infty$。
    \item $\cos x=\sum\limits_{n=0}^\infty(-1)^n\dfrac{x^{2n}}{(2n)!}=1-\dfrac{x^2}{2!}+\dfrac{x^4}{4!}+(-1)^n\dfrac{x^{2n}}{(2n)!}+\cdots$,$-\infty<x<+\infty$。
    \item $(1+x)^\alpha=1+\alpha x+\dfrac{\alpha(\alpha-1)}{2!}x^2+\cdots+\dfrac{a(a-1)\cdots(a-n+1)}{n!}x^n+\cdots$,$\left\{\begin{array}{l}
        x\in(-1,1),\text{当}\alpha\leqslant-1 \\
        x\in(-1,1],\text{当}-1<\alpha<0 \\
        x\in[-1,1],\text{当}\alpha>0
    \end{array}\right.$。
\end{enumerate}

其中第1、3、5都是直接的公式,其他公式是根据其推出的。

\subsubsection{求法}

\paragraph{直接法} \leavevmode \medskip

逐个计算$a_n=\dfrac{f^{(n)}(x_0)}{n!}$并代入,但是一般很麻烦。

\paragraph{间接法} \leavevmode \medskip

利用已知的七个幂级数展开式,通过变量代换、四则运算、逐项求导、逐项积分和待定系数等得到。

\textbf{例题:}求函数$f(x)=\arctan x$在$x=0$处的幂级数展开。

解:$f'(x)=(\arctan x)'=\dfrac{1}{1+x^2}=\dfrac{1}{1-(-x^2)}=\sum\limits_{n=0}^\infty(-1)^nx^{2n}$,$\vert-x^2\vert<1$。

已经求得求导后的函数的幂级数展开,所以求原函数的幂级数展开只需要积分,利用先导后积公式:$f(x)=f(0)+\int_0^xf'(t)\,\textrm{d}t=\int_0^x\sum\limits_{n=0}^\infty(-1)^nt^{2n}\,\textrm{d}t=\sum\limits_{n=0}^\infty(-1)^n\dfrac{t^{2n+1}}{2n+1}\bigg|_0^x=\sum\limits_{n=0}^\infty(-1)^n\dfrac{x^{2n+1}}{2n+1}$。

求导的级数要求$\vert x\vert<1$,代入$x=\pm1$到最后结果得到两个交错级数,所以收敛域其实为$[-1,1]$(可以不写)。

\subsection{幂级数求和函数}

\subsubsection{概念}

\textcolor{violet}{\textbf{定义:}}在收敛域上,记$S(x)=\sum\limits_{n=1}^\infty u_n(x)$,并称$S(x)$为$\sum\limits_{n=1}^\infty u_n(x)$的\textbf{和函数}。

\subsubsection{运算法则}

若幂级数$\sum\limits_{n=0}^\infty a_nx^n$与$\sum\limits_{n=0}^\infty b_nx^n$的收敛半径分别为$R_a$和$R_b$($R_a\neq R_b$),则:

\begin{itemize}
    \item $k\sum\limits_{n=0}^\infty a_nx^n=\sum\limits_{n=0}^\infty ka_nx^n$,$\vert x\vert<R$,$k$为常数。
    \item $\sum\limits_{n=0}^\infty a_nx^n\pm\sum\limits_{n=0}^\infty b_nx^n=\sum\limits_{n=0}^\infty (a_n\pm b_n)x^n$,$\vert x\vert<R=\min\{R_a,R_b\}$。
    \item $(\sum\limits_{n=0}^\infty a_nx^n)\cdot(\sum\limits_{n=0}^\infty b_nx^n)=\sum\limits_{n=0}^\infty(a_0b_n+a_1b_{n-1}+\cdots+a_nb)x^n$。
\end{itemize}

实际运算中,可能运算法则要求的起始$n$值不同,$a_nb_n$不为不包含$x$的常数,$x^n$的幂次不同,恒等变形方法如下:

\begin{enumerate}
    \item 通项,下标一起变化:$\sum\limits_{n=k}^\infty a_nx^n=\sum\limits_{n=k+l}^\infty a_{n-l}x^{n-l}$,其中$l$为整数。
    \item 只变下标,只变通项:$\sum\limits_{n=0}^\infty a_nx^n=a_kx^k+a_{k+1}x^{k+1}+\cdots+a_{k+l-1}x^{k+l-1}+\sum\limits_{n=k+l}^\infty a_nx^n$。
    \item 只变通项,不变下标:$\sum\limits_{n=0}^\infty a_nx^n=x^l\sum\limits_{n=0}^\infty a_nx^{n-l}$。
\end{enumerate}

如$\sum\limits_{n=0}^\infty a_nx^{2n}+\sum\limits_{n=0}^\infty b_{n+1}x^{2n+2}=\sum\limits_{n=0}^\infty a_nx^{2n}+\sum\limits_{n=1}^\infty b_nx^{2n}=a_0+\sum\limits_{n=1}^\infty a_nx^{2n}+\sum\limits_{n=1}^\infty b_nx^{2n}=a_0+\sum\limits_{n=0}^\infty(a_n+b_n)x^{2n}$。

\subsubsection{性质}

收敛域的扩大和缩小在于其端点是否通过求导或积分变得可取了。

\begin{itemize}
    \item 幂级数$\sum\limits_{n=0}^\infty a_nx^n$的和函数$S(n)$在其收敛区间$I$上连续,且如果幂级数在收敛区间的端点$x=\pm R$处收敛,则和函数$S(x)$在$(-R,R]$或$[-R,R)$上连续。
    \item 幂级数$\sum\limits_{n=0}^\infty a_nx^n$的和函数$S(x)$在其收敛域$I$上可积,且有逐项积分公式$\int_0^xS(t)\,\textrm{d}t=\int_0^x(\sum\limits_{n=0}^\infty a_nt^n)\,\textrm{d}t=\sum\limits_{n=0}^\infty a_n\int_0^xt^n\,\textrm{d}t=\sum\limits_{n=0}^\infty\dfrac{a_n}{n+1}x^{n+1}$($x\in I$),逐项积分后得到的幂级数和原级数有相同收敛半径,但是收敛域可能扩大。
    \item 幂级数$\sum\limits_{n=0}^\infty a_nx^n$的和函数$S(x)$在其收敛区间$(-R,R)$内可导,且有逐项求导公式$S'(x)=(\sum\limits_{n=0}^\infty a_nx^n)'=\sum\limits_{n=0}^\infty(a_nx^n)'=\sum\limits_{n=1}^\infty na_nx^{n-1}$($\vert x\vert<R$),逐项求导后得到的幂级数和原级数有相同收敛半径,但是收敛域可能缩小。
\end{itemize}

\section{傅里叶级数}

\subsection{* 三角级数}

\textcolor{violet}{\textbf{定义:}}将正弦函数$A_n\sin(n\omega t+\varphi_n)$按三角公式变形得到$A_n\sin\varphi_n\cos n\omega t+A_n\cos\varphi_n\sin n\omega t$,令$\dfrac{a_0}{2}=A_0$,$a_n=A\sin\varphi_n$,$b_n=A_n\cos\varphi_n$,$\omega=\dfrac{\pi}{l}$,则$\dfrac{a_0}{2}+\sum\limits_{n=1}^\infty\left(a_n\cos\dfrac{n\pi t}{l}+b_n\sin\dfrac{n\pi t}{l}\right)$。这个级数就是\textbf{三角级数}。

\subsection{函数展开为傅里叶级数}

\textcolor{violet}{\textbf{定义:}}设$f(x)$在$[-l,l]$上连续或只有有限个第一类间断点,且至多只有有限个真正的极值点,则$f(x)$的傅里叶级数处处收敛,记起和函数为$S(x)$,有$S(x)=\dfrac{a_0}{2}+\sum\limits_{n=1}^\infty\left(a_n\cos\dfrac{n\pi}{l}x+b_n\sin\dfrac{n\pi}{l}x\right)$。

$\displaystyle{a_n=\dfrac{1}{l}\int_{-l}^lf(x)\cos\dfrac{n\pi}{l}x\,\textrm{d}x}$,$\displaystyle{b_n=\dfrac{1}{l}\int_{-l}^lf(x)\sin\dfrac{n\pi}{l}x\,\textrm{d}x}$,$n=1,2,\cdots$。

其中三角函数也可以展开为幂级数,所以最后都能通过幂级数展开。

\textbf{例题:}将$f(x)=1-x^2$($-\pi\leqslant x\leqslant\pi$)展开为傅里叶级数。

解:$S(x)=\dfrac{a_0}{2}+\sum\limits_{n=1}^\infty\left(a_n\cos\dfrac{n\pi}{l}x+b_n\sin\dfrac{n\pi}{l}x\right)$。

其中$a_0=\dfrac{1}{\pi}\int\limits_{-\pi}^\pi(1-x^2)\,\textrm{d}x=\dfrac{2}{\pi}\int\limits_0^\pi(1-x^2)\,\textrm{d}x=2-\dfrac{2}{3}\pi^2$。

$a_n=\dfrac{1}{\pi}\int\limits_{-\pi}^\pi(1-x^2)\cos nx\,\textrm{d}x=\dfrac{2}{\pi}(\int\limits_0^\pi\cos nx\,\textrm{d}x-\int\limits_0^\pi x^2\cos nx\,\textrm{d}x)=\dfrac{4}{n^2}(-1)^{n+1}$。

$b_n=\dfrac{1}{\pi}\int\limits_{-\pi}^\pi(1-x^2)\sin nx\,\textrm{d}x=0$(奇函数乘偶函数为奇函数,且上下限对称)

$f(x)\sim S(x)=1-\dfrac{\pi^2}{3}+\sum\limits_{n=1}^\infty\dfrac{4}{n^2}(-1)^{n+1}\cos nx$。

\textcolor{violet}{\textbf{定义:}}当$f(x)$是偶函数,则$\sin$被消去,$f(x)\sim S(x)=\dfrac{a_0}{2}+\sum\limits_{n=1}^\infty a_n\cos\dfrac{n\pi}{l}x$,称为\textbf{余弦级数}。

\textcolor{violet}{\textbf{定义:}}当$f(x)$是奇函数,则$\cos$被消去,$f(x)\sim S(x)=\dfrac{a_0}{2}+\sum\limits_{n=1}^\infty b_n\sin\dfrac{n\pi}{l}x$,称为\textbf{正弦级数}。

若$f(x)$因为定义区间不对称导致无奇偶性,则补充定义域,使其称为奇偶函数。

迪利克雷定理\textcolor{violet}{\textbf{定义:}}$f(x)\sim S(x)=\left\{\begin{array}{ll}
    f(x), & x\text{为连续点} \\
    \dfrac{f(x-0)+f(x+0)}{2}, & x\text{为间断点} \\
    \dfrac{f(-l+0)+f(l-0)}{2}, & x=\pm l
\end{array}\right.$

\end{document}
