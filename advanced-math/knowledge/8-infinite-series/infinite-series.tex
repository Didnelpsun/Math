\documentclass[UTF8, 12pt]{ctexart}
% UTF8编码,ctexart现实中文
\usepackage{color}
% 使用颜色
\definecolor{orange}{RGB}{255,127,0} 
\definecolor{violet}{RGB}{192,0,255} 
\definecolor{aqua}{RGB}{0,255,255} 
\usepackage{geometry}
\setcounter{tocdepth}{4}
\setcounter{secnumdepth}{4}
% 设置四级目录与标题
\geometry{papersize={21cm,29.7cm}}
% 默认大小为A4
\geometry{left=3.18cm,right=3.18cm,top=2.54cm,bottom=2.54cm}
% 默认页边距为1英尺与1.25英尺
\usepackage{indentfirst}
\setlength{\parindent}{2.45em}
% 首行缩进2个中文字符
\usepackage{setspace}
\renewcommand{\baselinestretch}{1.5}
% 1.5倍行距
\usepackage{amssymb}
% 因为所以
\usepackage{amsmath}
% 数学公式
\usepackage[colorlinks,linkcolor=black,urlcolor=blue]{hyperref}
% 超链接
\usepackage{pifont}
% 圆圈序号
\author{Didnelpsun}
\title{无穷级数}
\date{}
\begin{document}
\maketitle
\pagestyle{empty}
\thispagestyle{empty}
\tableofcontents
\thispagestyle{empty}
\newpage
\pagestyle{plain}
\setcounter{page}{1}
\section{常数项级数}

\subsection{概念}

级数的经典悖论为芝诺悖论。

\subsubsection{基本概念}

\textcolor{violet}{\textbf{定义:}}给定义一个无穷数列$u_1,u_2,\cdots,u_n,\cdots$,将其各项用加号连起来的得到的记号$\sum\limits_{n=1}^\infty u_n$,即$\sum\limits_{n=1}^\infty u_n=u_1+u_2+\cdots+u_n+\cdots$叫做\textbf{无穷级数},简称\textbf{级数},其中$u_n$称为该级数的\textbf{通项}。

若$u_n$是常数而不是函数,则$\sum\limits_{n=1}^\infty u_n$就被称为\textbf{常数项无穷级数},简称\textbf{常数项级数}。

$S_n=u_1+u_2+\cdots+u_n$称为级数的\textbf{部分和},$\{S_n\}$是级数的\textbf{部分和数量}。

\textcolor{violet}{\textbf{定义:}}若$\lim\limits_{n\to\infty}S_n=S$,则$\sum\limits_{n=1}^\infty u_n$\textbf{收敛},并称$S$为该收敛级数$\sum\limits_{n=1}^\infty u_n$的\textbf{和};若$\lim\limits_{n\to\infty}S_n$不存在或为$\pm\infty$,则$\sum\limits_{n=1}^\infty u_n$\textbf{发散}。

研究$\sum\limits_{n=1}^\infty u_n$收敛还是发散,就是研究级数$\sum\limits_{n=1}^\infty u_n$的敛散性。

在级数$\sum\limits_{n=1}^\infty u_n$去掉前$m$项,得$\sum\limits_{n=m+1}^\infty u_n=u_{m+1}+u_{m+2}+\cdots$,称为级数$\sum\limits_{n=1}^\infty u_n$的\textbf{$m$项后余项}

\subsubsection{性质}

\begin{enumerate}
    \item 线性性质:若级数$\sum\limits_{n=1}^\infty u_n$,$\sum\limits_{n=1}^\infty v_n$均收敛,且其和分别为$S$,$T$,则任给常数$a,b$,有$\sum\limits_{n=1}^\infty(au_n+bv_n)$也收敛,且其和为$aS+bT$,即$\sum\limits_{n=1}^\infty(au_n+bv_n)=a\sum\limits_{n=1}^\infty u_n+b\sum\limits_{n=1}^\infty v_n$。
    \item 若级数$\sum\limits_{n=1}^\infty u_n$收敛,则其任意$m$项后余项$\sum\limits_{n=m+1}^\infty u_n$也收敛;若存在$m$项后余项$\sum\limits_{n=m+1}^\infty u_n$收敛,则$\sum\limits_{n=1}^\infty u_n$也收敛。
    \item 级数收敛必要条件:若级数$\sum\limits_{n=1}^\infty u_n$首先,则$\lim\limits_{n\to\infty}u_n=0$。
\end{enumerate}

证明性质三:$u_n=S_n-S_{n-1}$,所以$\lim\limits_{n\to\infty}=\lim\limits_{n\to\infty}(S_n-S_{n-1})=\lim\limits_{n\to\infty}S_n-\lim\limits_{n\to\infty}S_{n-1}=S-S=0$。极限为0不一定收敛。

\subsection{级数敛散性判别}

\subsubsection{正项级数}

\paragraph{概念} \leavevmode \medskip

\textcolor{violet}{\textbf{定义:}}若通项$u_n\geqslant0$,$n=1,2,\cdots$,则$\sum\limits_{n=1}^\infty u_n$为\textbf{正项级数}。

\paragraph{收敛原则} \leavevmode \medskip

\textcolor{aqua}{\textbf{定理:}}正项级数$\sum\limits_{n=1}^\infty u_n$收敛的充要条件是其部分和数列$\{S_n\}$有界。(某一函数在固定区间内变化率是有界的,则变化范围是有界的)

证明:必要性:由于$u_n\geqslant0$,$\therefore S_n=u_1+u_2+\cdots+u_n\geqslant0$,且$S_1\leqslant S_2\leqslant\cdots\leqslant S_n\leqslant\cdots$,$\{S_n\}$单调不减且下界为0。当$\sum\limits_{n=1}^\infty u_n$收敛时,$\lim\limits_{n\to\infty}S_n$存在,则$\{S_n\}$必有上界。有上界下界则$\{S_n\}$有界。(某一函数在固定区间内变化率是有界的,则变化范围是有界的)

充分性:由于$\{S_n\}$单调不减,所以根据单调有界准则,$\{S_n\}$收敛,即$\lim\limits_{n\to\infty}S_n$存在,于是$\sum\limits_{n=1}^\infty u_n$收敛。

基本就是使用放缩法判断是否有界。

\textbf{例题:}判断级数$\sum\limits_{n=1}^\infty\dfrac{1}{\sqrt{n}}$的敛散性。

解:$S_n=1+\dfrac{1}{\sqrt{2}}+\dfrac{1}{\sqrt{3}}+\cdots+\dfrac{1}{\sqrt{n}}>n\dfrac{1}{\sqrt{n}}=\sqrt{n}$,当$n\to\infty$时$\sqrt{n}\to\infty$,无上界所以发散。

\paragraph{比较判别法} \leavevmode \medskip

\textcolor{aqua}{\textbf{定理:}}给出两个正项级数$\sum\limits_{n=1}^\infty u_n$,$\sum\limits_{n=1}^\infty v_n$,若从某项开始有$u_n\leqslant v_n$成立,则:\ding{172}若$\sum\limits_{n=1}^\infty v_n$收敛,则$\sum\limits_{n=1}^\infty u_n$也收敛;\ding{173}若$\sum\limits_{n=1}^\infty u_n$发散,则$\sum\limits_{n=1}^\infty v_n$也发散。

即大的收敛小的也收敛,小的发散大的也发散。

\textbf{例题:}判断调和级数$\sum\limits_{n=1}^\infty\dfrac{1}{n}$的敛散性。

解:$\because x>0$,$x>\ln(1+x)$,$\therefore\dfrac{1}{n}>\ln\left(1+\dfrac{1}{n}\right)$。

又对于$\ln\left(1+\dfrac{1}{n}\right)=\ln\dfrac{n+1}{n}=\ln(n+1)-\ln n$。

$S_n=\ln\dfrac{2}{1}+\ln\dfrac{3}{2}+\cdots+\ln\dfrac{n+1}{n}=\ln2-\ln1+\ln3-\ln2+\cdots+\ln(n+1)-\ln n=\ln(n+1)$。当$\ln(n+1)$在$n\to\infty$时,$S_n\to\infty$。

所以$\sum\limits_{n=1}^\infty\ln\left(1+\dfrac{1}{n}\right)$发散,则$\sum\limits_{n=1}^\infty\dfrac{1}{n}$也发散。

\textcolor{aqua}{\textbf{定理:}}$p$级数:$\sum\limits_{n=1}^\infty\dfrac{1}{n^p}\left\{\begin{array}{l}
    p>1, \text{收敛} \\
    p\leqslant1, \text{发散}
\end{array}\right.$。

\paragraph{比较判别法极限性质} \leavevmode \medskip

是比较判别法的推论,利用极限的阶数来比较。

给出两个正项级数$\sum\limits_{n=1}^\infty u_n$,$\sum\limits_{n=1}^\infty v_n$,$v_n\neq0$,且$\lim\limits_{n\to\infty}\dfrac{u_n}{v_n}=A$:

\begin{enumerate}
    \item 若$A=0$,则当$\sum\limits_{n=1}^\infty v_n$收敛时,$\sum\limits_{n=1}^\infty u_n$也收敛。
    \item 当$A=+\infty$,当$\sum\limits_{n=1}^\infty v_n$发散时,$\sum\limits_{n=1}^\infty u_n$也发散。
    \item 若$0<A<+\infty$,则$\sum\limits_{n=1}^\infty v_n$与$\sum\limits_{n=1}^\infty u_n$具有相同敛散性。
\end{enumerate}

\textbf{例题:}判断$\sum\limits_{n=1}^\infty\left(\dfrac{1}{n}-\sin\dfrac{1}{n}\right)$敛散性。

解:令$\dfrac{1}{n}=x$,$n\to\infty$所以$x\to0^+$。当$x\to0^+$,$x-\sin x\sim\dfrac{1}{6}x^3$。

$\therefore\lim\limits_{n\to\infty}\dfrac{\dfrac{1}{n}-\sin\dfrac{1}{n}}{\dfrac{1}{n^3}}=\dfrac{1}{6}\neq0$,所以$\dfrac{1}{n}-\sin\dfrac{1}{n}$与$\dfrac{1}{n^3}$具有相同敛散性。

根据$p$级数定理,$p=3>1$,所以$\sum\limits_{n=1}^\infty\left(\dfrac{1}{n}-\sin\dfrac{1}{n}\right)$收敛。

\paragraph{比值判别法} \leavevmode \medskip

也称为达朗贝尔判别法。

\textcolor{aqua}{\textbf{定理:}}给出一正项级数$\sum\limits_{n=1}^\infty u_n$,若$\lim\limits_{n\to\infty}\dfrac{u_{n+1}}{u_n}=\rho$,则:\ding{172}若$\rho<1$,则$\sum\limits_{n=1}^\infty u_n$收敛;\ding{173}若$\rho>1$,则$\sum\limits_{n=1}^\infty u_n$发散。

\textcolor{orange}{注意:}$\rho=1$时无法根据此判断$\sum\limits_{n=1}^\infty u_n$敛散性,如$\sum\limits_{n=1}^\infty\dfrac{1}{n}$发散,但$\sum\limits_{n=1}^\infty\dfrac{1}{n^2}$收敛。

适用于含有$a^n$,$n!$,$n^n$的通项。

\textbf{例题:}判断级数$\sum\limits_{n=1}^\infty\dfrac{\vert a\vert^nn!}{n^n}$的敛散性,其中$a$为非零常数。

解:

\paragraph{根值判别法} \leavevmode \medskip

也称为柯西判别法。

\textcolor{aqua}{\textbf{定理:}}$\sum\limits_{n=1}^\infty u_n$

\subsubsection{交错级数}

\paragraph{概念} \leavevmode \medskip

\paragraph{莱布尼兹判别法} \leavevmode \medskip

\subsubsection{任意项级数}

\paragraph{绝对收敛} \leavevmode \medskip

\paragraph{条件收敛} \leavevmode \medskip

\section{幂级数}

\subsection{概念与收敛域}

\subsubsection{概念}

\subsubsection{阿贝尔定理}

\subsubsection{收敛域}

\subsection{幂级数求和函数}

\subsubsection{概念}

\subsubsection{运算法则}

\subsubsection{性质}

\subsubsection{重要展开式}

\subsection{函数展开为幂级数}

\subsubsection{概念}

\subsubsection{求法}

\paragraph{直接法} \leavevmode \medskip

\paragraph{间接法} \leavevmode \medskip

\end{document}
