\documentclass[UTF8, 12pt]{ctexart}
% UTF8编码,ctexart现实中文
\usepackage{xcolor}
% 使用颜色
\definecolor{orange}{RGB}{255,127,0} 
\definecolor{violet}{RGB}{192,0,255} 
\definecolor{aqua}{RGB}{0,255,255} 
\usepackage{geometry}
\setcounter{tocdepth}{4}
\setcounter{secnumdepth}{4}
% 设置四级目录与标题
\geometry{papersize={21cm,29.7cm}}
% 默认大小为A4
\geometry{left=3.18cm,right=3.18cm,top=2.54cm,bottom=2.54cm}
% 默认页边距为1英尺与1.25英尺
\usepackage{indentfirst}
\setlength{\parindent}{2.45em}
% 首行缩进2个中文字符
\usepackage{setspace}
\renewcommand{\baselinestretch}{1.5}
% 1.5倍行距
\usepackage{amssymb}
% 因为所以
\usepackage{amsmath}
% 数学公式
\usepackage[colorlinks,linkcolor=black,urlcolor=blue]{hyperref}
% 超链接
\usepackage{pifont}
% 圆圈序号
\usepackage{tikz}
% 绘图
\author{Didnelpsun}
\title{多元函数微分学}
\date{}
\begin{document}
\maketitle
\pagestyle{empty}
\thispagestyle{empty}
\tableofcontents
\thispagestyle{empty}
\newpage
\pagestyle{plain}
\setcounter{page}{1}

\section{基本概念}

\subsection{二元函数}

函数以$f(u,v)$的形式来出现,需要分别对其求偏导。

\textbf{例题:}设$z=e^{xy}+f(x+y,xy)$,$f(u,v)$有二阶连续偏导数,求$\dfrac{\partial^2z}{\partial x\partial y}$。

解:令$x+y$为$u$,$xy$为$v$,$f(u,v)$对$u$求导就是$f_1'$,对$v$求导就是$f_2'$,求$uv$依次求导就是$f_{12}''$,以此类推。

首先求一次偏导:$\dfrac{\partial z}{\partial x}=ye^{xy}+\dfrac{\partial f(u,v)}{\partial u}\dfrac{\partial u}{\partial x}+\dfrac{\partial f(u,v)}{\partial v}\dfrac{\partial v}{\partial x}=ye^{xy}+f_1'+f_2'y$。

接着对$y$求偏导:$\dfrac{\partial^2z}{\partial x\partial y}=e^{xy}+xye^{xy}+\dfrac{\partial f_1'}{\partial y}+\dfrac{\partial f_2'y}{\partial y}$。

$=e^{xy}+xye^{xy}+\dfrac{\partial f_1'}{\partial y}+\dfrac{\partial f_2'}{\partial y}y+f_2'\dfrac{\partial y}{\partial y}=e^{xy}+xye^{xy}+\dfrac{\partial f_1'}{\partial u}\dfrac{\partial u}{\partial y}+\dfrac{\partial f_1'}{\partial v}\dfrac{\partial v}{\partial y}+\dfrac{\partial f_2'}{\partial u}\dfrac{\partial u}{\partial y}y+\dfrac{\partial f_2'}{\partial v}\dfrac{\partial v}{\partial y}y+f_2'=e^{xy}+xye^{xy}+f_{11}''+f_{12}''x+f_{21}''y+f_{22}''xy+f_2'$。\medskip

又$f(u,v)$具有两阶连续偏导数,所以$f_{12}''=f_{21}''$。

即$=e^{xy}+xye^{xy}+f_{11}''+(x+y)f_{12}''+xyf_{22}''+f_2'$。

\subsection{复合函数}

函数以复合函数形式$f(g(x,y))$出现,函数的变量是一个整体。

\subsubsection{链式法则}

若是给出相应的不等式可以通过链式法则求出对应的表达式。

\textbf{例题:}设$u=u(\sqrt{x^2+y^2})$($r=\sqrt{x^2+y^2}>0$)有二阶连续的偏导数,且满足$\dfrac{\partial^2u}{\partial x^2}+\dfrac{\partial^2u}{\partial y^2}-\dfrac{1}{x}\dfrac{\partial u}{\partial x}+u=x^2+y^2$,则求$u(\sqrt{x^2+y^2})$。

解:这个函数是复合函数$u=u(r)$和$r=\sqrt{x^2+y^2}$而成。根据复合函数求导法则:

$\dfrac{\partial u}{\partial x}=\dfrac{\textrm{d}u}{\textrm{d}r}\dfrac{\partial r}{\partial x}=\dfrac{\textrm{d}u}{\textrm{d}r}\dfrac{x}{\sqrt{x^2+y^2}}=\dfrac{\textrm{d}u}{\textrm{d}r}\cdot\dfrac{x}{r}$,$\dfrac{1}{x}\cdot\dfrac{\partial u}{\partial x}=\dfrac{1}{r}\cdot\dfrac{\textrm{d}u}{\textrm{d}r}$。

$\dfrac{\partial^2u}{\partial x^2}=\dfrac{\partial}{\partial x}\left(\dfrac{\partial u}{\partial x}\right)=\dfrac{\partial}{\partial x}\left(\dfrac{\textrm{d}u}{\textrm{d}r}\cdot\dfrac{x}{r}\right)=\dfrac{x}{r}\cdot\dfrac{\partial}{\partial x}\left(\dfrac{\textrm{d}u}{\textrm{d}r}\right)+\dfrac{\textrm{d}u}{\textrm{d}r}\cdot\dfrac{\partial}{\partial x}\left(\dfrac{x}{r}\right)=\dfrac{x}{r}\cdot\dfrac{\partial}{\partial r}\left(\dfrac{\textrm{d}u}{\textrm{d}r}\right)\dfrac{\partial r}{\partial x}+\dfrac{\textrm{d}u}{\textrm{d}r}\cdot\dfrac{r-x\cdot(\partial r/\partial x)}{r^2}=\dfrac{x^2}{r^2}\cdot\dfrac{\textrm{d}^2u}{\textrm{d}r^2}+\dfrac{\textrm{d}u}{\textrm{d}r}\cdot\dfrac{r^2-x^2}{r^3}$。

$\dfrac{\partial^2u}{\partial y^2}=\dfrac{\partial}{\partial y}\left(\dfrac{\partial u}{\partial y}\right)=\dfrac{\partial}{\partial y}\left(\dfrac{\textrm{d}u}{\textrm{d}r}\cdot\dfrac{y}{r}\right)=\dfrac{y}{r}\cdot\dfrac{\partial}{\partial y}\left(\dfrac{\textrm{d}u}{\textrm{d}r}\right)+\dfrac{\textrm{d}u}{\textrm{d}r}\cdot\dfrac{\partial}{\partial y}\left(\dfrac{y}{r}\right)=\dfrac{y}{r}\cdot\dfrac{\partial}{\partial r}\left(\dfrac{\textrm{d}u}{\textrm{d}r}\right)\dfrac{\partial r}{\partial y}+\dfrac{\textrm{d}u}{\textrm{d}r}\cdot\dfrac{r-y\cdot(\partial r/\partial y)}{r^2}=\dfrac{y^2}{r^2}\cdot\dfrac{\textrm{d}^2u}{\textrm{d}r^2}+\dfrac{\textrm{d}u}{\textrm{d}r}\cdot\dfrac{r^2-x^2}{r^3}$

代入不等式:$\dfrac{x^2+y^2}{r^2}\cdot\dfrac{\textrm{d}u^2}{\textrm{d}r^2}+\dfrac{\textrm{d}u}{\textrm{d}r}\cdot\dfrac{2r^2-x^2-y^2}{r^3}-\dfrac{1}{r}\cdot\dfrac{\textrm{d}u}{\textrm{d}r}+u=x^2+y^2$。

代入$x^2+y^2=r^2$:$\dfrac{\textrm{d}^2u}{\textrm{d}r^2}+u=r^2$,为二阶线性常系数微分方程。

通解为$u=C_1\cos r+C_2\sin r+r^2-2$。

即$u(\sqrt{x^2+y^2})=C_1\cos\sqrt{x^2+y^2}+C_2\sin\sqrt{x^2+y^2}+x^2+y^2-2$。

\subsubsection{特殊值反代}

若是给出的不等式后还给出对应的特殊值,可以直接代入然后反代求出函数,而不用链式法则。

\textbf{例题:}设$z=e^x+y^2+f(x+y)$,且当$y=0$时,$z=x^3$,则求$\dfrac{\partial z}{\partial x}$。

解:已知$y=0$时,$z=e^x+f(x)=x^3$,$\therefore f(x)=x^3-e^x$,$f(x+y)=(x+y)^3-e^{x+y}$,$z=e^x+y^2+(x+y)^3-e^{x+y}$。

$\therefore\dfrac{\partial z}{\partial x}=e^x+3(x+y)^2-e^{x+y}$。

\subsection{积分与微分}

\subsubsection{积分到微分}

可能一个函数是积分的形式,又包含多个变量,要求其多元微分值。

$\dfrac{\textrm{d}}{\textrm{d}x}\int_{a(x)}^{b(x)}f(t)\,\textrm{d}t=b'(x)f[b(x)]-a'(x)f[a(x)]$。

\textbf{例题:}设$z=\int_0^1\vert xy-t\vert f(t)\,\textrm{d}t$,$0\leqslant x\leqslant1$,$0\leqslant y\leqslant1$,其中$f(x)$为连续函数,求$z_{xx}''+z_{yy}''$。

解:首先因为$z$是一个绝对值的形式,所以根据积分的性质可以拆开积分区间去掉绝对值:$z=\int_0^{xy}(xy-t)f(t)\,\textrm{d}t+\int_{xy}^1(t-xy)f(t)\,\textrm{d}t=xy\int_0^{xy}f(t)\,\textrm{d}t-\int_0^{xy}tf(t)\,\textrm{d}t+\int_{xy}^1tf(t)\,\textrm{d}t-xy\int_{xy}^1f(t)\,\textrm{d}t$。

$z_x'=y\int_0^{xy}f(t)\,\textrm{d}t+xy^2f(xy)-xy^2f(xy)-xy^2f(xy)-y\int_{xy}^1f(t)\,\textrm{d}t+xy^2f(xy)=y\int_0^{xy}f(t)\,\textrm{d}t-y\int_{xy}^1f(t)\,\textrm{d}t$。

$z_{xx}''=y^2f(xy)+y^2f(xy)=2y^2f(xy)$,同理根据变量对称性$z_{yy}''=2x^2f(xy)$,$z_{xx}''+z_{yy}''=2(x^2+y^2)f(xy)$。

\subsubsection{微分到积分}

注意多元函数进行积分的适合多出来的常数$C$不再是常数,而是与积分变量相关的$C(x)$,$C(y)$,因为对其中一个变量积分时,另一个变量是看作常数的。

\textbf{例题:}设$z=f(x,y)$满足$\dfrac{\partial^2z}{\partial x\partial y}=x+y$,且$f(x,0)=x$,$f(0,y)=y^2$,求$f(x,y)$。

解:

\section{多元函数微分应用}

\subsection{空间曲线的切线与法平面}

\subsubsection{参数方程}

设空间曲线$\varGamma$由参数方程$\left\{\begin{array}{l}
    x=\phi(t) \\
    y=\psi(t) \\
    z=\omega(t)
\end{array}\right.$给出,其中$\phi(t),\psi(t),\omega(t)$均可导,$P_0(x_0,y_0,z_0)$为$\varOmega$上的点,且当$t=t_0$时,$\phi'(t_0)$,$\psi'(t_0)$,$\omega'(t_0)$均不为0,则:

\begin{itemize}
    \item 曲线$\varGamma$在点$P_0(x_0,y_0,z_0)$处的切向量为$\vec{\tau}=(\phi'(t_0),\psi'(t_0),\omega'(t_0))$。
    \item 曲线$\varGamma$在点$P_0(x_0,y_0,z_0)$处的切线方程为$\dfrac{x-x_0}{\phi'(t_0)}=\dfrac{y-y_0}{\psi'(t_0)}=\dfrac{z-z_0}{\omega'(t_0)}$。
    \item 曲线$\varGamma$在点$P_0(x_0,y_0,z_0)$处的法平面(过$P_0$且与切线垂直的平面)方程为$\phi'(t_0)(x-x_0)+\psi'(t_0)(y-y_0)+\omega'(t_0)(z-z_0)=0$。
\end{itemize}

\subsubsection{交面式方程}

设空间曲线$\varGamma$由交面方程$\left\{\begin{array}{l}
    F(x,y,z)=0 \\
    G(x,y,z)=0
\end{array}\right.$给出,则:

\begin{itemize}
    \item 曲线$\varGamma$在点$P_0(x_0,y_0,z_0)$处的切向量为\\$\vec{\tau}=\left(\left\vert\begin{array}{cc}
        F_y' & F_z' \\
        G_y' & G_z'
    \end{array}\right\vert_{P_0},\left\vert\begin{array}{ll}
        F_z' & F_x' \\
        G_z' & G_x'
    \end{array}\right\vert_{P_0},\left\vert\begin{array}{ll}
        F_x' & F_y' \\
        G_x' & G_y'
    \end{array}\right\vert_{P_0}\right)$。
    \item 曲线$\varGamma$在点$P_0(x_0,y_0,z_0)$处的切线方程为\\$\dfrac{x-x_0}{\left\vert\begin{array}{cc}
        F_y' & F_z' \\
        G_y' & G_z'
    \end{array}\right\vert_{P_0}},\dfrac{y-y_0}{\left\vert\begin{array}{ll}
        F_z' & F_x' \\
        G_z' & G_x'
    \end{array}\right\vert_{P_0}},\dfrac{z-z_0}{\left\vert\begin{array}{ll}
        F_x' & F_y' \\
        G_x' & G_y'
    \end{array}\right\vert_{P_0}}$。
    \item 曲线$\varGamma$在点$P_0(x_0,y_0,z_0)$处的法平面方程为\\$\left\vert\begin{array}{cc}
        F_y' & F_z' \\
        G_y' & G_z'
    \end{array}\right\vert_{P_0}(x-x_0)+\left\vert\begin{array}{ll}
        F_z' & F_x' \\
        G_z' & G_x'
    \end{array}\right\vert_{P_0}(y-y_0)+\left\vert\begin{array}{ll}
        F_x' & F_y' \\
        G_x' & G_y'
    \end{array}\right\vert_{P_0}(z-z_0)=0$。
\end{itemize}

\subsection{空间曲面的切平面与法线}

\subsubsection{隐式}

设空间曲面$\varSigma$由方程$F(x,y,z)=0$给出,$P_0(x_0,y_0,z_0)$是$\varSigma$上的点,则:

\begin{itemize}
    \item 曲面$\varSigma$在点$P_0(x_0,y_0,z_0)$处的法向量为$\vec{n}=(F_x'(x_0,y_0,z_0),F_y'(x_0,y_0,z_0),$\\$F_z'(x_0,y_0,z_0))$且法线方程为$\dfrac{x-x_0}{F_x'(x_0,y_0,z_0)}=\dfrac{y-y_0}{F_y'(x_0,y_0,z_0)}=\dfrac{z-z_0}{F_z'(x_0,y_0,z_0}$。
    \item 曲面$\varSigma$在点$P_0(x_0,y_0,z_0)$处的切平面方程为$F_x'(x_0,y_0,z_0)(x-x_0)+F_y'$\\$(x_0,y_0,z_0)(y-y_0)+F_z'(x_0,y_0,z_0)(z-z_0)=0$。
\end{itemize}

\subsubsection{显式}

设空间曲面$\varSigma$由方程$z=f(x,y)$给出,令$F(x,y,z)=f(x,y)-z$,假定法向量的方向向下,即其余$z$轴正向所成的角为钝角,即$z$为-1,则:

\begin{itemize}
    \item 曲面$\varSigma$在点$P_0(x_0,y_0,z_0)$处的法向量为$\vec{n}=(f_x'(x_0,y_0),f_y'(x_0,y_0),-1)$,且法线方程为$\dfrac{x-x_0}{f_x'(x_0,y_0)}=\dfrac{y-y_0}{f_y'(x_0,y_0)}=\dfrac{z-z_0}{-1}$。
    \item 曲面$\varSigma$在点$P_0(x_0,y_0,z_0)$处的切平面方程为$f_x'(x_0,y_0)(x-x_0)+f_y'(x_0,y_0)$\\$(y-y_0)-(z-z_0)=0$。
\end{itemize}

若是反之成锐角,则将里面所有的-1都换成1。

若用$\alpha$,$\beta$,$\gamma$表示曲面$z=f(x,y)$在点$(x_0,y_0,z_0)$处的法向量的方向角,并这里假定法向量的方向是向上的,即其余$z$轴正向所成的角$\gamma$为锐角,则法向量\textbf{方向余弦}为$\cos\alpha=\dfrac{-f_x}{\sqrt{1+f_x^2+f_y^2}}$,$\cos\beta=\dfrac{-f_y}{\sqrt{1+f_x^2+f_y^2}}$,$\cos\gamma=\dfrac{1}{\sqrt{1+f_x^2+f_y^2}}$,其中$f_x=f_x'(x_0,y_0)$,$f_y=f_y'(x_0,y_0)$。

\textbf{例题:}设直线$L\left\{\begin{array}{l}
    x+y+b=0 \\
    x+ay-z-3=0
\end{array}\right.$在平面$\pi$上,而平面$\pi$与曲面$z=x^2+y^2$相切于$(1,-2,5)$,求$ab$的值。

解:$L$在$\pi$上且与曲面相切,则$\pi$为$L$的切平面。设曲面方程$F(x,y,z)=x^2+y^2-z$。

曲面法向量为$\vec{n}=\{F_x',F_y',F_z'\}=\{2x,2y,-1\}$,代入$(1,-2,5)$,则法向量为$\{2,-4,-1\}$。

又点法式:$\pi:2(x-1)-4(y+2)-(z-5)=0$,即$2x-4y-z-5=0$。

联立直线方程,得到:$(5+a)x+4b+ab-2=0$,又$x$是任意的。

解得$a=-5,b=-2$。

\end{document}
