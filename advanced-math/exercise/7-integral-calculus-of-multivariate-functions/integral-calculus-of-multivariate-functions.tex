\documentclass[UTF8, 12pt]{ctexart}
% UTF8编码,ctexart现实中文
\usepackage{color}
% 使用颜色
\usepackage{geometry}
\setcounter{tocdepth}{4}
\setcounter{secnumdepth}{4}
% 设置四级目录与标题
\geometry{papersize={21cm,29.7cm}}
% 默认大小为A4
\geometry{left=3.18cm,right=3.18cm,top=2.54cm,bottom=2.54cm}
% 默认页边距为1英尺与1.25英尺
\usepackage{indentfirst}
\setlength{\parindent}{2.45em}
% 首行缩进2个中文字符
\usepackage{setspace}
\renewcommand{\baselinestretch}{1.5}
% 1.5倍行距
\usepackage{amssymb}
% 因为所以
\usepackage{amsmath}
% 数学公式
\usepackage[colorlinks,linkcolor=black,urlcolor=blue]{hyperref}
% 超链接
\author{Didnelpsun}
\title{多元函数积分学}
\date{}
\begin{document}
\maketitle
\pagestyle{empty}
\thispagestyle{empty}
\tableofcontents
\thispagestyle{empty}
\newpage
\pagestyle{plain}
\setcounter{page}{1}
\section{二重积分}

\subsection{交换积分次序}

\subsubsection{直角坐标系}

\textbf{例题:}交换积分次序$\int_0^1\textrm{d}x\int_0^{x^2}f(x,y)\,\textrm{d}y+\int_1^3\textrm{d}x\int_0^{\frac{1}{2}(3-x)}f(x,y)\,\textrm{d}y$。

解:已知积分区域分为两个部分。将$X$型变为$Y$型。画出图形可以知道$y\in(0,1)$,$x$的上下限由$y=x^2$和$y=\dfrac{1}{2}(3-x)$转化为$\sqrt{y}$和$3-2y$。

所以转换为$\int_0^1\textrm{d}y\int_{\sqrt{y}}^{3-2y}f(x,y)\,\textrm{d}x$。

\subsubsection{极坐标系}

\subsection{极直互化}

\textbf{例题:}将$I=\int_0^{\frac{\sqrt{2}}{2}R}e^{-y^2}\textrm{d}y\int_0^ye^{-x^2}\,\textrm{d}x+\int_{\frac{\sqrt{2}}{2}R}^Re^{-y^2}\,\textrm{d}y\int_0^{\sqrt{R^2-y^2}}e^{-x^2}\,\textrm{d}x$转换为极坐标系并计算结果。

解:首先根据积分上下限得到积分区域$D=\left\{0\leqslant y\leqslant\dfrac{\sqrt{2}}{2}R,0\leqslant x\leqslant y\right\}\cup\left\{\dfrac{\sqrt{2}}{2}R\leqslant y\leqslant R,0\leqslant x\leqslant\sqrt{R^2-y^2}\right\}$,$D$为一个八分之一圆的扇形。

根据$x=r\cos\theta$,$y=r\sin\theta$替换得到$D=\left\{(x,y)\bigg|0\leqslant r\leqslant R,\dfrac{\pi}{4}\leqslant\theta\leqslant\dfrac{\pi}{2}\right\}$。

又$e^{-y^2}\cdot e^{-x^2}=e^{-(x^2+y^2)}=e^{-r^2}$。

$\therefore I=\int_{\frac{\pi}{4}}^{\frac{\pi}{2}}\textrm{d}\theta\int_0^Re^{-r^2}r\,\textrm{d}r$。

\end{document}
