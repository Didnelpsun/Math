\documentclass[UTF8, 12pt]{ctexart}
% UTF8编码,ctexart现实中文
\usepackage{color}
% 使用颜色
\definecolor{orange}{RGB}{255,127,0} 
\usepackage{geometry}
\setcounter{tocdepth}{4}
\setcounter{secnumdepth}{4}
% 设置四级目录与标题
\geometry{papersize={21cm,29.7cm}}
% 默认大小为A4
\geometry{left=3.18cm,right=3.18cm,top=2.54cm,bottom=2.54cm}
% 默认页边距为1英尺与1.25英尺
\usepackage{indentfirst}
\setlength{\parindent}{2.45em}
% 首行缩进2个中文字符
\usepackage{setspace}
\renewcommand{\baselinestretch}{1.5}
% 1.5倍行距
\usepackage{amssymb}
% 因为所以
\usepackage{amsmath}
% 数学公式
\usepackage{pifont}
% 圆圈序号
\usepackage{mathtools}
% 有字的长箭头
\usepackage[colorlinks,linkcolor=black,urlcolor=blue]{hyperref}
% 超链接
\author{Didnelpsun}
\title{极限}
\date{}
\begin{document}
\maketitle
\thispagestyle{empty}
\tableofcontents
\thispagestyle{empty}
\newpage
\pagestyle{plain}
\setcounter{page}{1}

\section{极限类型}

七种:$\dfrac{0}{0},\dfrac{\infty}{\infty},0\cdot\infty,\infty-\infty,\infty^0,0^0,1^\infty$。

\medskip

\ding{172}其中$\dfrac{0}{0}$为洛必达法则的基本型。$\dfrac{\infty}{\infty}$可以类比$\dfrac{0}{0}$的处理方式。$0\cdot\infty$可以转为$\dfrac{0}{\dfrac{1}{\infty}}=\dfrac{0}{0}=\dfrac{\infty}{\dfrac{1}{0}}=\dfrac{\infty}{\infty}$。设置分母有原则,简单因式才下放(简单:幂函数,e为底的指数函数)。 

\medskip

\ding{173}$\infty-\infty$可以提取公因式或通分,即和差化积。

\medskip

\ding{174}$\infty^0,0^0,1^\infty$,就是幂指函数。

\medskip

$
u^v=e^{v\ln u}=\left\{
\begin{array}{lcl}
    \infty^0 & \rightarrow & e^{0\cdot+\infty} \\
    0^0 & \rightarrow & e^{0\cdot-\infty} \\
    1^\infty & \rightarrow & e^{\infty\cdot 0} \\
\end{array} \right.
$

\medskip

$\therefore \lim u^v=e^{\lim v\cdot\ln u}=e^{\lim v(u-1)}$

综上,无论什么样的四则形式,都必须最后转换为商的形式。

\section{常用化简运算}

\subsection{对数法则}

\textbf{例题:}求$\lim\limits_{x\to 0}\dfrac{(e^{x^2}-1)(\sqrt{1+x}-\sqrt{1-x})}{[\ln(1-x)+\ln(1+x)]\sin\dfrac{x}{x+1}}$。

注意在积或商的时候不能把对应的部分替换为0,如分母部分的$[\ln(1-x)+\ln(1+x)]$就无法使用$\ln(1+x)\sim x$替换为$-x+x$,这样底就是0了,无法求得最后的极限。

这时可以尝试变形,如对数函数相加等于对数函数内部式子相乘:$\ln(1-x)+\ln(1+x)=\ln(1-x^2)\sim-x^2$。

\subsection{指数法则}

一般需要与洛必达法则配合使用。\medskip

\textbf{例题:}求$\lim\limits_{x\to 0}\left(\dfrac{a^x+b^x+c^x}{3}\right)^{\frac{1}{x}}(a>0,b>0,c>0)$。\medskip

$=e^{\lim\limits_{x\to 0}\frac{\ln\left(\frac{a^x+b^x+c^x}{3}\right)}{x}}=e^{\lim\limits_{x\to 0}\frac{\ln(a^x+b^x+c^x)-\ln 3}{x}}=e^{\lim\limits_{x\to 0}\frac{a^x\ln a+b^x\ln b+c^x\ln c}{a^x+b^x+c^x}}$(洛必达法则)

$=e^{\lim\limits_{x\to 0}\frac{\ln a+\ln b+\ln c}{1+1+1}}=e^{\lim\limits_{x\to 0}\frac{\ln(abc)}{3}}=\sqrt[3]{abc}$。

\textbf{例题:}求$\lim\limits_{n\to\infty}n\left[\left(1+\dfrac{1}{n}\right)^{\frac{n}{2}}-\sqrt{e}\right]$。

首先对于幂指函数需要取指数,所以$\left(1+\dfrac{1}{n}\right)^{\frac{n}{2}}=e^{\frac{n}{2}\ln(1+\frac{1}{n})}$。\medskip

而后面的多一个$\sqrt{e}$导致整个式子变为一个复杂的式子,而与$e^x$相关的是$e^x-1\sim x$。

所以$e^{\frac{n}{2}\ln(1+\frac{1}{n})}-\sqrt{e}=e^{\frac{1}{2}}\cdot\left(e^{\frac{n}{2}\ln(1+\frac{1}{n})-\frac{1}{2}}-1\right)=e^{\frac{1}{2}}\cdot\left[\dfrac{n}{2}\ln(1+\dfrac{1}{n})-\dfrac{1}{2}\right]$。

综上:

$\lim\limits_{n\to\infty}n\left[\left(1+\dfrac{1}{n}\right)^{\frac{n}{2}}-\sqrt{e}\right]=\lim\limits_{n\to\infty}n\left(e^{\frac{n}{2}\ln(1+\frac{1}{n})}-\sqrt{e}\right)$ \medskip

$=\lim\limits_{n\to\infty}n\left[e^{\frac{1}{2}}\cdot\left(e^{\frac{n}{2}\ln(1+\frac{1}{n})-\frac{1}{2}}-1\right)\right]=\dfrac{e^{\frac{1}{2}}}{2}\lim\limits_{n\to\infty}n^2\left[\ln\left(1+\frac{1}{n}\right)-\dfrac{1}{n}\right]$

$=\dfrac{e^{\frac{1}{2}}}{2}\lim\limits_{n\to\infty}\dfrac{\dfrac{1}{n}-\dfrac{1}{2n^2}-\dfrac{1}{n}}{\dfrac{1}{n^2}}=\dfrac{e^{\frac{1}{2}}}{2}\cdot\left(-\dfrac{1}{2}\right)=-\dfrac{\sqrt{e}}{4}$

\subsection{三角函数关系式}

\textbf{例题:}求极限$\lim\limits_{x\to 0}\left(\dfrac{1}{\sin^2x}-\dfrac{\cos^2x}{x^2}\right)$。\medskip

$\lim\limits_{x\to 0}\left(\dfrac{1}{\sin^2x}-\dfrac{\cos^2x}{x^2}\right)=\lim\limits_{x\to 0}\dfrac{x^2-\sin^2x\cos^2x}{\sin^2x\cdot x^2} (\sin x\sim x)$ \medskip

$=\lim\limits_{x\to 0}\dfrac{x^2-\sin^2x\cos^2x}{x^4} (\sin x\cos x\sim\dfrac{1}{2}\sin 2x)=\lim\limits_{x\to 0}\dfrac{x^2-\dfrac{1}{4}\sin^22x}{x^4}$ \medskip

$=\lim\limits_{x\to 0}\dfrac{2x-\dfrac{1}{4}\cdot 2\sin 2x\cdot\cos 2x\cdot 2}{4x^3} (\sin x\cos x\sim\dfrac{1}{2}\sin 2x)=\lim\limits_{x\to 0}\dfrac{2x-\dfrac{1}{2}\sin 4x}{4x^3}$ \medskip

$=\lim\limits_{x\to 0}\dfrac{2-\dfrac{1}{2}\cos 4x\cdot 4}{12x^2}=\dfrac{1}{6}\lim\limits_{x\to 0}\dfrac{1-\cos 4x}{x^2} (1-\cos x\sim \dfrac{1}{2}x^2)=\dfrac{4}{3}$

\subsection{提取常数因子}

提取常数因子就是提取出能转换为常数的整个极限式子的因子。这个因子必然在自变量的趋向时会变为非0的常数,那么这个式子就可以作为常数提出。

\subsection{提取公因子}

当作为商的极限式子上下都具有公因子时可以提取公因子然后相除,从而让未知数集中在分子或分母上。

\textbf{例题:}求$\lim\limits_{x\to 4}\dfrac{x^2-6x+8}{x^5-5x+4}$。

需要先提取公因子:

$=\lim\limits_{x\to 4}\dfrac{(x-2)(x-4)}{(x-1)(x-4)}=\lim\limits_{x\to 4}\dfrac{x-2}{x-1}=\dfrac{2}{3}$。

(当然可以使用洛必达法则得到极限为$\lim\limits_{x\to 4}\dfrac{2x-6}{2x-5}=\lim\limits_{x\to 4}\dfrac{8-6}{8-5}$)

\textcolor{orange}{注意:}提取公因子的时候应该注意开平方等情况下符号的问题。如果极限涉及倒正负两边则必须都讨论。

当趋向为负且式子中含有根号的时候最好提取负因子,从而让趋向变为正。\medskip

\textbf{例题:}求$\lim\limits_{x\to-\infty}\left[\sqrt{4x^2+x}\ln\left(2+\dfrac{1}{x}\right)+2\ln 2x\right]$。\medskip

题目的形式为$\infty-\infty$,所以必须使用后面的倒代换转换为商的形式。\medskip

$=\lim\limits_{x\to-\infty}-x\left[\sqrt{4+\dfrac{1}{x}}\ln\left(2+\dfrac{1}{x}\right)-2\ln 2\right]$。 \medskip

这里就需要注意到因为$\sqrt{4x^2+x}$的限制导致这个式子必然为正数,而$x\to-\infty$代表自变量为负数,所以提出来的$x$必然是负数,而原式是正数,所以就需要添加一个负号,而后面的$2\ln 2x$则没有要求,所以直接变成$-2\ln 2$就可以了。

将$x$下翻变成分母为$\dfrac{1}{x}$,并令$t=\dfrac{1}{x}$。\medskip

$=\lim\limits_{t\to 0^-}\dfrac{\sqrt{t+4}\ln\left(2+\dfrac{1}{x}\right)-2\ln 2}{-t}$。\medskip

幂次不高可以尝试洛必达:\medskip

$=\lim\limits_{t\to 0^-}\left(\dfrac{1}{2}\cdot\dfrac{\ln(2+t)}{\sqrt{t+4}}+\dfrac{\sqrt{t+4}}{2+t}\right)=-\left(\dfrac{1}{2}\cdot\dfrac{\ln 2}{2}+\dfrac{2}{2}\right)=-\dfrac{\ln 2}{4}-1$。

\subsection{幂指函数}

当出现$f(x)^{g(x)}$的类似幂函数与指数函数类型的式子,需要使用$u^v=e^{v\ln u}$。

\textbf{例题:}求极限$\lim\limits_{x\to+\infty}(x+\sqrt{1+x^2})^{\frac{1}{x}}$。

$\lim\limits_{x\to+\infty}(x+\sqrt{1+x^2})^{\frac{1}{x}}=e^{\lim\limits_{x\to+\infty}\frac{\ln(x+\sqrt{1+x^2})}{x}} \left(\ln(x+\sqrt{1+x^2})'=\dfrac{1}{\sqrt{1+x^2}}\right)$\medskip

$=e^{\lim\limits_{x\to+\infty}\frac{1}{\sqrt{1+x^2}}}=e^0=1$

\subsection{有理化}

当遇到带有根号的式子可以使用等价无穷小,但是只针对形似$(1+x)a-1\sim ax$的式子,而针对$x^a\pm x^b$的式子则无法替换,必须使用有理化来将单个式子变为商的形式。

如$\sqrt{a}\pm\sqrt{b}=\dfrac{a+b}{\sqrt{a}\mp\sqrt{b}}$。\medskip

\textbf{例题:}求极限$\lim\limits_{x\to-\infty}x(\sqrt{x^2+100}+x)$。

首先定性分析:$\lim\limits_{x\to-\infty}x\cdot(\sqrt{x^2+100}+x)$。

在$x\to-\infty$趋向时,$x$就趋向无穷大。

而$\sqrt{x^2+100}$为一次,所以$\sqrt{x^2+100}+x$趋向0。

又$\sqrt{x^2+100}$在$x\to-\infty$时本质为根号差,所以有理化:

$\lim\limits_{x\to-\infty}x(\sqrt{x^2+100}+x)=\lim\limits_{x\to-\infty}x\dfrac{x^2+100-x^2}{\sqrt{x^2+100}-x}=\lim\limits_{x\to-\infty}\dfrac{100x}{\sqrt{x^2+100}-x}$\medskip

$\xRightarrow{\text{令}x=-t}\lim\limits_{t\to+\infty}\dfrac{-100t}{\sqrt{t^2+100}+t}=\lim\limits_{t\to+\infty}\dfrac{-100}{\sqrt{1+\dfrac{100}{t^2}}+1}=-50$\medskip

\textbf{例题:}求$\lim\limits_{x\to 0}\dfrac{\sqrt{1+\tan x}-\sqrt{1+\sin x}}{x\sqrt{1+\sin^2x}-x}$。

$=\lim\limits_{x\to 0}\sqrt{1+\tan x}-\sqrt{1+\sin x}\cdot\lim\limits_{x\to 0}\dfrac{1}{x\sqrt{1+\sin^2x}-x}$\medskip

$=\lim\limits_{x\to 0}\dfrac{\tan x-\sin x}{\sqrt{1+\tan x}+\sqrt{1+\sin x}}\cdot\lim\limits_{x\to 0}\dfrac{x\sqrt{1+\sin^2x}+x}{x^2(1+\sin^2x)-x^2}$\medskip

$=\lim\limits_{x\to 0}\dfrac{\tan x-\sin x}{\sqrt{1+\tan x}+\sqrt{1+\sin x}}\cdot\lim\limits_{x\to 0}\dfrac{\sqrt{1+\sin^2x}+1}{x\sin^2x}$\medskip

$=\lim\limits_{x\to 0}\dfrac{\tan x-\sin x}{2}\cdot\lim\limits_{x\to 0}\dfrac{2}{x\sin^2x}$

$=\lim\limits_{x\to 0}\dfrac{\tan x-\sin x}{x\sin^2x}=\lim\limits_{x\to 0}\dfrac{1-\cos x}{x\cos x\sin x}=\lim\limits_{x\to 0}\dfrac{\dfrac{1}{2}x^2}{x^2}=\dfrac{1}{2}$。

\subsection{换元法}

换元法本身没什么技巧性,主要是更方便计算。最重要的是获取到共有的最大因子进行替换。

\textbf{例题:}求极限$\lim\limits_{x\to 1^-}\ln x\ln(1-x)$。

当$x\to 1^-$时,$\ln x$趋向0,$\ln(1-x)$趋向$-\infty$。

又$x\to 0$,$\ln(1+x)\sim x$,所以$x\to 1$,$\ln x\sim x-1$:

$\lim\limits_{x\to 1^-}\ln x\ln(1-x)=\lim\limits_{x\to 1^-}(x-1)\ln(1-x)\xRightarrow{令t=1-x} =-\lim\limits_{t\to 0^+}t\ln t$

$=-\lim\limits_{t\to 0^+}\dfrac{\ln t}{\dfrac{1}{t}}=-\lim\limits_{t\to 0^+}\dfrac{\dfrac{1}{t}}{-\dfrac{1}{t^2}}=\lim\limits_{t\to 0^+}t=0$

\subsection{倒代换}

\subsubsection{含有分式}

当极限式子中含有分式中一般都需要用其倒数,把分式换成整式方便计算。

\textbf{例题:}求极限$\lim\limits_{x\to 0}\dfrac{e^{-\frac{1}{x^2}}}{x^{100}}$ \medskip

$\lim\limits_{x\to 0}\dfrac{e^{-\frac{1}{x^2}}}{x^{100}}=\lim\limits_{x\to 0}\dfrac{e^{-\frac{1}{x^2}}\cdot 2x^{-3}}{100x^{99}}=\lim\limits_{x\to 0}\dfrac{1}{50}\lim\limits_{x\to 0}\dfrac{e^{-\frac{1}{x^2}}}{x^{102}}$

\medskip

使用洛必达法则下更复杂,因为分子的幂次为负数,导致求导后幂次绝对值越来越大,不容易计算。

使用倒代换再洛必达降低幂次,令$t=\dfrac{1}{x^2}$

$\lim\limits_{x\to 0}\dfrac{e^{-\frac{1}{x^2}}}{x^{100}}=\lim\limits_{t\to+\infty}\dfrac{e^{-t}}{t^{-50}}=\lim\limits_{t\to+\infty}\dfrac{t^{50}}{e^t}=\lim\limits_{t\to+\infty}\dfrac{50t^{49}}{e^t}$

$=\cdots$

$=\lim\limits_{t\to+\infty}\dfrac{50!}{e^t}=0$

\textbf{例题:}求极限$\lim\limits_{x\to+\infty}[x^2(e^{\frac{1}{x}}-1)-x]$。

该式子含有分数,所以尝试使用倒数代换:\medskip

$\lim\limits_{x\to+\infty}[x^2(e^{\frac{1}{x}}-1)-x]\xRightarrow{\text{令}x=\frac{1}{t}}\lim\limits_{t\to 0^+}\left(\dfrac{e^t-1}{t^2}-\dfrac{1}{t}\right)=\lim\limits_{t\to 0^+}\dfrac{e^t-1-t}{t^2}$

$\xRightarrow{\text{泰勒展开}e^t}\lim\limits_{t\to 0^+}\dfrac{\dfrac{1}{2}t^2}{t^2}=\dfrac{1}{2}$

\subsubsection{\texorpdfstring{$\infty-\infty$}\ 型}

\subsection{拆项}

拆项需要根据式子形式进行,所以很难找到普遍规律。

\textbf{例题:}求$\lim\limits_{n\to\infty}\dfrac{(n+1)(n+2)(n+3)\cdots(n+6)}{6n^6}$。

需要将分子和分母都拆为6项:

$=\dfrac{1}{6}\lim\limits_{n\to\infty}\dfrac{n+1}{n}\times\dfrac{n+2}{n}\times\cdots\dfrac{n+6}{n}=\dfrac{1}{6}\lim\limits_{n\to\infty}(1+\dfrac{1}{n})(1+\dfrac{2}{n})\cdots(1+\dfrac{6}{n})=\dfrac{1}{6}$。

当极限式子中出现不知道项数的$n$时,一般需要使用拆项,把项重新组合。一般的组合是根据等价无穷小。

而对于复杂的具有同一结构的式子也可以考虑拆项。

\textbf{例题:}求极限$\lim\limits_{x\to 0}\left(\dfrac{e^x+e^{2x}+\cdots+e^{nx}}{n}\right)^{\frac{e}{x}}$。($n\in N^+$)

这里可以使用等价无穷小$e^x-1\sim x$。

$\lim\limits_{x\to 0}\left(\dfrac{e^x+e^{2x}+\cdots+e^{nx}}{n}\right)^{\frac{e}{x}}=e^{\lim\limits_{x\to 0}\frac{e}{x}\ln\left(\frac{e^x+e^{2x}+\cdots+e^{nx}}{n}\right)}$

$=e^{\lim\limits_{x\to 0}\frac{e}{x}\left(\frac{e^x+e^{2x}+\cdots+e^{nx}}{n}-1\right)}=e^{\lim\limits_{x\to 0}\frac{e}{x}\left(\frac{e^x+e^{2x}+\cdots+e^{nx}-n}{n}\right)}$

$=e^{\frac{e}{n}\lim\limits_{x\to 0}\left(\frac{e^x-1}{x}+\frac{e^{2x}-1}{x}+\cdots+\frac{e^{nx}-1}{x}\right)}=e^{\frac{e}{n}[1+2+\cdots+n]}=e^{\frac{e}{n}\cdot\frac{n(1+n)}{2}}=e^{\frac{e(1+n)}{2}}$

\textbf{例题:}求$\lim\limits_{x\to 0}\dfrac{1-\cos x\sqrt{\cos 2x}\sqrt[3]{\cos 3x}}{\ln\cos x}$\medskip

可以使用$\cos x-1\sim\dfrac{x^2}{2}$。\medskip

$\lim\limits_{x\to 0}\dfrac{1-\cos x\sqrt{\cos 2x}\sqrt[3]{\cos 3x}}{\ln\cos x}$

$=\lim\limits_{x\to 0}\dfrac{1-\cos x+\cos x-\cos x\sqrt{\cos 2x}\sqrt[3]{\cos 3x}}{-\dfrac{x^2}{2}}$

$=\lim\limits_{x\to 0}\dfrac{\dfrac{x^2}{2}}{-\dfrac{x^2}{2}}+\dfrac{\cos x(1-\sqrt{\cos 2x}\sqrt[3]{\cos 3x})}{-\dfrac{x^2}{2}}$

$=-1+\lim\limits_{x\to 0}\dfrac{(1-\sqrt{\cos 2x})+\sqrt{\cos 2x}-\sqrt{\cos 2x}\sqrt[3]{\cos 3x}}{-\dfrac{x^2}{2}}$

$=-1+\lim\limits_{x\to 0}\dfrac{-\dfrac{1}{2}(\cos 2x-1)+\sqrt{\cos 2x}(1-\sqrt[3]{\cos 3x})}{-\dfrac{x^2}{2}}$

$=-1+\lim\limits_{x\to 0}\dfrac{-\dfrac{1}{2}(-\dfrac{4x^2}{2})+\left(-\dfrac{1}{3}\right)\left(-\dfrac{9x^2}{2}\right)}{-\dfrac{x^2}{2}}=-6$

\section{基本计算方式}

课本上极限计算可以使用的主要计算方式:

\subsection{基础四则运算}

只有式子的极限各自存在才能使用四则运算,使用的频率较少。

\subsection{重要极限}

重要极限有两个,但是$\lim\limits_{x\to 0}\dfrac{\sin x}{x}=1$这个很少用,因为往往用等价无穷小替代了,而$\lim\limits_{x\to\infty}\left(1+\dfrac{1}{x}\right)^x=e$则用的较多,当出现分数幂的幂指函数时,不要先去取对数,而是使用重要极限看看能不能转换。\medskip

\textbf{例题:}求$\lim\limits_{x\to\infty}\left(\dfrac{3+x}{6+x}\right)^{\frac{x-1}{2}}$。\medskip

$=\lim\limits_{x\to\infty}\left(1-\dfrac{3}{6+3}\right)^{\frac{6+x}{-3}\cdot\frac{-3}{6+x}\cdot\frac{x-1}{2}}=\lim\limits_{x\to\infty}e^{\frac{-3}{6+x}\cdot\frac{x-1}{2}}=\lim\limits_{x\to\infty}e^{-\frac{3}{2}\cdot\frac{x-1}{x+6}}=e^{-\frac{3}{2}}$。\medskip

\textbf{例题:}求$\lim\limits_{x\to\infty}\left(\dfrac{2x+3}{2x+1}\right)^{x+1}$。\medskip

$=\lim\limits_{x\to\infty}\left(\dfrac{2x+3}{2x+1}\right)^x\cdot\dfrac{2x+3}{2x+1}=\lim\limits_{x\to\infty}\left(\dfrac{2x+3}{2x+1}\right)^x$\medskip

$=\lim\limits_{x\to\infty}\left(\dfrac{1+\dfrac{3}{2x}}{1+\dfrac{1}{2x}}\right)^x=\lim\limits_{x\to\infty}\dfrac{\left(1+\dfrac{3}{2x}\right)^x}{\left(1+\dfrac{1}{2x}\right)^x}$

$=\lim\limits_{x\to\infty}\dfrac{\left[\left(1+\dfrac{3}{2x}\right)^{\frac{2x}{3}}\right]^{\frac{3}{2}}}{\left[\left(1+\dfrac{1}{2x}\right)^{2x}\right]^{\frac{1}{2}}}=\dfrac{e^{\frac{3}{2}}}{e^{\frac{1}{2}}}=e$。

\subsection{导数定义}

极限转换以及连续性的时候会用到,但是使用的频率也较小。

\subsection{等价无穷小替换}

当看到复杂的式子,且不论要求的极限值的趋向,而只要替换的式子是$\Delta\to 0$时的无穷小,就使用等价无穷小进行替换。

\textcolor{orange}{注意:}替换的必然是整个求极限的乘或除的因子,一般加减法与部分的因子不能进行等价无穷小替换。

对于无法直接得出变换式子的,可以对对应参数进行凑,以达到目标的可替换的等价无穷小。

\subsection{夹逼准则}

夹逼准则可以用来证明不等式也可以用来计算极限。但是最重要的是找到能夹住目标式子的两个式子。\medskip

\textbf{例题:}求极限$\lim\limits_{x\to 0}x\left[\dfrac{10}{x}\right]$,其中$[\cdot]$为取整符号。

取整函数公式:$x-1<[x]\leqslant x$,所以$\dfrac{10}{x}-1<\left[\dfrac{10}{x}\right]\leqslant\dfrac{10}{x}$。

当$x>0$时,$x\to 0^+$,两边都乘以10,$10-x<x\cdot\left[\dfrac{10}{x}\right]\leqslant x\cdots\dfrac{10}{x}=10$,而左边在$x\to 0^+$时极限也为10,所以夹逼准则,中间$x\cdot\left[\dfrac{10}{x}\right]$极限也为10。\medskip

当$x>0$时,$x\to 0^-$,同样也是夹逼准则得到极限为10。\medskip

$\therefore \lim\limits_{x\to 0}x\left[\dfrac{10}{x}\right]=10$。

\subsection{拉格朗日中值定理}

对于形如$f(a)-f(b)$的极限式子就可以使用拉格朗日中值定理,这个$f(x)$为任意的函数。使用拉格朗日中值定理最重要的还是找到这个$f(x)$。

\subsubsection{证明不等式}

证明不等式最重要的还是找到$f(x)$,有时候不等式不存在$f(a)-f(b)$这种式子,就需要我们转换。

\paragraph{对数函数特性} \leavevmode \medskip

对于对数函数,要记住其特定的性质:$\log_n(\dfrac{a}{b})=\log_na-\log_nb$。

\textbf{例题:}设$a>b>0$,证明:$\dfrac{a-b}{a}<\ln\dfrac{a}{b}<\dfrac{a-b}{b}$。

因为$\ln\dfrac{a}{b}=\ln a-\ln b$,所以令$f(x)=\ln x$。

所以根据拉格朗日中值定理:$\ln a-\ln b=f'(\xi)(a-b)$($\xi\in(b,a)$)。

又$f'(\xi)=\dfrac{1}{\xi}$,所以$\ln a-\ln b=\dfrac{a-b}{\xi}$。

又$\xi\in(b,a)$,所以$\dfrac{1}{\xi}\in(\dfrac{1}{a},\dfrac{1}{b})$。

所以$\dfrac{a-b}{a}<\dfrac{a-b}{\xi}<\dfrac{a-b}{b}$,从而$\dfrac{a-b}{a}<\ln\dfrac{a}{b}<\dfrac{a-b}{b}$,得证。

\paragraph{查找特定值} \leavevmode \medskip

对于证明一种不等式,如果里面没有差式,也无法转换为差式,那么就可以考虑制造差式,对于$f(x)$一般选择更高阶的,$a$选择$x$,$b$要根据题目和不等式设置一个常数。

一般是0或1。可以先尝试1。

\textbf{例题:}当$x>1$时,证明$e^x>ex$。

题目中没有差式,所以需要选择一个函数作为基准函数,里面有一个指数函数和一个幂函数,所以选择$e^x$作为基准函数。

然后选择一个常数作为$b$值,可以先选一个1作为$b$值:$f(x)-f(1)=f'(\xi)(x-1)$。

从而$e^x-e=e^\xi(x-1)$,$\xi\in(1,x)$,所以$e^x-e>e(x-1)$,即$e^x>ex$,得证。

\subsubsection{极限计算}

可以将极限式子中形如$f(a)-f(b)$的极限部分使用拉格朗日中值定理进行替换。

\textbf{例题:}求极限$\lim\limits_{n\to\infty}n^2\left(\arctan\dfrac{2}{n}-\arctan\dfrac{2}{n+1}\right)$。\medskip

因为式子不算非常复杂,其实也可以通过洛必达法则来完成,但是求导会很复杂。而$\arctan x$可以认定为$f(x)$。

从而$\arctan\dfrac{2}{n}-\arctan\dfrac{2}{n+1}$为$f(\dfrac{2}{n})-f(\dfrac{2}{n+1})=f'(\xi)\left(\dfrac{2}{n}-\dfrac{2}{n+1}\right)$。

其中$\dfrac{2}{n+1}<\xi<\dfrac{2}{n}$,而当$n\to\infty$时,$f'(\xi)=\dfrac{1}{1+\xi^2}\to 1$。

$\therefore\arctan\dfrac{2}{n}-\arctan\dfrac{2}{n+1}\sim\dfrac{2}{n}-\dfrac{2}{n+1}=\dfrac{2}{n(n+1)}$。

$\therefore\lim\limits_{n\to\infty}n^2\left(\arctan\dfrac{2}{n}-\arctan\dfrac{2}{n+1}\right)=\lim\limits_{n\to\infty}n^2\cdot\dfrac{2}{n(n+1)}=2$。

\subsection{洛必达法则}

洛必达法则的本质是降低商形式的极限式子的幂次。

洛必达在处理一般的极限式子比较好用,但是一旦式子比较复杂最好不要使用洛必达法则,最好是对求导后有规律或幂次较低的式子进行上下求导。

对于幂次高的式子必然使用洛必达法则。

洛必达法则必须使用在分式都趋向0或$\infty$时,如果不是这样的趋向则不能使用。如:

\textbf{例题:}求$\lim\limits_{x\to 1}\dfrac{x^2-x+1}{(x-1)^2}$。

如果使用洛必达法则,则会得到结果为1,这是错误的,因为分子在$x\to 1$时结果为常数1。正确的计算方式:

$=\lim\limits_{x\to 1}\dfrac{1}{(x-1)^2}=\infty$。

\subsection{泰勒公式}

泰勒公式一般会使用趋向0的麦克劳林公式,且一般只作为极限计算的一个小部分,用来替代一个部分。

且一般只有麦克劳林公式表上的基本初等函数才会使用倒泰勒公式,复合函数最好不要使用。

\textbf{例题:}求极限$\lim\limits_{x\to 0}\dfrac{\arcsin x-\arctan x}{\sin x-\tan x}$。\medskip

分析:该题目使用洛必达法则会比较麻烦且难以计算,所以先考虑是否能用泰勒展开:

$x\to 0$,$\sin x=x-\dfrac{1}{6}x^3+o(x^3)$,$\tan x=x+\dfrac{1}{3}x^3+o(x^3)$,$\arcsin x=x+\dfrac{1}{6}x^3+o(x^3)$,$\arctan x=x-\dfrac{1}{3}x^3+o(x^3)$。

$\therefore \sin x-\tan x=-\dfrac{1}{2}x^3+o(x^3)$,$\arcsin x-\arctan x=\dfrac{1}{2}x^3+o(x^3)$

$\therefore \text{原式}=\dfrac{\dfrac{1}{x}x^3+o(x^3)}{-\dfrac{1}{2}x^3+o(x^3)}=-1$。

\section{极限计算形式}

极限相关计算形式主要分为下面六种:

\begin{enumerate}
    \item 未定式:直接根据式子计算极限值。
    \item 极限转换:根据已知的极限值计算目标极限值。
    \item 求参数:已知式子的极限值,计算式子中未知的参数。
    \item 极限存在性:根据式子以及极限存在性计算极限或参数。
    \item 极限唯一性:式子包含参数,根据唯一性计算两侧极限并求出参数与极限值。
    \item 函数连续性:根据连续性与附加条件计算极限值或参数。
    \item 迭代式数列:根据数列迭代式计算极限值。
    \item 变限积分:根据变限积分计算极限值。
\end{enumerate}

\subsection{极限转换}

\subsubsection{整体换元}

最常用的方式就是将目标值作为一个部分,然后对已知的式子进行替换。

\textbf{例题:}已知$\lim\limits_{x\to 0}\dfrac{\ln(1-x)+xf(x)}{x^2}=0$,求$\lim\limits_{x\to 0}\dfrac{f(x)-1}{x}$。\medskip

令目标$\dfrac{f(x)-1}{x}=t$,$\therefore f(x)=tx+1$。\medskip

$\lim\limits_{x\to 0}\dfrac{\ln(1-x)+xf(x)}{x^2}=\lim\limits_{x\to 0}\dfrac{\ln(1-x)+tx^2+x}{x^2} (\text{泰勒展开})$\medskip

$=\lim\limits_{x\to 0}\dfrac{-x-\dfrac{x^2}{2}+tx^2+x}{x^2}=\lim\limits_{x\to 0}\dfrac{\left(t-\dfrac{1}{2}\right)x^2}{x^2}=\lim\limits_{x\to 0}\left(t-\dfrac{1}{2}\right)=0$

$\therefore\lim\limits_{x\to 0}t=\lim\limits_{x\to 0}\dfrac{f(x)-1}{x}=\dfrac{1}{2}$。

\subsubsection{关系转换}

\textbf{例题:}如果$\lim\limits_{x\to 0}\dfrac{x-\sin x+f(x)}{x^4}$存在,则$\lim\limits_{x\to 0}\dfrac{x^3}{f(x)}$为常数多少?

由$\lim\limits_{x\to 0}\dfrac{x\sin x+f(x)}{x^4}=A$,而目标是$x^3$,所以需要变形:

$\lim\limits_{x\to 0}\dfrac{x\sin x+f(x)}{x^4}=A$

$\lim\limits_{x\to 0}\dfrac{x\sin x+f(x)\cdot x}{x^4}=A\cdot\lim\limits_{x\to 0}x=0$

$\lim\limits_{x\to 0}\dfrac{x-\sin x}{x^3}+\lim\limits_{x\to 0}\dfrac{f(x)}{x^3}=0$

$\text{泰勒展开:}x-\sin x=\dfrac{1}{6}x^3$

$\lim\limits_{x\to 0}\dfrac{f(x)}{x^3}=-\dfrac{1}{6}$

$\lim\limits_{x\to 0}\dfrac{x^3}{f(x)}=-6$

\subsubsection{脱帽法}

$\lim\limits_{x\to x_0}f(x)\Leftrightarrow f(x)=A+\alpha(x),\lim\limits_{x\to x_0}\alpha(x)=0$。

\textbf{例题:}如果$\lim\limits_{x\to 0}\dfrac{x-\sin x+f(x)}{x^4}$存在,则$\lim\limits_{x\to 0}\dfrac{x^3}{f(x)}$为常数多少?

由$\lim\limits_{x\to 0}\dfrac{x\sin x+f(x)}{x^4}=A$脱帽:$\dfrac{x\sin x+f(x)}{x^4}=A+\alpha$。

得到:$f(x)=Ax^4+\alpha\cdot x^4-(x-\sin x)$。

反代入:$\lim\limits_{x\to 0}\dfrac{f(x)}{x^3}=\lim\limits_{x\to 0}\dfrac{Ax^4+\alpha\cdot x^4-x+\sin x}{x^3}=0+0-\dfrac{1}{6}=-\dfrac{1}{6}$。

$\therefore \lim\limits_{x\to 0}\dfrac{x^3}{f(x)}=-6$。

\subsection{求参数}

因为求参数类型的题目中式子是未知的,所以求导后也是未知的,所以一般不要使用洛必达法则,而使用泰勒展开。

一般极限式子右侧等于一个常数,或是表明高阶或低阶。具体的关系参考无穷小比阶。

在求参数的时候要注意与0的关系。\medskip

\textbf{例题:}设$\lim\limits_{x\to 0}\dfrac{\ln(1+x)-(ax+bx^2)}{x^2}=2$,求常数a,b。

根据泰勒展开式:$x\to 0,\ln(1+x)=x-\dfrac{x^2}{x}+o(x^2)$,$x-\ln(1+x)\sim\dfrac{1}{2}x^2\sim 1-\cos x$。

$\lim\limits_{x\to 0}\dfrac{\ln(1+x)-(ax+bx^2)}{x^2}=2$

$=\lim\limits_{x\to 0}\dfrac{(1-a)x-\left(\dfrac{1}{2}+b\right)x^2+o(x^2)}{x^2}=2\neq 0$

$1-a=0;-\left(\dfrac{1}{2}+b\right)=2$\medskip

$\therefore a=1;b=-\dfrac{5}{2}$。

\subsection{极限存在性}

一般会给出带有参数的例子,并给定一个点指明在该点极限存在,求参数。

若该点极限存在,则该点两侧的极限都相等。\medskip

\textbf{例题:}设函数$f(x)=\left\{\begin{array}{lcl}
    \dfrac{\sin x(b\cos x-1)}{e^x+a}, & & x>0 \\
    \dfrac{\sin x}{\ln(1+3x)}, & & x<0
\end{array}
\right.$在$x=0$处极限存在,则$a$,$b$分别为。

解:首先根据极限在$x=0$存在,且极限的唯一性。分段函数在0两侧的极限值必然相等。

$\because\lim\limits_{x\to 0^-}\dfrac{\sin x}{\ln(1+3x)}=\lim\limits_{x\to 0^-}\dfrac{\sin x}{3x}=\dfrac{1}{3}=\lim\limits_{x\to 0^+}\dfrac{\sin x(b\cos x-1)}{e^x+a}$。

\medskip

又$\lim\limits_{x\to 0^+}\dfrac{\sin x(b\cos x-1)}{e^x+a}$的分母的$e^x$当$x\to 0^+$时$e^x\to 1$,假如$a\neq-1$,则$e^x+a\neq 0$,则为一个常数。

从而提取常数因子:$\lim\limits_{x\to 0^+}\dfrac{\sin x(b\cos x-1)}{e^x+a}=\dfrac{1}{1+a}\lim\limits_{x\to 0^+}\sin x(b\cos x-1)$,这时候$\sin x$是趋向0的,而$b\cos x-1$无论其中的$b$为何值都是趋向一个常数或0,这时候他们的乘积必然为无穷小,从而无法等于$\dfrac{1}{3}$这个常数。

$\therefore a=-1$,从而让极限式子变为一个商的形式:\medskip

$\lim\limits_{x\to 0^+}\dfrac{\sin x(b\cos x-1)}{e^x+a}=\lim\limits_{x\to 0^+}\dfrac{\sin x(b\cos x-1)}{e^x-1}=\lim\limits_{x\to 0^+}\dfrac{\sin x(b\cos x-1)}{x}$\medskip

$=\lim\limits_{x\to 0^+}b\cos x-1=b-1=\dfrac{1}{3}$\medskip

$\therefore a=-1,b=\dfrac{4}{3}$。

\subsection{极限唯一性}

若极限存在则必然唯一。

\textbf{例题:}设$a$为常数,$\lim\limits_{x\to 0}\left(\dfrac{e^{\frac{1}{x}}-\pi}{e^{\frac{2}{x}}+1}+a\cdot\arctan\dfrac{1}{x}\right)$存在,求出极限值。

因为求$x\to 0$,所以需要分两种情况讨论:

\medskip

$\lim\limits_{x\to 0^+}\left(\dfrac{e^{\frac{1}{x}}-\pi}{e^{\frac{2}{x}}+1}+a\cdot\arctan\dfrac{1}{x}\right)$

$=\lim\limits_{x\to 0^+}\left(\dfrac{e^{\frac{1}{x}}-\pi}{e^{\frac{2}{x}}+1}\right)+\lim\limits_{x\to 0^+}\left(a\cdot\arctan\dfrac{1}{x}\right)$

$=\lim\limits_{x\to 0^+}\left(\dfrac{0\cdot\left(e^{\frac{2}{x}}\right)^2+e^{\frac{1}{x}}-\pi}{1\cdot\left(e^{\frac{2}{x}}\right)^2+1}\right)+a\cdot\dfrac{\pi}{2}=a\cdot\dfrac{\pi}{2}$

\medskip

$\lim\limits_{x\to 0^-}\left(\dfrac{e^{\frac{1}{x}}-\pi}{e^{\frac{2}{x}}+1}+a\cdot\arctan\dfrac{1}{x}\right)=-\pi+a\cdot\left(-\dfrac{\pi}{2}\right)=-\pi-\dfrac{\pi}{2}\cdot a$

因为极限值具有唯一性,所以$-\pi-\dfrac{\pi}{2}a=\dfrac{\pi}{2}a$,所以$a=-1$,极限值为$-\dfrac{\pi}{2}$。

\subsection{函数连续性}

函数的连续性代表:极限值=函数值。

\subsubsection{判断函数连续}

题目给出函数,往往是分段函数,然后判断分段点的连续性。\medskip

\textbf{例题:}讨论函数$f(x)=\left\{\begin{array}{lcl}
    \left[\dfrac{(1+x)^{\frac{1}{x}}}{e}\right]^{\frac{1}{x}},& & x>0 \\
    e^{-\frac{1}{2}}, & & x\leqslant 0
\end{array}\right.$在$x=0$处的连续性。

因为$\lim\limits_{x\to 0^+}f(x)=\lim\limits_{x\to 0^+}\left[\dfrac{(1+x)^{\frac{1}{x}}}{e}\right]^{\frac{1}{x}}=e^{\lim\limits_{x\to 0^+}\frac{1}{x}\ln[\frac{(1+x)^{\frac{1}{x}}}{e}]}$。

又$\lim\limits_{x\to 0^+}\dfrac{1}{x}\ln\left[\dfrac{(1+x)^{\frac{1}{x}}}{e}\right]=\lim\limits_{x\to 0^+}\dfrac{1}{x}\left[\dfrac{1}{x}\ln(1+x)-1\right]=\lim\limits_{x\to 0^+}\dfrac{\ln(1+x)-x}{x^2}$

$=\lim\limits_{x\to 0^+}\dfrac{\dfrac{1}{1+x}-1}{2x}=\lim\limits_{x\to 0^+}-\dfrac{1}{2(1+x)}=-\dfrac{1}{2}$。

$\therefore\lim\limits_{x\to 0^+}f(x)=e^{-\frac{1}{2}}$。

又$\lim\limits_{x\to 0^-}f(x)=\lim\limits_{x\to 0^-}e^{-\frac{1}{2}}=e^{-\frac{1}{2}}$,且$f(0)=e^{-\frac{1}{2}}$。

从而$\lim\limits_{x\to 0^+}f(x)=\lim\limits_{x\to 0^-}f(x)=f(0)$,所以$f(x)$在$x=0$处连续。

\subsubsection{连续性求极限}

\textbf{例题:}函数在$f(x)$在$x=1$处连续,且$f(1)=1$,求$\lim\limits_{x\to+\infty}\ln\left[2+f\left(x^{\frac{1}{x}}\right)\right]$。

根据题目,所求的$\lim\limits_{x\to+\infty}\ln\left[2+f\left(x^{\frac{1}{x}}\right)\right]$中,唯一未知的且会随着$x\to+\infty$而变换就是$f\left(x^{\frac{1}{x}}\right)$。如果我们可以求出这个值就可以了。

而我们对于$f(x)$的具体的关系是未知的,只知道$f(1)=1$。那么先需要考察$\lim\limits_{x\to+\infty}x^{\frac{1}{x}}$的整数最大值。

$\lim\limits_{x\to+\infty}x^{\frac{1}{x}}=e^{\lim\limits_{x\to+\infty}\frac{\ln x}{x}}=e^{\lim\limits_{x\to+\infty}\frac{1}{x}}=e^0=1$

$\therefore\lim\limits_{x\to+\infty}f(x^{\frac{1}{x}})=f(1)=1$。

\subsection{迭代式数列}

\subsubsection{数列表达式}

最重要的是将迭代式进行变形。

\textbf{例题:}数列$\{a_n\}$满足$a_0=0,a_1=1,2a_{n+1}=a_n+a_{n-1},n=1,2,\cdots$。计算$\lim\limits_{n\to\infty}a_n$。

首先看题目,给出的递推式设计到二阶递推,即存在三个数列变量,所以我们必须先求出对应的数列表达式。因为这个表达式涉及三个变量,所以尝试对其进行变型:

$a_{n+1}-a_n=\dfrac{a_{n-1}-a_n}{2}=\left(-\dfrac{1}{2}\right)(a_n-a_{n-1})=\left(-\dfrac{1}{2}\right)^2(a_{n-1}-a_{n-2})$

$=\cdots$

$=\left(-\dfrac{1}{2}\right)^n(a_1-a_0)=\left(-\dfrac{1}{2}\right)^n$

然后得到了$a_{n+1}-a_n=\left(-\dfrac{1}{2}\right)^n$,而需要求极限,所以使用列项相消法的逆运算:

$a_n=(a_n-a_{n-1})+(a_{n-1}-a_{n-2})+\cdots+(a_1-a_0)+a_0$\medskip

$=\left(-\dfrac{1}{2}\right)^{n-1} + \left(-\dfrac{1}{2}\right)^{n-2} + \cdots + \left(-\dfrac{1}{2}\right)^0$\medskip

$=\dfrac{1\cdot\left(1-\left(-\dfrac{1}{2}\right)^n\right)}{1-\left(-\dfrac{1}{2}\right)}=\dfrac{2}{3}\left[1-\left(-\dfrac{1}{2}\right)^n\right]$\medskip

$\therefore\lim\limits_{n\to\infty}a_n=\dfrac{2}{3}$

\subsubsection{单调有界准则}

对于无法将关系式通过变形归纳为一般式的关系式,对于其极限就必须使用单调有界准则来求出。

单调有界的数列必有极限。需要证明单调性和有界性,然后对式子求极限就能求出目标极限。

\textbf{例题:}$x_0=0$,$x_n=\dfrac{1+2x_{n-1}}{1+x_{n-1}}(n\in N*)$,求$\lim\limits_{n\to\infty}x_n$。\medskip

首先应该知道数列的趋向都是趋向正无穷。

然后对关系式进行变形:$x_n=\dfrac{1+2x_{n-1}}{1+x_{n-1}}=1+\dfrac{x_{n-1}}{1+x_{n-1}}=2-\dfrac{1}{1+x_{n-1}}$。

首先证明单调性,令$f(x)=2-\dfrac{1}{1+x}$。

$\therefore f'(x)=\dfrac{1}{(x+1)^2}>0$,则$f(x)$单调递增。

所以不管$x=x_{n-1}$或其他,$f'(x)>0$,$x_n$都是单调递增,则$x_n\geqslant x_0=0$。

然后证明有界性,$\because x_n\geqslant 0$且单调,$\therefore x_n=2-\dfrac{1}{1+x_{n-1}}\in[0,2]$。

从而$x_n$有界。

所以根据单调有界定理,$x_n$的极限存在。

对于关系式两边取极限:

$\lim\limits_{n\to\infty}x_n=\lim\limits_{n\to\infty}\dfrac{1+2x_{n-1}}{1+x_{n-1}}=\dfrac{1+2\lim\limits_{n\to\infty}x_{n-1}}{1+\lim\limits_{n\to\infty}x_{n-1}}=\dfrac{1+2\lim\limits_{n\to\infty}x_n}{1+\lim\limits_{n\to\infty}x_n}$。

解该一元二次方程:$\lim\limits_{n\to\infty}x_n=\dfrac{1\pm\sqrt{5}}{2}$,又根据保号性,$\lim\limits_{n\to\infty}x_n>0$。

$\therefore\lim\limits_{n\to\infty}x_n=\dfrac{1+\sqrt{5}}{2}$。

而许多题目只给出样子,连通项公式都不会给出。\medskip

\textbf{例题:}求出数列$\sqrt{2}$,$\sqrt{2+\sqrt{2}}$,$\sqrt{2+\sqrt{2+\sqrt{2}}}$$\cdots$的极限。

根据数列样式,无法通过普通的通项公式来表达,所以需要考虑使用递推式来表示:$x_{n+1}=\sqrt{2+x_n}$。

首先证明有界性:

给定一个任意的正整数$k$,再根据递推式,假定$x_k<2$,所以$x_{k+1}=\sqrt{2+x_k}<\sqrt{2+2}=2$。且$x_1=\sqrt{2}$满足假定,所以$x_k<2$对于任意的正整数$k$都成立,所以$x_n$存在上界2。

然后证明单调性,根据其递推式:

$x_{n+1}-x_n=\sqrt{2+x_n}-x_n=\dfrac{2+x_n-x_n^2}{\sqrt{2+x_n}+x_n}=\dfrac{-(x_n-2)(x_n+1)}{\sqrt{2+x_n}+x_n}$。\medskip

又$0<x_n<2$,从而上式子大于0,从而数列单调递增。

所以根据单调有界定理,数列$x_{n+1}=\sqrt{2+x_n}$一定存在极限,令其极限值$\lim\limits_{n\to\infty}x_n=a$。

将递推式两边平方并取极限:$\lim\limits_{n\to\infty}x^2_{n+1}=\lim\limits_{n\to\infty}(2+x_n)$。

从而$a^2=2+a$,得出$a=2$(根据极限的保号性$-1$被舍去)。

\subsection{变限积分极限}

已知更改区间限制的积分$s(x)=\int_{\varphi_1(x)}^{\varphi_2(x)}g(t)\,\textrm{d}x$,$s'(x)=g[\varphi_2(x)]\cdot\varphi_2'(x)-g[\varphi_1(x)]\cdot\varphi_1'(x)$。

\end{document}
