\documentclass[UTF8, 12pt]{ctexart}
% UTF8编码,ctexart现实中文
\usepackage{color}
% 使用颜色
\usepackage{geometry}
\setcounter{tocdepth}{4}
\setcounter{secnumdepth}{4}
% 设置四级目录与标题
\geometry{papersize={21cm,29.7cm}}
% 默认大小为A4
\geometry{left=3.18cm,right=3.18cm,top=2.54cm,bottom=2.54cm}
% 默认页边距为1英尺与1.25英尺
\usepackage{indentfirst}
\setlength{\parindent}{2.45em}
% 首行缩进2个中文字符
\usepackage{setspace}
\renewcommand{\baselinestretch}{1.5}
% 1.5倍行距
\usepackage{amssymb}
% 因为所以
\usepackage{amsmath}
% 数学公式
\usepackage[colorlinks,linkcolor=black,urlcolor=blue]{hyperref}
% 超链接
\author{Didnelpsun}
\title{导数与微分}
\date{}
\begin{document}
\maketitle
\pagestyle{empty}
\thispagestyle{empty}
\tableofcontents
\thispagestyle{empty}
\newpage
\pagestyle{plain}
\setcounter{page}{1}
\section{一阶导数}

\subsection{分段函数导数}

当给出一个分段函数,要求求出该函数的导数时,最重要的就是分段点是否可导,计算分段点的导数,如果两边的导数不相等,则需要挖去该点。\medskip

\textbf{例题:}设$f(x)=\left\{\begin{array}{lcl}
    \arctan x, & & x\leqslant 1 \\
    \dfrac{1}{2}(e^{x^2-1}-x)+\dfrac{\pi}{4}, & & x>1
\end{array}
\right.$,求$f'(x)$。

当$x\leqslant 1$时,$f'(x)=\dfrac{1}{1+x^2}$,当$x>1$时,$f'(x)=xe^{x^2-1}-\dfrac{1}{2}$。

然后需要查看分段点两边的导数是否一样:$f'_-(1)=\dfrac{1}{1+x^2}\,\bigg\vert_{x=1}=\dfrac{1}{1+1}=\dfrac{1}{2}$,$f'_+(1)=xe^{x^2-1}-\dfrac{1}{2}\,\bigg\vert_{x=1}=1\cdot e^{1-1}-\dfrac{1}{2}=\dfrac{1}{2}$。\medskip

$\therefore f'_-(1)=f'_+(1)$,所以该点可导。\medskip

$f(x)=\left\{\begin{array}{lcl}
    \dfrac{1}{1+x^2}, & & x\leqslant 1 \\
    xe^{x^2-1}-\dfrac{1}{2}, & & x>1
\end{array}
\right.$。

\subsection{导数存在性}

导数存在即可导。而该点左右导数都相等该点才可导。

可导必连续,连续不一定可导。

导数的定义:$\lim\limits_{x\to x_0}\dfrac{f(x)-f(x_0)}{x-x_0}$或$\lim\limits_{\Delta x\to 0}\dfrac{f(x+\Delta x)-f(x)}{\Delta x}$。

导数的存在性:若$\lim\limits_{x\to x_0}f'(x)$存在,则$f'(x_0)=\lim\limits_{x\to x_0}f'(x)$。\medskip

\textbf{例题:}设$f(x)=\left\{\begin{array}{lcl}
    \dfrac{\ln(1+bx)}{x}, & & x\neq 0 \\
    -1, & & x=0
\end{array}
\right.$,其中$b$为某常数,$f(x)$在定义域上处处可导,求$f'(x)$。

首先需要求出参数$b$,而定义域上可导则在分段点$x=0$处也必然可导。

而可导必连续,所以当$x=0$时$f(x)$也是连续的,而连续的定义就是两边极限相等,且两边极限等于该点函数值。\medskip

$\lim\limits_{x\to 0}\dfrac{\ln(1+bx)}{x}=\lim\limits_{x\to 0}\dfrac{bx}{x}=b=-1$。从而可以完善函数与定义域。\medskip

$\therefore f(x)=\left\{\begin{array}{lcl}
    \dfrac{\ln(1-x)}{x}, & & x<1,x\neq 0 \\
    -1, & & x=0
\end{array}
\right.$。

这样就能转换为直接求导数问题。

对于定义域的$x<1,x\neq 0$部分:\medskip

$f'(x)=\dfrac{\dfrac{-x}{1-x}-\ln(1-x)}{x^2}=\dfrac{x-(x-1)\ln(1-x)}{x^2(x-1)}\,(x<1,x\neq 0)$。

然后需要求分段点$x=0$处的导数。

可以由导数的定义:、

根据导数的定义是某点偏移量的极限值$\lim\limits_{x\to x_0}\dfrac{f(x)-f(x_0)}{x-x_0}$:

$f'(0)=\lim\limits_{x\to 0}\dfrac{f(x)-f(0)}{x-0}$\medskip

$=\dfrac{\dfrac{\ln(1-x)}{x}-(-1)}{x-0}$\medskip

$=\dfrac{\dfrac{\ln(1-x)}{x}+1}{x}$\medskip

$=\dfrac{\ln(1-x)+x}{x^2}$

泰勒公式:$=\dfrac{-x-\dfrac{1}{2}x^2+o(x^2)+x}{x^2}=-\dfrac{1}{2}$。\medskip

$\therefore f'(x)=\left\{\begin{array}{lcl}
    \dfrac{x-(x-1)\ln(1-x)}{x^2(x-1)}, & & x<1,x\neq 0 \\
    -\dfrac{1}{2}, & & x=0
\end{array}
\right.$。\medskip

同样也可以使用导数的存在性:

$\because f(x)$在$x=0$处连续,$\therefore x=0$的空心邻域上可导。从而$\lim\limits_{x\to x_0}f'(x)$存在。

$\therefore f'(0)=\lim\limits_{x\to 0}f'(x)$。计算过程类似。

\subsection{导数连续性}

导数具有连续性与之前的函数连续性类似,不过要对函数求导数罢了。

要求导数两侧的极限并让其相等。\medskip

\textbf{例题:}设$f(x)=\left\{\begin{array}{lcl}
    x^2, & & x\leqslant 0 \\
    x^\alpha\sin\dfrac{1}{x}, & & x>0
\end{array}
\right.$,若$f'(x)$连续,则$\alpha$应该满足? 

若导数连续,则两侧导数相等。

$\lim\limits_{x\to 0^-}f'(x)=\lim\limits_{x\to 0^-}2x=0$。

$\lim\limits_{x\to 0^+}f'(x)=\lim\limits_{x\to 0^+}\alpha x^{\alpha-1}\sin\dfrac{1}{x}-x^{\alpha-2}\cos\dfrac{1}{x}=\lim\limits_{x\to 0^+}x^{\alpha-2}\left(\alpha x\sin\dfrac{1}{x}-\cos\dfrac{1}{x}\right)$。

$\because x\to 0^+$时,$\sin\dfrac{1}{x}\in[-1,1]$,$\therefore\alpha x\sin\dfrac{1}{x}=0$,$-\cos\dfrac{1}{x}\in[-1,1]$,$\therefore \alpha x\sin\dfrac{1}{x}-\cos\dfrac{1}{x}$为一个不为0的常数。

又$\lim\limits_{x\to 0^+}f'(x)=\lim\limits_{x\to 0^+}x^{\alpha-2}\left(\alpha x\sin\dfrac{1}{x}-\cos\dfrac{1}{x}\right)=\lim\limits_{x\to 0^-}f'(x)=0$。

$\therefore\lim\limits_{x\to 0^+}x^{\alpha-2}=0$。

$\therefore\alpha-2>0$,从而$\alpha>2$。

\subsection{已知导数求极限}

题目会给出对应的导数以及相关条件,并要求求一个极限,这个极限式子并不是个随机的式子,而一个是与导数定义相关的极限式子,所需要的就是将极限式子转换为导数定义的相关式子。

\subsubsection{导数定义式子}

有时极限式子可以直接转换为导数定义式子,稍微变换就可以代入导数。

\textbf{例题:}设$f(x)$是以3为周期的可导函数,且是偶函数,$f'(-2)=-1$,求$\lim\limits_{h\to 0}\dfrac{h}{f(5-2\sin h)-f(5)}$。\medskip

根据导数与函数的基本性质,原函数为偶函数,则其导函数为奇函数,所以$f'(5)=f'(2)=-f'(-2)=1$。

然后需要转换目标的极限式子,因为目标式子倒过来的式子类似于导数定义的$f'(x)=\lim\limits_{\Delta x\to 0}\dfrac{f(x+\Delta x)-f(x)}{\Delta x}$结构。所以我们可以先求其倒数式子:\medskip

$=\lim\limits_{h\to 0}\dfrac{f(5-2\sin h)-f(5)}{h}$

$=\lim\limits_{h\to 0}\dfrac{f(5-2\sin h)-f(5)}{-2\sin h}\cdot\dfrac{-2\sin h}{h}$

$=-2f'(5)=-2\times 1=-2$

$\therefore\lim\limits_{h\to 0}\dfrac{h}{f(5-2\sin h)-f(5)}=-\dfrac{1}{2}$。

\subsubsection{定义近似式子}

有时候极限式子不为导数定义的近似式子,这时候就需要先根据求极限的计算方式简化目标极限式子。



\section{高阶导数}

\subsection{导数存在性}

\section{微分}

\section{隐函数与参数方程}

\section{导数应用}

\subsection{单调性与凹凸性}

\subsection{极值与最值}

\subsection{函数图像}

\subsection{曲率}

曲率公式:$k=\left\lvert\dfrac{\rm{d}\alpha}{\rm{d}s}\right\rvert=\dfrac{\vert y''\vert}{(1+y'^2)^{\frac{3}{2}}}$。

\subsubsection{一般计算}

\textbf{例题:}求$y=\sin xx$在$x=\dfrac{\pi}{4}$对应的曲率

$y'=\cos x$,$y'(\dfrac{\pi}{4})=\dfrac{\sqrt{2}}{2}$。

$y''=-\sin x$,$y''(\dfrac{\pi}{4})=-\dfrac{\sqrt{2}}{2}$。

$\therefore k=\dfrac{\dfrac{\sqrt{2}}{2}}{\dfrac{3}{2}\cdot\sqrt{\dfrac{3}{2}}}=\dfrac{2\sqrt{3}}{9}$。

所以$y=\sin x$在$x=\dfrac{\pi}{4}$的点$(\dfrac{\pi}{4},\dfrac{\sqrt{2}}{2})$的曲率为$\dfrac{2\sqrt{3}}{9}$。

\subsubsection{最值}

\textbf{例题:}求$y=x^2-4x+11$曲率最大值所在的点。

简单得$y'=2x-4$,$y''=2$。

曲率为$\dfrac{2}{[1+(2x-4)^2]^{\frac{3}{2}}}$。

当$2x-4=0$时即在$(2,7)$时曲率最大为2。

\end{document}
