\documentclass[UTF8, 12pt]{ctexart}
% UTF8编码,ctexart现实中文
\usepackage{color}
% 使用颜色
\usepackage{geometry}
\setcounter{tocdepth}{4}
\setcounter{secnumdepth}{4}
% 设置四级目录与标题
\geometry{papersize={21cm,29.7cm}}
% 默认大小为A4
\geometry{left=3.18cm,right=3.18cm,top=2.54cm,bottom=2.54cm}
% 默认页边距为1英尺与1.25英尺
\usepackage{indentfirst}
\setlength{\parindent}{2.45em}
% 首行缩进2个中文字符
\usepackage{setspace}
\renewcommand{\baselinestretch}{1.5}
% 1.5倍行距
\usepackage{amssymb}
% 因为所以
\usepackage{amsmath}
% 数学公式
\author{Didnelpsun}
\title{导数与微分}
\date{}
\begin{document}
\maketitle
\pagestyle{empty}
\thispagestyle{empty}
\tableofcontents
\thispagestyle{empty}
\newpage
\pagestyle{plain}
\setcounter{page}{1}
\section{一阶导数}

\subsection{导数存在性}

导数存在即可导。

\subsection{导数连续性}

\subsection{已知导数求极限}

\section{高阶导数}

\subsection{导数存在性}

\section{微分}

\section{隐函数与参数方程}

\section{导数应用}

\subsection{单调性与凹凸性}

\subsection{极值与最值}

\subsection{函数图像}

\subsection{曲率}

曲率公式:$k=\left\lvert\dfrac{\rm{d}\alpha}{\rm{d}s}\right\rvert=\dfrac{\vert y''\vert}{(1+y'^2)^{\frac{3}{2}}}$。

\subsubsection{一般计算}

\textbf{例题:}求$y=\sin xx$在$x=\dfrac{\pi}{4}$对应的曲率

$y'=\cos x$,$y'(\dfrac{\pi}{4})=\dfrac{\sqrt{2}}{2}$。

$y''=-\sin x$,$y''(\dfrac{\pi}{4})=-\dfrac{\sqrt{2}}{2}$。

$\therefore k=\dfrac{\dfrac{\sqrt{2}}{2}}{\dfrac{3}{2}\cdot\sqrt{\dfrac{3}{2}}}=\dfrac{2\sqrt{3}}{9}$。

所以$y=\sin x$在$x=\dfrac{\pi}{4}$的点$(\dfrac{\pi}{4},\dfrac{\sqrt{2}}{2})$的曲率为$\dfrac{2\sqrt{3}}{9}$。

\subsubsection{最值}

\textbf{例题:}求$y=x^2-4x+11$曲率最大值所在的点。

简单得$y'=2x-4$,$y''=2$。

曲率为$\dfrac{2}{[1+(2x-4)^2]^{\frac{3}{2}}}$。

当$2x-4=0$时即在$(2,7)$时曲率最大为2。

\end{document}
