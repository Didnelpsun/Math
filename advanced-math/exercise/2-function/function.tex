\documentclass[UTF8, 12pt]{ctexart}
% UTF8编码,ctexart现实中文
\usepackage{color}
% 使用颜色
\usepackage{geometry}
\setcounter{tocdepth}{4}
\setcounter{secnumdepth}{4}
% 设置四级目录与标题
\geometry{papersize={21cm,29.7cm}}
% 默认大小为A4
\geometry{left=3.18cm,right=3.18cm,top=2.54cm,bottom=2.54cm}
% 默认页边距为1英尺与1.25英尺
\usepackage{indentfirst}
\setlength{\parindent}{2.45em}
% 首行缩进2个中文字符
\usepackage{setspace}
\renewcommand{\baselinestretch}{1.5}
% 1.5倍行距
\usepackage{amssymb}
% 因为所以
\usepackage{amsmath}
% 数学公式
\usepackage[colorlinks,linkcolor=black,urlcolor=blue]{hyperref}
% 超链接
\author{Didnelpsun}
\title{函数}
\date{}
\begin{document}
\maketitle
\pagestyle{empty}
\thispagestyle{empty}
\tableofcontents
\thispagestyle{empty}
\newpage
\pagestyle{plain}
\setcounter{page}{1}
\section{函数性质}

\subsection{连续}

连续则极限值等于函数值。

\subsubsection{求连续区间}

若要考察一个函数的连续区间,必须要了解函数的所有部分,一般会给出分段函数,所以要了解分段函数的每段函数的性质。

对于函数$f(x)$是个极限表达形式,我们要简化这个极限,最好得到一个$x$的表达式,从而才能判断其连续区间。\medskip

\textbf{例题:}$f(x)=\lim\limits_{n\to\infty}\dfrac{x+x^2e^{nx}}{1+e^{nx}}$,求函数连续区间。\medskip

解:注意到函数的形式为一个极限值,其极限趋向的变量为$n$($n\to\infty$指$n\to+\infty$)。所以在该极限式子中将$x$当作类似$t$的常数。

需要先求出极限形式的$f(x)$,而$x$变量的取值会影响到极限,且求的就是$x$的取值范围。所以将其分为三段:

当$x<0$时,$nx\to-\infty$,$\therefore e^{nx}\to 0$,$x^2$在这个极限式子为一个常数,$\therefore x^2e^{nx}\to 0$,$f(x)=\lim\limits_{n\to\infty}\dfrac{x+x^2e^{nx}}{1+e^{nx}}=\dfrac{x+0}{1+0}=x$。\medskip

当$x=0$时,$f(x)=\lim\limits_{n\to\infty}\dfrac{x+x^2e^{nx}}{1+e^{nx}}=\dfrac{0}{2}=0$。\medskip

当$x>0$时,$e^{nx}$在$n\to\infty$时为$\infty$,上下都有这个无穷大的因子,所以上下都除以$e^{nx}$,$f(x)=\lim\limits_{n\to\infty}\dfrac{x+x^2e^{nx}}{1+e^{nx}}=f(x)=\lim\limits_{n\to\infty}\dfrac{xe^{-nx}+x^2}{1+e^{-nx}}=\dfrac{0+x^2}{1}=x^2$。\medskip

从而得到了$f(x)$关于$x$的表达式:\medskip

$f(x)=\left\{\begin{array}{lcl}
        x,   &  & x<0 \\
        0,   &  & x=0 \\
        x^2, &  & x>0
    \end{array}
    \right.$\medskip

又$\lim\limits_{x\to 0^-}f(x)=\lim\limits_{x\to 0^-}x=\lim\limits_{x\to 0^+}f(x)=\lim\limits_{x\to 0^-}x^2=f(0)=0$。

$f(x)$在$R$上连续。

\subsubsection{已知连续区间求参数}

一般会给出带有参数的分段函数,要计算参数就必须了解连续区间与函数之间的关系。

\textbf{例题:}$f(x)=\left\{\begin{array}{lcl}
        6,                               &  & x\leqslant 0 \\
    \dfrac{e^{ax^3}-1}{x-\arcsin x}, &  & x>0
    \end{array}
    \right.$,$g(x)=\left\{\begin{array}{lcl}
        \dfrac{3\sin(x-1)}{x-1}, &  & x<1          \\
        e^{bx}+1,                &  & x\geqslant 1
    \end{array}
    \right.$,\smallskip \\ 若$f(x)+g(x)$在$R$上连续,则求$a,b$。

解:已知$f(x)+g(x)$在$R$上连续,但是不能判断$f(x)$与$g(x)$的连续性。

所以分开讨论。

对于$f(x)$因为左侧为常数函数,所以若是$f(x)$连续,则必然:\medskip

$\lim\limits_{x\to 0^+}\dfrac{e^{ax^3}-1}{x-\arcsin x}=6$\medskip

$\therefore\lim\limits_{x\to 0^+}\dfrac{e^{ax^3}-1}{x-\arcsin x}=\lim\limits_{x\to 0^+}\dfrac{ax^3}{x-\arcsin x}$\medskip

$\text{令}t=\arcsin x\Rightarrow=\lim\limits_{x\to 0^+}\dfrac{a\sin^3t}{\sin t-t}=a\lim\limits_{x\to 0^+}\dfrac{t^3}{\sin t-t}=a\lim\limits_{x\to 0^+}\dfrac{3t^2}{\cos t-1}$

$=-6a=6$。

$\therefore a=-1$时$f(x)$在$R$上连续。\medskip

对于$g(x)$,当$x<1$时,$\lim\limits_{x\to 1^-}\dfrac{3\sin(x-1)}{x-1}=\lim\limits_{t\to 0^-}\dfrac{3\sin t}{t}=3$。\medskip

$\therefore\lim\limits_{x\to 1^+}e^{bx}+1=e^b+1=3$。\medskip

$\therefore b=\ln 2$时$g(x)$在$R$上连续。\medskip

$\therefore a=-1,b=\ln 2$时$f(x)+g(x)$在$R$上连续。而$a\neq -1$时$f(x)+g(x)$在$x=0$时不连续,$b\neq\ln 2$时$f(x)+g(x)$在$x=1$时不连续。

\subsection{间断}

\subsubsection{求间断点}

求间断点需要首先分析函数的表达形式。

\textbf{例题:}设$f(x)=\lim\limits_{n\to\infty}\dfrac{1+x}{1+x^{2n}}$,求其间断点并分析其类型。

解:根据函数形式,我们需要首先回顾一下幂函数的性质,幂函数的变化趋势取决于底数。

当$x=1$时,$x^n\equiv 1$,当$x\in(-\infty,-1)\cup(1,+\infty)$时,当$n\to\infty$时,$x^n\to\infty$,而$x\in(-1,1)$时,当$n\to\infty$时,$x^n\to 0$。

$\therefore\lim\limits_{n\to\infty}\dfrac{1+x}{1+x^{2n}}=\left\{\begin{array}{lcl}
        0,   &  & x\in(-\infty,-1]\cup(1,+\infty) \\
        1,   &  & x=1                                        \\
        x+1, &  & x\in(-1,1)
    \end{array}
    \right.$

所以分段点为$x=\pm 1$。

当$x=-1$时,$f(-1^+)=f(-1^-)=f(-1)=0$,所以在此处连续。

当$x=1$时,$f(1^+)=0\neq f(1^-)=2$,所以在此处简短,为跳跃间断点。

\subsubsection{已知间断点求参数}

这种题目已知间断点,而未知式子中的参数,只用将间断点代入式子并利用极限计算间断点的类型就可以了。

\textbf{例题:}$f(x)=\dfrac{e^x-b}{(x-a)(x-b)}$有无穷间断点$x=e$,可去间断点$x=1$,求$ab$的值。

解:已知有两个间断点$x=a,x=b$,其中无穷间断点指极限值为无穷的点,可去间断点表示极限值存在且两侧相等,但是与函数值不相等的点。

已经给出两个间断点的值为$x=1$和$x=e$,所以$ab$必然对应其中一个,但是不清楚到底谁是谁。

当$a=1,b=e$时,$f(x)=\dfrac{e^x-e}{(x-1)(x-e)}$。\medskip

当$x\to 1$时,$\lim\limits_{x\to 1}\dfrac{e^x-e}{(x-1)(x-e)}$$=\dfrac{1}{1-e}\lim\limits_{x\to 1}\dfrac{e^x-e}{x-1}$$=\dfrac{e}{1-e}\lim\limits_{x\to 1}\dfrac{e^{x-1}-1}{x-1}$$=\dfrac{e}{1-e}\lim\limits_{x\to 1}\dfrac{x-1}{x-1}$$=\dfrac{e}{1-e}$。\medskip

$\therefore x=1$为可去间断点。\medskip

    当$x\to e$时,$\lim\limits_{x\to e}\dfrac{e^x-e}{(x-1)(x-e)}$$=\dfrac{1}{e-1}\lim\limits_{x\to e}\dfrac{e^x-e}{x-e}$$=\dfrac{e}{e-1}\lim\limits_{x\to e}\dfrac{e^{x-1}-1}{x-e}$\medskip$=\dfrac{e}{e-1}\lim\limits_{x\to e}\dfrac{x-1}{x-e}$$=\dfrac{e(e-1)}{e-1}\lim\limits_{x\to e}\dfrac{1}{x-e}=\infty$。\medskip

$\therefore x=e$为无穷间断点。\medskip

    当$a=e,b=1$时,$f(x)=\dfrac{e^x-1}{(x-e)(x-1)}$。\medskip

    而作为分子的$e^x-1$必然为一个常数,当式子趋向$1$或$e$的时候分母两个不等式中的一个不等式必然为一个常数,从而另一个不等式则变为了无穷小,所以$\lim\limits_{x\to 1}f(x)=\lim\limits_{x\to e}f(x)=\infty$。

$\therefore a=1,b=e$。

\subsection{有界性}

对于一个函数和其导函数,其函数有界性和其导函数有界性不存在必然关系。

对于有限区间,导函数有界是原函数有界的充分必要条件。如$f(x)=\sqrt{x}$在$[0,a]$上有界,$f'(x)=\dfrac{1}{2\sqrt{x}}$在靠近0的一侧无界。如果导函数有界,则在有限区间内增加的增量必然是有界的,所以原函数必然有界。

对于无限区间,导函数有界和原函数有界既非充分也非必要。如$f(x)=x$在$R$上无界,但是$f'(x)=1$有界;$f(x)=\sin x^2$在$R$上有界,而$f'(x)=2x\cos x^2$在$R$上无界。

\section{中值定理}

中值定理一般用于判断不等式。

一般分为连续函数闭区间上的性质(零点定理、介值定理)和微分中值定理(罗尔定理、拉格朗日中值定理、柯西中值定理、泰勒中值定理)。

\subsection{介值定理}

适用于$\exists\xi\in[a,b]$,或函数和$f(a)+f(b)$的形式。

\begin{enumerate}
    \item 利用最值定理$m\leqslant f(x)\leqslant M$。
    \item 利用介值定理。
\end{enumerate}

\textbf{例题:}$f(x)$、$g(x)$在$[a,b]$上连续,且$g(x)>0$,证明第二积分中值定理:存在一点$\xi\in[a,b]$,使得$\int_a^bf(x)g(x)\,\textrm{d}=f(\xi)\int_a^bg(x)\,\textrm{d}x$。

解:首先看式子形式,是积分形式,没有导数,所以不会是四大微分中值定理。

首先将未知数提到一边:$f(\xi)=\dfrac{\int_a^bf(x)g(x)\,\textrm{d}x}{\int_a^bg(x)\,\textrm{d}x}$,这是一个数。

又$f(x)$、$g(x)$在$[a,b]$上连续,由最值定理$m\leqslant f(x)\leqslant M$。

由于已知$g(x)>0$,所以就要对$f(x)$进行处理,这样移动$g(x)$时不等式不会变号,如果要移动$f(x)$则要讨论$f(x)$的正负性。

由保号性,对其积分$\int_a^bg(x)\,\textrm{d}x>0$,从而$m\cdot g(x)\leqslant f(x)g(x)\leqslant M\cdot g(x)$。

再对其积分:$m\int_a^bg(x)\,\textrm{d}x\leqslant\int_a^bf(x)g(x)\,\textrm{d}x\leqslant M\int_a^bg(x)\,\textrm{d}x$。

从而$m\leqslant f(\xi)=\dfrac{\int_a^bf(x)g(x)\,\textrm{d}x}{\int_a^bg(x)\,\textrm{d}x}\leqslant M$。

\subsection{零点定理}

适用于$\exists\xi\in[a,b]$,且$f(a)$、$f(b)$已知。

\begin{enumerate}
    \item 令$\xi=x$,移项得$f(x)=0$。
    \item 令辅助函数$F(x)=f(x)$,验证是否满足零点定理,若满足则得证。
    \item 若不满足,令$F'(x)=f(x)$,对$f(x)$积分取$C=0$,$F(x)=\int f(x)\,\textrm{d}x$,此时需要使用罗尔定理,若满足则得证。
    \item (基本上不会考到)若不满足,令$F''(x)=f(x)$,两次积分得到$F(x)$,将$F(x)$使用泰勒展开为一阶,使用拉格朗日余项。
\end{enumerate}

\subsubsection{直接零点定理}

\textbf{例题:}已知函数$f(x)$在$[0,1]$上连续,且$f(0)=0$,$f(1)=1$,证明$\exists\xi\in(0,1)$使得$f(\xi)=1-\xi$。

解:构造辅助函数$F(x)=f(x)+x-1$。

$F(0)=f(0)+0-1=-1<0$,$F(1)=f(1)+1-1=1>0$。

根据零点定理$\exists\xi\in(0,1)$使得$F(\xi)=0$。

即$\exists\xi\in(0,1)$使得$f(\xi)=1-\xi$。

\subsubsection{罗尔定理}

\textbf{例题:}设$f(x)$、$g(x)$在$[a,b]$上连续,证明$\exists\xi\in(a,b)$,使得$g(\xi)\int_a^\xi f(x)\,\textrm{d}x=f(\xi)\int_\xi^bg(x)\,\textrm{d}x$。

解:由于不含导数,所以不使用微分中值定理。

首先令$\xi=x$,移项$g(x)\int_a^xf(t)\,\textrm{d}t-f(x)\int_x^bg(t)\,\textrm{d}t=0$。

令$F(x)=g(x)\int_a^xf(t)\,\textrm{d}t-f(x)\int_x^bg(t)\,\textrm{d}t$,验证是否满足零点定理。

$F(a)=-f(a)\int_a^bg(t)\,\textrm{d}t$,$F(b)=g(b)\int_a^bf(t)\,\textrm{d}t$。零点定理无法判断。

改令$F'(x)=g(x)\int_a^xf(t)\,\textrm{d}t+f(x)\int_b^xg(t)\,\textrm{d}t$。

所以能看出$F(x)=\int_a^xf(t)\,\textrm{d}t\int_b^xg(t)\,\textrm{d}t$

$F(a)=0$,$F(b)=0$,满足罗尔定理,所以存在$\xi$使得$F'(\xi)=0$,得证。

\subsection{罗尔定理}

重点是找到两点$f(a)=f(b)$。

\subsubsection{寻找原函数}

% 即证明什么就构造他的原函数为函数式子。

\begin{itemize}
    \item 简单型:直接可以看出。
    \item 公式型:$f'(\xi)+g(\xi)f(\xi)=0$,则令$G(x)=\int g(x)\,\textrm{d}x$,$F(x)=e^{G(x)}f(x)$。
\end{itemize}

通过乘积求导公式$(uv)'=u'v+uv'$的逆运算来构造辅助函数。

如$f(x)f'(x)$,作$F(x)=f^2(x)$,$[f'(x)]^2+f(x)f''(x)$,作$F(x)=f(x)f'(x)$,$f'(x)+f(x)\varphi'(x)$,作$F(x)=f(x)e^{\varphi(x)}$。

\textbf{例题:}设$f(x)$在$[a,b]$上连续,在$(a,b)$内可导,$f(a)=b$,$f(b)=a$,$ab$同号,证明$\exists\xi\in(a,b)$使得$f'(\xi)=\dfrac{-f(\xi)}{\xi}$。

解:对$f'(x)+\dfrac{1}{x}f(x)=0$,即$g(x)=\dfrac{1}{x}$,$G(x)=\int\dfrac{1}{x}\textrm{d}x=\ln x$,辅助函数为$F(x)=e^{\ln x}f(x)=xf(x)$。

又$f(a)=b$,$f(b)=a$,$F(a)=af(a)=ab$,$F(b)=bf(b)=ab$,所以$F(a)=F(b)$。根据罗尔定理$\exists\xi\in(a,b)$使得$F'(\xi)=0$,即$f(\xi)+\xi f'(\xi)=0$。

又$ab$同号,所以$ab\neq0$,$\xi\neq0$,从而$f'(x)=\dfrac{-f(\xi)}{\xi}$。

\textbf{例题:}设函数$f(x)$、$g(x)$在$[a,b]$上二阶可导,且$g''(x)\neq0$,$f(a)=f(b)=g(a)=g(b)=0$,证明:

(1)在$(a,b)$内$g(x)\neq0$。

(2)在$(a,b)$内至少有一点$\xi$,使得$\dfrac{f(\xi)}{g(\xi)}=\dfrac{f''(\xi)}{g''(x)}$。

解:

(1)为了简单使用反证法。

假设$\exists c\in(a,b)$,使得$g(c)=0$。对两边区间使用罗尔定理:

$\exists\xi_1\in(a,c)$使得$g'(\xi_1)=0$,$\exists\xi_2\in(a,c)$使得$g'(\xi_2)=0$。

再次使用罗尔定理$\exists\xi\in(\xi_1,\xi_2)\in(a,b)$使得$g''(\xi)=0$,与题设矛盾,则假设不成立。

(2)对式子交叉相乘,令$\xi=x$,即证明$f(x)g''(x)-f''(x)g(x)=0$成立。

由于公式只针对$f(x)$和$f'(x)$没有二阶导,所以将式子加上一阶导的式子并减去,来补全一阶导。

即$f(x)g''(x)+f'(x)g'(x)-f'(x)g'(x)-f''(x)g(x)=0$。将其拆为两个部分$R_1-R_2$。

对于$R_1=f(x)g''(x)+f'(x)g'(x)=0$即得$f'(x)+\dfrac{g''(x)}{g'(x)}f(x)=0$。

$G_1'(x)=\ln(g'(x))$所以$R_1$原函数为$g'(x)f(x)$。

对于$R2=f'(x)g'(x)-f''(x)g(x)=0$即得$f''(x)+\dfrac{g'(x)}{g(x)}f'(x)=0$。

令辅助函数为$F(x)=g'(x)f(x)-g(x)f'(x)$。

$F(a)=g'(a)f(a)-g(a)f'(a)=0$,$F(b)=g'(b)f(b)-g(b)f'(b)=0$,所以根据罗尔定理$\exists\xi\in(a,b)$,使得$F'(\xi)=0$。

即根据上面过程反推回去得证。

\subsubsection{导数关系式}

\paragraph{导数值为常量} \leavevmode \medskip

即$f^{(n)}(\xi)=0$:

\begin{enumerate}
    \item $f^{(n-1)}(x)$在$x=\xi$处满足费马定理。
    \item $f^{(n-1)}$在包含$x=\xi$于其内的闭区间上满足罗尔定理(最常使用)。
    \item 利用泰勒公式展开(基本不会考)。
\end{enumerate}

\textbf{例题:}设$f(x)$在$[0,3]$上连续,$(0,3)$上可导,且$f(0)+f(1)+f(2)=3$,$f(3)=1$,证明存在$\xi\in(0,3)$,使得$f'(\xi)=0$。

解:对于一个导数$f'(\xi)=0$,只会想到罗尔定理和拉格朗日中值定理两种方式。由于这里没有差式,所以很可能是罗尔定理。所以必须找到$f(a)=f(b)$。

由于题目只给出特定条件而没有给出$f(x)$的表达形式,所以无法使用求原函数的方式证明。$f(3)=1$是唯一已知固定常数值的,所以想在$(0,3)$上找到一点$c$,使得$f(c)=f(3)=1$。

所以此时能用的条件就只有$f(0)+f(1)+f(2)=3$,如何转换为存在$f(c)=1$呢?

已知$f(x)$在$[0,3]$上连续,$(0,3)$上可导,则一定存在最大值$M$,最小值$m$,使得$m\leqslant f(0)\leqslant M$,$m\leqslant f(1)\leqslant M$,$m\leqslant f(2)\leqslant M$。\medskip

所以相加$m\leqslant\dfrac{f(0)+f(1)+f(2)}{3}=1\leqslant M$。

由介值定理,可得一定存在某点$c\in[0,2]$,使得$f(c)=\dfrac{f(0)+f(1)+f(2)}{3}=1$。

所以$f(c)=f(3)=1$,根据罗尔定理,$\xi\in(c,3)$上,得证。

\textbf{例题:}$f(x)$在$[0,2]$上连续,在$(0,2)$内二阶可导,$f(0)=f(1)$,$f(2)=2\int_1^{\frac{3}{2}}f(x)\,\textrm{d}x$,证明$\exists\xi\in(0,2)$使得$f''(\xi)=0$。

解:遇到积分式子首先利用积分中值定理$f(2)=2\int_1^{\frac{3}{2}}f(x)\,\textrm{d}x=2f(\xi')(\dfrac{3}{2}-1)=f(\xi')$($1<\xi'<\dfrac{3}{2}$)。

所以根据罗尔定理$\exists\xi_1\in(\xi',2)$使得$f'(\xi_1)=0$。

又$f(0)=f(1)$,$\therefore\exists\xi_2\in(0,1)$,罗尔定理$f'(x_2)=0$。

$\therefore f'(\xi_1)=f'(\xi_2)$,罗尔定理$\exists\xi\in(\xi_1,\xi_2)$使得$f''(\xi)=0$。

\paragraph{导数方程} \leavevmode \medskip

类似微分方程,代数式由$a$、$b$、$f(a)$、$f(b)$、$f(\xi)$、$f'(\xi)$、$\cdots$、$f^{(n)}(\xi)$构成。

若$\xi$是$ab$可分离的式子:

\begin{itemize}
    \item 拉格朗日中值定理:分离出$f(b)-f(a)$或$\dfrac{f(b)-f(a)}{b-a}$。
    \item 柯西中值定理:分离出$\dfrac{f(b)-f(a)}{F(b)-F(a)}$。
\end{itemize}

若$\xi$是$ab$不可分离的式子:

\begin{enumerate}
    \item 构造辅助函数。
    \item 利用罗尔定理。
\end{enumerate}

\subsubsection{零点情况}

\paragraph{直接式子} \leavevmode \medskip

需要证明所给式子的导数是否在该区间为0即可。

\textbf{例题:}证明多项式$f(x)=x^3-3x+a$在$[0,1]$上不可能有两个零点。

证明:假设$f(x)=x^3-3x+a$在$[0,1]$有两个零点$x_1$和$x_2$,其中$x_1<x_2$。

因为$f(x)=x^3-3x+a$在$[0,1]$内连续,所以$f(x)=x^3-3x+a$在$[0,1]$内可导。

由罗尔定理得知$\exists\xi\in(x_1,x_2)\subset(0,1)$,使得$f'(\xi)=0$,但是$f'(x)=3x^2-3$在$(0,1)$上不超过0,所以$\xi$不存在,从而多项式$f(x)=x^3-3x+a$在$[0,1]$上不可能有两个零点。

\paragraph{含参数式子} \leavevmode \medskip

若所求式子是一个含参数,那么其一定还有另一个式子约束参数,此时我们就需要构建一个新的式子来利用所给的条件,然后将新式子转换为旧式子。

\textbf{例题:}设$a_0+\dfrac{a_1}{2}+\cdots+\dfrac{a_n}{n+1}=0$,证明多项式$f(x)=a_0+a_1x+\cdots+a_nx^n$在$(0,1)$中至少有一个零点。

证明:因为所要证明零点,所以一定使用罗尔定理。所给出的约束参数式子$a_0+\dfrac{a_1}{2}+\cdots+\dfrac{a_n}{n+1}=0$与所求$f(x)$之间存在一个关系。

设$F(x)=a_0x+a_1\dfrac{x^2}{2}+\cdots+a_n\dfrac{x^{n+1}}{n+1}$,$F'(x)=a_0+a_1x+\cdots+a_nx^n=f(x)$。

又$F(0)=0$,$F(1)=a_0+\dfrac{a_1}{2}+\cdots+\dfrac{a_n}{n+1}=0$,又罗尔定理一定存在一个$\xi\in(0,1)$,使得$F'(\xi)=f(\xi)=0$。

从而$f(x)=a_0+a_1x+\cdots+a_nx^n$在$(0,1)$中至少有一个零点。

\subsection{拉格朗日中值定理}

证明不等式最重要的还是找到$f(x)$,即出现差值$f(a)-f(b)$,那么$f(x)$就是我们的目标函数,有时候不等式不存在$f(a)-f(b)$这种式子,就需要我们转换,常用的就是$f(b)=0$。

注意$a$和$b$可以是常量也可以是变量,$\xi$依赖$a$和$b$,所以$f'(\xi)$不为常数不能随便提出来。

如$f(x)-f(a)=f'(\xi)(x-a)\Rightarrow f(x)=f(a)+f'(\xi)(x-a)\nRightarrow\int_a^bf(x)\,\textrm{d}x=\int_a^bf(a)\,\textrm{d}x+f'(\xi)\int_a^b(x-a)\,\textrm{d}x$。

\subsubsection{式子转换}

使用初等运算将目标式子转换减式。

\textbf{例题:}设$f(x)$在闭区间$[0,c]$上连续,其导数$f'(x)$在开区间$(0,c)$内存在且单调减少,又$f(0)=0$,证明$f(a+b)\leqslant f(a)+f(b)$,$0\leqslant a\leqslant b\leqslant a+b\leqslant c$。

解:不存在两端点相等的条件,所以使用拉格朗日中值定理。

因为所要证明的式子中含有$a$、$b$、$a+b$,$f(0)=0$,所以对这几个区间进行拉格朗日中值定理。

证明式子中没有减的形式只有和的形式,所以需要对其转换。

$f(a)-f(0)=f'(\xi_1)(a-0)$,$f(a+b)-f(b)=f'(\xi_2)(a+b-b)$。

从而$f(a)=f'(\xi_1)a$,$f(a+b)-f(b)=f'(\xi_2)a$。

又$f'(x)$单调减少,所以$f'(\xi_1)>f'(\xi_2)$。

$f(a)\geqslant f(a+b)-f(b)$,所以$f(a+b)\leqslant f(a)+f(b)$。

\subsubsection{求原函数}

这种题目就是证明某个式子成立,式子一边是常数一边是导数式子,要证明,就要将导数式子转换为原函数,方法跟罗尔定理使用的转换原函数的技巧一样。

\textbf{例题:}设$f(x)$在$[0,1]$上连续且可导,证明存在一点$\xi\in(0,1)$,使得$f(1)=3\xi^2f(\xi)+\xi^3f'(\epsilon)$。

证明:由$3\xi^2f(\xi)+\xi^3f'(\epsilon)$,可推出原函数为$x^3f(x)$,令$F(x)=x^3f(x)$,则其在$(0,1)$也可导。

即使用拉格朗日中值定理,$F(1)-F(0)=F'(\xi)$,$\xi\in(0,1)$。即$f(1)=3\xi^2f(\xi)+\xi^3f'(\epsilon)$。

\subsubsection{对数函数特性}

对于对数函数,要记住其特定的性质:$\log_n(\dfrac{a}{b})=\log_na-\log_nb$。

\textbf{例题:}设$a>b>0$,证明:$\dfrac{a-b}{a}<\ln\dfrac{a}{b}<\dfrac{a-b}{b}$。

证明:因为$\ln\dfrac{a}{b}=\ln a-\ln b$,所以令$f(x)=\ln x$。

所以根据拉格朗日中值定理:$\ln a-\ln b=f'(\xi)(a-b)$($\xi\in(b,a)$)。

又$f'(\xi)=\dfrac{1}{\xi}$,所以$\ln a-\ln b=\dfrac{a-b}{\xi}$。

又$\xi\in(b,a)$,所以$\dfrac{1}{\xi}\in(\dfrac{1}{a},\dfrac{1}{b})$。

所以$\dfrac{a-b}{a}<\dfrac{a-b}{\xi}<\dfrac{a-b}{b}$,从而$\dfrac{a-b}{a}<\ln\dfrac{a}{b}<\dfrac{a-b}{b}$,得证。

\subsubsection{划分区间}

证明存在两个不同的点在同一个区间满足一个不等式。如果两个点彼此存在一定关系,如上面式子转换的例子$a+b$,$a$,$b$,那么我们可以使用转换,如果两个完全独立的变量,则这种方式没用,我们可以考虑划分区间,假定这两个点在不同的区间,中间以一个区间变量分隔,由于拉格朗日中值定理中两个变量只会出现一次,而间隔变量会出现多次,所以对其分别拉格朗日中值定理,就可以把两个变量换成以间隔变量表示的形式,将两个无关变量的式子变成一个变量的式子。

\textbf{例题:}设函数$f(x)$在$[0,1]$上连续,在$(0,1)$内可导,且$f(0)=0$,$f(1)=1$,证明存在不同的$\xi_1$、$\xi_2$,使得$\dfrac{1}{f'(\xi_1)}+\dfrac{1}{f'(\xi_2)}=2$。

证明:使用$\xi$将$[0,1]$划分为$[0,\xi]$和$[\xi,1]$两个区间,假定$\xi_1$、$\xi_2$分别在这两个区间上。

分别对其进行拉格朗日:$f(\xi)-f(0)=f'(\xi_1)(\xi-0)$,即$\dfrac{1}{f'(\xi_1)}=\dfrac{\xi}{f(\xi)}$,$f(1)-f(\xi)=f'(\xi_2)(1-\xi)$,即$\dfrac{1}{f'(\xi_2)}=\dfrac{1-\xi}{1-f(\xi)}$。

即$\dfrac{1}{f'(\xi_1)}+\dfrac{1}{f'(\xi_2)}=\dfrac{\xi}{f(\xi)}+\dfrac{1-\xi}{1-f(\xi)}$,任取$f(\xi)=\dfrac{1}{2}$,原式等于2,得证。

\subsubsection{查找特定值}

对于证明一种不等式,如果里面没有差式,也无法转换为差式(没有相同的$f(x)$),那么就可以考虑制造差式,对于$f(x)$一般选择更高阶的,$a$选择$x$,$b$要根据题目和不等式设置一个常数。

一般是0或1。可以先尝试1。

对于这种不等式子看上去一般不会想到拉格朗日中值定理。

\textbf{例题:}当$x>1$时,证明$e^x>ex$。

证明:题目中没有差式,所以需要选择一个函数作为基准函数,里面有一个指数函数和一个幂函数,所以选择$e^x$作为基准函数。

然后选择一个常数作为$b$值,可以先选一个1作为$b$值:$f(x)-f(1)=f'(\xi)(x-1)$。

从而$e^x-e=e^\xi(x-1)$,$\xi\in(1,x)$,所以$e^x-e>e(x-1)$,即$e^x>ex$,得证。

\subsubsection{目标为常数}

使用拉格朗日中值定理的推论:$f(x)$在$I$上连续且可导,则$I$上$f(x)=C\Leftrightarrow f'(x)\equiv 0$。

\textbf{例题:}证明$\arctan e^x+\arctan e^{-x}=\dfrac{\pi}{2}$。

解:令$F(x)=\arctan e^x+\arctan e^{-x}-\dfrac{\pi}{2}$。

$F'(x)=\dfrac{e^x}{1+e^{2x}}+\dfrac{-e^{-x}}{1+e^{-2x}}=0$。

根据拉格朗日中值定理,$F(x)\equiv C$。

又$F(0)=\arctan 1+\arctan 1-\dfrac{\pi}{2}=0$,$C=0$,得证。

\subsection{柯西中值定理}

需要找到两个函数,使得$\dfrac{f(b)-f(a)}{F(b)-F(a)}=\dfrac{f'(\xi)}{F'(\xi)}$。

\textbf{例题:}设$0<a<b$,函数$f(x)$在$[a,b]$上连续,在$(a,b)$内可导,证明存在一点$\xi\in(a,b)$使得$f(b)-f(a)=\xi f'(\xi)\ln\dfrac{b}{a}$。

证明:由对数函数的特性可以知道$\dfrac{b}{a}=\ln b-\ln a$,所以可以令$F(x)=\ln x$,所以$F'(x)=\dfrac{1}{x}$。

$f(b)-f(a)=\xi f'(\xi)\ln\dfrac{b}{a}=\dfrac{f(b)-f(a)}{\ln b-\ln a}=\dfrac{f'(\xi)}{\dfrac{1}{\xi}}\dfrac{f(b)-f(a)}{F(b)-F(a)}=\dfrac{f'(\xi)}{F'(\xi)}$。

根据柯西中值定理得证。

\subsection{泰勒中值定理}

\begin{enumerate}
    \item $n$阶可导则只展开到$n-1$阶,第$n$阶留给余项。
    \item 只要给出二阶或二阶以上可导,可以先泰勒展开看看。
\end{enumerate}

证明$\exists\xi\in(a,b)$使得$F(\xi,f^{(n)}(\xi))\geqslant0$($n\geqslant2$)使用带拉格朗日余项的泰勒公式。

\begin{enumerate}
    \item 把$f(x)$在给定信息最多的点上泰勒展开。
    \item 若两点信息一样多,则优先在有导数信息的一点展开。
    \item 若两点都是导数信息或函数信息,则对两点分别展开。
\end{enumerate}

\textbf{例题:}$f(x)$在$[a,b]$上二阶可导,$f'(a)=f'(b)=0$,证明$\exists\xi\in(a,b)$,使得$\vert f''(\xi)\vert\geqslant4\dfrac{\vert f(b)-f(a)\vert}{(b-a)^2}$。

解:由于$ab$两点信息一样多,所以对两点进行泰勒一阶展开:

$f(x)=f(a)+\dfrac{f''(\xi_1)}{2!}(x-a)^2$,$f(x)=f(b)+\dfrac{f''(\xi_2)}{2!}(x-b)^2$。

因为最后要证明的式子里面$ab$都要有,所以代入$x=\dfrac{a+b}{2}$:

$f\left(\dfrac{a+b}{2}\right)=f(a)+\dfrac{f''(\xi_1')}{8}(b-a)^2$,$f\left(\dfrac{a+b}{2}\right)=f(b)+\dfrac{f''(\xi_2')}{8}(b-a)^2$($\xi_1$和$\xi_2$是关于$ab$的变量,此时都发生变化了)

两式相减得$f(b)-f(a)=\dfrac{(b-a)^2}{8}[f''(\xi_1')-f''(\xi_2)]$,取绝对值$\vert f(b)-f(a)\vert=\dfrac{(b-a)^2}{8}\vert f''(\xi_1')-f''(\xi_2)\vert\leqslant\dfrac{(b-a)}{8}\vert f''(\xi_1')\vert+\vert f''(\xi_2')\vert$。

取$\xi=\max\{\xi_1',\xi_2'\}$,$\dfrac{(b-a)}{4}\vert f''(\xi')\vert\geqslant\vert f(b)-f(a)\vert$,得证。

\subsection{双中值问题}

即有两个值$\xi$和$\eta$,分离$\xi$和$\eta$,两边都使用一次中值定理:

\begin{itemize}
    \item 不要求$\xi\neq\eta$,对$a,b$使用两次中值定理,要么两次拉格朗日中值定理,要么一次拉格朗日中值定理一次柯西中值定理。
    \item 要求$\xi\neq\eta$,找点$c$,在$(a,c)$、$c,b$区间上分别使用两次拉格朗日中值定理或一次拉格朗日中值定理一次柯西中值定理。对于$c$的取值,要么是第一问给出,要么逆推假设$c$已知根据要证的结论推出$c$。(基本上不会考到)
\end{itemize}

重点关注中值项,其余不用管。

\subsubsection{两次拉格朗日}

\textbf{例题:}$f(x)$在$[a,b]$上连续,$(a,b)$内可导,且$f(a)=f(b)=1$,证明存在$\xi,\eta\in(a,b)$使得$e^{\eta-\xi}[f(\eta)+f'(\eta)]=1$。

解:首先将$\xi$和$\eta$分开:$e^\eta[f(\eta)+f'(\eta)]=e^\xi$。根据拉格朗日中值定理$f'(\xi)=\dfrac{f(b)-f(a)}{b-a}$:

对于右边$e^\xi=t'(\xi)$,$t(x)=\int e^x\,\textrm{d}x=e^x$,$\therefore e^\xi=\dfrac{e^b-e^a}{b-a}$。

对于左边$e^\eta[f(\eta)+f'(\eta)]=[e^xf(x)]'_{x=\eta}$,$\therefore e^\eta[f(\eta)+f'(\eta)]=\dfrac{e^bf(b)-e^af(a)}{b-a}\\=\dfrac{e^b-e^a}{b-a}=e^\xi$,所以等式成立,得证。

\subsubsection{一次拉格朗日一次柯西}

\textbf{例题:}$f(x)$在$[a,b]$上连续,$(a,b)$内可导,且$f'(x)\neq0$,证明存在$\xi,\eta\in(a,b)$使得$\dfrac{f'(\xi)}{f'(\eta)}=\dfrac{e^b-e^a}{b-a}e^{-\eta}$。

解:首先将$\xi$和$\eta$分开:$f'(\xi)=\dfrac{e^b-e^a}{b-a}\dfrac{e^{-\eta}}{f'(\eta)}$。

对$f'(\xi)$使用拉格朗日,$=\dfrac{f(b)-f(a)}{b-a}$。对$\dfrac{e^{-\eta}}{f'(\eta)}$使用柯西,$=\dfrac{f(b)-f(a)}{e^b-e^a}$。

左右相等,得证。

\section{导数应用}

\subsection{单调性}

\textbf{例题:}求$y=x+\vert\sin 2x\vert$的单调区间。

解:因为函数的定义域为$R$。

又$y=\left\{\begin{array}{lcl}
    x+\sin 2x, & & n\pi\leqslant x\leqslant n\pi+\dfrac{\pi}{2} \\
    x-\sin 2x, & &n\pi+\dfrac{\pi}{2}\leqslant x\leqslant (n+1)\pi
\end{array}\right.$($n=0,\pm 1,\pm2,\cdots$)。

$\therefore y'=\left\{\begin{array}{lcl}
    1+2\cos 2x, & & n\pi\leqslant x\leqslant n\pi+\dfrac{\pi}{2} \\
    1-2\cos 2x, & &n\pi+\dfrac{\pi}{2}\leqslant x\leqslant (n+1)\pi
\end{array}\right.$($n=0,\pm 1,\pm2,\cdots$)。

令$y'=0$,所以得到驻点为$x=n\pi+\dfrac{\pi}{3}$和$x=n\pi+\dfrac{5\pi}{6}$。

分割区间:$\left[n\pi,n\pi+\dfrac{\pi}{3}\right]$,$\left[n\pi+\dfrac{\pi}{3},n\pi+\dfrac{\pi}{2}\right]$,$\left[n\pi+\dfrac{\pi}{2},x=n\pi+\dfrac{5\pi}{6}\right]$,

$\left[x=n\pi+\dfrac{5\pi}{6},(n+1)\pi\right]$($n=0,\pm 1,\pm2,\cdots$)。

当$x\in\left[n\pi,n\pi+\dfrac{\pi}{3}\right]$,$y'>0$,所以函数在区间上单调递增。

当$x\in\left[n\pi+\dfrac{\pi}{3},n\pi+\dfrac{\pi}{2}\right]$,$y'<0$,所以函数在区间上单调递减。

当$x\in\left[n\pi+\dfrac{\pi}{2},x=n\pi+\dfrac{5\pi}{6}\right]$,$y'>0$,所以函数在区间上单调递增。

当$x\in\left[x=n\pi+\dfrac{5\pi}{6},(n+1)\pi\right]$,$y'<0$,所以函数在区间上单调递减。

从而函数在$\left[\dfrac{k\pi}{2},\dfrac{k\pi}{2}+\dfrac{\pi}{3}\right]$时单调增加,在$\left[\dfrac{k\pi}{2}+\dfrac{\pi}{3},\dfrac{k\pi}{2}+\dfrac{\pi}{2}\right]$上单调减少($k=0,\pm 1,\pm2,\cdots$)。

\subsection{凹凸性}

二阶导数为0处就是拐点。

\textbf{例题:}决定曲线$y=ax^3+bx^2+cx+d$中参数,使得$x=-2$处曲线有水平切线,$(1,-10)$为拐点,且点$(-2,44)$在曲线上。

解:$y'=3ax^2+2bx+c$,$y''=6ax+2b$。

因为$x=-2$处曲线有水平切线,即$y'\vert_{x=-2}=12a-4b+c=0$。

$(1,-10)$为拐点,代入:$y''\vert_{x=1}=6a+2b=0$,$y\vert_{x=1}=a+b+c+d=-10$。

又点$(-2,44)$在曲线上,所以$y\vert_{x=-2}=-8a+4b-2c+d=44$。

解得四个方程:$a=1$,$b=-3$,$c=-24$,$d=16$。

\subsection{极值与最值}

求极值需要考虑$y'$与点两边正负号,如果$y''$存在则可以考虑,$y''<0$则取极大值,$y''>0$则取极小值。

对于最值需要考虑极值和闭区间端点两个部分。

\subsection{渐近线}

\subsection{拐点}

\subsubsection{已知二阶导函数}

不仅要关注$f''(x)$函数,还要关注$f''(x)$未定义的点。

\begin{itemize}
    \item $f''(x_0)=0$,且左右两边正负号发生改变。
    \item $f''(x_0)$在$x_0$处无定义,但是左右两边正负号发生改变。
\end{itemize}

\subsubsection{已知导函数}

要求原函数的拐点(求二阶导),就要求导函数(一阶导)的驻点(一阶导),即关心导函数的极值点。

如果驻点两端的单调性相反,即是极值点,则代表这里就是拐点。

\subsubsection{已知原函数}

要关注的是函数图像凹凸性发生变化的点。

\subsection{零点}

\subsubsection{零点定理}

若$f(x)$在$[a,b]$上连续,且$f(a)f(b)<0$,则$f(x)=0$在$(a,b)$内至少有一个根。其中$ab$是具体数也可以是无穷大。如果是无穷大时需要求$f(x)$到正负无穷的极限值。

用于证明存在某一个零点。

\subsubsection{单调性}

若$f(x)$在$(a,b)$内单调($f'(x)$存在且不恒等于0),则$f(x)=0$在$(a,b)$内至多有一个根。

用于证明只有一个零点。

当用于证明有且仅有一个实根时需要将零点定理和单调性一同使用。

\textbf{例题:}

\subsubsection{罗尔原话}

若$f^{(n)}(x)=0$至多有$k$个根,则$f(x)=0$至多有$k+n$个根。是罗尔定理的推论。

即若$f(x)=0$至少有两个根,则$f'(x)$至少有一个根。一般证明$f(x)$至多有$k$个实根,就要求$f(x)$的$k$次导。

由于罗尔原话只能推断出至多多少个实根,所以我们往往要带值进入$f(x)$计算出至少多少个实根。

\textbf{例题:}证明方程$2^x-x^2=1$有且仅有3个实根。

解:令$f(x)=2^x-x^2-1$,则$f'(x)=\ln22^x-2x$,$f''(x)=(\ln2)^22^x-2$,$f'''(x)=(\ln2)^32^x\neq 0$。

所以$f'''(x)=0$至多0个根。所以根据罗尔原话$f(x)=0$至多三个根。

又观察法$f(0)=0$,$f(1)=0$得到两个实根。

$f(4)=-1$,$f(5)=6$,所以$(4,5)$内存在一个实根,从而一共与三个根。

\subsubsection{实系数奇次方程}

实系数奇次方程至少有一个实根。即$x^{2n+1}+a_1x^{2n}+\cdots+a_{2n}x+a_{2n+1}=0$至少与一个实根。

这个判断法则往往要和罗尔原话一起使用确定是否有且仅有常数个实根。

\textbf{例题:}若$3a^2-5b<0$,则方程$x^5+2ax^3+3bx+4c=0$()。

$A.\text{无实根}$\qquad$B.\text{有唯一实根}$\qquad$C.\text{有三个不同实根}$\qquad$D.\text{与五个不同实根}$

解:令$f(x)=x^5+2ax^3+3bx+4c$,该实系数奇次方程至少有一个根。

$f'(x)=5x^4+6ax^2+3b$,令$t=x^2$,$5t^2+6at+3b=0$。

$\Delta=36a^2-4\cdot5\cdot3b=36a^2-60b=12(3a^2-5b)<0$。

$\therefore f'(x)$无实根,所以$t=x^2$解不出来,所以$f'(x)\neq0$。

$f'(x)=0$至多0个根。所以根据罗尔原话$f(x)=0$至多一个根,又由上面至少一个根,所以只有一个根,选择$B$。

\subsubsection{函数含参导数不含参}

参数是一个加在式子上的常数,函数求导后参数就被消掉了,所以可以在计算过程中不考虑参数,等到了最后的结果再讨论参数。

\textbf{例题:}设常数$k>0$,函数$f(x)=\ln x-\dfrac{x}{e}+k$在$(0,+\infty)$内的零点个数为()。

$A.3$\qquad$B.2$\qquad$C.1$\qquad$D.0$

解:$f'(x)=\dfrac{1}{x}-\dfrac{1}{e}$,令其为0,则$x=e$。

$x\in(0,e)$,$f'(x)>0$,$f(x)\nearrow$,$x\in(e,+\infty)$,$f'(x)<0$,$f(x)\searrow$。

又$f(e)=k>0$,$\lim\limits_{x\to0^+}f(x)=\lim\limits_{x\to0^+}(\ln x-\dfrac{x}{e}+k)=-\infty$,所以左边有一个根,$\lim\limits_{x\to+\infty}f(x)=\lim\limits_{x\to+\infty}(\ln x-\dfrac{x}{e}+k)=-\infty$,所以一共有两个根。

\subsubsection{函数导数含参}

参数与自变量进行运算,从而求导后参数仍在式子中,计算时需要携带参数来思考。

\textbf{例题:}求方程$k\arctan x-x=0$的不同实根的个数,其中$k$为参数。

解:令$f(x)=k\arctan x-x$$,\because f(-x)=-f(x)$,所以$f(x)$是一个奇函数,所以可以只要考虑一边的情况。$x=0$是函数的一个根。

$f'(x)=\dfrac{k}{1+x^2}-1=\dfrac{k-1-x^2}{1+x^2}$。

若$k-1\leqslant0$即$k<1$则$f'(x)\leqslant0$,所以$f(x)$单调减少,从而只有一个根。

若$k-1>0$即$k>1$,令$f'(x)=0$,即$k-1-x^2=0$,$x=\sqrt{k-1}$。

$x\in(0,\sqrt{k-1})$,$f'(x)>0$,$f(x)\nearrow$。$x\in(\sqrt{k-1},+\infty)$,$f'(x)<0$,$f(x)\searrow$。

$\lim\limits_{x\to+\infty}(k\arctan x-x)=-\infty$,所以在0的右侧一定存在一个零点,同理左边也因为奇函数对称存在一个零点,所以一共有三个根。

\end{document}
