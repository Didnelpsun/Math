\documentclass[UTF8, 12pt]{ctexart}
% UTF8编码,ctexart现实中文
\usepackage{color}
% 使用颜色
\definecolor{orange}{RGB}{255,127,0} 
\usepackage{geometry}
\setcounter{tocdepth}{4}
\setcounter{secnumdepth}{4}
% 设置四级目录与标题
\geometry{papersize={21cm,29.7cm}}
% 默认大小为A4
\geometry{left=3.18cm,right=3.18cm,top=2.54cm,bottom=2.54cm}
% 默认页边距为1英尺与1.25英尺
\usepackage{indentfirst}
\setlength{\parindent}{2.45em}
% 首行缩进2个中文字符
\usepackage{setspace}
\renewcommand{\baselinestretch}{1.5}
% 1.5倍行距
\usepackage{amssymb}
% 因为所以
\usepackage{amsmath}
% 数学公式
\usepackage[colorlinks,linkcolor=black,urlcolor=blue]{hyperref}
% 超链接
\author{Didnelpsun}
\title{不定积分与定积分}
\date{}
\begin{document}
\maketitle
\pagestyle{empty}
\thispagestyle{empty}
\tableofcontents
\thispagestyle{empty}
\newpage
\pagestyle{plain}
\setcounter{page}{1}
\section{不定积分}
\subsection{基本积分}

\textbf{例题:}汽车以20m/s的速度行驶,刹车后匀减速行驶了50m停止,求刹车加速度。

已知题目含有两个变量:距离和时间,设距离为$s$,时间为$t$。

因为汽车首先按20m/s匀速运动,所以$\dfrac{\textrm{d}s}{\textrm{d}t}\bigg\vert_{t=0}=20$,最开始距离为0,所以$s\vert_{t=0}=0$。

又因为是匀减速的,所以速度形如:$v=\dfrac{s}{t}=kt+b$,从而令二阶导数下$\dfrac{\textrm{d}^2s}{\textrm{d}t^2}=k$。

所以$\displaystyle{\dfrac{\textrm{d}s}{\textrm{d}t}=\int\dfrac{\textrm{d}^2s}{\textrm{d}t^2}\,\textrm{d}t=\int k\,\textrm{d}t}=kt+C_1$。

代入$\dfrac{\textrm{d}s}{\textrm{d}t}\bigg\vert_{t=0}=20$,所以$C_1=20$,即$\dfrac{\textrm{d}s}{\textrm{d}t}=kt+20$。

所以$\textrm{d}s=(kt+20)\,\textrm{d}t$,从而$s=\displaystyle{\int(kt+20)\,\textrm{d}t}=\dfrac{1}{2}kt^2+20t+C_2$。

又$s\vert_{t=0}=0$,所以代入得$C_2=0$,所以$s=\dfrac{1}{2}kt^2+20t$。

当$s=50$时停住,所以此时$\dfrac{\textrm{d}s}{\textrm{d}t}=0$,得到$t=-\dfrac{20}{k}$。

代入$s$:$50=\dfrac{1}{2}k\left(-\dfrac{20}{k}\right)^2+20\left(-\dfrac{20}{k}\right)$,解得$k=-4$,即加速度为-4m/$s^2$。

\subsection{换元积分}

使用换元法做了换元之后是要带回式子中的,也就是说要保证反函数的存在才能代入有意义。为了保证反函数的存在,因此要保证原函数的单调性,所以要有一个规定的范围来使原函数保证单调。

对于第二类换元特别要考虑这方面,而第一类换元一般不考虑。

\subsubsection{第一类换元}

\paragraph{聚集因式} \leavevmode \medskip

将复杂的式子转换为简单的一个因式放到$\textrm{d}$后面看作一个整体,然后利用基本积分公式计算。

\textbf{例题:}$\displaystyle{\int\dfrac{\textrm{d}x}{x\ln x\ln\ln x}}$。 \medskip

$=\displaystyle{\int\dfrac{\textrm{d}(\ln x)}{\ln x\ln\ln x}}=\displaystyle{\int\dfrac{\textrm{d}(\ln\ln x)}{\ln\ln x}}=\ln\vert\ln\ln x\vert+C$。\medskip

\textbf{例题:}$\displaystyle{\int\dfrac{10^{2\arccos x}}{\sqrt{1-x^2}}\,\textrm{d}x}$。

$=-\displaystyle{\int10^{2\arccos x}\,\textrm{d}(\arccos x)}=-\dfrac{1}{2}\displaystyle{\int10^{2\arccos x}\,\textrm{d}(2\arccos x)}=-\dfrac{10^{2\arccos x}}{2\ln10}+C$。

\paragraph{积化和差} \leavevmode \medskip

对于两个三角函数的乘积可以使用积化和差简单计算。

\textbf{例题:}$\displaystyle{\int\sin2x\cos3x\,\textrm{d}x}$。

$=\displaystyle{\int\cos3x\sin2x\,\textrm{d}x=\dfrac{1}{2}\int(\sin5x-\sin x)\,\textrm{d}x}$

$=\dfrac{1}{2}\int\sin5x\,\textrm{d}x-\dfrac{1}{2}\int\sin x\,\textrm{d}x=-\dfrac{1}{10}\cos5x+\dfrac{1}{2}\cos x+C$。

\paragraph{三角拆分} \leavevmode \medskip

主要用于$\sec^2-1=\tan^2x$,当出现$\tan^2$、$\tan^3$等与$\sec x$在一起作为乘积时可以考虑拆分。

\textbf{例题:}$\displaystyle{\int\tan^3x\sec x\,\textrm{d}x}$。

$=\displaystyle{\int(\sec^2x-1)}\tan x\sec x\,\textrm{d}x=\displaystyle{\int(\sec^2x-1)}\,\textrm{d}(\sec x)=\dfrac{1}{3}\sec^3x-\sec x+C$。

\subsubsection{第二类换元}

当使用第二类换元时需要考虑$x$定义域与换元后式子的正负号的问题,因为根号一定为正,而在定义域上的不同部分,换元式子正负号可能不同。

\paragraph{\texorpdfstring{$\sqrt{a^2-x^2}$:$x=a\sin t(a\cos t)$}\ } \leavevmode \medskip

若令$x=a\sin t$,则根据$\sin t\in(-1,1)$得到主区间:$t\in\left(-\dfrac{\pi}{2},\dfrac{\pi}{2}\right)$,从而代入式子$\sqrt{a^2-x^2}=a\cos t$,根据$t\in\left(-\dfrac{\pi}{2},\dfrac{\pi}{2}\right)$,所以$\cos t>0$,所以正负号无误。

若令$x=a\cos t$,则根据$\cos t\in(-1,1)$,得到主区间:$t\in(0,\pi)$,从而代入式子$\sqrt{a^2-x^2}=a\sin t$,根据$t\in(0,\pi)$,所以$\sin t>0$,所以正负号无误。

所以这种情况不用考虑正负号。

\textbf{例题:}求$\displaystyle{\int\dfrac{\textrm{d}x}{1+\sqrt{1-x^2}}}$。

令$x=\sin t$($t\in\left(-\dfrac{\pi}{2},\dfrac{\pi}{2}\right)$),所以$\sqrt{1-x^2}=\cos t$,$\textrm{d}x=\cos t\,\textrm{d}t$,$t=\arcsin x$。

$=\displaystyle{\int\dfrac{\cos t}{1+\cos t}\textrm{d}t=\int\dfrac{2\cos^2\dfrac{t}{2}-1}{2\cos^2\dfrac{t}{2}}\textrm{d}t=\int\textrm{d}t-\int\sec^2\dfrac{t}{2}\,\textrm{d}t=t-\tan}\dfrac{t}{2}+C$

$=t-\dfrac{\sin\dfrac{t}{2}}{\cos\dfrac{t}{2}}+C=t-\dfrac{\sin\dfrac{t}{2}\cos\dfrac{t}{2}}{\cos^2\dfrac{t}{2}}+C=t-\dfrac{\sin t}{1+\cos t}+C$

$=\arcsin x-\dfrac{x}{1+\sqrt{1-x^2}}+C$。

\paragraph{\texorpdfstring{$\sqrt{a^2+x^2}$:$x=a\tan t$}\ } \leavevmode \medskip

根据$\tan t\in R$,从而得到主空间:$t\in\left(-\dfrac{\pi}{2},\dfrac{\pi}{2}\right)$,代入$\sqrt{a^2+x^2}=a\sec t$,根据$t\in\left(-\dfrac{\pi}{2},\dfrac{\pi}{2}\right)$,所以$\cos t>0$,$\sec t>0$,所以正负号无误。

所以这种情况不用考虑正负号。

\paragraph{\texorpdfstring{$\sqrt{x^2-a^2}$:$x=a\sec t$}\ } \leavevmode \medskip

根据$\sec t\in(-1,1)$,所以从而得到主空间:$t\in\left(-\dfrac{\pi}{2},\dfrac{\pi}{2}\right)$,代入$\sqrt{x^2-a^2}=a\tan t$,根据$t\in\left(-\dfrac{\pi}{2},\dfrac{\pi}{2}\right)$,所以$\tan t\in R$,所以此时不能保证转换后的$a\tan t>0$,此时必须对$x$分正负情况讨论。\medskip

\textbf{例题:}求$\displaystyle{\int\dfrac{\sqrt{x^2-9}}{x}\textrm{d}x}$。

令$x=3\sec t$。$\therefore\sqrt{x^2-9}=3\tan t$,$\textrm{d}x=3\sec t\tan t\,\textrm{d}t$。

当$x>3$时,$\sec t>1$,即$t\in\left(0,\dfrac{\pi}{2}\right)$。

$=\displaystyle{\int3\tan^2t\,\textrm{d}t=3\int(\sec^2t-1)\textrm{d}t}$

$=3\tan t-3t+C=\sqrt{x^2-9}-3\arccos\dfrac{3}{x}+C$。

当$x<-3$时,$\sec t<-1$,即$t\in\left(-\dfrac{\pi}{2},0\right)$。

$=\displaystyle{-\int3\tan^2t\,\textrm{d}t=-3\int(\sec^2t-1)\textrm{d}t}$

$=-3\tan t+3t+C=\sqrt{x^2-9}+3\arccos\dfrac{3}{x}+C$($\tan t<0$)

$=\sqrt{x^2-9}-3\arccos\dfrac{3}{-x}+3\pi+C$($3\arccos\dfrac{3}{x}=3\pi-3\arccos-\dfrac{3}{x}$)

$=\sqrt{x^2-9}-3\arccos\dfrac{3}{-x}+C$。

\paragraph{辅助换元} \leavevmode \medskip

在使用换元法的时候有可能单个式子不能求出积分,而使用其他辅助式子加减在一起积分可以得到结果,从而能得到原式和辅助式子的积分结果。对于这类题目需要观察什么样的式子能让积分简单。

\textbf{例题:}求$\displaystyle{\int\dfrac{\textrm{d}x}{x+\sqrt{1-x^2}}}$。

令$x=\sin t$,所以$\sqrt{1-x^2}=\cos t$,$\textrm{d}x=\cos t\,\textrm{d}t$。

$\therefore=\displaystyle{\int\dfrac{\cos t}{\sin t+\cos t}\textrm{d}t}$。

\subsection{分部积分}

\subsubsection{基本分部}

\subsubsection{多次分部}

\subsection{有理积分}

\subsubsection{高阶多项式分配}

当不定积分式子形如$\displaystyle{\int\dfrac{f(x)}{g(x)}\,\textrm{d}x}$,且$f(x)$、$g(x)$都为与$x$相关的多项式,$f(x)$阶数高于或等于$g(x)$,则$f(x)$可以按照$g(x)$的形式分配,约去式子,得到最简单的表达。

\textbf{例题:}$\displaystyle{\int\dfrac{x^3}{x^2+9}\,\textrm{d}x}$。 \medskip

$=\displaystyle{\int\dfrac{x^3+9x-9x}{x^2+9}\,\textrm{d}x=\int\dfrac{x^3+9x}{x^2+9}\,\textrm{d}x-\int\dfrac{9x}{x^2+9}\,\textrm{d}x}$ \medskip

$\displaystyle{=\int x\,\textrm{d}x-\dfrac{9}{2}\int\dfrac{\textrm{d}(x^2+9)}{x^2+9}}=\dfrac{x^2}{2}-\dfrac{9}{2}\ln(9+x^2)+C$。

\subsubsection{低阶多项式分解}

当不定积分式子形如$\displaystyle{\int\dfrac{f(x)}{g(x)}\,\textrm{d}x}$,且$f(x)$、$g(x)$都为与$x$相关的多项式,$f(x)$阶数低于$g(x)$,则可以分解式子:$\displaystyle{\int\dfrac{f(x)}{g(x)}\,\textrm{d}x=\int\dfrac{f_1(x)}{g_1(x)}\,\textrm{d}x+\int\dfrac{f_2(x)}{g_2(x)}\,\textrm{d}x}$。



\section{定积分}

\subsection{变限积分}

\subsection{牛莱公式}

\subsection{换元积分}

\subsection{分部积分}

\subsection{反常积分}

\section{积分应用}

\subsection{面积}

\subsection{体积}

\subsection{弧长}

\end{document}
