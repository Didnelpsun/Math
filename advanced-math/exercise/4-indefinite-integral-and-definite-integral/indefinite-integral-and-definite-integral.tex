\documentclass[UTF8, 12pt]{ctexart}
% UTF8编码,ctexart现实中文
\usepackage{color}
% 使用颜色
\definecolor{orange}{RGB}{255,127,0} 
\usepackage{geometry}
\setcounter{tocdepth}{4}
\setcounter{secnumdepth}{4}
% 设置四级目录与标题
\geometry{papersize={21cm,29.7cm}}
% 默认大小为A4
\geometry{left=3.18cm,right=3.18cm,top=2.54cm,bottom=2.54cm}
% 默认页边距为1英尺与1.25英尺
\usepackage{indentfirst}
\setlength{\parindent}{2.45em}
% 首行缩进2个中文字符
\usepackage{setspace}
\renewcommand{\baselinestretch}{1.5}
% 1.5倍行距
\usepackage{amssymb}
% 因为所以
\usepackage{amsmath}
% 数学公式
\usepackage{pifont}
% 圆圈序号
\usepackage[colorlinks,linkcolor=black,urlcolor=blue]{hyperref}
% 超链接
\author{Didnelpsun}
\title{不定积分与定积分}
\date{}
\begin{document}
\maketitle
\pagestyle{empty}
\thispagestyle{empty}
\tableofcontents
\thispagestyle{empty}
\newpage
\pagestyle{plain}
\setcounter{page}{1}
\section{不定积分}
\subsection{基本积分}

\textbf{例题:}汽车以20m/s的速度行驶,刹车后匀减速行驶了50m停止,求刹车加速度。

已知题目含有两个变量:距离和时间,设距离为$s$,时间为$t$。

因为汽车首先按20m/s匀速运动,所以$\dfrac{\textrm{d}s}{\textrm{d}t}\bigg\vert_{t=0}=20$,最开始距离为0,所以$s\vert_{t=0}=0$。

又因为是匀减速的,所以速度形如:$v=\dfrac{s}{t}=kt+b$,从而令二阶导数下$\dfrac{\textrm{d}^2s}{\textrm{d}t^2}=k$。

所以$\displaystyle{\dfrac{\textrm{d}s}{\textrm{d}t}=\int\dfrac{\textrm{d}^2s}{\textrm{d}t^2}\,\textrm{d}t=\int k\,\textrm{d}t}=kt+C_1$。

代入$\dfrac{\textrm{d}s}{\textrm{d}t}\bigg\vert_{t=0}=20$,所以$C_1=20$,即$\dfrac{\textrm{d}s}{\textrm{d}t}=kt+20$。

所以$\textrm{d}s=(kt+20)\,\textrm{d}t$,从而$s=\displaystyle{\int(kt+20)\,\textrm{d}t}=\dfrac{1}{2}kt^2+20t+C_2$。

又$s\vert_{t=0}=0$,所以代入得$C_2=0$,所以$s=\dfrac{1}{2}kt^2+20t$。

当$s=50$时停住,所以此时$\dfrac{\textrm{d}s}{\textrm{d}t}=0$,得到$t=-\dfrac{20}{k}$。

代入$s$:$50=\dfrac{1}{2}k\left(-\dfrac{20}{k}\right)^2+20\left(-\dfrac{20}{k}\right)$,解得$k=-4$,即加速度为-4m/$s^2$。

\subsection{换元积分}

\subsubsection{第一类换元}

\paragraph{聚集因式} \leavevmode \medskip

将复杂的式子转换为简单的一个因式放到$\textrm{d}$后面看作一个整体,然后利用基本积分公式计算。

\textbf{例题:}$\displaystyle{\int\dfrac{\textrm{d}x}{x\ln x\ln\ln x}}$。 \medskip

$=\displaystyle{\int\dfrac{\textrm{d}(\ln x)}{\ln x\ln\ln x}}=\displaystyle{\int\dfrac{\textrm{d}(\ln\ln x)}{\ln\ln x}}=\ln\vert\ln\ln x\vert+C$。\medskip

\textbf{例题:}$\displaystyle{\int\dfrac{10^{2\arccos x}}{\sqrt{1-x^2}}\,\textrm{d}x}$。

$=-\displaystyle{\int10^{2\arccos x}\,\textrm{d}(\arccos x)}=-\dfrac{1}{2}\displaystyle{\int10^{2\arccos x}\,\textrm{d}(2\arccos x)}=-\dfrac{10^{2\arccos x}}{2\ln10}+C$。

\paragraph{积化和差} \leavevmode \medskip

对于两个三角函数的乘积可以使用积化和差简单计算。

\textbf{例题:}$\displaystyle{\int\sin2x\cos3x\,\textrm{d}x}$。

$=\displaystyle{\int\cos3x\sin2x\,\textrm{d}x=\dfrac{1}{2}\int(\sin5x-\sin x)\,\textrm{d}x}$

$=\dfrac{1}{2}\int\sin5x\,\textrm{d}x-\dfrac{1}{2}\int\sin x\,\textrm{d}x=-\dfrac{1}{10}\cos5x+\dfrac{1}{2}\cos x+C$。

\paragraph{三角拆分} \leavevmode \medskip

主要用于$\sec^2-1=\tan^2x$,当出现$\tan^2$、$\tan^3$等与$\sec x$在一起作为乘积时可以考虑拆分。

\textbf{例题:}$\displaystyle{\int\tan^3x\sec x\,\textrm{d}x}$。

$=\displaystyle{\int(\sec^2x-1)}\tan x\sec x\,\textrm{d}x=\displaystyle{\int(\sec^2x-1)}\,\textrm{d}(\sec x)=\dfrac{1}{3}\sec^3x-\sec x+C$。

\subsubsection{第二类换元}

使用换元法做了换元之后是要带回式子中的,也就是说要保证反函数的存在才能代入有意义。为了保证反函数的存在,因此要保证原函数的单调性,所以要有一个规定的范围来使原函数保证单调。

\paragraph{\texorpdfstring{$\sqrt{a^2-x^2}$:$x=a\sin t(a\cos t)$}\ } \leavevmode \medskip

若令$x=a\sin t$,则根据$\sin t\in(-1,1)$得到主区间:$t\in\left(-\dfrac{\pi}{2},\dfrac{\pi}{2}\right)$。

若令$x=a\cos t$,则根据$\cos t\in(-1,1)$,得到主区间:$t\in(0,\pi)$。

\textbf{例题:}求$\displaystyle{\int\dfrac{\textrm{d}x}{1+\sqrt{1-x^2}}}$。\medskip

令$x=\sin t$($t\in\left(-\dfrac{\pi}{2},\dfrac{\pi}{2}\right)$),所以$\sqrt{1-x^2}=\cos t$,$\textrm{d}x=\cos t\,\textrm{d}t$,$t=\arcsin x$。

因为式子$\dfrac{1}{1+\sqrt{1-x^2}}>0$,单调递减,所以不用讨论正负号。

$=\displaystyle{\int\dfrac{\cos t}{1+\cos t}\textrm{d}t=\int\dfrac{2\cos^2\dfrac{t}{2}-1}{2\cos^2\dfrac{t}{2}}\textrm{d}t=\int\textrm{d}t-\int\sec^2\dfrac{t}{2}\,\textrm{d}t=t-\tan}\dfrac{t}{2}+C$

$=t-\dfrac{\sin\dfrac{t}{2}}{\cos\dfrac{t}{2}}+C=t-\dfrac{\sin\dfrac{t}{2}\cos\dfrac{t}{2}}{\cos^2\dfrac{t}{2}}+C=t-\dfrac{\sin t}{1+\cos t}+C$

$=\arcsin x-\dfrac{x}{1+\sqrt{1-x^2}}+C$。

\paragraph{\texorpdfstring{$\sqrt{a^2+x^2}$:$x=a\tan t$}\ } \leavevmode \medskip

根据$\tan t\in R$,从而得到主空间:$t\in\left(-\dfrac{\pi}{2},\dfrac{\pi}{2}\right)$。

\textbf{例题:}求$\displaystyle{\int\dfrac{x^3+1}{(x^2+1)^2}\textrm{d}x}$。\medskip

虽然本题目看着可以从分母解开平方,然后低阶分配,但是这分母是平方的式子很难分配,所以需要使用换元法。

令$x=\tan t$,$t\in\left(-\dfrac{\pi}{2},\dfrac{\pi}{2}\right)$,$x^2+1=\sec^2t$,$\textrm{d}x=\sec^2t\,\textrm{d}t$。

因为$(x^2+1)^2>0$,虽然$x^3+1$可能为负可能为正,但是都是单调递增的,所以不用考虑正负号。

$=\displaystyle{\int\dfrac{\tan^3t+1}{\sec^2t}\textrm{d}t=\int\dfrac{\sin^3t+\cos^3t}{\cos t}\textrm{d}t=\int\dfrac{\sin t(1-\cos^2t)+\cos^3t}{\cos t}\textrm{d}t}$

$=\displaystyle{\int\dfrac{\cos^2t-1}{\cos t}\textrm{d}(\cos t)+\int\dfrac{1+\cos2t}{2}\textrm{d}t}$

$=\displaystyle{\int\cos t\,\textrm{d}(\cos t)-\int\dfrac{1}{\cos t}\,\textrm{d}(\cos t)+\dfrac{1}{2}\int\textrm{d}t+\dfrac{1}{4}\int\cos2t\,\textrm{d}(2t)}$

$=\displaystyle{\int\cos^t-\ln\cos t+\dfrac{t}{2}+\dfrac{1}{4}\sin2t+C}$($\cos t$在$t\in\left(-\dfrac{\pi}{2},\dfrac{\pi}{2}\right)$中为正)

$\because\tan t=x$,$\therefore\sin t=\dfrac{x}{\sqrt{1+x^2}}$,$\cos t=\dfrac{1}{\sqrt{1+x^2}}$。

$=\dfrac{1+x}{2(1+x^2)}+\dfrac{1}{2}\ln(1+x^2)+\dfrac{1}{2}\textrm{arctan}\,x+C$。

\paragraph{\texorpdfstring{$\sqrt{x^2-a^2}$:$x=a\sec t$}\ } \leavevmode \medskip

根据$\sec t\in(-1,1)$,所以从而得到主空间:$t\in\left(-\dfrac{\pi}{2},\dfrac{\pi}{2}\right)$。\medskip

\textbf{例题:}求$\displaystyle{\int\dfrac{\sqrt{x^2-9}}{x}\textrm{d}x}$。

令$x=3\sec t$。$\therefore\sqrt{x^2-9}=3\tan t$,$\textrm{d}x=3\sec t\tan t\,\textrm{d}t$。\medskip

因为式子$\dfrac{\sqrt{x^2-9}}{x}$的分子必然为为正,而对于分子在0两边的单调性不同,所以需要对$x$进行正负区分,又$x\in(-\infty,-3]\cup[3,+\infty)$,所以:

当$x>3$时,$\sec t>1$,即$t\in\left(0,\dfrac{\pi}{2}\right)$。

$=\displaystyle{\int3\tan^2t\,\textrm{d}t=3\int(\sec^2t-1)\textrm{d}t}$

$=3\tan t-3t+C=\sqrt{x^2-9}-3\arccos\dfrac{3}{x}+C$。

当$x<-3$时,$\sec t<-1$,即$t\in\left(-\dfrac{\pi}{2},0\right)$。

$=\displaystyle{-\int3\tan^2t\,\textrm{d}t=-3\int(\sec^2t-1)\textrm{d}t}$

$=-3\tan t+3t+C=\sqrt{x^2-9}+3\arccos\dfrac{3}{x}+C$($\tan t<0$)

$=\sqrt{x^2-9}-3\arccos\dfrac{3}{-x}+3\pi+C$($3\arccos\dfrac{3}{x}=3\pi-3\arccos-\dfrac{3}{x}$)

$=\sqrt{x^2-9}-3\arccos\dfrac{3}{-x}+C$。

综上结果为$\sqrt{x^2-9}-3\arccos\dfrac{3}{\vert x\vert}+C$。

\paragraph{辅助换元} \leavevmode \medskip

在使用换元法的时候有可能单个式子不能求出积分,而使用其他辅助式子加减在一起积分可以得到结果,从而能得到原式和辅助式子的积分结果。对于这类题目需要观察什么样的式子能让积分简单。

\textbf{例题:}求$\displaystyle{\int\dfrac{\textrm{d}x}{x+\sqrt{1-x^2}}}$。\medskip

令$x=\sin t$,所以$\sqrt{1-x^2}=\cos t$,$\textrm{d}x=\cos t\,\textrm{d}t$。

$\because x+\sqrt{1-x^2}$可能为正可能为负,正负时单调性不同,所以令$ x+\sqrt{1-x^2}=0$,即$\sin t=-\dfrac{\sqrt{2}}{2}$,从而$t\in(-\dfrac{\pi}{2},-\dfrac{\pi}{4})\cup(-\dfrac{\pi}{4},\dfrac{\pi}{2})$。

$\therefore=\displaystyle{\int\dfrac{\cos t}{\sin t+\cos t}\textrm{d}t}$($t\in(-\dfrac{\pi}{4},\dfrac{\pi}{2})$)。 \medskip

这时你会发现使用积化和差、万能公式、倍角公式都无法解出这个积分,所以这时候就需要另外一个辅助积分式子加上或减去这个式子,从而让和以及差更容易解出积分。这里根据式子特点让辅助式子分子为$\sin t$:

令$I_1=\displaystyle{\int\dfrac{\cos t}{\sin t+\cos t}\textrm{d}t}$,$I_2=\displaystyle{\int\dfrac{\sin t}{\sin t+\cos t}\textrm{d}t}$。

$I_1+I_2=\displaystyle{\int\dfrac{\sin t+\cos t}{\sin t+\cos t}\textrm{d}t=\int\textrm{d}t=t}$。

$I_1-I_2=\displaystyle{\int\dfrac{\cos t-\sin t}{\sin t+\cos t}\textrm{d}t=\int\dfrac{\textrm{d}(\sin t+\cos t)}{\sin t+\cos t}}=\ln\vert\sin t+\cos t\vert +C$。

所以$I_1=\dfrac{1}{2}(\arcsin x+\ln\vert x+\sqrt{1-x^2}\vert)+C$。

同理$t\in(-\dfrac{\pi}{2},-\dfrac{\pi}{4})$也得到同样结果。

\subsection{分部积分}

因为分部积分法使用$\int u\,\textrm{d}v=uv-\int v\,\textrm{d}u$,所以基本上用于两项乘积形式的积分式子。

\subsubsection{基本分部}

\paragraph{非幂函数优先} \leavevmode \medskip

当幂函数与一些微分后能降低幂函数幂次的函数在一起时,先对非幂函数优先分部积分,结果与幂函数相乘可以消去幂次,以达到降低幂次的作用。

如$\int x^n\ln x\,\textrm{d}x$,$\int x^n\arctan x\,\textrm{d}x$,$\int x^n\arcsin x\,\textrm{d}x$。

\textbf{例题:}求$\int x^2\arctan x\,\textrm{d}x$。

$=\dfrac{1}{3}\int\arctan x\,\textrm{d}(x^3)=\dfrac{1}{3}x^3\arctan x-\dfrac{1}{3}\int x^3\,\textrm{d}(\arctan x)$

$=\dfrac{1}{3}x^3\arctan x-\dfrac{1}{3}\displaystyle{\int\dfrac{x^3}{1+x^2}\textrm{d}x}=\dfrac{1}{3}x^3\arctan x-\dfrac{1}{3}\displaystyle{\int\dfrac{x+x^3-x}{1+x^2}\textrm{d}x}$

$=\dfrac{1}{3}x^3\arctan x-\dfrac{1}{3}\int x\,\textrm{d}x+\displaystyle{\dfrac{1}{6}\int\dfrac{\textrm{d}(1+x^2)}{1+x^2}}$

$=\dfrac{1}{3}x^3\arctan x-\dfrac{1}{6}x^2+\dfrac{1}{6}\ln(1+x^2)+C$。

\paragraph{幂函数优先} \leavevmode \medskip

当幂函数与三角函数在一起微分时,因为三角函数无论如何积分都不会被消去,所以应该优先消去幂函数部分,从而降低幂次。

如$\int x^a\sin x\,\textrm{d}x$,$\int x^a\cos x\,\textrm{d}x$。

\textbf{例题:}求$\int x\tan^2x\,\textrm{d}x$。

$=\int x(\sec^2-1)\,\textrm{d}x=\int x\,\textrm{d}(\tan x)-\dfrac{x^2}{2}=x\tan x+\ln\vert\cos x\vert-\dfrac{x^2}{2}+C$。

\subsubsection{多次分部}

对于一部分通过微分形式不会发生变化的函数,所以需要多次积分,然后利用等式求出目标值。

如:$\int e^x\sin x\,\textrm{d}x$,$\int e^x\cos x\,\textrm{d}x$。

\textbf{例题:}求$\int e^x\sin^2x\,\textrm{d}x$。

$=\sin^2x\cdot e^x-\int e^x\,\textrm{d}(\sin^2x)=\sin^2x\cdot e^x-\int e^x\cdot\sin 2x\,\textrm{d}x$

$=\sin^2x\cdot e^x-\int\sin2x\,\textrm{d}(e^x)=\sin^2x\cdot e^x-\sin2x\cdot e^x+\int e^x\,\textrm{d}(\sin2x)$

$=\sin^2x\cdot e^x-\sin2x\cdot e^x+2\int e^x\cdot\cos2x\,\textrm{d}x$ (\ding{172})

$=\sin^2x\cdot e^x-\sin2x\cdot e^x+2\int\cos2x\,\textrm{d}(e^x)$

$=\sin^2x\cdot e^x-\sin2x\cdot e^x+2e^x\cos2x-2\int e^x\,\textrm{d}(\cos2x)$

$=\sin^2x\cdot e^x-\sin2x\cdot e^x+2e^x\cos2x+4\int e^x\cdot\sin2x\,\textrm{d}x$

$=\sin^2x\cdot e^x-\sin2x\cdot e^x+2e^x\cos2x+4\int\sin2x\,\textrm{d}(e^x)$

$=\sin^2x\cdot e^x-\sin2x\cdot e^x+2e^x\cos2x+4\sin2x\cdot e^x-4\int e^x\,\textrm{d}(\sin2x)$

$=\sin^2x\cdot e^x-\sin2x\cdot e^x+2e^x\cos2x+4\sin2x\cdot e^x-8\int e^x\cdot\cos2x\,\textrm{d}x$ (\ding{173})

然后\ding{172}=\ding{173}:$\int e^x\cdot\cos2x\,\textrm{d}x=\dfrac{e^x(\cos2x+2\sin2x)}{5}+C$

代入\ding{172}:$=\dfrac{e^x(5\sin^2x-5\sin2x+2\cos2x+4\sin2x)}{5}+C$

$=\dfrac{e^x(5\sin^2x-\sin2x+2\cos2x)}{5}+C=e^x\left(\dfrac{1}{2}-\dfrac{1}{5}\sin2x-\dfrac{1}{10}\cos2x\right)+C$

\subsubsection{分部与换元}

分部积分法和换元积分法经常一起使用。

\textbf{例题:}求$\int e^{\sqrt[3]{x}}\,\textrm{d}x$。

令$\sqrt[3]{x}=u$,从而$x=u^3$,$\textrm{d}x=3u^2\,\textrm{d}u$。

$=3\int e^uu^2\,\textrm{d}u=3\int u^2\,\textrm{d}(e^u)=3u^2e^u-3\int e^u\,\textrm{d}(u^2)=3u^2e^u-6\int e^uu\,\textrm{d}u$

$=3u^2e^u-6\int u\,\textrm{d}(e^u)=3u^2e^u-6ue^u+6\int e^u\,\textrm{d}u=3u^2e^u-6ue^u+6e^u+C$

$=3e^u(u^2-2u+2)+C=3e^{\sqrt[3]{x}}(x^{\frac{2}{3}}-2x^{\frac{1}{3}}+2)+C$。

\textbf{例题:}求$\int e^{\sqrt{3x+9}}\,\textrm{d}x$。

令$\sqrt{3x+9}=u$,从而$x=\dfrac{1}{3}(u^2-9)$,$\textrm{d}x=\dfrac{2}{3}u\,\textrm{d}u$:

$=\displaystyle{\dfrac{2}{3}\int ue^u\,\textrm{d}u=\dfrac{2}{3}\int u\,\textrm{d}(e^u)=\dfrac{2}{3}ue^u-\int\dfrac{2}{3}e^u\,\textrm{d}u=\dfrac{2}{3}ue^u-\dfrac{2}{3}e^u+C}$

$=\dfrac{2}{3}e^{\sqrt{3x+9}}(\sqrt{3x+9}-1)+C$。

\subsection{有理积分}

\subsubsection{高阶多项式分配}

当不定积分式子形如$\displaystyle{\int\dfrac{f(x)}{g(x)}\,\textrm{d}x}$,且$f(x)$、$g(x)$都为与$x$相关的多项式,$f(x)$阶数高于或等于$g(x)$,则$f(x)$可以按照$g(x)$的形式分配,约去式子,得到最简单的表达。

\textbf{例题:}$\displaystyle{\int\dfrac{x^3}{x^2+9}\,\textrm{d}x}$。 \medskip

$=\displaystyle{\int\dfrac{x^3+9x-9x}{x^2+9}\,\textrm{d}x=\int\dfrac{x^3+9x}{x^2+9}\,\textrm{d}x-\int\dfrac{9x}{x^2+9}\,\textrm{d}x}$ \medskip

$\displaystyle{=\int x\,\textrm{d}x-\dfrac{9}{2}\int\dfrac{\textrm{d}(x^2+9)}{x^2+9}}=\dfrac{x^2}{2}-\dfrac{9}{2}\ln(9+x^2)+C$。

\subsubsection{低阶多项式分配}

当不定积分式子形如$\displaystyle{\int\dfrac{f(x)}{g(x)}\,\textrm{d}x}$,且$f(x)$、$g(x)$都为与$x$相关的多项式,$f(x)$阶数低于$g(x)$,且$g(x)$不能因式分解为$g(x)=g_1(x)g_2(x)\cdots$时,则可以分解式子:$\displaystyle{\int\dfrac{f(x)}{g(x)}\,\textrm{d}x=a_1\int\dfrac{\textrm{d}(f_1(x))}{g_1(x)}+a_2\int\dfrac{\textrm{d}(f_2(x))}{g_2(x)}}+\cdots$,将积分式子组合成积分结果为分式的函数,如$\ln x$、$\arcsin x$、$\arctan x$等。

\textbf{例题:}求$\displaystyle{\int\dfrac{x-1}{x^2+2x+3}\textrm{d}x}$。

因为$x^2+2x+3$不能因式分解,所以考虑将分子按照分母形式进行分配。优先对高阶的$x$进行分配。

首先因为分子最高阶为$x$只比分母最高阶$x^2$低一阶,所以考虑将$x-1$分配到微分号内。

$\because\textrm{d}(x^2+2x+3)=2x+2$,而现在是$x-1$,所以:

$=\displaystyle{\dfrac{1}{2}\int\dfrac{2x+2}{x^2+2x+3}\textrm{d}x-2\int\dfrac{1}{x^2+2x+3}\textrm{d}x}=\displaystyle{\dfrac{1}{2}\int\dfrac{\textrm{d}(x^2+2x+3)}{x^2+2x+3}}$

$-\displaystyle{\int\dfrac{1}{\left(\dfrac{x+1}{\sqrt{2}}\right)^2+1}\textrm{d}x}=\displaystyle{\dfrac{1}{2}\ln(x^2x+3)-\sqrt{2}\int\dfrac{\textrm{d}\left(\dfrac{x+1}{\sqrt{2}}\right)}{\left(\dfrac{x+1}{\sqrt{2}}\right)^2+1}}$

$=\dfrac{1}{2}\ln(x^2+2x+3)-\sqrt{2}\arctan\dfrac{x+1}{\sqrt{2}}+C$。

\section{定积分}

\subsection{变限积分}

\subsection{牛莱公式}

\subsection{换元积分}

\subsection{分部积分}

\subsection{反常积分}

\section{积分应用}

\subsection{面积}

\subsection{体积}

\subsection{弧长}

\end{document}
