\documentclass[UTF8, 12pt]{ctexart}
% UTF8编码,ctexart现实中文
\usepackage{color}
% 使用颜色
\usepackage{geometry}
\setcounter{tocdepth}{4}
\setcounter{secnumdepth}{4}
% 设置四级目录与标题
\geometry{papersize={21cm,29.7cm}}
% 默认大小为A4
\geometry{left=3.18cm,right=3.18cm,top=2.54cm,bottom=2.54cm}
% 默认页边距为1英尺与1.25英尺
\usepackage{indentfirst}
\setlength{\parindent}{2.45em}
% 首行缩进2个中文字符
\usepackage{setspace}
\renewcommand{\baselinestretch}{1.5}
% 1.5倍行距
\usepackage{amssymb}
% 因为所以
\usepackage{amsmath}
% 数学公式
\usepackage[colorlinks,linkcolor=black,urlcolor=blue]{hyperref}
% 超链接
\author{Didnelpsun}
\title{无穷级数}
\date{}
\begin{document}
\maketitle
\pagestyle{empty}
\thispagestyle{empty}
\tableofcontents
\thispagestyle{empty}
\newpage
\pagestyle{plain}
\setcounter{page}{1}
\section{求和函数}

可以利用展开式求和函数,但是很多展开式的通项都不是公式中的,就需要对通项进行变形。

\subsection{先导后积}

$n$在分母上,先导后积。使用变限积分:$\int_{x_0}^xS'(t)\,\textrm{d}t=S(x)-S(x_0)$,即$S(x)=S(x_0)+\int_{x_0}^xS'(t)\,\textrm{d}t$。一般选择$x_0$为展开点。

\textbf{例题:}求级数$\sum\limits_{n=1}^\infty\dfrac{x^n}{n}$的和函数。

解:已知$\sum\limits_{n=0}^\infty x^n=\dfrac{1}{1-x}$,而这里求和是$\dfrac{x^n}{n}$,所以需要对其进行转换。

对$\dfrac{x^n}{n}$求导就得到了$x^{n-1}$消去了分母的$n$,所以使用先导后积的方法。

记$S(x)=\sum\limits_{n=1}^\infty\dfrac{x^n}{n}$,则$x^n=(x-0)^n$,取$x_0=0$。

$\therefore S(x)=S(0)+\displaystyle{\int_0^x\left(\sum\limits_{n=1}^\infty\dfrac{t^n}{n}\right)_t'\,\textrm{d}t}=0+\int_0^x(\sum\limits_{n=1}^\infty t^{n-1})\,\textrm{d}t=\displaystyle{\int_0^x\dfrac{1}{1-t}\textrm{d}t}=-\ln(1-x)$。收敛域为$(-1,1)$。


\subsection{先积后导}

$n$在分子上,先积后导。$(\int S(x)\,\textrm{d}x)'=S(x)$。

\textbf{例题:}求级数$\sum\limits_{n=1}^\infty nx^n$的和函数。

解:记$S(x)=\sum\limits_{n=1}^\infty nx^n=x\sum\limits_{n=1}^\infty x^{n-1}=x(\int\sum\limits_{n=1}^\infty nx^{n-1}\,\textrm{d}x)'=x(\sum\limits_{n=1}^\infty x^n)'=x\left(\dfrac{x}{1-x}\right)'=\dfrac{x}{(1-x)^2}$。收敛域为$[-1,1]$。

\end{document}
