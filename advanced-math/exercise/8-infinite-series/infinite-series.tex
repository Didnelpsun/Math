\documentclass[UTF8, 12pt]{ctexart}
% UTF8编码,ctexart现实中文
\usepackage{color}
% 使用颜色
\usepackage{geometry}
\setcounter{tocdepth}{4}
\setcounter{secnumdepth}{4}
% 设置四级目录与标题
\geometry{papersize={21cm,29.7cm}}
% 默认大小为A4
\geometry{left=3.18cm,right=3.18cm,top=2.54cm,bottom=2.54cm}
% 默认页边距为1英尺与1.25英尺
\usepackage{indentfirst}
\setlength{\parindent}{2.45em}
% 首行缩进2个中文字符
\usepackage{setspace}
\renewcommand{\baselinestretch}{1.5}
% 1.5倍行距
\usepackage{amssymb}
% 因为所以
\usepackage{amsmath}
% 数学公式
\usepackage[colorlinks,linkcolor=black,urlcolor=blue]{hyperref}
% 超链接
\author{Didnelpsun}
\title{无穷级数}
\date{}
\begin{document}
\maketitle
\pagestyle{empty}
\thispagestyle{empty}
\tableofcontents
\thispagestyle{empty}
\newpage
\pagestyle{plain}
\setcounter{page}{1}
\section{常数项级数}

\subsection{正项级数}

如果题目中没有说明,要首先证明多项式为正数,否则不能使用正项级数的方法。

\subsubsection{放缩法}

即根据收敛准则来进行判断。如果要判断原级数收敛,则辅助级数应该是对其放大,判断原级数发散,则辅助级数应该是对其缩小。

\subsubsection{比较判别法}

都需要找到一个好的级数进行比较。常用的只有两个:

$p$级数:$\sum\limits_{n=1}^\infty\dfrac{1}{n^p}\left\{\begin{array}{l}
    p>1, \text{收敛} \\
    p\leqslant1, \text{发散}
\end{array}\right.$。

等比级数(几何级数):$\sum\limits_{n=1}^\infty\dfrac{1}aq^{n-1}\left\{\begin{array}{l}
    \vert q\vert<1, \text{收敛} \\
    \vert q\vert\geqslant 1, \text{发散}
\end{array}\right.$。

当不知道用哪个时可以使用洛必达先计算一下极限值。

\subsubsection{比值判别法}

适用于含有$a^n$,$n!$,$n^n$的通项。主要是$n!$。

\textbf{例题:}判断$\sum\limits_{n=1}^\infty\dfrac{n!}{n^n}$。

解:利用比值判别法,令$a_n=\dfrac{n!}{n^n}$,$\lim\limits_{n\to\infty}\dfrac{a_{n+1}}{a_n}=\lim\limits_{n\to\infty}\dfrac{(n+1)!}{(n+1)^{n+1}}\dfrac{n^n}{n!}=\lim\limits_{n\to\infty}\dfrac{n^n}{(n+1)^n}$,注意这里幂也为变量,不是等于1而是上下同时除以$n^n$,$=\lim\limits_{n\to\infty}\dfrac{1}{(1+\frac{1}{n})^n}$,根据两个重要极限得到$=\dfrac{1}{e}<1$,所以收敛。

\subsubsection{根值判别法}

适用于含有$a^n$,$n^n$的通项。

$\lim\limits_{n\to\infty}\sqrt[n]{n}=1$。

\subsubsection{积分判别法}

\textbf{例题:}判断级数$\sum\limits_{n=2}^\infty\dfrac{1}{n\ln n}$的敛散性。

解:因为$\dfrac{1}{n\ln n}<\dfrac{1}{n}$,调和级数发散,所以比较判别法找不到一个较好的辅助级数。同理根据级数形式比值和根值判别法都无法使用。

令$f(x)=\dfrac{1}{x\ln x}$,$a_n=f(n)$,在$[2,+\infty)$上$\dfrac{1}{n\ln n}$单调减且非负。

级数$\sum\limits_{n=2}^\infty\dfrac{1}{n\ln n}$与$\int_2^{+\infty}\dfrac{\textrm{d}x}{x\ln x}$同敛散。

$=\ln\ln x\vert_2^{+\infty}=+\infty$,所以原级数发散。

\subsection{交错级数}

\section{幂级数}

\subsection{收敛域}

\subsubsection{基本方法}

使用比值或根值法进行求解。

\textbf{例题:}求幂级数$\sum\limits_{n=1}^\infty\dfrac{e^n-(-1)^n}{n^2}x^n$的收敛半径。

解:

比值法:

$\lim\limits_{n\to\infty}\left\vert\dfrac{a_{n+1}}{a_n}\right\vert=\lim\limits_{n\to\infty}\dfrac{e^{n+1}-(-1)^{n+1}}{(n+1)^2}\dfrac{n^2}{e^n-(-1)^n}=\lim\limits_{n\to\infty}\dfrac{e+(\frac{-1}{e})^n}{1-(\frac{-1}{e})^n}$。

又$\lim\limits_{n\to\infty}x^n=\left\{\begin{array}{ll}
    0 & \vert x\vert<1 \\
    \infty & \vert x\vert\geqslant1
\end{array}\right.$,$\lim\limits_{n\to\infty}\left(\dfrac{-1}{e}\right)^n=0$,原式$=e$。$R=\dfrac{1}{e}$。

根值法:

$\lim\limits_{n\to\infty}\sqrt[n]{\vert a_n\vert}=\lim\limits_{n\to\infty}\sqrt[n]{\dfrac{e^n-(-1)^n}{n^2}}=\lim\limits_{n\to\infty}\dfrac{e\sqrt[n]{1-(-\frac{-1}{e})^n}}{\sqrt[n]{n}\sqrt[n]{n}}$,又$\lim\limits_{n\to\infty}\sqrt[n]{n}=1$。

$=\lim\limits_{n\to\infty}\dfrac{e\sqrt[n]{1-0}}{1\cdot1}=e$,所以$R=\dfrac{1}{e}$。

\subsubsection{缺项变换}

若求$\sum\limits_{n=0}^\infty a_nx^{2n+1}$或$\sum\limits_{n=0}^\infty a_nx^{2n}$,则求出其$\rho$,$R=\sqrt{\dfrac{1}{\rho}}$。

\textbf{例题:}求幂级数$\sum\limits_{n=1}^\infty\dfrac{n}{2^n+(-3)^n}x^{2n-1}$的收敛半径。

解:由于分母都是幂函数,所以使用根值法:$=\lim\limits_{n=1}^\infty\sqrt[n]{\vert a_n\vert}=\lim\limits_{n=1}^\infty\dfrac{\sqrt[n]{n}}{\sqrt[n]{3^n+(-2)^n}}\\=\lim\limits_{n=1}^\infty\dfrac{1}{\sqrt[n]{1+(-\frac{2}{3})^n}}=\dfrac{1}{3}$。

所以$R=3$。注意这里是错误的,因为之前求收敛域时都是$x^n$,而这里是$x^{2n-1}$,只有奇数次项,所以幂级数的一半都没有了。

$\sum\limits_{n=1}^\infty a_nx^{2n}=\sum\limits_{n=1}^\infty a_nx^{2n-1}=\sum\limits_{n=1}^\infty a_n(x^2)^n$,当前已知收敛半径为$3$,即$\vert x^2\vert<3$,即$\vert x\vert<\sqrt{3}$。

\subsubsection{收敛域变换}

\textbf{例题:}已知幂级数$\sum\limits_{n=0}^\infty a_n(x+2)^n$在$x=0$处收敛,在$x=-4$处发散,求$\sum\limits_{n=0}^\infty a_n(x-3)^n$的收敛域。

解:根据阿贝尔定理,已知在$x=0$处收敛,且中心点在$x=-2$,则收敛区间为$(-4,0)$,在$x=-4$处发散,则$x<-4$,$x>0$处发散。

然后确定两端端点敛散性,$x=0$处收敛则收敛域包括$x=0$,$x=-4$处发散则收敛域不包括$x=-4$,得到收敛域$(-4,0]$。

对于$\sum\limits_{n=0}^\infty a_n(x-3)^n$的中心点为$x=3$,则根据相对位置收敛域为$(1,5]$。

\subsubsection{常数项级数变换}

可以代入特殊点确定收敛点,将幂级数转换为常数项级数。

\textbf{例题:}若级数$\sum\limits_{n=0}^\infty a_n$条件收敛,求幂级数$\sum\limits_{n=0}^\infty na_n(x-1)^n$的收敛区间。

解:已知$\sum\limits_{n=0}^\infty a_n$条件收敛,则对于幂级数$\sum\limits_{n=0}^\infty na_n(x-1)^n$而言在$x=2$处条件收敛,即得到以中心点$x=1$的收敛区间$(0,2)$。

\subsection{函数展开}

\subsubsection{因式分解}

\textbf{例题:}将函数$f(x)=\dfrac{1}{x^2-3x-4}$展开为$x-1$的幂级数并指出收敛区间。

解:$\dfrac{1}{x^2-3x-4}=\dfrac{1}{5}\left(\dfrac{1}{x-4}-\dfrac{1}{x+1}\right)$。

$\dfrac{1}{x-4}=\dfrac{1}{(x-1)-3}=-\dfrac{1}{3}\dfrac{1}{1-\frac{x-1}{3}}=-\dfrac{1}{3}\sum\limits_{n=0}^\infty\left(\dfrac{x-1}{3}\right)^n$,$\left\vert\dfrac{x-1}{3}\right\vert<1$,$x\in(-2,4)$。

$\dfrac{1}{x+1}=\dfrac{1}{(x-1)+2}=\dfrac{1}{2}\dfrac{1}{1+\frac{x-1}{2}}=\dfrac{1}{2}\sum\limits_{n=0}^\infty\left(-\dfrac{x-1}{2}\right)^n$,$\left\vert-\dfrac{x-1}{2}\right\vert<1$,$x\in(-1,3)$。

所以其幂级数就是其加和,收敛区间为$(-2,4)\cap(-1,3)=(-1,3)$。

\subsubsection{先导后积}

\textbf{例题:}求函数$f(x)=\arctan x$在$x=0$处的幂级数展开。

解:$f'(x)=(\arctan x)'=\dfrac{1}{1+x^2}=\dfrac{1}{1-(-x^2)}=\sum\limits_{n=0}^\infty(-1)^nx^{2n}$,$\vert-x^2\vert<1$。

已经求得求导后的函数的幂级数展开,所以求原函数的幂级数展开只需要积分,利用先导后积公式:$f(x)=f(0)+\int_0^xf'(t)\,\textrm{d}t=\int_0^x\sum\limits_{n=0}^\infty(-1)^nt^{2n}\,\textrm{d}t=\sum\limits_{n=0}^\infty(-1)^n\dfrac{t^{2n+1}}{2n+1}\bigg|_0^x=\sum\limits_{n=0}^\infty(-1)^n\dfrac{x^{2n+1}}{2n+1}$。

求导的级数要求$\vert x\vert<1$,代入$x=\pm1$到最后结果得到两个交错级数,所以收敛域其实为$[-1,1]$(可以不写)。

\subsection{级数求和}

即对展开式进行逆运算,根据幂级数展开式反推原幂级数。

可以利用展开式求和函数,但是很多展开式的通项都不是公式中的,就需要对通项进行变形。

在求和之前要先计算收敛半径和收敛域。

无论是哪个方法都要求求导和积分后系数$n+a$与幂次$n$相等,所以求导或积分的目的就是为了让他们相同,从而能被看成一个整体。

\subsubsection{先导后积}

$\sum\dfrac{x^{f(n)}}{P(n)}$:$n$在分母上,先导后积。使用变限积分:$\int_{x_0}^xS'(t)\,\textrm{d}t=S(x)-S(x_0)$,即$S(x)=S(x_0)+\int_{x_0}^xS'(t)\,\textrm{d}t$。一般选择$x_0$为展开点。

主要公式:$\sum\limits_{n=1}^\infty\dfrac{(-1)^{n-1}}{n}x^n=\ln(1+x)$($(-1,1]$);$\sum\limits_{n=1}^\infty\dfrac{x^n}{n}=-\ln(1-x)$($[-1,1)$)。

目的是让$P(n)=f(n)$。

% \textbf{例题:}求级数$\sum\limits_{n=1}^\infty\dfrac{x^n}{n}$的和函数。

% 解:已知$\sum\limits_{n=0}^\infty x^n=\dfrac{1}{1-x}$,而这里求和是$\dfrac{x^n}{n}$,所以需要对其进行转换。

% 对$\dfrac{x^n}{n}$求导就得到了$x^{n-1}$消去了分母的$n$,所以使用先导后积的方法。

% 记$S(x)=\sum\limits_{n=1}^\infty\dfrac{x^n}{n}$,则$x^n=(x-0)^n$,取$x_0=0$。

% $\therefore S(x)=S(0)+\displaystyle{\int_0^x\left(\sum\limits_{n=1}^\infty\dfrac{t^n}{n}\right)_t'\,\textrm{d}t}=0+\int_0^x(\sum\limits_{n=1}^\infty t^{n-1})\,\textrm{d}t=\displaystyle{\int_0^x\dfrac{1}{1-t}\textrm{d}t}=-\ln(1-x)$。收敛域为$[-1,1)$。

\textbf{例题:}求级数$\sum\limits_{n=0}^\infty\dfrac{(-1)^n}{2n+1}x^{2n}$的和函数。

解:首先$\lim\limits_{n\to\infty}\left\vert\dfrac{a_{n+1}}{a_n}\right\vert=1$,$R=1$。

当$x=\pm1$时,$x^2n=1$,所以原式$=\sum\limits_{n=0}^\infty\dfrac{(-1)^n}{2n+1}$,为交错级数,由莱布尼茨判别法可知极限为0且单调递减,从而该级数收敛。从而收敛域为$[-1,1]$。

令$S(x)=\sum\limits_{n=0}^\infty\dfrac{(-1)^n}{2n+1}x^{2n}$。易得$x=0$时$S(x)=1$。

当$x\neq0$时,$S(x)=\dfrac{1}{x}\sum\limits_{n=0}^\infty\dfrac{(-1)^n}{2n+1}x^{2n+1}=\dfrac{1}{x}\sum\limits_{n=0}^\infty\int_0^x(-1)^nx^{2n}\,\textrm{d}x\\=\dfrac{1}{x}\int_0^x[\sum\limits_{n=0}^\infty(-1)^nx^{2n}]\textrm{d}x$。

所以$(-1)^nx^{2n}$为一个几何级数,所以$q=\dfrac{(-1)^{n+1}x^{2n+2}}{(-1)^nx^{2n}}=-x^2$。

从而$=\displaystyle{\dfrac{1}{x}\int_0^x\dfrac{1}{1+x^2}\textrm{d}x=\dfrac{\arctan x}{x}}$。

\subsubsection{先积后导}

$\sum P(n)x^{f(n)}$:$n$在分子上,先积后导。$(\int S(x)\,\textrm{d}x)'=S(x)$。

主要公式:$\sum\limits_{n=0}^\infty(-1)^nx^n=\dfrac{1}{1+x}$($(-1,1)$);$\sum\limits_{n=0}^\infty x^n=\dfrac{1}{1-x}$($(-1,1)$)。

目的是让$P(n)=f^{(n)}(n)\cdots f'(n)$。

\textbf{例题:}求级数$\sum\limits_{n=1}^\infty nx^n$的和函数。

解:记$S(x)=\sum\limits_{n=1}^\infty nx^n=x\sum\limits_{n=1}^\infty x^{n-1}=x(\int\sum\limits_{n=1}^\infty nx^{n-1}\,\textrm{d}x)'=x(\sum\limits_{n=1}^\infty x^n)'=x\left(\dfrac{x}{1-x}\right)'=\dfrac{x}{(1-x)^2}$。收敛域为$[-1,1]$。

\textbf{例题:}求级数$\sum\limits_{n=0}^\infty(n+1)(n+3)x^n$的和函数。

解:$(n+1)(n+3)$的形式可以推出$(n+1)(n+2)$是求两次导的结果,而这里是$(n+1)(n+3)$,所以拆开:$\sum\limits_{n=0}^\infty(n+1)(n+2)x^n+\sum\limits_{n=0}^\infty(n+1)x^n=\left(\sum\limits_{n=0}^\infty x^{n+2}\right)''+\left(\sum\limits_{n=0}^\infty x^{n+1}\right)=\left(\dfrac{x^2}{1-x}\right)''+\left(\dfrac{x}{1-x}\right)'=\dfrac{3-x}{(1-x)^3}$,$x\in(-1,1)$。

\section{傅里叶级数}

\end{document}
